\documentclass[11pt,headsepline,a4paper]{scrreprt}
%onehalfspacing
\usepackage[utf8]{inputenc} 
\usepackage[german]{babel}
\usepackage[T1]{fontenc} %Schöner trennen
\usepackage{amsmath}
\usepackage{amsfonts}
\usepackage{amssymb}
\usepackage{makeidx}
\usepackage{graphicx}
\usepackage{lmodern}
\usepackage{float}
%\usepackage{kpfonts}

\usepackage{longtable}
\usepackage{svg}
\usepackage{listings,xcolor}
\usepackage{inconsolata}

\usepackage[colorlinks,
pdfpagelabels,
pdfstartview = FitH,
bookmarksopen = true,
bookmarksnumbered = true,
linkcolor = black,
plainpages = false,
hypertexnames = false,
citecolor = black] {hyperref}%Paket/Befehl um verlinkungen im Inhaltsverzeichnis zu bekommen

\usepackage{scrlayer-scrpage}
\automark{chapter}
\automark*{section}

\definecolor{dkgreen}{rgb}{0,.6,0}
\definecolor{dkblue}{rgb}{0,0,.6}
\definecolor{dkyellow}{cmyk}{0,0,.8,.3}

\lstdefinelanguage{JavaScript}{
	keywords={typeof, new, true, false, catch, function, return, null, catch, switch, var, if, in, while, do, else, case, break},
	keywordstyle    = \color{dkblue},
	ndkeywords={class, export, boolean, throw, implements, import, this},
	ndkeywordstyle=\color{darkgray},
	identifierstyle=\color{dkgreen},
	sensitive=false,
	comment=[l]{//},
	morecomment=[s]{/*}{*/},
	commentstyle    = \color{gray},
	stringstyle     = \color{red},
	morestring=[b]',
	morestring=[b]"
}

\lstset{
	language        = php,
	basicstyle      = \small\ttfamily,
	keywordstyle    = \color{dkblue},
	stringstyle     = \color{red},
	identifierstyle = \color{dkgreen},
	commentstyle    = \color{gray},
	emph            =[1]{php},
	emphstyle       =[1]\color{black},
	emph            =[2]{if,and,or,else},
	emphstyle       =[2]\color{dkyellow},
	numbers=left,
	stepnumber=1,  
	numberfirstline=true,
	numberstyle=\footnotesize,
	xleftmargin=4.0ex,
	upquote=true,
	showlines=true,
	extendedchars=true,
	literate={ä}{{\"a}}1 {ö}{{\"o}}1 {ü}{{\"u}}1 {Ä}{{\"A}}1 {Ö}{{\"O}}1 {Ü}{{\"U}}1,
	breaklines=true,
	postbreak=\mbox{\textcolor{red}{$\hookrightarrow$}\space},
}

\begin{document}
\setcounter{tocdepth}{3}
\title{Dokumentation  \\ \normalsize{Citizen Science Platform {\glqq COSP\grqq}} }
\date{\today}
\author{Projektteam \\
	\normalsize{
		\begin{tabular}[t]{ll}
			Version:  & \quad Beta 1.5.0 \\[1.2ex]
		\end{tabular}
	}
}
\maketitle

%Titelseite
%Inhaltsverzeichnis
\newpage
\tableofcontents
\thispagestyle{empty}
\newpage
\setcounter{page}{1}
\chapter{Rahmenbedingungen}


\section{Technische Rahmenbedingungen}
Die Website soll, aufgrund der verschiedenen Bedingungen der Anwender, mindestens auf folgenden Browsern fehlerfrei dargestellt werden:
\begin{itemize}
\item Mozilla Firefox (Version 66.0.3)
\item Edge (Version 44.17763.1.0)
\item Google Chrome (74.0.3729.108)
\item Apple Safari (12 (13606.2.11))
\end{itemize}
\vspace{\baselineskip}
Als Server-Betriebssystem wird Debian im Stable-Release verwendet. Desweiteren wird unter dem genannten Betriebssystem MariaDB als Datenbank und PHP7 als Plugin für einen aktuellen Apache2 Webserver verwendet.

\section{Entwicklungsumgebung}
Als Entwicklungsumgebung dient GitLab. Die Dokumentation wird automatisiert aus den Sources mittels der Gitlab-CI und zwei Gitlabrunnern erstellt.

\section{Verwendete Software}
Es wird Bootstrap 4 eingesetzt.

\chapter{Farbkonventionen}
Um die Nutzung unserer Website angenehm zu gestalten, haben wir uns für schlichte und gedeckte Farben entschieden. Die Navigationsleiste ist dunkel mit weißer Schrift.


\chapter{Konfiguration}\label{chapter:config}
\section{Grundlegendes}
In der Konfigurationsdatei, welche im Ordner {\glqq bin\grqq} zu finden ist, können alle wichtigen Parameter der Website eingestellt werden. Hierzu ist im ersten Schritt die Datei mit dem Pfad {\glqq bin/config-sample.php\grqq} nach {\glqq bin/config.php\grqq} zu kopieren. Anschließend sind alle entsprechenden Einstellungen zur Kommunikation mit der Datenbank und zum Mailversand zu treffen. Hiernach sollten alle wichtigen Parameter bereits korrekt vordefiniert sein.
\section{Parameter}
\subsection{Auflistung}
In Nachfolgender Tabelle sind alle Konfigurationsparameter mit einer entsprechenden Kurzbeschreibung zusammengefasst. Im Anschluss an die Tabelle finden Sie eine ausführliche Beschreibung.
\begin{longtable}[H]{|c|l|}
		\hline
		\textbf{Parameter}   & \textbf{Kurzbeschreibung}                                                                                                     \\ \hline
		\$SQL\_SERVER        & Adresse des SQL-Servers                                                                                                     \\ \hline	
		\$SQL\_USER          & Nutzername am SQL-Server                                                                                                     \\ \hline	
		\$SQL\_PASSWORD      & Passwort am SQL-Server                                                                                                     \\ \hline	
		\$SQL\_SCHEMA        & Schema auf dem SQL-Servers                                                                                                     \\ \hline	
		\$SQL\_PREFIX        & Tabellennamenprefix im Schema auf dem SQL-Servers                                                                                                     \\ \hline	
		\$SQL\_Connector     & Angabe des zu nutzenden SQL-Connectors                                                                                                     \\ \hline	
		\$BASE\_DMN          & Grunddomain des Webservers                                                                                                     \\ \hline	
		\$SELF\_REGISTRATION & Freischaltung der Selbstregistrierung                                                                                                     \\ \hline	
		\$MAIN\_CAPTION      & Hauptschriftzug oder Namenskürzel der Plattform                                                                                                     \\ \hline	
		\$TAGLINE\_CAPTION   & Vollständiger Name der Plattform                                                                                                     \\ \hline	
		\$DEBUG              & Versetzt Plattform in Debug-Modus                                                                                                     \\ \hline	
		\$DEBUG\_LEVEL       & Tiefe der Debug-Ausgaben                                                                                                     \\ \hline	
		\$PWD\_LENGTH        & Mindestpasswortlänge                                                                                                     \\ \hline	
		\$PWD\_ALGORITHM     & Algorithmus zur Verschlüsselung neuer Passwörter                                                                                                     \\ \hline	
		\$RANDOM\_STRING\_LENGTH     & Länge zufallsgenerierter Zeichenketten                                                                                                     \\ \hline	
		\$DOMAIN            & Domain der Plattform                                                                                                     \\ \hline	
		\$SENDER\_ADDRESS   & Von-Adresse versendeter Mails                                                                                                     \\ \hline	
		\$HMAC\_SECRET      & Geheimnis für diverse kryptographische Funktionen                                                                                                     \\ \hline	
		\$UPLOAD\_DIR       & Unterverzeichnis in das hochgeladene Bilder gespeichert werden                                                                                                   \\ \hline	
		\$ZENTRAL\_MAIL     & Mailadresse an die Administrationsmails versendet werden                                                                                                     \\ \hline	
		\$ROLE\_GUEST       & Rollenwert den ein Gast benötigt                                                                                                    \\ \hline	
		\$ROLE\_UNAUTH\_USER         & Rollenwert den ein unauthentifizierter Nutzer benötigt                                                                                                    \\ \hline	
		\$ROLE\_AUTH\_USER           & Rollenwert den ein authentifizierter Nutzer benötigt                                                                                                    \\ \hline	
		\$ROLE\_EMPLOYEE             & Rollenwert den ein Mitarbeiter benötigt                                                                                                    \\ \hline	
		\$ROLE\_ADMIN                & Rollenwert den ein Administrator benötigt                                                                                                    \\ \hline	
		\$BETA              & Versetzt Plattform in Beta-Modus                                                                                                     \\ \hline	
		\$MAINTENANCE       & Versetzt Plattform in Wartungsmodus                                                                                                     \\ \hline	
		\$IMPRESSUM\_NAME   & Name des Inhaltsverantwortlichen                                                                                                     \\ \hline
		\$IMPRESSUM\_STREET & Straße und Hausnummer der Adresse des Inhaltsverantwortlichen                                                                                                     \\ \hline	
		\$IMPRESSUM\_CITY   & Stadt und Postleitzahl der Adresse des Inhaltsverantwortlichen                                                                                                     \\ \hline	
		\$SPECIAL\_CHARS\_CAPTCHA    & Schaltet Sonderzeichen für Captcha-Generator ein                                                                                                     \\ \hline
		\$PRIVACY\_COMPANY\_NAME  & Name der Firma/Organisation                                                                                                     \\ \hline
		\$PRIVACY\_COMPANY\_STREET & Straße und Hausnummer der Adresse der Firma/Organisation                                                                                                     \\ \hline	
		\$PRIVACY\_COMPANY\_CITY   & Stadt und Postleitzahl der Adresse der Firma/Organisation                                                                                                     \\ \hline	
		\$PRIVACY\_COMPANY\_FON    & Telefonnummer der Firma/Organisation                                                                                                     \\ \hline	
		\$PRIVACY\_COMPANY\_FAX    & Faxnummer der Firma/Organisation                                                                                                     \\ \hline	
		\$PRIVACY\_COMPANY\_MAIL   & Mailadresse der Firma/Organisation                                                                                                     \\ \hline	
		\$PRIVACY\_REP\_NAME      & Name des Datenschutzbeauftragten                                                                                                     \\ \hline
		\$PRIVACY\_REP\_POS       & Position der Datenschutzbeauftragten 																										\\ \hline
		\$PRIVACY\_REP\_STREET    & Straße und Hausnummer der Adresse des Datenschutzbeauftragten                                                                                                     \\ \hline	
		\$PRIVACY\_REP\_CITY      & Stadt und Postleitzahl der Adresse des Datenschutzbeauftragten                                                                                                     \\ \hline	
		\$PRIVACY\_REP\_FON       & Telefonnummer des Datenschutzbeauftragten                                                                                                     \\ \hline	
		\$PRIVACY\_REP\_FAX       & Faxnummer des Datenschutzbeauftragten                                                                                                     \\ \hline	
		\$PRIVACY\_REP\_MAIL      & Mailadresse des Datenschutzbeauftragten                                                                                                     \\ \hline		
		\$DIRECT\_DELETE          & Daten direkt löschen                                                                                                     \\ \hline
		\$ADDITIONAL\_MAIL\_PARAM & Zusätzlich Parameter zum Senden einer Mail 	                                                                             \\ \hline
\end{longtable}
\newpage
\subsection{Parameterbeschreibung}
\subsubsection{\$SQL\_SERVER}
\paragraph{Beschreibung}Dieser Parameter setzt die Adresse des zu nutzenden SQL-Servers. Hier sollte daher eine Valide URI oder IP zu einem MySQL- oder MariaDB-Server angegeben werden. Es wird ein MariaDB-Server bevorzugt.
\paragraph{Standardwert}Es gibt keinen Standardwert.

\subsubsection{\$SQL\_USER}
\paragraph{Beschreibung}Dieser Parameter setzt den am SQL-Server zu nutzenden Nutzernamen.
\paragraph{Standardwert}Es gibt keinen Standardwert.

\subsubsection{\$SQL\_PASSWORD}
\paragraph{Beschreibung}Dieser Parameter setzt das am SQL-Server zu nutzenden Passwort für den angegebenen Nutzernamen. 
\paragraph{Standardwert}Es gibt keinen Standardwert.

\subsubsection{\$SQL\_SCHEMA}
\paragraph{Beschreibung}Dieser Parameter setzt das am SQL-Server zu nutzende Schema.
\paragraph{Standardwert}Es gibt keinen Standardwert.

\subsubsection{\$SQL\_PREFIX}
\paragraph{Beschreibung}Dieser Parameter setzt das vor jedem Tabellennamen stehende Prefix. Es sollte üblicherweise mit einem {\glqq \_\grqq} enden.
\paragraph{Standardwert}Der Standartwert ist {\glqq dload\_\_\grqq}.

\subsubsection{\$SQL\_Connector}
\paragraph{Beschreibung}Dieser Parameter setzt den durch PHP zu nutzenden SQL-Connector. Aktuell ist nur PHP-PDO als Connector implementiert. 
\paragraph{Parameterwert}Es kann einer der folgenden Werte eingetragen werden: {\glqq pdo\grqq}.
\paragraph{Standardwert}Der Standartwert ist {\glqq pdo\grqq}.

\subsubsection{\$BASE\_DMN}
\paragraph{Beschreibung}Dieser Parameter gibt die Basisdomain an, auf welcher dise Plattform erreichbar ist.
\paragraph{Parameterwert}Sei {\glqq plattform.beliebieg.tld\grqq} die Domain dieser Plattform, dann ist der Wert für diesen Parameter {\glqq .beliebieg.tld\grqq}.
\paragraph{Standardwert}Es gibt keinen Standardwert.

\subsubsection{\$SELF\_REGISTRATION}
\paragraph{Beschreibung}Dieser Parameter gibt an, ob die Selbstregistrierung möglich ist.
\paragraph{Parameterwert}Es kann einer der folgenden Werte eingetragen werden: {\glqq true\grqq}, {\glqq false\grqq}.
\paragraph{Standardwert}Der Standartwert ist {\glqq true\grqq}.

\subsubsection{\$MAIN\_CAPTION}
\paragraph{Beschreibung}Dieser Parameter setzt den Haupttitel der Plattform. Dieser ist üblicherweise eine Abkürzung.
\paragraph{Standardwert}Der Standartwert ist {\glqq COSP\grqq}.

\subsubsection{\$TAGLINE\_CAPTION}
\paragraph{Beschreibung}Dieser Parameter setzt den vollständigen Titel der Plattform. Dieser ist üblicherweise keine Abkürzung.
\paragraph{Standardwert}Der Standartwert ist {\glqq Citizen Open Science Plattform\grqq}.

\subsubsection{\$DEBUG}
\paragraph{Beschreibung}Dieser Parameter kann den Debug-Modus der Plattform aktivieren. Im Produktiveinsatz sollte der Wert des Parameters stets {\glqq false\grqq} sein.
\paragraph{Parameterwert}Es kann einer der folgenden Werte eingetragen werden: {\glqq true\grqq}, {\glqq false\grqq}.
\paragraph{Standardwert}Der Standartwert ist {\glqq true\grqq}.

\subsubsection{\$DEBUG\_LEVEL} \label{config:debug-level}
\paragraph{Beschreibung}Dieser Parameter gibt die Anzahl und tiefe der Debug-Ausgaben an. Es ist ein Integerwert zu wählen. Dieser Parameter ist nur bei aktivierten Debug-Modus von Relevanz. Bei einem Wert über 3 können API-Anfragen möglicherweise Debug-Ausgaben beinhalten und daher nicht vom Anfragenden System verarbeitet werden.
\paragraph{Parameterwert}Es kann eine Werte zwischen 0 und 8 eingetragen werden.
\paragraph{Standardwert}Der Standartwert ist {\glqq 3\grqq}.

\subsubsection{\$PWD\_LENGTH}
\paragraph{Beschreibung}Dieser Parameter gibt die Mindestlänge des Passwortes an. Er sollte stets größer 3 sein.
\paragraph{Parameterwert}Es kann ein Wert der Natürlichen Zahlen, welcher größer 3 ist angegeben werden.
\paragraph{Standardwert}Der Standartwert ist {\glqq 8\grqq}.

\subsubsection{\$PWD\_ALGORITHM}
\paragraph{Beschreibung}Dieser Parameter gibt zum Passwort speichern zu verwenden Hashalgorithmus an. 
\paragraph{Parameterwert}Es kann einer der folgenden Werte eingetragen werden: {\glqq PASSWORD\_DEFAULT\grqq}, {\glqq PASSWORD\_BCRYPT\grqq}, {\glqq PASSWORD\_ARGON2I\grqq}, {\glqq PASSWORD\_ARGON2ID\grqq}.
\paragraph{Standardwert}Der Standartwert ist {\glqq PASSWORD\_ARGON2ID\grqq}. Unter PHP 7.2 sollte als Standartwert {\glqq PASSWORD\_ARGON2I\grqq} gewählt werden.

\subsubsection{\$RANDOM\_STRING\_LENGTH}
\paragraph{Beschreibung}Dieser Parameter gibt die Länge zufallsgenerierter Zeichenketten an.
\paragraph{Parameterwert}Es kann ein Wert der Natürlichen Zahlen, welcher größer 20 ist angegeben werden.
\paragraph{Standardwert}Der Standartwert ist {\glqq 170\grqq}.

\subsubsection{\$DOMAIN}
\paragraph{Beschreibung}Dieser Parameter setzt die Domain, unter welcher die Plattform im Internet mittels eines geeigneten Webbrowsers aufrufbar ist.
\paragraph{Standardwert}Es gibt keinen Standardwert.

\subsubsection{\$SENDER\_ADDRESS}
\paragraph{Beschreibung}Dieser Parameter setzt die Absenderadresse für Systemmails
\paragraph{Standardwert}Es gibt keinen Standardwert.

\subsubsection{\$HMAC\_SECRET} \label{config:hmac-secret}
\paragraph{Beschreibung}Dieser Parameter setzt das genutzte Geheimnis für multiple kryptographische Funktionen.
\paragraph{Standardwert}Es gibt keinen Standardwert.

\subsubsection{\$UPLOAD\_DIR}
\paragraph{Beschreibung}Dieser Parameter setzt das Verzeichnis zum Speichern von hochgeladenen Bildern.
\paragraph{Parameterwert}Es kann jeder beliebige Unterpfad im Webverzeichnis hier relativ zum Wurzelverzeichnis der Website angegeben werden. Der Pfad darf nicht absolut sein.
\paragraph{Standardwert}Der Standartwert ist {\glqq images/uploadMat\grqq}.

\subsubsection{\$ZENTRAL\_MAIL} \label{config:zentral-mail}
\paragraph{Beschreibung}Dieser Parameter Mailadresse an welche Mails Systemmails für administrative Anfragen gesendet werden. Diese Mailadresse erhält auch alle Anfragen, welche über das Kontaktformular eingesendet werden.
\paragraph{Standardwert}Es gibt keinen Standardwert.

\subsubsection{\$ROLE\_GUEST}
\paragraph{Beschreibung}Dieser Parameter setzt den mindest Rollenwert für den Gastnutzer.
\paragraph{Standardwert}Der Standartwert ist {\glqq 0\grqq}.

\subsubsection{\$ROLE\_UNAUTH\_USER}
\paragraph{Beschreibung}Dieser Parameter setzt den mindest Rollenwert für den unautorisierten Nutzer.
\paragraph{Standardwert}Der Standartwert ist {\glqq 1\grqq}.

\subsubsection{\$ROLE\_AUTH\_USER}
\paragraph{Beschreibung}Dieser Parameter setzt den mindest Rollenwert für den autorisierten Nutzer.
\paragraph{Standardwert}Der Standartwert ist {\glqq 2\grqq}.

\subsubsection{\$ROLE\_EMPLOYEE}
\paragraph{Beschreibung}Dieser Parameter setzt den mindest Rollenwert für einen Mitarbeiter.
\paragraph{Standardwert}Der Standartwert ist {\glqq 10\grqq}.

\subsubsection{\$ROLE\_ADMIN}
\paragraph{Beschreibung}Dieser Parameter setzt den mindest Rollenwert für einen Administrator.
\paragraph{Standardwert}Der Standartwert ist {\glqq 20\grqq}.

\subsubsection{\$BETA}
\paragraph{Beschreibung}Dieser Parameter versetzt die Plattform in den Beta-Modus. Der Zugriff auf die Plattform ist dann nur noch mit einem entsprechenden Link möglich. Grundsätzlich ist für einen Zugriff auf die Website das Suffix {\glqq ?b=0\grqq} an die im Parameter {\glqq \$DOMAIN\grqq} genannte URI anzuhängen.
\paragraph{Parameterwert}Es kann einer der folgenden Werte eingetragen werden: {\glqq true\grqq}, {\glqq false\grqq}.
\paragraph{Standardwert}Der Standartwert ist {\glqq false\grqq}.

\subsubsection{\$MAINTENANCE}
\paragraph{Beschreibung}Dieser Parameter versetzt die Plattform in den Wartungsmodus. Der Zugriff auf die Plattform ist dann nur noch mit einem entsprechenden Link möglich. Grundsätzlich ist für einen Zugriff auf die Website das Suffix {\glqq ?m=0\grqq} an die im Parameter {\glqq \$DOMAIN\grqq} genannte URI anzuhängen.
\paragraph{Parameterwert}Es kann einer der folgenden Werte eingetragen werden: {\glqq true\grqq}, {\glqq false\grqq}.
\paragraph{Standardwert}Der Standartwert ist {\glqq false\grqq}.

\subsubsection{\$IMPRESSUM\_NAME} \label{config:impressum-name}
\paragraph{Beschreibung}Dieser Parameter setzt den Name des Inhaltsverantwortlichen der Plattform im Impressum.
\paragraph{Standardwert}Es gibt keinen Standardwert.

\subsubsection{\$IMPRESSUM\_STREET} \label{config:impressum-street}
\paragraph{Beschreibung}Dieser Parameter setzt den Straßenname und die Hausnummer der Adresse des Inhaltsverantwortlichen im Impressum.
\paragraph{Standardwert}Es gibt keinen Standardwert.

\subsubsection{\$IMPRESSUM\_CITY} \label{config:impressum-city}
\paragraph{Beschreibung}Dieser Parameter setzt den Ort und die Postleitzahl der Adresse des Inhaltsverantwortlichen im Impressum.
\paragraph{Standardwert}Es gibt keinen Standardwert.

\subsubsection{\$SPECIAL\_CHARS\_CAPTCHA}
\paragraph{Beschreibung}Dieser Parameter ermöglicht die Nutzung von Sonderzeichen in Captcha-Codes.
\paragraph{Parameterwert}Es kann einer der folgenden Werte eingetragen werden: {\glqq true\grqq}, {\glqq false\grqq}.
\paragraph{Standardwert}Der Standartwert ist {\glqq true\grqq}.

\subsubsection{\$PRIVACY\_COMPANY\_NAME} \label{config:privacy-comp-name}
\paragraph{Beschreibung}Dieser Parameter setzt den Name der Firma oder Organisation der Plattform in der Datenschutzerklärung.
\paragraph{Standardwert}Es gibt keinen Standardwert.

\subsubsection{\$PRIVACY\_COMPANY\_STREET} \label{config:privacy-comp-street}
\paragraph{Beschreibung}Dieser Parameter setzt den Straßennamen und die Hausnummer der Adresse der Firma oder Organisation der Plattform in der Datenschutzerklärung.
\paragraph{Standardwert}Es gibt keinen Standardwert.

\subsubsection{\$PRIVACY\_COMPANY\_CITY} \label{config:privacy-comp-city}
\paragraph{Beschreibung}Dieser Parameter setzt den Ortsnamen und die Postleitzahl der Adresse der Firma oder Organisation der Plattform in der Datenschutzerklärung.
\paragraph{Standardwert}Es gibt keinen Standardwert.

\subsubsection{\$PRIVACY\_COMPANY\_FON} \label{config:privacy-comp-fon}
\paragraph{Beschreibung}Dieser Parameter setzt die Telefonnummer der Firma oder Organisation der Plattform in der Datenschutzerklärung.
\paragraph{Standardwert}Es gibt keinen Standardwert.

\subsubsection{\$PRIVACY\_COMPANY\_FAX} \label{config:privacy-comp-fax}
\paragraph{Beschreibung}Dieser Parameter setzt die Faxnummer der Firma oder Organisation der Plattform in der Datenschutzerklärung.
\paragraph{Standardwert}Es gibt keinen Standardwert.

\subsubsection{\$PRIVACY\_COMPANY\_MAIL} \label{config:privacy-comp-mail}
\paragraph{Beschreibung}Dieser Parameter setzt die Mailadresse der Firma oder Organisation der Plattform in der Datenschutzerklärung.
\paragraph{Standardwert}Es gibt keinen Standardwert.

\subsubsection{\$PRIVACY\_REP\_NAME} \label{config:privacy-rep-name}
\paragraph{Beschreibung}Dieser Parameter setzt den Name des Datenschutzverantwortlichen der Plattform in der Datenschutzerklärung.
\paragraph{Standardwert}Es gibt keinen Standardwert.

\subsubsection{\$PRIVACY\_REP\_POS}\label{config:privacy-rep-pos}
\paragraph{Beschreibung}Dieser Parameter setzt die Position des Datenschutzverantwortlichen der Plattform in der Datenschutzerklärung.
\paragraph{Standardwert}Es gibt keinen Standardwert.

\subsubsection{\$PRIVACY\_REP\_STREET} \label{config:privacy-rep-street}
\paragraph{Beschreibung}Dieser Parameter setzt den Straßennamen und die Hausnummer der Adresse des Datenschutzverantwortlichen der Plattform in der Datenschutzerklärung.
\paragraph{Standardwert}Es gibt keinen Standardwert.

\subsubsection{\$PRIVACY\_REP\_CITY} \label{config:privacy-rep-city}
\paragraph{Beschreibung}Dieser Parameter setzt den Ortsnamen und die Postleitzahl der Adresse des Datenschutzverantwortlichen der Plattform in der Datenschutzerklärung.
\paragraph{Standardwert}Es gibt keinen Standardwert.

\subsubsection{\$PRIVACY\_REP\_FON} \label{config:privacy-rep-fon}
\paragraph{Beschreibung}Dieser Parameter setzt die Telefonnummer des Datenschutzverantwortlichen der Plattform in der Datenschutzerklärung.
\paragraph{Standardwert}Es gibt keinen Standardwert.

\subsubsection{\$PRIVACY\_REP\_FAX} \label{config:privacy-rep-fax}
\paragraph{Beschreibung}Dieser Parameter setzt die Faxnummer des Datenschutzverantwortlichen der Plattform in der Datenschutzerklärung.
\paragraph{Standardwert}Es gibt keinen Standardwert.

\subsubsection{\$PRIVACY\_REP\_MAIL} \label{config:privacy-rep-mail}
\paragraph{Beschreibung}Dieser Parameter setzt die Mailadresse des Datenschutzverantwortlichen der Plattform in der Datenschutzerklärung.
\paragraph{Standardwert}Es gibt keinen Standardwert.

\subsubsection{\$DIRECT\_DELETE}\label{config:direct-delete}
\paragraph{Beschreibung}Dieser Parameter legt fest, ob Daten direkt gelöscht werden oder nur als gelöscht markiert werden.
\paragraph{Parameterwert}Es kann einer der folgenden Werte eingetragen werden: {\glqq true\grqq} (direkt löschen), {\glqq false\grqq} (markieren).
\paragraph{Standardwert}Der Standartwert ist {\glqq false\grqq}.

\subsubsection{\$ADDITIONAL\_MAIL\_PARAM} \label{config:additional-mail-header}
\paragraph{Beschreibung}Dieser Parameter ermöglicht es, zusätzliche Mail-Header zu setzen.
\paragraph{Parameterwert}Es werden Schlüssel-Werte-Paare erwartet. Zum Beispiel: "CC": "name@domain.tld".
\paragraph{Standardwert}Der Standartwert ist ein Leeres Array.
\subsection{Allgemeines} Diese Datei ist der Endpunkt der Modul-API.
Die Datei ist direkt durch den Nutzer aufrufbar. Sie setzt auch die entsprechende Konstante und bindet alle notwendigen Dateien ein:
\begin{lstlisting}[language=php]
define('NICE_PROJECT', true);
require_once "bin/inc.php";
\end{lstlisting}
\subsection{Allgemeines}
Diese Seite verteilt die API-Anfragen auf die verschiedenen Funktionen auf. Für eine Liste aller möglichen API-Befehle siehe \autoref{api}.
\subsection{Besonderheiten}
Diese Seite bietet keine Graphische Nutzeroberfläche an. Alle Antworten auf Anfragen sind im JSON-Format.

\newpage
\section{Nutzer API-Spezifikation}\label{uapi}
\subsection{Beschreibung}Diese API dient der Kommunikation zwischen einem Nutzer eines Moduls und den Inhalten die durch {\glqq COSP\grqq} bereit gestellt werden. Hauptsächlich unterstützt sie das Abrufen von Bildern und anderen Informationen. Durch diese API können keine Informationen geändert, gelöscht oder neu hinzugefügt werden. Jedoch können Geschichten und Bilder validiert werden. Diese API wird durch ein HMAC-Verfahren geschützt.
\subsection{Befehlsübersicht}
\begin{longtable}[H]{|c|p{12cm}|}
		\hline
		\textbf{Api-Befehl} & \textbf{Kurzbeschreibung}              \\ \hline
		gpp                 & Vollbild laden          \\ \hline
		gpf                 & Vorschaubild laden            \\ \hline
		gus                 & Geschichte laden \\ \hline
		gas                 & Alle Geschichten einer Liste laden \\ \hline
		vas                 & Geschichte Validieren \\ \hline
		vap                 & Bild validieren \\ \hline
\end{longtable}
\newpage
\subsection{Befehle}
\subsubsection{Vollbildladen}
\paragraph{Kurzbeschreibung}Dieser API-Request wird dazu genutzt um ein Vollbild zu laden.
\paragraph{Anfrage}Folgende Daten werden zu Anfrage benötigt:
\begin{table}[H]
	\begin{tabular}{|c|c|c|p{6.5cm}|}
		\hline
		\textbf{Paramtername} & \textbf{Datentyp} & \textbf{Konstante} & \textbf{Kurzbeschreibung}                                                                                               \\ \hline
		type                & string            & gpp                & Vollbild abrufen \\ \hline
		data                & string            &                    & Token \\ \hline
		seccode             & string            &                    & Security Code \\ \hline
		time                & int               &                    & Timestamp \\ \hline
	\end{tabular}
\end{table}
\paragraph{Antwort}Die Antwort ist das Bild mit einem entsprechendem Header.
\subsubsection{Vorschaubild laden}
\paragraph{Kurzbeschreibung}Dieser API-Request wird dazu genutzt um ein Vorschaubild zu laden.
\paragraph{Anfrage}Folgende Daten werden zu Anfrage benötigt:
\begin{table}[H]
	\begin{tabular}{|c|c|c|p{6.5cm}|}
		\hline
		\textbf{Paramtername} & \textbf{Datentyp} & \textbf{Konstante} & \textbf{Kurzbeschreibung}                                                                                               \\ \hline
		type                & string            & gpf                & Vorschaubild abrufen \\ \hline
		data                & string            &                    & Token \\ \hline
		seccode             & string            &                    & Security Code \\ \hline
		time                & int               &                    & Timestamp \\ \hline
	\end{tabular}
\end{table}
\paragraph{Antwort}Die Antwort ist das Bild mit einem entsprechendem Header.

\subsubsection{Geschichte laden}
\paragraph{Kurzbeschreibung}Dieser API-Request wird dazu genutzt um eine einzelne Geschichte zu laden.
\paragraph{Anfrage}Folgende Daten werden zu Anfrage benötigt:
\begin{table}[H]
	\begin{tabular}{|c|c|c|p{6.5cm}|}
		\hline
		\textbf{Paramtername} & \textbf{Datentyp} & \textbf{Konstante} & \textbf{Kurzbeschreibung}                                                                                               \\ \hline
		type                & string            & gus                & Geschichte abrufen \\ \hline
		data                & string            &                    & Token \\ \hline
		seccode             & string            &                    & Security Code \\ \hline
		time                & int               &                    & Timestamp \\ \hline
	\end{tabular}
\end{table}
\paragraph{Antwort}Die Antwort ist wie folgt aufgebaut:
\begin{table}[H]
	\begin{tabular}{|c|c|c|p{6.5cm}|}
		\hline
		\textbf{Paramtername} & \textbf{Datentyp} & \textbf{Konstante} & \textbf{Kurzbeschreibung}            \\ \hline                
		result              & string           &                 & Erfolgreich wenn Wert {\glqq ack\grqq} ist \\ \hline
		Code                & int              &                 & Erfolgreich wenn Wert {\glqq 0\grqq} ist \\ \hline
		data                & array            &                 & Abgefragter Inhalt \\ \hline
	\end{tabular}
\end{table}
\subparagraph{data}Dieses Array enthält Einträge in der nachstehend dargestellten Form haben:
\begin{table}[H]
	\begin{tabular}{|c|c|c|p{6.5cm}|}
		\hline
		\textbf{Paramtername} & \textbf{Datentyp} & \textbf{Konstante} & \textbf{Kurzbeschreibung}    \\ \hline
		token              & string            &                 & Identifikator der Geschichte \\ \hline
		story              & string            &                 & Inhalt der Geschichte \\ \hline
		title              & string            &                 & Titel der Geschichte \\ \hline
		name               & string            &                 & Nutzername des Erstellers \\ \hline
		date               & timestamp         &                 & Erstellungsdatum \\ \hline
		validatedByUser    & bool              &                 & Wahr, wenn Nutzer bereits Geschichte validiert hat \\ \hline
		validate           & bool              &                 & Wahr, Geschichte validiert ist \\ \hline
		valLink            & string            &                 & Validierungslink \\ \hline
		approval           & bool              &                 & Freischaltstatus \\ \hline
		editable           & bool              &                 & Wahr, wenn Geschichte änderbar \\ \hline
		deleted            & bool              &                 & Wahr, wenn Geschichte als gelöscht gilt \\ \hline
	\end{tabular}
\end{table}

\subsubsection{Alle Geschichten einer Liste laden}
\paragraph{Kurzbeschreibung}Dieser API-Request wird dazu genutzt um  alle Geschichten einer Liste zu laden.
\paragraph{Anfrage}Folgende Daten werden zu Anfrage benötigt:
\begin{table}[H]
	\begin{tabular}{|c|c|c|p{6.5cm}|}
		\hline
		\textbf{Paramtername} & \textbf{Datentyp} & \textbf{Konstante} & \textbf{Kurzbeschreibung}                                                                                               \\ \hline
		type                & string            & gas                & Geschichten abrufen \\ \hline
		data                & string            &                    & Token \\ \hline
		seccode             & string            &                    & Security Code \\ \hline
		time                & int               &                    & Timestamp \\ \hline
	\end{tabular}
\end{table}
\paragraph{Antwort}Die ist ein Array mit Elementen, welche folgende Einträge haben:
\begin{table}[H]
	\begin{tabular}{|c|c|c|p{6.5cm}|}
		\hline
		\textbf{Paramtername} & \textbf{Datentyp} & \textbf{Konstante} & \textbf{Kurzbeschreibung}            \\ \hline                
		token              & string            &                 & Identifikator der Geschichte \\ \hline
		story              & string            &                 & Inhalt der Geschichte \\ \hline
		title              & string            &                 & Titel der Geschichte \\ \hline
		name               & string            &                 & Nutzername des Erstellers \\ \hline
		date               & timestamp         &                 & Erstellungsdatum \\ \hline
		validatedByUser    & bool              &                 & Wahr, wenn Nutzer bereits Geschichte validiert hat \\ \hline
		validate           & bool              &                 & Wahr, Geschichte validiert ist \\ \hline
		valLink            & string            &                 & Validierungslink \\ \hline
		approval           & bool              &                 & Freischaltstatus \\ \hline
		editable           & bool              &                 & Wahr, wenn Geschichte änderbar \\ \hline
		deleted            & bool              &                 & Wahr, wenn Geschichte als gelöscht gilt \\ \hline
	\end{tabular}
\end{table}

\subsubsection{Geschichte validieren}
\paragraph{Kurzbeschreibung}Dieser API-Request wird dazu genutzt um eine einzelne Geschichte zu validieren.
\paragraph{Anfrage}Folgende Daten werden zu Anfrage benötigt:
\begin{table}[H]
	\begin{tabular}{|c|c|c|p{6.5cm}|}
		\hline
		\textbf{Paramtername} & \textbf{Datentyp} & \textbf{Konstante} & \textbf{Kurzbeschreibung}                                                                                               \\ \hline
		type                & string            & vas                & Geschichte validieren \\ \hline
		data                & string            &                    & Token \\ \hline
		seccode             & string            &                    & Security Code \\ \hline
		time                & int               &                    & Timestamp \\ \hline
	\end{tabular}
\end{table}
\paragraph{Antwort}Die Antwort ist wie folgt aufgebaut:
\begin{table}[H]
	\begin{tabular}{|c|c|c|p{6.5cm}|}
		\hline
		\textbf{Paramtername} & \textbf{Datentyp} & \textbf{Konstante} & \textbf{Kurzbeschreibung}            \\ \hline                
		result              & string           &                 & Erfolgreich wenn Wert {\glqq ack\grqq} ist \\ \hline
		Code                & int              &                 & Erfolgreich wenn Wert {\glqq 0\grqq} ist \\ \hline
	\end{tabular}
\end{table}

\subsubsection{Bild validieren}
\paragraph{Kurzbeschreibung}Dieser API-Request wird dazu genutzt um eine einzelnes Bild zu validieren.
\paragraph{Anfrage}Folgende Daten werden zu Anfrage benötigt:
\begin{table}[H]
	\begin{tabular}{|c|c|c|p{6.5cm}|}
		\hline
		\textbf{Paramtername} & \textbf{Datentyp} & \textbf{Konstante} & \textbf{Kurzbeschreibung}                                                                                               \\ \hline
		type                & string            & vap                & Bild validieren \\ \hline
		data                & string            &                    & Token \\ \hline
		seccode             & string            &                    & Security Code \\ \hline
		time                & int               &                    & Timestamp \\ \hline
	\end{tabular}
\end{table}
\paragraph{Antwort}Die Antwort ist wie folgt aufgebaut:
\begin{table}[H]
	\begin{tabular}{|c|c|c|p{6.5cm}|}
		\hline
		\textbf{Paramtername} & \textbf{Datentyp} & \textbf{Konstante} & \textbf{Kurzbeschreibung}            \\ \hline                
		result              & string           &                 & Erfolgreich wenn Wert {\glqq ack\grqq} ist \\ \hline
		Code                & int              &                 & Erfolgreich wenn Wert {\glqq 0\grqq} ist \\ \hline
	\end{tabular}
\end{table}
\newpage
\section{Steuerunsg API-Spezifikation}\label{mapi}
\subsection{Beschreibung}Diese API dient der Kommunikation zwischen dem Frontend und dem Backend von {\glqq COSP\grqq}. Sie unterstützt Funktionen zur Nutzer-, Rollen-, Ränge- und Modul-Verwaltung. Des weiteren unterstützt sie auch das Abfragen einiger statistischer Daten. Diese API ist nur für angemeldete und authorisierte Nutzer verfügbar.
\subsection{Befehlsübersicht}
\begin{longtable}[H]{|c|p{12cm}|}
		\hline
		\textbf{Api-Befehl} & \textbf{Kurzbeschreibung}              \\ \hline
		cur                 & Nutzerrolle ändern          \\ \hline
		anr                 & Neue Rolle hinzufügen            \\ \hline
		eer                 & Bestehende Rolle ändern \\ \hline
		der                 & Rolle löschen \\ \hline
		cup                 & Nutzerpasswort ändern \\ \hline
		teu                 & Nutzer Aktivieren oder Deaktivieren (Toggle) \\ \hline
		rup                 & Reset Nutzerpasswort \\ \hline
		adr                 & Neuen Rang anlegen \\ \hline
		dra                 & Rang löschen \\ \hline
		era                 & Bestehenden Rang ändern \\ \hline
		gsd                 & Statistische Daten abfragen \\ \hline
		gar                 & Alle Rollennamen abfragen \\ \hline
		grn                 & Alle Rangnamen abfragen \\ \hline
		cpa                 & Anfordern eines Captcha-Codes als Base64 endocdiertes Bild \\ \hline
        cmg                 & Kontaktnachricht absenden \\ \hline
        rns                 & Alle Module, welche keinen spezialisierten Namen bei einem Rang haben \\ \hline
        rna                 & Modulbasierte Namen eines Ranges abfragen \\ \hline
        imr                 & Modulbasierte Namen eines Ranges hinzufügen\\ \hline
        dmr                 & Modulbasierte Namen eines Ranges löschen\\ \hline
        gap                 & Abfrage aller Daten einer API\\ \hline
        sap                 & Speichert Daten einer API\\ \hline
        gmr                 & Fragt alle Module ab, für welche ein Nutzer Rechte hat\\ \hline
        gam                 & Name und ID aller APIs abfragen\\ \hline
        sar                 & Gibt alle für den Nutzer vergebbaren Rollen zurück\\ \hline
        smr                 & Speichert eine Modul-Rolle eines Benutzers ab\\ \hline
        drm                 & Löscht eine Modul-Rolle\\ \hline
        smv                 & Mailvalidierungsstatus setzen\\ \hline
        umr					& Aktualisiere Modulberechtigungen \\ \hline
        dsr					& Deaktivierungsstatus eines Nutzers für ein Modul setzen \\ \hline
        cna					& Anlegen eines neuen Moduls beziehungsweise einer neuen API. \\ \hline
        cma					& Prüft ob eine Mailadresse bereits verwendet wird. \\ \hline
\end{longtable}
\newpage
\subsection{Befehle}
\subsubsection{Nutzerrolle ändern}
\paragraph{Kurzbeschreibung}Dieser API-Request wird dazu genutzt um die Rolle eines Nutzers zu ändern.
\paragraph{Anfrage}Folgende Daten werden zu Anfrage benötigt:
\begin{table}[H]
	\begin{tabular}{|c|c|c|p{6.5cm}|}
		\hline
		\textbf{Paramtername} & \textbf{Datentyp} & \textbf{Konstante} & \textbf{Kurzbeschreibung}                                                                                               \\ \hline
		type                & string            & cur                & Nutzerrolle ändern \\ \hline
		username            & string            &                    & Nutzername \\ \hline
		role                & int               &                    & Identifikator der Rolle \\ \hline
	\end{tabular}
\end{table}
\paragraph{Antwort}Die Antwort ist wie folgt aufgebaut:
\begin{table}[H]
	\begin{tabular}{|c|c|c|p{6.5cm}|}
		\hline
		\textbf{Paramtername} & \textbf{Datentyp} & \textbf{Konstante} & \textbf{Kurzbeschreibung}            \\ \hline                
		success             & bool             &                 & Erfolgreich wenn Wert {\glqq true\grqq} ist \\ \hline
		type                & string           & cur             & Nutzerrolle ändern \\ \hline
		username            & string           &                 & Nutzername \\ \hline
		role                & int              &                 & Identifikator der Rolle \\ \hline
	\end{tabular}
\end{table}
\subsubsection{Neue Rolle anlegen}
\paragraph{Kurzbeschreibung}Dieser API-Request wird dazu genutzt um eine neue Rolle hinzuzufügen.
\paragraph{Anfrage}Folgende Daten werden zu Anfrage benötigt:
\begin{table}[H]
	\begin{tabular}{|c|c|c|p{6.5cm}|}
		\hline
		\textbf{Paramtername} & \textbf{Datentyp} & \textbf{Konstante} & \textbf{Kurzbeschreibung}                                                                                               \\ \hline
		type                & string            & anr                & Neue Rolle anlegen \\ \hline
		name                & string            &                    & Rollenname \\ \hline
		value               & int               &                    & Rollenwert \\ \hline
	\end{tabular}
\end{table}
\paragraph{Antwort}Die Antwort ist wie folgt aufgebaut:
\begin{table}[H]
	\begin{tabular}{|c|c|c|p{6.5cm}|}
		\hline
		\textbf{Paramtername} & \textbf{Datentyp} & \textbf{Konstante} & \textbf{Kurzbeschreibung}            \\ \hline                
		success             & bool             &                 & Erfolgreich wenn Wert {\glqq true\grqq} ist \\ \hline
		payload             & bool             &                 & Wahr, wenn Erfolgreich \\ \hline
	\end{tabular}
\end{table}
\subsubsection{Bestehende Rolle ändern}
\paragraph{Kurzbeschreibung}Dieser API-Request wird dazu genutzt um eine einzelne bestehende Rolle zu ändern.
\paragraph{Anfrage}Folgende Daten werden zu Anfrage benötigt:
\begin{table}[H]
	\begin{tabular}{|c|c|c|p{6.5cm}|}
		\hline
		\textbf{Paramtername} & \textbf{Datentyp} & \textbf{Konstante} & \textbf{Kurzbeschreibung}                                                                                               \\ \hline
		type                & string            & eer                & Rolle ändern \\ \hline
		name                & string            &                    & Rollenname \\ \hline
		value               & int               &                    & Rollenwert \\ \hline
		id                  & int               &                    & Identifikator der Rolle \\ \hline
	\end{tabular}
\end{table}
\paragraph{Antwort}Die Antwort ist wie folgt aufgebaut:
\begin{table}[H]
	\begin{tabular}{|c|c|c|p{6.5cm}|}
		\hline
		\textbf{Paramtername} & \textbf{Datentyp} & \textbf{Konstante} & \textbf{Kurzbeschreibung}            \\ \hline                
		success             & bool             &                 & Erfolgreich wenn Wert {\glqq true\grqq} ist \\ \hline
		payload             & array            &                 & Leeres Array \\ \hline
	\end{tabular}
\end{table}
\subsubsection{Rolle löschen}
\paragraph{Kurzbeschreibung}Dieser API-Request wird dazu genutzt um eine einzelne Rolle zu löschen.
\paragraph{Anfrage}Folgende Daten werden zu Anfrage benötigt:
\begin{table}[H]
	\begin{tabular}{|c|c|c|p{6.5cm}|}
		\hline
		\textbf{Paramtername} & \textbf{Datentyp} & \textbf{Konstante} & \textbf{Kurzbeschreibung}                                                                                               \\ \hline
		type                & string            & der                & Rolle löschen \\ \hline
		id                  & int               &                    & Identifikator der Rolle \\ \hline
	\end{tabular}
\end{table}
\paragraph{Antwort}Die Antwort ist wie folgt aufgebaut:
\begin{table}[H]
	\begin{tabular}{|c|c|c|p{6.5cm}|}
		\hline
		\textbf{Paramtername} & \textbf{Datentyp} & \textbf{Konstante} & \textbf{Kurzbeschreibung}            \\ \hline                
		success             & bool             &                 & Erfolgreich wenn Wert {\glqq true\grqq} ist \\ \hline
		payload             & string           &                 & Bei Erfolg: {\glqq Role successfully deleted\grqq} \\ \hline
	\end{tabular}
\end{table}
\subsubsection{Nutzerpasswort ändern}
\paragraph{Kurzbeschreibung}Dieser API-Request wird dazu genutzt um ein Nutzerpasswort zu ändern.
\paragraph{Anfrage}Folgende Daten werden zu Anfrage benötigt:
\begin{table}[H]
	\begin{tabular}{|c|c|c|p{6.5cm}|}
		\hline
		\textbf{Paramtername} & \textbf{Datentyp} & \textbf{Konstante} & \textbf{Kurzbeschreibung}                                                                                               \\ \hline
		type                & string            & cup                & Passwort ändern \\ \hline
		id                  & int               &                    & Identifikator des Nutzers \\ \hline
		pwd1                & string            &                    & Inhalt Passwortbox 1 \\ \hline
		pwd2                & string            &                    & Inhalt Passwortbox 2 \\ \hline
	\end{tabular}
\end{table}
\paragraph{Antwort}Die Antwort ist wie folgt aufgebaut:
\begin{table}[H]
	\begin{tabular}{|c|c|c|p{6.5cm}|}
		\hline
		\textbf{Paramtername} & \textbf{Datentyp} & \textbf{Konstante} & \textbf{Kurzbeschreibung}            \\ \hline                
		success             & bool             &                 & Erfolgreich wenn Wert {\glqq true\grqq} ist \\ \hline
		payload             & string           &                 & Bei Erfolg: {\glqq Successfully updated Password!\grqq} \\ \hline
	\end{tabular}
\end{table}
\subsubsection{Nutzeraktivierung umschalten}
\paragraph{Kurzbeschreibung}Dieser API-Request wird dazu genutzt um das Nutzerpasswort eines Nutzers neu zu setzen.
\paragraph{Anfrage}Folgende Daten werden zu Anfrage benötigt:
\begin{table}[H]
	\begin{tabular}{|c|c|c|p{6.5cm}|}
		\hline
		\textbf{Paramtername} & \textbf{Datentyp} & \textbf{Konstante} & \textbf{Kurzbeschreibung}                                                                                               \\ \hline
		type                & string            & teu                & Nutzeraktivierung umschalten \\ \hline
		id                  & int               &                    & Identifikator des Nutzers \\ \hline
	\end{tabular}
\end{table}
\paragraph{Antwort}Die Antwort ist wie folgt aufgebaut:
\begin{table}[H]
	\begin{tabular}{|c|c|c|p{6.5cm}|}
		\hline
		\textbf{Paramtername} & \textbf{Datentyp} & \textbf{Konstante} & \textbf{Kurzbeschreibung}            \\ \hline                
		success             & bool             &                 & Erfolgreich wenn Wert {\glqq true\grqq} ist \\ \hline
		payload             & string           &                 & Bei Erfolg: {\glqq Successfully updated User!\grqq} \\ \hline
	\end{tabular}
\end{table}
\subsubsection{Nutzerpasswort Reset via Mail}
\paragraph{Kurzbeschreibung}Dieser API-Request wird dazu genutzt um eine ein Nutzerpasswort zu ändern.
\paragraph{Anfrage}Folgende Daten werden zu Anfrage benötigt:
\begin{table}[H]
	\begin{tabular}{|c|c|c|p{6.5cm}|}
		\hline
		\textbf{Paramtername} & \textbf{Datentyp} & \textbf{Konstante} & \textbf{Kurzbeschreibung}                                                                                               \\ \hline
		type                & string            & rup                & Nutzerpasswort Reset \\ \hline
		id                  & int               &                    & Identifikator des Nutzers \\ \hline
	\end{tabular}
\end{table}
\paragraph{Antwort}Die Antwort ist wie folgt aufgebaut:
\begin{table}[H]
	\begin{tabular}{|c|c|c|p{6.5cm}|}
		\hline
		\textbf{Paramtername} & \textbf{Datentyp} & \textbf{Konstante} & \textbf{Kurzbeschreibung}            \\ \hline                
		success             & bool             &                 & Erfolgreich wenn Wert {\glqq true\grqq} ist \\ \hline
		payload             & array            &                 & Bei Erfolg: Leeres Array \\ \hline
	\end{tabular}
\end{table}
\subsubsection{Neuen Rang hinzufügen}
\paragraph{Kurzbeschreibung}Dieser API-Request wird dazu genutzt um einen neuen Rang hinzuzufügen.
\paragraph{Anfrage}Folgende Daten werden zu Anfrage benötigt:
\begin{table}[H]
	\begin{tabular}{|c|c|c|p{6.5cm}|}
		\hline
		\textbf{Paramtername} & \textbf{Datentyp} & \textbf{Konstante} & \textbf{Kurzbeschreibung}                                                                                               \\ \hline
		type                & string            & adr                & Rang anlegen \\ \hline
		name                & string            &                    & Rangname \\ \hline
		value               & int               &                    & Rangwert \\ \hline
	\end{tabular}
\end{table}
\paragraph{Antwort}Die Antwort ist wie folgt aufgebaut:
\begin{table}[H]
	\begin{tabular}{|c|c|c|p{6.5cm}|}
		\hline
		\textbf{Paramtername} & \textbf{Datentyp} & \textbf{Konstante} & \textbf{Kurzbeschreibung}            \\ \hline                
		success             & bool             &                 & Erfolgreich wenn Wert {\glqq true\grqq} ist \\ \hline
		payload             & bool             &                 & Bei Erfolg: {\glqq true\grqq} \\ \hline
	\end{tabular}
\end{table}
\subsubsection{Rang löschen}
\paragraph{Kurzbeschreibung}Dieser API-Request wird dazu genutzt um einen Rang zu löschen.
\paragraph{Anfrage}Folgende Daten werden zu Anfrage benötigt:
\begin{table}[H]
	\begin{tabular}{|c|c|c|p{6.5cm}|}
		\hline
		\textbf{Paramtername} & \textbf{Datentyp} & \textbf{Konstante} & \textbf{Kurzbeschreibung}                                                                                               \\ \hline
		type                & string            & dra                & Rang löschen \\ \hline
		id                  & int               &                    & Identifikator des Rangs \\ \hline
	\end{tabular}
\end{table}
\paragraph{Antwort}Die Antwort ist wie folgt aufgebaut:
\begin{table}[H]
	\begin{tabular}{|c|c|c|p{6.5cm}|}
		\hline
		\textbf{Paramtername} & \textbf{Datentyp} & \textbf{Konstante} & \textbf{Kurzbeschreibung}            \\ \hline                
		success             & bool             &                 & Erfolgreich wenn Wert {\glqq true\grqq} ist \\ \hline
		payload             & bool             &                 & Bei Erfolg: {\glqq Rank successfully deleted\grqq} \\ \hline
	\end{tabular}
\end{table}
\subsubsection{Bestehenden Rang ändern}
\paragraph{Kurzbeschreibung}Dieser API-Request wird dazu genutzt um einen bestehenden Rang zu ändern.
\paragraph{Anfrage}Folgende Daten werden zu Anfrage benötigt:
\begin{table}[H]
	\begin{tabular}{|c|c|c|p{6.5cm}|}
		\hline
		\textbf{Paramtername} & \textbf{Datentyp} & \textbf{Konstante} & \textbf{Kurzbeschreibung}                                                                                               \\ \hline
		type                & string            & era                & Rang ändern \\ \hline
		id                  & int               &                    & Identifikator des Rangs \\ \hline
		name                & string            &                    & Rangname \\ \hline
		value               & int               &                    & Rangwert \\ \hline
	\end{tabular}
\end{table}
\paragraph{Antwort}Die Antwort ist wie folgt aufgebaut:
\begin{table}[H]
	\begin{tabular}{|c|c|c|p{6.5cm}|}
		\hline
		\textbf{Paramtername} & \textbf{Datentyp} & \textbf{Konstante} & \textbf{Kurzbeschreibung}            \\ \hline                
		success             & bool             &                 & Erfolgreich wenn Wert {\glqq true\grqq} ist \\ \hline
		payload             & array            &                 & Bei Erfolg: Leeres Array \\ \hline
	\end{tabular}
\end{table}
\subsubsection{Abrufen von Statistiken}
\paragraph{Kurzbeschreibung}Dieser API-Request wird dazu genutzt um Statistiken abzufragen.
\paragraph{Anfrage}Folgende Daten werden zu Anfrage benötigt:
\begin{table}[H]
	\begin{tabular}{|c|c|c|p{6.5cm}|}
		\hline
		\textbf{Paramtername} & \textbf{Datentyp} & \textbf{Konstante} & \textbf{Kurzbeschreibung}                                                                                               \\ \hline
		type                & string            & gsd                & Statistiken anfordern \\ \hline
		data                & array             &                    & Strukturierter Request \\ \hline
	\end{tabular}
\end{table}
\subparagraph{data}Dieses Array enthält Einträge in der nachstehend dargestellten Form haben:
\begin{table}[H]
	\begin{tabular}{|c|c|c|p{6.5cm}|}
		\hline
		\textbf{Paramtername} & \textbf{Datentyp} & \textbf{Konstante} & \textbf{Kurzbeschreibung}    \\ \hline
		data               & Array             &                 & Liste der angeforderten Daten \\ \hline
		src                & string            &                 & Quelle der Daten \\ \hline
	\end{tabular}
\end{table}
\subparagraph{data}Dieses Array enthält Elemente mit Einträgen in der nachstehend dargestellten Form haben:
\begin{table}[H]
	\begin{tabular}{|c|c|c|p{6.5cm}|}
		\hline
		\textbf{Paramtername} & \textbf{Datentyp} & \textbf{Konstante} & \textbf{Kurzbeschreibung}    \\ \hline
		type               & string            &                 & Zeiteinheit (D:Tage, W:Wochen, M:Monate, Y:Jahre) \\ \hline
		Amount             & int               &                 & Anzahl an Einheiten \\ \hline
		ID                 & int               &                 & Identifikator \\ \hline
	\end{tabular}
\end{table}
\paragraph{Antwort}Die Antwort ist wie folgt aufgebaut:
\begin{table}[H]
	\begin{tabular}{|c|c|c|p{6.5cm}|}
		\hline
		\textbf{Paramtername} & \textbf{Datentyp} & \textbf{Konstante} & \textbf{Kurzbeschreibung}            \\ \hline                
		code                & int              &                 & Erfolgreich wenn Wert {\glqq 0\grqq} ist \\ \hline
		result              & string           &                 & Bei Erfolg: {\glqq ack\grqq} \\ \hline
		data                & array            &                 & Strukturiertes Ergebnis \\ \hline
	\end{tabular}
\end{table}
\subparagraph{data}Dieses Array enthält Elemente mit Einträgen in der nachstehend dargestellten Form haben:
\begin{table}[H]
	\begin{tabular}{|c|c|c|p{6.5cm}|}
		\hline
		\textbf{Paramtername} & \textbf{Datentyp} & \textbf{Konstante} & \textbf{Kurzbeschreibung}    \\ \hline
		type               & string            &                 & Zeiteinheit (D:Tage, W:Wochen, M:Monate, Y:Jahre) \\ \hline
		Amount             & int               &                 & Anzahl an Einheiten \\ \hline
		ID                 & int               &                 & Identifikator \\ \hline
		data               & array             &                 & Eintrag entsprechend Doku zu Chart.js \\ \hline
	\end{tabular}
\end{table}
\subsubsection{Abfrage aller Rollennamen}
\paragraph{Kurzbeschreibung}Dieser API-Request wird dazu genutzt um eine Liste aller Rollennamen abzurufen.
\paragraph{Anfrage}Folgende Daten werden zu Anfrage benötigt:
\begin{table}[H]
	\begin{tabular}{|c|c|c|p{6.5cm}|}
		\hline
		\textbf{Paramtername} & \textbf{Datentyp} & \textbf{Konstante} & \textbf{Kurzbeschreibung}                                                                                               \\ \hline
		type                & string            & gar                & Rollennamen abfragen \\ \hline
	\end{tabular}
\end{table}
\paragraph{Antwort}Die Antwort ist wie folgt aufgebaut:
\begin{table}[H]
	\begin{tabular}{|c|c|c|p{6.5cm}|}
		\hline
		\textbf{Paramtername} & \textbf{Datentyp} & \textbf{Konstante} & \textbf{Kurzbeschreibung}            \\ \hline                
		success             & bool             &                 & Erfolgreich wenn Wert {\glqq true\grqq} ist \\ \hline
		payload             & array            &                 & Bei Erfolg: Leeres Array \\ \hline
		data                & array            &                 & Liste mit allen Rollennamen \\ \hline
	\end{tabular}
\end{table}
\subsubsection{Abfrage aller Rangnamen}
\paragraph{Kurzbeschreibung}Dieser API-Request wird dazu genutzt um eine Liste aller Rangnamen abzurufen.
\paragraph{Anfrage}Folgende Daten werden zu Anfrage benötigt:
\begin{table}[H]
	\begin{tabular}{|c|c|c|p{6.5cm}|}
		\hline
		\textbf{Paramtername} & \textbf{Datentyp} & \textbf{Konstante} & \textbf{Kurzbeschreibung}                                                                                               \\ \hline
		type                & string            & grn                & Rangnamen abfragen \\ \hline
	\end{tabular}
\end{table}
\paragraph{Antwort}Die Antwort ist wie folgt aufgebaut:
\begin{table}[H]
	\begin{tabular}{|c|c|c|p{6.5cm}|}
		\hline
		\textbf{Paramtername} & \textbf{Datentyp} & \textbf{Konstante} & \textbf{Kurzbeschreibung}            \\ \hline                
		success             & bool             &                 & Erfolgreich wenn Wert {\glqq true\grqq} ist \\ \hline
		payload             & array            &                 & Bei Erfolg: Leeres Array \\ \hline
		data                & array            &                 & Liste mit allen Rangnamen \\ \hline
	\end{tabular}
\end{table}
\subsubsection{Anfordern Captcha-Bild}
\paragraph{Kurzbeschreibung}Dieser API-Request wird dazu genutzt um ein Base64 encodiertes Captcha-Bild anzufordern.
\paragraph{Anfrage}Folgende Daten werden zu Anfrage benötigt:
\begin{table}[H]
	\begin{tabular}{|c|c|c|p{6.5cm}|}
		\hline
		\textbf{Paramtername} & \textbf{Datentyp} & \textbf{Konstante} & \textbf{Kurzbeschreibung}                                                                                               \\ \hline
		type                & string            & cpa                & Captcha-Bild anfordern \\ \hline
	\end{tabular}
\end{table}
\paragraph{Antwort}Die Antwort ist wie folgt aufgebaut:
\begin{table}[H]
	\begin{tabular}{|c|c|c|p{6.5cm}|}
		\hline
		\textbf{Paramtername} & \textbf{Datentyp} & \textbf{Konstante} & \textbf{Kurzbeschreibung}            \\ \hline                
		success             & bool             &                 & Erfolgreich wenn Wert {\glqq true\grqq} ist \\ \hline
		payload             & array            &                 & Bei Erfolg: Leeres Array \\ \hline
		data                & string           &                 & Base64 codiertes Captcha-JPEG \\ \hline
	\end{tabular}
\end{table}
\subsubsection{Kontaktnachricht senden}
\paragraph{Kurzbeschreibung}Dieser API-Request wird dazu genutzt um eine Kontaktnachricht zu versenden.
\paragraph{Anfrage}Folgende Daten werden zu Anfrage benötigt:
\begin{table}[H]
	\begin{tabular}{|c|c|c|p{6.5cm}|}
		\hline
		\textbf{Paramtername} & \textbf{Datentyp} & \textbf{Konstante} & \textbf{Kurzbeschreibung}                                                                                               \\ \hline
		type                & string            & cmg                & Captcha-Bild anfordern \\ \hline
		cap                 & string            &                    & Nutzereingabe des Captchas \\ \hline
		title               & string            &                    & Betreff der Nachricht \\ \hline
		msg                 & string            &                    & Nachricht \\ \hline
	\end{tabular}
\end{table}
\paragraph{Antwort}Die Antwort ist wie folgt aufgebaut:
\begin{table}[H]
	\begin{tabular}{|c|c|c|p{6.5cm}|}
		\hline
		\textbf{Paramtername} & \textbf{Datentyp} & \textbf{Konstante} & \textbf{Kurzbeschreibung}            \\ \hline                
		success             & bool             &                 & Erfolgreich wenn Wert {\glqq true\grqq} ist \\ \hline
		payload             & array            &                 & Bei Erfolg: Leeres Array \\ \hline
	\end{tabular}
\end{table}
\subsubsection{Liste mit Modulen ohne Rangnamenspezifikation}
\paragraph{Kurzbeschreibung}Dieser API-Request wird dazu genutzt um eine Liste von Modulen zu bekommen, welche keinen spezialisierten Namen eines Ranges haben.
\paragraph{Anfrage}Folgende Daten werden zu Anfrage benötigt:
\begin{table}[H]
	\begin{tabular}{|c|c|c|p{6.5cm}|}
		\hline
		\textbf{Paramtername} & \textbf{Datentyp} & \textbf{Konstante} & \textbf{Kurzbeschreibung}                                                                                               \\ \hline
		type                & string            & rns                & Modul-Liste anfordern \\ \hline
		id                  & int               &                    & Identifikator eines Ranges \\ \hline
	\end{tabular}
\end{table}
\paragraph{Antwort}Die Antwort ist wie folgt aufgebaut:
\begin{table}[H]
	\begin{tabular}{|c|c|c|p{6.5cm}|}
		\hline
		\textbf{Paramtername} & \textbf{Datentyp} & \textbf{Konstante} & \textbf{Kurzbeschreibung}            \\ \hline                
		success             & bool             &                 & Erfolgreich wenn Wert {\glqq true\grqq} ist \\ \hline
		payload             & array            &                 & Liste der Apis \\ \hline
	\end{tabular}
\end{table}
\subparagraph{payload}Dieses Array enthält Elemente mit Einträgen in der nachstehend dargestellten Form haben:
\begin{table}[H]
	\begin{tabular}{|c|c|c|p{6.5cm}|}
		\hline
		\textbf{Paramtername} & \textbf{Datentyp} & \textbf{Konstante} & \textbf{Kurzbeschreibung}    \\ \hline
		name               & string            &                 & Name des Moduls \\ \hline
		id                 & int               &                 & Identifikator des Moduls \\ \hline
	\end{tabular}
\end{table}
\subsubsection{Liste aller modulbasierten Rangnamen}
\paragraph{Kurzbeschreibung}Dieser API-Request wird dazu genutzt um eine Liste mit modulbasierten Namen eines Ranges ab zu fragen.
\paragraph{Anfrage}Folgende Daten werden zu Anfrage benötigt:
\begin{table}[H]
	\begin{tabular}{|c|c|c|p{6.5cm}|}
		\hline
		\textbf{Paramtername} & \textbf{Datentyp} & \textbf{Konstante} & \textbf{Kurzbeschreibung}                                                                                               \\ \hline
		type                & string            & rna                & Modul-Liste anfordern \\ \hline
		id                  & int               &                    & Identifikator eines Ranges \\ \hline
	\end{tabular}
\end{table}
\paragraph{Antwort}Die Antwort ist wie folgt aufgebaut:
\begin{table}[H]
	\begin{tabular}{|c|c|c|p{6.5cm}|}
		\hline
		\textbf{Paramtername} & \textbf{Datentyp} & \textbf{Konstante} & \textbf{Kurzbeschreibung}            \\ \hline                
		success             & bool             &                 & Erfolgreich wenn Wert {\glqq true\grqq} ist \\ \hline
		payload             & array            &                 & Liste der Apis \\ \hline
	\end{tabular}
\end{table}
\subparagraph{payload}Dieses Array enthält Elemente mit Einträgen in der nachstehend dargestellten Form haben:
\begin{table}[H]
	\begin{tabular}{|c|c|c|p{6.5cm}|}
		\hline
		\textbf{Paramtername} & \textbf{Datentyp} & \textbf{Konstante} & \textbf{Kurzbeschreibung}    \\ \hline
		id                      &                   &                 & Identifikator des modulbasierten Rangnamens \\ \hline
		rankname                & string            &                 & Name des Moduls \\ \hline
		modulename              & string            &                 & Name des Moduls \\ \hline
	\end{tabular}
\end{table}
\subsubsection{Modulbasierten Rangnamen hinzufügen}
\paragraph{Kurzbeschreibung}Dieser API-Request wird dazu genutzt um eine Liste mit modulbasierten Namen eines Ranges ab zu fragen.
\paragraph{Anfrage}Folgende Daten werden zu Anfrage benötigt:
\begin{table}[H]
	\begin{tabular}{|c|c|c|p{6.5cm}|}
		\hline
		\textbf{Paramtername} & \textbf{Datentyp} & \textbf{Konstante} & \textbf{Kurzbeschreibung}                                                                                               \\ \hline
		type                & string            & imr                & Modul-Liste anfordern \\ \hline
		rid                 & int               &                    & Identifikator eines Ranges \\ \hline
		aid                 & int               &                    & Identifikator eines Moduls \\ \hline
		name                & string            &                    & modulbasierter Name des Ranges  \\ \hline
	\end{tabular}
\end{table}
\paragraph{Antwort}Die Antwort ist wie folgt aufgebaut:
\begin{table}[H]
	\begin{tabular}{|c|c|c|p{6.5cm}|}
		\hline
		\textbf{Paramtername} & \textbf{Datentyp} & \textbf{Konstante} & \textbf{Kurzbeschreibung}            \\ \hline                
		success             & bool             &                 & Erfolgreich wenn Wert {\glqq true\grqq} ist \\ \hline
		payload             & array            &                 & Leeres Array \\ \hline
	\end{tabular}
\end{table}
\subsubsection{Modulbasierten Rangnamen löschen}
\paragraph{Kurzbeschreibung}Dieser API-Request wird dazu genutzt um eine Liste mit modulbasierten Namen eines Ranges ab zu fragen.
\paragraph{Anfrage}Folgende Daten werden zu Anfrage benötigt:
\begin{table}[H]
	\begin{tabular}{|c|c|c|p{6.5cm}|}
		\hline
		\textbf{Paramtername} & \textbf{Datentyp} & \textbf{Konstante} & \textbf{Kurzbeschreibung}                                                                                               \\ \hline
		type                & string            & imr                & Modul-Liste anfordern \\ \hline
		id                  & int               &                    & Identifikator eines modulbasierten Ranganmens \\ \hline
	\end{tabular}
\end{table}
\paragraph{Antwort}Die Antwort ist wie folgt aufgebaut:
\begin{table}[H]
	\begin{tabular}{|c|c|c|p{6.5cm}|}
		\hline
		\textbf{Paramtername} & \textbf{Datentyp} & \textbf{Konstante} & \textbf{Kurzbeschreibung}            \\ \hline                
		success             & bool             &                 & Erfolgreich wenn Wert {\glqq true\grqq} ist \\ \hline
		payload             & array            &                 & Leeres Array \\ \hline
	\end{tabular}
\end{table}
\subsubsection{Abfrage aller Daten einer API}
\paragraph{Kurzbeschreibung}Dieser API-Request wird dazu genutzt um alle Daten einer API abzufragen.
\paragraph{Anfrage}Folgende Daten werden zu Anfrage benötigt:
\begin{table}[H]
	\begin{tabular}{|c|c|c|p{6.5cm}|}
		\hline
		\textbf{Paramtername} & \textbf{Datentyp} & \textbf{Konstante} & \textbf{Kurzbeschreibung}                                                                                               \\ \hline
		type                & string            & gap                & API-Daten anfordern \\ \hline
		id                  & int               &                    & Identifikator einer API \\ \hline
	\end{tabular}
\end{table}
\paragraph{Antwort}Die Antwort ist wie folgt aufgebaut:
\begin{table}[H]
	\begin{tabular}{|c|c|c|p{6.5cm}|}
		\hline
		\textbf{Paramtername} & \textbf{Datentyp} & \textbf{Konstante} & \textbf{Kurzbeschreibung}            \\ \hline                
		success             & bool             &                 & Erfolgreich wenn Wert {\glqq true\grqq} ist \\ \hline
		payload             & array            &                 & Daten der API \\ \hline
	\end{tabular}
\end{table}
\subparagraph{payload}Dieses Array enthält Elemente mit Einträgen in der nachstehend dargestellten Form haben:
\begin{table}[H]
	\begin{tabular}{|c|c|c|p{6.5cm}|}
		\hline
		\textbf{Paramtername} & \textbf{Datentyp} & \textbf{Konstante} & \textbf{Kurzbeschreibung}    \\ \hline
		id                      &                   &                 & Identifikator der API \\ \hline
		name                    & string            &                 & Name der API \\ \hline
		url                     & string            &                 & Reverse-API-URL der API \\ \hline
	\end{tabular}
\end{table}
\subsubsection{Speichert Daten einer API}
\paragraph{Kurzbeschreibung}Dieser API-Request wird dazu genutzt um alle Daten einer API abzufragen.
\paragraph{Anfrage}Folgende Daten werden zu Anfrage benötigt:
\begin{table}[H]
	\begin{tabular}{|c|c|c|p{6.5cm}|}
		\hline
		\textbf{Paramtername} & \textbf{Datentyp} & \textbf{Konstante} & \textbf{Kurzbeschreibung}                                                                                               \\ \hline
		type                & string            & sap                & API-Daten speichern \\ \hline
		id                  & int               &                    & Identifikator einer API \\ \hline
		name                & string            &                    & Name einer API \\ \hline
		url                 & string            &                    & Url der Reverse-API des Moduls \\ \hline
	\end{tabular}
\end{table}
\paragraph{Antwort}Die Antwort ist wie folgt aufgebaut:
\begin{table}[H]
	\begin{tabular}{|c|c|c|p{6.5cm}|}
		\hline
		\textbf{Paramtername} & \textbf{Datentyp} & \textbf{Konstante} & \textbf{Kurzbeschreibung}            \\ \hline                
		success             & bool             &                 & Erfolgreich wenn Wert {\glqq true\grqq} ist \\ \hline
	\end{tabular}
\end{table}
\subsubsection{Modulbasierte Rechte abfragen}
\paragraph{Kurzbeschreibung}Dieser API-Request wird dazu genutzt um Module abzufragen, für die der Nutzer rechte besitzt.
\paragraph{Anfrage}Folgende Daten werden zu Anfrage benötigt:
\begin{table}[H]
	\begin{tabular}{|c|c|c|p{6.5cm}|}
		\hline
		\textbf{Paramtername} & \textbf{Datentyp} & \textbf{Konstante} & \textbf{Kurzbeschreibung}                                                                                               \\ \hline
		type                & string            & gmr                & API-Daten speichern \\ \hline
		id                  & int               &                    & Identifikator eines Nutzers\\ \hline
	\end{tabular}
\end{table}
\paragraph{Antwort}Die Antwort ist wie folgt aufgebaut:
\begin{table}[H]
	\begin{tabular}{|c|c|c|p{6.5cm}|}
		\hline
		\textbf{Paramtername} & \textbf{Datentyp} & \textbf{Konstante} & \textbf{Kurzbeschreibung}            \\ \hline                
		success             & bool             &                 & Erfolgreich wenn Wert {\glqq true\grqq} ist \\ \hline
		payload             & array            &                 & Daten der API \\ \hline
	\end{tabular}
\end{table}
\subparagraph{payload}Dieses Array enthält Elemente mit Einträgen in der nachstehend dargestellten Form haben:
\begin{table}[H]
	\begin{tabular}{|c|c|c|p{6.5cm}|}
		\hline
		\textbf{Paramtername} & \textbf{Datentyp} & \textbf{Konstante} & \textbf{Kurzbeschreibung}    \\ \hline
		id                      & int               &                 & Identifikator der Berechtigung \\ \hline
		api                     & string            &                 & Name der API \\ \hline
		name                    & string            &                 & Name der Rolle \\ \hline
		apiid                   & int               &                 & Identifikator eines Moduls\\ \hline
		enabled                 & boolean           &                 & Freischaltungsstatus auf Modul\\ \hline
	\end{tabular}
\end{table}
\subsubsection{Name und ID aller APIs abfragen}
\paragraph{Kurzbeschreibung}Dieser API-Request wird dazu genutzt um Module abzufragen, für die der Nutzer rechte besitzt.
\paragraph{Anfrage}Folgende Daten werden zu Anfrage benötigt:
\begin{table}[H]
	\begin{tabular}{|c|c|c|p{6.5cm}|}
		\hline
		\textbf{Paramtername} & \textbf{Datentyp} & \textbf{Konstante} & \textbf{Kurzbeschreibung}                                                                                               \\ \hline
		type                & string            & gam                & API-Daten speichern \\ \hline
	\end{tabular}
\end{table}
\paragraph{Antwort}Die Antwort ist wie folgt aufgebaut:
\begin{table}[H]
	\begin{tabular}{|c|c|c|p{6.5cm}|}
		\hline
		\textbf{Paramtername} & \textbf{Datentyp} & \textbf{Konstante} & \textbf{Kurzbeschreibung}            \\ \hline                
		success             & bool             &                 & Erfolgreich wenn Wert {\glqq true\grqq} ist \\ \hline
		payload             & array            &                 & Daten der API \\ \hline
	\end{tabular}
\end{table}
\subparagraph{payload}Dieses Array enthält Elemente mit Einträgen in der nachstehend dargestellten Form haben:
\begin{table}[H]
	\begin{tabular}{|c|c|c|p{6.5cm}|}
		\hline
		\textbf{Paramtername} & \textbf{Datentyp} & \textbf{Konstante} & \textbf{Kurzbeschreibung}    \\ \hline
		id                      & int               &                 & Identifikator der API \\ \hline
		name                    & string            &                 & Name der API \\ \hline
	\end{tabular}
\end{table}
\subsubsection{Alle erlaubten Rollen abfragen}
\paragraph{Kurzbeschreibung}Dieser API-Request wird dazu genutzt alle Rollen die der aktuelle Benutzer vergeben darf ab zu fragen.
\paragraph{Anfrage}Folgende Daten werden zu Anfrage benötigt:
\begin{table}[H]
	\begin{tabular}{|c|c|c|p{6.5cm}|}
		\hline
		\textbf{Paramtername} & \textbf{Datentyp} & \textbf{Konstante} & \textbf{Kurzbeschreibung}                                                                                               \\ \hline
		type                & string            & sar                & API-Daten speichern \\ \hline
	\end{tabular}
\end{table}
\paragraph{Antwort}Die Antwort ist wie folgt aufgebaut:
\begin{table}[H]
	\begin{tabular}{|c|c|c|p{6.5cm}|}
		\hline
		\textbf{Paramtername} & \textbf{Datentyp} & \textbf{Konstante} & \textbf{Kurzbeschreibung}            \\ \hline                
		success             & bool             &                 & Erfolgreich wenn Wert {\glqq true\grqq} ist \\ \hline
		payload             & array            &                 & Daten der API \\ \hline
	\end{tabular}
\end{table}
\subparagraph{payload}Dieses Array enthält Elemente mit Einträgen in der nachstehend dargestellten Form haben:
\begin{table}[H]
	\begin{tabular}{|c|c|c|p{6.5cm}|}
		\hline
		\textbf{Paramtername} & \textbf{Datentyp} & \textbf{Konstante} & \textbf{Kurzbeschreibung}    \\ \hline
		id                      & int               &                 & Identifikator der Rolle \\ \hline
		name                    & string            &                 & Name der Rolle \\ \hline
		value                   & int               &                 & Wert der Rolle \\ \hline
	\end{tabular}
\end{table}
\subsubsection{Modulbasierte Rolle speichern}
\paragraph{Kurzbeschreibung}Dieser API-Request wird dazu genutzt um Rechte für ein Modul der Datenbank hinzuzufügen.
\paragraph{Anfrage}Folgende Daten werden zu Anfrage benötigt:
\begin{table}[H]
	\begin{tabular}{|c|c|c|p{6.5cm}|}
		\hline
		\textbf{Paramtername} & \textbf{Datentyp} & \textbf{Konstante} & \textbf{Kurzbeschreibung}                                                                                               \\ \hline
		type                & string            & smr                & Modul-Rolle speichern \\ \hline
		module              & int               &                    & Identifikator eines Moduls \\ \hline
		role                & int               &                    & Identifikator einer Rolle \\ \hline
		user                & int               &                    & Identifikator eines Nutzers \\ \hline
	\end{tabular}
\end{table}
\paragraph{Antwort}Die Antwort ist wie folgt aufgebaut:
\begin{table}[H]
	\begin{tabular}{|c|c|c|p{6.5cm}|}
		\hline
		\textbf{Paramtername} & \textbf{Datentyp} & \textbf{Konstante} & \textbf{Kurzbeschreibung}            \\ \hline                
		success             & bool             &                 & Erfolgreich wenn Wert {\glqq true\grqq} ist \\ \hline
		payload             & array            &                 & Leeres Array \\ \hline
	\end{tabular}
\end{table}
\subsubsection{Modulbasierte Rolle löschen}
\paragraph{Kurzbeschreibung}Dieser API-Request wird dazu genutzt um Rechte für ein Modul der Datenbank hinzuzufügen.
\paragraph{Anfrage}Folgende Daten werden zu Anfrage benötigt:
\begin{table}[H]
	\begin{tabular}{|c|c|c|p{6.5cm}|}
		\hline
		\textbf{Paramtername} & \textbf{Datentyp} & \textbf{Konstante} & \textbf{Kurzbeschreibung}                                                                                               \\ \hline
		type                & string            & drm                & Modul-Rolle löschen \\ \hline
		id                  & int               &                    & Identifikator einer Berechtigung \\ \hline
	\end{tabular}
\end{table}
\paragraph{Antwort}Die Antwort ist wie folgt aufgebaut:
\begin{table}[H]
	\begin{tabular}{|c|c|c|p{6.5cm}|}
		\hline
		\textbf{Paramtername} & \textbf{Datentyp} & \textbf{Konstante} & \textbf{Kurzbeschreibung}            \\ \hline                
		success             & bool             &                 & Erfolgreich wenn Wert {\glqq true\grqq} ist \\ \hline
		payload             & array            &                 & Leeres Array \\ \hline
	\end{tabular}
\end{table}
\subsubsection{Mailvalidierungsstatus setzen}
\paragraph{Kurzbeschreibung}Dieser API-Request wird dazu genutzt um die Validierung einer Mailadresse manuell zu setzen.
\paragraph{Anfrage}Folgende Daten werden zu Anfrage benötigt:
\begin{table}[H]
	\begin{tabular}{|c|c|c|p{6.5cm}|}
		\hline
		\textbf{Paramtername} & \textbf{Datentyp} & \textbf{Konstante} & \textbf{Kurzbeschreibung}                                                                                               \\ \hline
		type                & string            & smv                & Mailvalidierungsstatus setzen \\ \hline
		name                & string            &                    & Nutzername \\ \hline
		state               & bool              &                    & Status der Validierung der Mailadresse \\ \hline
	\end{tabular}
\end{table}
\paragraph{Antwort}Die Antwort ist wie folgt aufgebaut:
\begin{table}[H]
	\begin{tabular}{|c|c|c|p{6.5cm}|}
		\hline
		\textbf{Paramtername} & \textbf{Datentyp} & \textbf{Konstante} & \textbf{Kurzbeschreibung}            \\ \hline                
		success             & bool             &                 & Erfolgreich wenn Wert {\glqq true\grqq} ist \\ \hline
		payload             & array            &                 & Leeres Array \\ \hline
	\end{tabular}
\end{table}
\subsubsection{Modulberechtigungen aktualisieren}
\paragraph{Kurzbeschreibung}Dieser API-Request wird dazu genutzt um Rechte für ein Modul in der Datenbank zu ändern.
\paragraph{Anfrage}Folgende Daten werden zu Anfrage benötigt:
\begin{table}[H]
	\begin{tabular}{|c|c|c|p{6.5cm}|}
		\hline
		\textbf{Paramtername} & \textbf{Datentyp} & \textbf{Konstante} & \textbf{Kurzbeschreibung}                                                                                               \\ \hline
		type                & string            & umr                & Modulberechtigungen aktualisieren \\ \hline
		roleid              & int               &                    & Identifikator eines Nutzers \\ \hline
		rightid             & int               &                    & Identifikator einer Rolle \\ \hline
	\end{tabular}
\end{table}
\paragraph{Antwort}Die Antwort ist wie folgt aufgebaut:
\begin{table}[H]
	\begin{tabular}{|c|c|c|p{6.5cm}|}
		\hline
		\textbf{Paramtername} & \textbf{Datentyp} & \textbf{Konstante} & \textbf{Kurzbeschreibung}            \\ \hline                
		success             & bool             &                 & Erfolgreich wenn Wert {\glqq true\grqq} ist \\ \hline
		payload             & string            &                 & Link zum Neuladen der Seite \\ \hline
	\end{tabular}
\end{table}
\subsubsection{Deaktivierungsstatus für Modulrechte eines Nutzer setzen}
\paragraph{Kurzbeschreibung}Dieser API-Request wird dazu genutzt um den Deaktivierungsstatus eines Nutzers für ein Modul in der Datenbank zu ändern.
\paragraph{Anfrage}Folgende Daten werden zu Anfrage benötigt:
\begin{table}[H]
	\begin{tabular}{|c|c|c|p{6.5cm}|}
		\hline
		\textbf{Paramtername} & \textbf{Datentyp} & \textbf{Konstante} & \textbf{Kurzbeschreibung}                                                                                               \\ \hline
		type                & string            & dsr                & Ändern Deaktivierungsstatus eines Nutzers für ein Modul \\ \hline
		state               & bool              &                    & Status der Deaktivierung \\ \hline
		rightid             & int               &                    & Identifikator eines Rechtes \\ \hline
	\end{tabular}
\end{table}
\paragraph{Antwort}Die Antwort ist wie folgt aufgebaut:
\begin{table}[H]
	\begin{tabular}{|c|c|c|p{6.5cm}|}
		\hline
		\textbf{Paramtername} & \textbf{Datentyp} & \textbf{Konstante} & \textbf{Kurzbeschreibung}            \\ \hline                
		success             & bool             &                 & Erfolgreich wenn Wert {\glqq true\grqq} ist \\ \hline
		payload             & string           &                 & Link zum Neuladen der Seite \\ \hline
	\end{tabular}
\end{table}
\subsubsection{Nutzer ohne Rechte für Modul suchen}
\paragraph{Kurzbeschreibung}Dieser API-Request wird dazu genutzt um Nutzer ohne Rechte für ein Modul ab zu fragen.
\paragraph{Anfrage}Folgende Daten werden zu Anfrage benötigt:
\begin{table}[H]
	\begin{tabular}{|c|c|c|p{6.5cm}|}
		\hline
		\textbf{Paramtername} & \textbf{Datentyp} & \textbf{Konstante} & \textbf{Kurzbeschreibung}                                                                                               \\ \hline
		type                & string            & gum                & Liste von Nutzern abfragen \\ \hline
		module              & int               &                    & Identifikator eines Moduls \\ \hline
	\end{tabular}
\end{table}
\paragraph{Antwort}Die Antwort ist wie folgt aufgebaut:
\begin{table}[H]
	\begin{tabular}{|c|c|c|p{6.5cm}|}
		\hline
		\textbf{Paramtername} & \textbf{Datentyp} & \textbf{Konstante} & \textbf{Kurzbeschreibung}            \\ \hline                
		success             & bool             &                 & Erfolgreich wenn Wert {\glqq true\grqq} ist \\ \hline
		payload             & string           &                 & Link zum Neuladen der Seite \\ \hline
	\end{tabular}
\end{table}
\subparagraph{payload}Dieses Array enthält Elemente mit Einträgen in der nachstehend dargestellten Form haben:
\begin{table}[H]
	\begin{tabular}{|c|c|c|p{6.5cm}|}
		\hline
		\textbf{Paramtername} & \textbf{Datentyp} & \textbf{Konstante} & \textbf{Kurzbeschreibung}    \\ \hline
		id                      & int               &                 & Identifikator der Rolle \\ \hline
		name                    & string            &                 & Name des Nutzers \\ \hline
		firstname               & string            &                 & Vorname \\ \hline
		lastname                & string            &                 & Nachname \\ \hline
	\end{tabular}
\end{table}
\subsubsection{Neues Modul anlegen}
\paragraph{Kurzbeschreibung}Dieser API-Request wird dazu genutzt um ein neues Modul an zu legen.
\paragraph{Anfrage}Folgende Daten werden zu Anfrage benötigt:
\begin{table}[H]
	\begin{tabular}{|c|c|c|p{6.5cm}|}
		\hline
		\textbf{Paramtername} & \textbf{Datentyp} & \textbf{Konstante} & \textbf{Kurzbeschreibung}                                                                                               \\ \hline
		type                & string            & cna                & Modul anlegen \\ \hline
		name                & string            &                    & Name des neuen Moduls \\ \hline
		url					& string			&					 & Url der Reverse-Api des Moduls \\ \hline
	\end{tabular}
\end{table}
\paragraph{Antwort}Die Antwort ist wie folgt aufgebaut:
\begin{table}[H]
	\begin{tabular}{|c|c|c|p{6.5cm}|}
		\hline
		\textbf{Paramtername} & \textbf{Datentyp} & \textbf{Konstante} & \textbf{Kurzbeschreibung}            \\ \hline                
		success             & bool             &                 & Erfolgreich wenn Wert {\glqq true\grqq} ist \\ \hline
		payload             & string           &                 & generierter Authentifizierungstoken \\ \hline
	\end{tabular}
\end{table}
\subsubsection{Existenz Mailadresse prüfen}
\paragraph{Kurzbeschreibung}Dieser API-Request wird dazu genutzt um ein neues Modul an zu legen.
\paragraph{Anfrage}Folgende Daten werden zu Anfrage benötigt:
\begin{table}[H]
	\begin{tabular}{|c|c|c|p{6.5cm}|}
		\hline
		\textbf{Paramtername} & \textbf{Datentyp} & \textbf{Konstante} & \textbf{Kurzbeschreibung}                                                                                               \\ \hline
		type                & string            & cma                & Mailadresse prüfen \\ \hline
		email               & string            &                    & Mailadresse \\ \hline
	\end{tabular}
\end{table}
\paragraph{Antwort}Die Antwort ist wie folgt aufgebaut:
\begin{table}[H]
	\begin{tabular}{|c|c|c|p{6.5cm}|}
		\hline
		\textbf{Paramtername} & \textbf{Datentyp} & \textbf{Konstante} & \textbf{Kurzbeschreibung}            \\ \hline                
		success             & bool             &                 & Erfolgreich wenn Wert {\glqq true\grqq} ist \\ \hline
		payload             & string           &                 & Wahr, wenn Mailadresse bereits verwendet wird \\ \hline
	\end{tabular}
\end{table}
\chapter{Datenbank-Spezifikation}
\section{Tabellen-Übersicht}
\begin{longtable}[H]{|l|p{9cm}|}
	\hline
	\textbf{Api-Befehl} 	   & \textbf{Kurzbeschreibung}              \\ \hline
	dload\_\_api-token         & Daten für Kommunikation mit Modulen    \\ \hline
	dload\_\_module-rank       & Daten für modulbasierte Ränge          \\ \hline
	dload\_\_pictures          & Daten zu hochgeladenen Bildern         \\ \hline
	dload\_\_pictures-validate & Validierungsdaten zu Bildern           \\ \hline
	dload\_\_point-origin      & Herkunft für Rang-Punkte               \\ \hline
	dload\_\_ranks             & Verfügbar Ränge und Daten zu diesen    \\ \hline
	dload\_\_ranksystem        & Rangpunkte der Nutzer und Herkunft     \\ \hline
	dload\_\_rights-tools      & Tabelle für toolspezifische Rechte     \\ \hline
	dload\_\_roles             & Daten für Rollen                       \\ \hline
	dload\_\_session           & Sessiondaten der Nutzer                \\ \hline
	dload\_\_source\_type      & Typen von Quellen                      \\ \hline
	dload\_\_stories           & Daten von hochgeladenen Geschichten    \\ \hline
	dload\_\_stories-validate  & Validierungsdaten von Geschichten      \\ \hline
	dload\_\_user-login        & Nutzerdaten für Login                  \\ \hline
	dload\_\_visitors          & Statistische Nutzungsdaten             \\ \hline
	dload\_\_user\_requests    & Statistiken zur Nutzung des Kontaktformulars \\ \hline
\end{longtable}
\newpage
\section{Erläuterung}
\subsection{Abkürzungen}
\begin{table}[H]
	\begin{tabular}{|c|p{12cm}|}
		\hline
		\textbf{Abkürzung} & \textbf{Bedeutung} \\ \hline
		PRI & Primary Key \\ \hline
		FOR & Foreign Key / Fremdschlüssel \\ \hline
		UNI & Unique Key \\ \hline
	\end{tabular}
\end{table}
\subsection{Aufbau}
In den folgenden Abschnitten wird zuerst etwas über die Verwendung der Tabelle gesagt. Anschließend ist noch der Aufbau detailliert geschildert. Das Feld {\glqq Null\grqq} besagt, ob dieser Wert in der Tabelle den Wert {\glqq null\grqq} annehmen darf. Im Feld {\glqq Key\grqq} ist zu sehen ob dieser Wert als Schlüssel verwendet wird. Sofern dies ein Fremdschlüssel ({\glqq FOR\grqq}) ist, ist in der nächsten Tabelle zu finden auf welches Feld welcher Tabelle dieser sich bezieht.
\section{Tabellen}
\subsection{dload\_\_api-token}
\subsubsection{Verwendung} Diese Tabelle wird verwendet um alle Daten zu verbundenen Modulen zu speichern.
\subsubsection{Inhalt}
\begin{table}[H]
	\begin{tabular}{|c|c|c|c|c|p{3.5cm}|}
		\hline
		\textbf{Feldname} & \textbf{Datentyp} & \textbf{Null} & \textbf{Standardwert} & \textbf{Key}   & \textbf{Besonderheiten} \\ \hline
		id & int & NO &  & PRI & auto\_increment \\ \hline
		token & varchar & NO &  & UNI & \\ \hline
		name & mediumtext & NO &  &  & \\ \hline
		apiuri & mediumtext & NO &  &  & \\ \hline
	\end{tabular}
\end{table}
\subsubsection{Beschreibung}
\begin{table}[H]
	\begin{tabular}{|c|p{12cm}|}
		\hline
		\textbf{Feldname} & \textbf{Beschreibung} \\ \hline
		id & numerischer Identifikator des Moduls \\ \hline
		token & Eindeutiger alphanumerischer Identifikator des Moduls \\ \hline
		name & Modulname \\ \hline
		apiuri & URL der Reverse-API \\ \hline
	\end{tabular}
\end{table}
\subsubsection{Fremdschlüssel}
In dieser Tabelle sind keine Fremdschlüssel vorhanden.
\subsection{dload\_\_module-rank}
\subsubsection{Verwendung} Diese Tabelle wird verwendet um alle Daten zu Rangnamen zu speichern, welche auf ein Modul angepasst wurden.
\subsubsection{Inhalt}
\begin{table}[H]
	\begin{tabular}{|c|c|c|c|c|p{3.5cm}|}
		\hline
		\textbf{Feldname} & \textbf{Datentyp} & \textbf{Null} & \textbf{Standardwert} & \textbf{Key}   & \textbf{Besonderheiten} \\ \hline
		id & int & NO &  & PRI & auto\_increment \\ \hline
		aid & int & NO &  & FOR & \\ \hline
		rid & int & NO &  & FOR & \\ \hline
		name & varchar & NO &  &  & \\ \hline
	\end{tabular}
\end{table}
\subsubsection{Beschreibung}
\begin{table}[H]
	\begin{tabular}{|c|p{12cm}|}
		\hline
		\textbf{Feldname} & \textbf{Beschreibung} \\ \hline
		id & Identifikator des Modul basierten Rangnamens \\ \hline
		aid & Identifikator eines Moduls \\ \hline
		rid & Identifikator eines Ranges \\ \hline
		name & Name des Ranges im Modul \\ \hline
	\end{tabular}
\end{table}
\subsubsection{Fremdschlüssel}
\begin{table}[H]
	\begin{tabular}{|c|p{12.5cm}|}
		\hline
		\textbf{Feldname} & \textbf{Fremd-Feld} \\ \hline
		aid & dload\_\_api-token.id \\ \hline
		rid & dload\_\_ranks.id \\ \hline
	\end{tabular}
\end{table}
\subsection{dload\_\_pictures}
\subsubsection{Verwendung} Diese Tabelle wird verwendet um alle Daten zu hochgeladenen Bildern zu speichern.
\subsubsection{Inhalt}
\begin{table}[H]
	\begin{tabular}{|c|c|c|c|c|p{3.5cm}|}
		\hline
		\textbf{Feldname} & \textbf{Datentyp} & \textbf{Null} & \textbf{Standardwert} & \textbf{Key}   & \textbf{Besonderheiten} \\ \hline
		id & int & NO &  & PRI & auto\_increment \\ \hline
		token & varchar & NO &  & UNI & \\ \hline
		title & mediumtext & NO &  &  & \\ \hline
		description & longtext & YES & NULL &  & \\ \hline
		picurl & mediumtext & NO &  &  & \\ \hline
		preview & longtext & NO &  &  & \\ \hline
		source & mediumtext & NO &  &  & \\ \hline
		sourcetype & int & NO &  & FOR & \\ \hline
		uid & int & NO &  & FOR & \\ \hline
		aid & int & NO & 5 & FOR & \\ \hline
		creationdate & timestamp & NO & current\_timestamp() &  & \\ \hline
		deleted & tinyint & NO & 0 &  & \\ \hline
	\end{tabular}
\end{table}
\subsubsection{Beschreibung}
\begin{table}[H]
	\begin{tabular}{|c|p{12cm}|}
		\hline
		\textbf{Feldname} & \textbf{Beschreibung} \\ \hline
		id & Eindeutiger Identifikator des Bildes \\ \hline
		token & Eindeutiger String Identifikator des Bildes \\ \hline
		title & Titel des Bildes \\ \hline
		description & Beschreibung des Bildes \\ \hline
		picurl & Dateiname des Bildes \\ \hline
		preview & Vorschau des Bildes, base64 encodiert \\ \hline
		source & Quellenangabe des Bildes \\ \hline
		sourcetype & Typ der Quelle des Bildes \\ \hline
		uid & Nutzerid des hochladenden \\ \hline
		aid & Id des hochladenden Moduls \\ \hline
		creationdate & Zeitstempel des Eintragens \\ \hline
		deleted & Zustandsmarkierung {\glqq gelöscht\grqq} \\ \hline
	\end{tabular}
\end{table}
\subsubsection{Fremdschlüssel}
\begin{table}[H]
	\begin{tabular}{|c|p{12.5cm}|}
		\hline
		\textbf{Feldname} & \textbf{Fremd-Feld} \\ \hline
		aid & dload\_\_api-token.id \\ \hline
		uid & dload\_\_user-login.id \\ \hline
		sourcetype & dload\_\_source\_type.id \\ \hline
	\end{tabular}
\end{table}
\subsection{dload\_\_pictures-validate}
\subsubsection{Verwendung} Diese Tabelle wird verwendet um alle Validierungen zu einem Bild zu speichern. Hierzu wird die ID des Bildes und die Nutzer-ID des validierenden Nutzers benötigt.
\subsubsection{Inhalt}
\begin{table}[H]
	\begin{tabular}{|c|c|c|c|c|p{3.5cm}|}
		\hline
		\textbf{Feldname} & \textbf{Datentyp} & \textbf{Null} & \textbf{Standardwert} & \textbf{Key}   & \textbf{Besonderheiten} \\ \hline
		id & int & NO &  & PRI & auto\_increment \\ \hline
		picture-id & int & NO &  & FOR & \\ \hline
		user-id & int & NO &  & FOR & \\ \hline
		value & int & NO &  &  & \\ \hline
		date & datetime & NO & current\_timestamp() &  & \\ \hline
	\end{tabular}
\end{table}
\subsubsection{Beschreibung}
\begin{table}[H]
	\begin{tabular}{|c|p{12cm}|}
		\hline
		\textbf{Feldname} & \textbf{Beschreibung} \\ \hline
		id & Identifikator der Validierung \\ \hline
		picture-id & Identifikator des validierten Bildes \\ \hline
		user-id & Identifikator des validierenden Nutzers \\ \hline
		value & Validierungswert, mit dem Nutzer validiert hat \\ \hline
		date & Zeitstempel der Validierung \\ \hline
	\end{tabular}
\end{table}
\subsubsection{Fremdschlüssel}
\begin{table}[H]
	\begin{tabular}{|c|p{12.5cm}|}
		\hline
		\textbf{Feldname} & \textbf{Fremd-Feld} \\ \hline
		user-id & dload\_\_user-login.id \\ \hline
		picture-id & dload\_\_pictures.id \\ \hline
	\end{tabular}
\end{table}
\subsection{dload\_\_point-origin}
\subsubsection{Verwendung} Diese Tabelle wird verwendet um den Ursprung der Rangpunkte zu speichern.
\subsubsection{Inhalt}
\begin{table}[H]
	\begin{tabular}{|c|c|c|c|c|p{3.5cm}|}
		\hline
		\textbf{Feldname} & \textbf{Datentyp} & \textbf{Null} & \textbf{Standardwert} & \textbf{Key}   & \textbf{Besonderheiten} \\ \hline
		id & int & NO &  & PRI & auto\_increment \\ \hline
		name & mediumtext & NO &  &  & \\ \hline
	\end{tabular}
\end{table}
\subsubsection{Beschreibung}
\begin{table}[H]
	\begin{tabular}{|c|p{12cm}|}
		\hline
		\textbf{Feldname} & \textbf{Beschreibung} \\ \hline
		id & Identifikator des Punkteursprungs \\ \hline
		name & Bezeichnung des Punkteursprungs \\ \hline
	\end{tabular}
\end{table}
\subsubsection{Fremdschlüssel}
In dieser Tabelle sind keine Fremdschlüssel vorhanden.
\subsection{dload\_\_ranks}
\subsubsection{Verwendung} Diese Tabelle wird verwendet um alle Daten Rängen zu speichern.
\subsubsection{Inhalt}
\begin{table}[H]
	\begin{tabular}{|c|c|c|c|c|p{3.5cm}|}
		\hline
		\textbf{Feldname} & \textbf{Datentyp} & \textbf{Null} & \textbf{Standardwert} & \textbf{Key}   & \textbf{Besonderheiten} \\ \hline
		id & int & NO &  & PRI & auto\_increment \\ \hline
		value & int & NO & 50 &  & \\ \hline
		name & mediumtext & NO &  &  & \\ \hline
		icon & varchar & NO & '''newbie.svg''' &  & \\ \hline
	\end{tabular}
\end{table}
\subsubsection{Beschreibung}
\begin{table}[H]
	\begin{tabular}{|c|p{12cm}|}
		\hline
		\textbf{Feldname} & \textbf{Beschreibung}\\ \hline
		id & Identifikator des Ranges \\ \hline
		value & Rangwert \\ \hline
		name & Name des Ranges \\ \hline
		icon & Icon des Ranges \\ \hline
	\end{tabular}
\end{table}
\subsubsection{Fremdschlüssel}
In dieser Tabelle sind keine Fremdschlüssel vorhanden.
\subsection{dload\_\_ranksystem}
\subsubsection{Verwendung} Diese Tabelle wird verwendet um alle erhaltenen und vergebenen Punkten Nutzern zu zuordnen, um diese anschließend auf einen Rang abzubilden.
\subsubsection{Inhalt}
\begin{table}[H]
	\begin{tabular}{|c|c|c|c|c|p{3.5cm}|}
		\hline
		\textbf{Feldname} & \textbf{Datentyp} & \textbf{Null} & \textbf{Standardwert} & \textbf{Key}   & \textbf{Besonderheiten} \\ \hline
		id & int & NO &  & PRI & auto\_increment \\ \hline
		uid & int & NO &  & FOR & \\ \hline
		aid & int & NO &  & FOR & \\ \hline
		value & int & NO &  &  & \\ \hline
		date & timestamp & NO & current\_timestamp() &  & on update current\_timestamp()\\ \hline
		oid & int & NO &  & FOR & \\ \hline
	\end{tabular}
\end{table}
\subsubsection{Beschreibung}
\begin{table}[H]
	\begin{tabular}{|c|p{12cm}|}
		\hline
		\textbf{Feldname} & \textbf{Beschreibung} \\ \hline
		id & Identifikator des zugeschriebenen Punktes \\ \hline
		uid & Nutzer-ID des zugeschriebenen Nutzers \\ \hline
		aid & Identifikator des Ursprungsmoduls \\ \hline
		value & Zugeschriebene Punkteanzahl \\ \hline
		date & Zeitstempel des hinzufügens \\ \hline
		oid & Grund der hinzugefügten Punkte \\ \hline
	\end{tabular}
\end{table}
\subsubsection{Fremdschlüssel}
\begin{table}[H]
	\begin{tabular}{|c|p{12.5cm}|}
		\hline
		\textbf{Feldname} & \textbf{Fremd-Feld} \\ \hline
		aid & dload\_\_api-token.id \\ \hline
		uid & dload\_\_user-login.id \\ \hline
		oid & dload\_\_point-origin.id \\ \hline
	\end{tabular}
\end{table}
\subsection{dload\_\_rights-tools}
\subsubsection{Verwendung} Diese Tabelle wird verwendet um toolspezifische Rechte zu vergeben. Diese Tabelle wird aktuell nicht genutzt.
\subsubsection{Inhalt}
\begin{table}[H]
	\begin{tabular}{|c|c|c|c|c|p{3.5cm}|}
		\hline
		\textbf{Feldname} & \textbf{Datentyp} & \textbf{Null} & \textbf{Standardwert} & \textbf{Key}   & \textbf{Besonderheiten} \\ \hline
		id & int & NO &  & PRI & auto\_increment \\ \hline
		uid & int & NO &  & FOR & \\ \hline
		aid & int & NO &  & FOR & \\ \hline
		role & int & NO &  & FOR & \\ \hline
		disabled & tinyint & NO & 1 &  & \\ \hline
	\end{tabular}
\end{table}
\subsubsection{Beschreibung}
\begin{table}[H]
	\begin{tabular}{|c|p{12cm}|}
		\hline
		\textbf{Feldname} & \textbf{Beschreibung} \\ \hline
		id & Identifikator der Rollenzuweisung \\ \hline
		uid & Identifikator des Nutzers dem eine Rolle zugewiesen wird \\ \hline
		aid & Modulidentifikator für welches Rolle zugewiesen wird \\ \hline
		role & Identifikator der zugewiesenen Rolle \\ \hline
		disabled & Deaktivierung des Users in Modul \\ \hline
	\end{tabular}
\end{table}
\subsubsection{Fremdschlüssel}
\begin{table}[H]
	\begin{tabular}{|c|p{12.5cm}|}
		\hline
		\textbf{Feldname} & \textbf{Fremd-Feld} \\ \hline
		aid & dload\_\_api-token.id \\ \hline
		uid & dload\_\_user-login.id \\ \hline
		role & dload\_\_roles.id \\ \hline
	\end{tabular}
\end{table}
\subsection{dload\_\_roles}
\subsubsection{Verwendung} Diese Tabelle wird verwendet um Daten zu Rollen zu speichern.
\subsubsection{Inhalt}
\begin{table}[H]
	\begin{tabular}{|c|c|c|c|c|p{3.5cm}|}
		\hline
		\textbf{Feldname} & \textbf{Datentyp} & \textbf{Null} & \textbf{Standardwert} & \textbf{Key}   & \textbf{Besonderheiten} \\ \hline
		id & int & NO &  & PRI & auto\_increment \\ \hline
		value & int & NO & 0 &  & \\ \hline
		name & varchar & NO &  & UNI & \\ \hline
	\end{tabular}
\end{table}
\subsubsection{Beschreibung}
\begin{table}[H]
	\begin{tabular}{|c|p{12cm}|}
		\hline
		\textbf{Feldname} & \textbf{Beschreibung} \\ \hline
		id & Identifikator der Rolle \\ \hline
		value & Wert der Rolle \\ \hline
		name & Name der Rolle \\ \hline
	\end{tabular}
\end{table}
\subsubsection{Fremdschlüssel}
In dieser Tabelle sind keine Fremdschlüssel vorhanden.
\subsection{dload\_\_session}
\subsubsection{Verwendung} Diese Tabelle wird verwendet um Daten zu Nutzersessions zu speichern.
\subsubsection{Inhalt}
\begin{table}[H]
	\begin{tabular}{|c|c|c|c|c|p{3.5cm}|}
		\hline
		\textbf{Feldname} & \textbf{Datentyp} & \textbf{Null} & \textbf{Standardwert} & \textbf{Key}   & \textbf{Besonderheiten} \\ \hline
		ses\_id & varchar & NO &  & PRI & \\ \hline
		ses\_time & int & NO &  &  & \\ \hline
		ses\_value & varchar & NO &  &  & \\ \hline
	\end{tabular}
\end{table}
\subsubsection{Fremdschlüssel}
In dieser Tabelle sind keine Fremdschlüssel vorhanden.
\subsection{dload\_\_stories}
\subsubsection{Verwendung} Diese Tabelle wird verwendet um alle Daten zu hochgeladenen Geschichten zu speichern.
\subsubsection{Inhalt}
\begin{table}[H]
	\begin{tabular}{|c|c|c|c|c|p{3.5cm}|}
		\hline
		\textbf{Feldname} & \textbf{Datentyp} & \textbf{Null} & \textbf{Standardwert} & \textbf{Key}   & \textbf{Besonderheiten} \\ \hline
		id & int & NO &  & PRI & auto\_increment \\ \hline
		name & mediumtext & NO &  &  & \\ \hline
	\end{tabular}
\end{table}
\subsubsection{Beschreibung}
\begin{table}[H]
	\begin{tabular}{|c|p{12cm}|}
		\hline
		\textbf{Feldname} & \textbf{Beschreibung}\\ \hline
		id   & Identifikator des Typs der Quelle \\ \hline
		name & Name des Typs der Quelle \\ \hline
	\end{tabular}
\end{table}
\subsubsection{Fremdschlüssel}
In dieser Tabelle sind keine Fremdschlüssel vorhanden.
\subsection{dload\_\_stories}
\subsubsection{Verwendung} Diese Tabelle wird verwendet um alle Daten zu hochgeladenen Geschichten zu speichern.
\subsubsection{Inhalt}
\begin{table}[H]
	\begin{tabular}{|c|c|c|c|c|p{3.5cm}|}
		\hline
		\textbf{Feldname} & \textbf{Datentyp} & \textbf{Null} & \textbf{Standardwert} & \textbf{Key}   & \textbf{Besonderheiten} \\ \hline
		id & int & NO &  & PRI & auto\_increment \\ \hline
		user\_id & int & NO &  & FOR & \\ \hline
		storie\_token & varchar & NO &  & UNI & \\ \hline
		title & mediumtext & NO &  &  & \\ \hline
		story & longtext & NO &  &  & \\ \hline
		aid & int & NO &  & FOR & \\ \hline
		approved & tinyint & NO & 0 &  & \\ \hline
		date & datetime & NO & current\_timestamp() &  & \\ \hline
		points\_received & tinyint & NO & 0 &  & \\ \hline
		deleted & tinyint & NO & 0 &  & \\ \hline
	\end{tabular}
\end{table}
\subsubsection{Beschreibung}
\begin{table}[H]
	\begin{tabular}{|c|p{12cm}|}
		\hline
		\textbf{Feldname} & \textbf{Beschreibung}\\ \hline
		id & numerischer Identifikator der Geschichte \\ \hline
		user\_id & ID des hochladenden Nutzers \\ \hline
		storie\_token & alphanumerischer Identifikator der Geschichte \\ \hline
		title & Titel der Geschichte \\ \hline
		story & Inhalt der Geschichte \\ \hline
		aid & numerischer Modulidentifikator des hochladenden Moduls \\ \hline
		approved & Status der Freischaltung \\ \hline
		date & Datum des Hochladens oder Änderns \\ \hline
		points\_received & Status der Punktevergabe \\ \hline
		deleted & Status, ob Objekt als gelöscht gilt \\ \hline
	\end{tabular}
\end{table}
\subsubsection{Fremdschlüssel}
\begin{table}[H]
	\begin{tabular}{|c|p{12.5cm}|}
		\hline
		\textbf{Feldname} & \textbf{Fremd-Feld} \\ \hline
		aid & dload\_\_api-token.id \\ \hline
		user\_id & dload\_\_user-login.id \\ \hline
	\end{tabular}
\end{table}
\subsection{dload\_\_stories-validate}
\subsubsection{Verwendung} Diese Tabelle wird verwendet um alle Validerungsdaten zu hochgeladenen Bildern zu speichern.
\subsubsection{Inhalt}
\begin{table}[H]
	\begin{tabular}{|c|c|c|c|c|p{3.5cm}|}
		\hline
		\textbf{Feldname} & \textbf{Datentyp} & \textbf{Null} & \textbf{Standardwert} & \textbf{Key}   & \textbf{Besonderheiten} \\ \hline
		id & int & NO &  & PRI & auto\_increment \\ \hline
		sid & int & NO &  & FOR & \\ \hline
		uid & int & NO &  & FOR & \\ \hline
		value & int & NO &  &  & \\ \hline
		date & datetime & NO & current\_timestamp() &  & \\ \hline
	\end{tabular}
\end{table}
\subsubsection{Beschreibung}
\begin{table}[H]
	\begin{tabular}{|c|p{12cm}|}
		\hline
		\textbf{Feldname} & \textbf{Beschreibung} \\ \hline
		id & Identifikator der Validierung \\ \hline
		sid & numerischer Identifikator der Geschichte \\ \hline
		uid & Nutzeridentifikator des Validierenden \\ \hline
		value & Wert der Validierung \\ \hline
		date & Zeitstempel der Validierung \\ \hline
	\end{tabular}
\end{table}
\subsubsection{Fremdschlüssel}
\begin{table}[H]
	\begin{tabular}{|c|p{12.5cm}|}
		\hline
		\textbf{Feldname} & \textbf{Fremd-Feld} \\ \hline
		uid & dload\_\_user-login.id \\ \hline
		sid & dload\_\_stories.id \\ \hline
	\end{tabular}
\end{table}
\subsection{dload\_\_user-login}
\subsubsection{Verwendung} Diese Tabelle wird verwendet um alle Daten zu Nutzern zu speichern.
\subsubsection{Inhalt}
\begin{table}[H]
	\begin{tabular}{|c|c|c|c|c|p{3.5cm}|}
		\hline
		\textbf{Feldname} & \textbf{Datentyp} & \textbf{Null} & \textbf{Standardwert} & \textbf{Key}   & \textbf{Besonderheiten} \\ \hline
		id & int & NO &  & PRI & auto\_increment \\ \hline
		name & varchar & NO &  & UNI & \\ \hline
		password & longtext & NO &  &  & \\ \hline
		firstname & mediumtext & YES & NULL &  & \\ \hline
		lastname & mediumtext & YES & NULL &  & \\ \hline
		email & mediumtext & NO &  &  & \\ \hline
		enabled & tinyint & NO & 1 &  & \\ \hline
		mailvalidated & tinyint & NO & 0 &  & \\ \hline
		role & int & NO &  & FOR & \\ \hline
		creationdate & datetime & NO & current\_timestamp() &  & \\ \hline
	\end{tabular}
\end{table}
\subsubsection{Beschreibung}
\begin{table}[H]
	\begin{tabular}{|c|p{12cm}|}
		\hline
		\textbf{Feldname} & \textbf{Beschreibung} \\ \hline
		id & Identifikator des Nutzers \\ \hline
		name & Nutzername \\ \hline
		password & Hash des Passwortes \\ \hline
		firstname & Vorname des Nutzers \\ \hline
		lastname & Nachname des Nutzers \\ \hline
		email & E-Mailadresse des Nutzers \\ \hline
		enabled & Status der Freischaltung \\ \hline
		mailvalidated & Status der E-Mail-Validierung \\ \hline
		role & Rollenidentifikator \\ \hline
		creationdate & Zeitstempel des Anlegens des Nutzers \\ \hline
	\end{tabular}
\end{table}
\subsubsection{Fremdschlüssel}
\begin{table}[H]
	\begin{tabular}{|c|p{12.5cm}|}
		\hline
		\textbf{Feldname} & \textbf{Fremd-Feld} \\ \hline
		role & dload\_\_roles.id \\ \hline
	\end{tabular}
\end{table}
\subsection{dload\_\_visitors}
\subsubsection{Verwendung} Diese Tabelle wird verwendet um statistische Nutzungsdaten zur Plattform zu sammeln.
\subsubsection{Inhalt}
\begin{table}[H]
	\begin{tabular}{|c|c|c|c|c|p{3.5cm}|}
		\hline
		\textbf{Feldname} & \textbf{Datentyp} & \textbf{Null} & \textbf{Standardwert} & \textbf{Key}   & \textbf{Besonderheiten} \\ \hline
		id & int & NO &  & PRI & auto\_increment \\ \hline
		ip & varchar & NO &  &  & \\ \hline
		date & timestamp & NO & current\_timestamp() &  & \\ \hline
		type & varchar & NO &  &  & \\ \hline
	\end{tabular}
\end{table}
\subsubsection{Beschreibung}
\begin{table}[H]
	\begin{tabular}{|c|p{12cm}|}
		\hline
		\textbf{Feldname} & \textbf{Beschreibung} \\ \hline
		id & Identifikator des Eintrags \\ \hline
		ip & IP-Adresse des Aufrufers \\ \hline
		date & Zeitstempel des Aufrufs \\ \hline
		type & Login-Typ des Aufrufs (Gast oder Nutzer) \\ \hline
	\end{tabular}
\end{table}
\subsubsection{Fremdschlüssel}
In dieser Tabelle sind keine Fremdschlüssel vorhanden.
\subsection{dload\_\_user\_requests}
\subsubsection{Verwendung} Diese Tabelle wird verwendet um statistische Nutzungsdaten des Kontaktformulars zu sammeln.
\subsubsection{Inhalt}
\begin{table}[H]
	\begin{tabular}{|c|c|c|c|c|p{3.5cm}|}
		\hline
		\textbf{Feldname} & \textbf{Datentyp} & \textbf{Null} & \textbf{Standardwert} & \textbf{Key}   & \textbf{Besonderheiten} \\ \hline
		id & int & NO &  & PRI & auto\_increment \\ \hline
		date & timestamp & NO & current\_timestamp() &  & \\ \hline
		ip & varchar & NO &  &  & \\ \hline
		module & int & YES & NULL & FOR & \\ \hline
	\end{tabular}
\end{table}
\subsubsection{Beschreibung}
\begin{table}[H]
	\begin{tabular}{|c|p{12cm}|}
		\hline
		\textbf{Feldname} & \textbf{Beschreibung} \\ \hline
		id & Identifikator des Eintrags \\ \hline
		date & Zeitstempel der Nutzung des Kontaktformulars \\ \hline
		ip & IP-Adresse des Nutzenden \\ \hline
		module & Modul für welchen bei welchem das Formular genutzt wurde \\ \hline
	\end{tabular}
\end{table}
\subsubsection{Fremdschlüssel}
\begin{table}[H]
	\begin{tabular}{|c|p{12.5cm}|}
		\hline
		\textbf{Feldname} & \textbf{Fremd-Feld} \\ \hline
		module & dload\_\_api-token.id \\ \hline
	\end{tabular}
\end{table}
\chapter{Datei-Übersicht}
\section{Auflistung}
\begin{longtable}[H]{|c|p{10cm}|}
	\hline
	\textbf{Dateiname} & \textbf{Beschreibung} \\ \hline
	api                          & API-Endpunkt der Modula-API \\ \hline
	apimgmt                      & Datei mit Seite zum Management der verschiedenen APIs \\ \hline
	api-functions                & Datei mit Funktionen für Modul-API \\ \hline
	authSystem                   & Funktionen des Authentifizierungsystems \\ \hline
	captcha                      & Funktionen zum generieren eines Captchas \\ \hline
	config-sample                & Beispiel Konfigurationsdatei \\ \hline
	config                       & Konfigurationsdatei (gleicher Aufbau wie config-sample.php) \\ \hline
	api-db                       & Datei mit Funktionen für Moduldatenbank \\ \hline
	basic-db                     & Datei mit grundlegenden Funktionen zum Datenbankzugriff \\ \hline
	inc-db-sub                   & Datei zum Einbinden aller Dateien für Datenbankzugriff für Dateien in Unterordnern \\ \hline
	inc-db                       & Datei zum Einbinden aller Dateien für Datenbankzugriff \\ \hline
	logging                      & Datei mit Funktionen für Statistik-Datenbank für Nutzungsanalyse \\ \hline
	module-rank-db               & Datei mit Funktionen für Datenbank für modulbasierte Rangnamen \\ \hline
	module-rights-db             & Datei mit Funktionen für Datenbank für modulbasierte Rechte \\ \hline
	picture-db                   & Datei mit Funktionen für Datenbank für Bilder \\ \hline
	point-reason-db              & Datei mit Funktionen für Datenbank mit Rangpunktursprüngen \\ \hline
	rank-point-db                & Datei mit Funktionen für Datenbank mit Rangpunkten \\ \hline
	ranks-db                     & Datei mit Funktionen für Datenbank mit Rängen \\ \hline
	role-db                      & Datei mit Funktionen für Datenbank mit Rollen \\ \hline
	session-db                   & Datei mit Funktionen für Datenbank mit Session-Daten \\ \hline
	statistics-basic-dbfunctions & Datei mit grundlegenden Funktionen für Statistik \\ \hline
	statistics-pictures-db       & Datei mit statistischen Funktionen für Datenbank mit Bildern \\ \hline
	statistics-story-db          & Datei mit statistischen Funktionen für Datenbank mit Geschichten \\ \hline
	statistics-user              & Datei mit statistischen Funktionen für Datenbank mit Geschichten \\ \hline
	statistics-contact           & Datei mit Funktionen für Statistik-Datenbank für Nutzung des Kontaktformulars \\ \hline
	story-db                     & Datei mit Funktionen für Datenbank mit Geschichten \\ \hline
	user-db                      & Datei mit Funktionen für Datenbank mit Nutzerdaten \\ \hline
	validate-picture-db          & Datei mit Funktionen für Datenbank mit Validierungsdaten zu Bildern \\ \hline
	validate-story-db            & Datei mit Funktionen für Datenbank mit Validierungsdaten zu Geschichten \\ \hline
	deletions                    & Datei mit Wrapper-Funktionen zum Löschen \\ \hline
	functionLib                  & Datei mit grundlegenden Funktionen, welche an multiplen Stellen benutzt werden \\ \hline
	inc-sub                      & Datei zum Einbinden aller Dateien für Funktionen für Dateien in Unterordnern \\ \hline
	inc                          & Datei zum Einbinden aller Dateien für Funktionen \\ \hline
	mailer                       & Datei mit Funktionen zum versenden von Mails \\ \hline
	MailTemplates                & Datei mit Templates für Mails \\ \hline
	mapi-functions               & Datei für Funktionen der Management-API \\ \hline
	SessionValues                & Datei für Funktionen zum verwalten von Sessions \\ \hline
	settings                     & Datei mit grundlegenden Settings \\ \hline
	statistic-calc               & Datei mit statistischen Funktionen \\ \hline
	uapi-functions               & Datei mit Funktionen für User-API \\ \hline
	changePwd                    & Datei mit Seite zum Ändern des Passworts \\ \hline
	contact                      & Datei mit Kontaktformular \\ \hline
	hub                          & Datei mit Seite für Funktionsübersicht \\ \hline
	impressum                    & Datei mit Seite für Impressum \\ \hline
	index                        & Datei mit Startseite \\ \hline
	mapi                         & API-Endpunkt der Management-API \\ \hline
	myUser                       & Datei mit Seite zum Anzeigen der eigenen Nutzerdaten \\ \hline
	privacy-policy               & Datei mit Seite für Datenschutzerklärung \\ \hline
	rankMgmt                     & Datei mit Seite zum Rang-Management \\ \hline
	registration                 & Datei mit Seite zur Registrierung \\ \hline
	resetPwd                     & Datei mit Seite zum Passwort neusetzen \\ \hline
	roleMgmt                     & Datei mit Seite zum Rollen-Management \\ \hline
	statistics                   & Datei mit Seite für Statistiken \\ \hline
	uapi                         & API-Endpunkt der User-API \\ \hline
	usermgmt                     & Datei mit Seite zum User-Management \\ \hline
	moduleRights                 & Datei mit Seite zum modulbasierte Rechte \\ \hline
	validate                     & Datei mit Seite zur Validierung der Mailadresse \\ \hline
	validations                  & Datei mit Seite zur Anzeige der vergebenen Rangpunkte \\ \hline
	addRank                      & Datei mit JavaScript Funktionen zum Hinzufügen eines Ranges \\ \hline
	addRole                      & Datei mit JavaScript Funktionen zum Hinzufügen einer Rolle \\ \hline
	coseMainLib                  & Datei mit JavaScript Funktionen, welche auf mehreren Seiten genutzt werden \\ \hline
	coseMainLibNg                & Datei mit TypeScript Funktionen, welche auf mehreren Seiten genutzt werden \\ \hline
	DropDownMenuApi              & Datei mit JavaScript Funktionen für DropDownMenus des Nutzer-Management \\ \hline
	loadCaptcha                  & Datei mit JavaScript Funktionen zum Laden eines Captchas \\ \hline
	loadCookie                   & Datei mit JavaScript Funktionen zum Anzeigen des Cookie-Banner \\ \hline
	statistics                   & Datei mit JavaScript Funktionen für Statistik-Seite \\ \hline
	modulerights                 & Datei mit TypeScript Funktionen für Modulberechtigungsseite \\ \hline
	Usermanagement               & Datei mit TypeScript Funktionen für Nutzerverwaltungsseite \\ \hline
	registration                 & Datei mit TypeScript Funktionen für Registrierungsseite \\ \hline
\end{longtable}
\section{Speicherort}
\subsection{Eigene oder Geänderte Dateien}
Die oben genannten Dateien sind unter folgenden Pfaden zu finden:
\begin{itemize}
	\item {\glqq ./\grqq}
	\item {\glqq ./bin\grqq}
	\item {\glqq ./bin/database\grqq}
	\item {\glqq ./js\grqq}
	\item {\glqq ./tjs\grqq}
\end{itemize}
Die dem Projekt zugehörigen Cascading-Stylesheets, welche bearbeitet oder geschrieben wurden, befinden sich im Ordner {\glqq ./css\grqq}. Alle verwendeten Icons befinden sich im Ordner {\glqq ./images\grqq}.
\subsection{Weitere Dateien}
Alle Dateien unter {\glqq ./jse\grqq} sind aus Bibliotheken übernommen und nicht verändert worden. Sie werden für Bootstrap, Fontawesome, Leaflet, Lightbox oder als Abhängigkeit dieser gebraucht. Für alle Dateien unter {\glqq ./csse\grqq} gilt dies ebenso. Die Dokumentation entsprechender Funktionen ist nicht teil dieser Dokumentation. Für diese Dateien ist die jeweilige Dokumentation zu verwenden. 

\setcounter{secnumdepth}{3}
\chapter{Funktionsübersicht}
Die Funktionen sind nach den Dateien sortiert in denen die Funktionen enthalten sind. Des weiteren ist auch die Reihenfolge der Funktionen mit denen in der Datei übereinstimmend. \\
Hier finden Sie insbesondere eine Kurzbeschreibung aller verfügbaren Funktionen innerhalb des Projektes sowie eine Beschreibung des Aufbaus der meisten Seiten. Die Funktionen sind alle durch einen Dokumentationsblog mit folgenden Bestandteilen Beschrieben:
\begin{itemize}
	\item Parameter
	\item Beschreibung
	\item Vorgehensweise
\end{itemize}
Diese Bestandteile enthalten die Nachfolgend beschriebenen Informationen:
\paragraph{Parameter} Hier finden Sie eine Kurzbeschreibung aller verfügbaren Parameter der Funktion, inklusive aller optionalen. Diese Beschreibung ist meist tabellarisch gehalten und hat folgendes Schema:
\begin{table}[H]
	\begin{tabular}{|c|p{11cm}|}
		\hline
		\textbf{Parametername} & \textbf{Parameterbeschreibung} \\ \hline
		\$Parameter1 & Beschreibung 1 \\ \hline
		\$Parameter2 & Beschreibung 2 \\ \hline
	\end{tabular}
\end{table}
Sollte eine Funktionen einen Parameter des Typs Array haben, so wird dieser im Anschluss bei Bedarf näher aufgeschlüsselt. Dies geschieht in einer tabellarisch ähnlichen Form wie die Beschreibung und Nennung der Parameter.
\paragraph{Beschreibung} Hier finden Sie eine Beschreibung der Funktion, sowie die Quellen für Informationen, welche die Funktion verarbeitet. Dabei können die Quellen direkt oder Indirekt abgefragt werden. Die Quellen sind stets als Auflistung dargestellt.
\paragraph{Vorgehensweise} Bei einigen größeren Funktionen ist unter diesem Punkt eine Beschreibung der Vorgehensweise der Funktion zu finden. Diese ist meist Abstrakt gehalten und dient dem groben Verständnis der Funktion.
\newpage
\section{api}
\subsection{Allgemeines} Diese Datei ist der Endpunkt der Modul-API.
Die Datei ist direkt durch den Nutzer aufrufbar. Sie setzt auch die entsprechende Konstante und bindet alle notwendigen Dateien ein:
\begin{lstlisting}[language=php]
define('NICE_PROJECT', true);
require_once "bin/inc.php";
\end{lstlisting}
\subsection{Allgemeines}
Diese Seite verteilt die API-Anfragen auf die verschiedenen Funktionen auf. Für eine Liste aller möglichen API-Befehle siehe \autoref{api}.
\subsection{Besonderheiten}
Diese Seite bietet keine Graphische Nutzeroberfläche an. Alle Antworten auf Anfragen sind im JSON-Format.

\newpage
\section{apimgmt}
\subsection{Allgemeines} Diese Datei dient der Verwaltung angelegter Module.
Die Datei ist direkt durch den Nutzer aufrufbar. Sie setzt auch die entsprechende Konstante und bindet alle notwendigen Dateien ein:
\begin{lstlisting}[language=php]
define('NICE_PROJECT', true);
require_once "bin/inc.php";
\end{lstlisting}
Des Weiteren wird noch die Berechtigung des Aufrufers geprüft:
\begin{lstlisting}[language=php]
checkLoginDeny($LOGIN);
checkPermission(config::$ROLE_ADMIN);
\end{lstlisting}
Dieser muss mindestens die Rolle eines Administrators besitzen.
\subsection{Allgemeines}
Die Seite verfügt über die Möglichkeit ein bestehende Verbindung zu einem Modul zu löschen. Das Löschen eines Moduls ist irreversibel.
\subsection{Besonderheiten}
Es erfolgt auch eine Auflistung aller Module inklusive der Links zu den Reverse-APIs.

\newpage
\section{api-functions}
\subsection{Allgemeines} Diese Datei enthält alle durch die Modul-API aufgerufene Funktionen sowie zusätzliche Funktionen, um die Datenstrukturierung der Antworten zu vereinheitlichen.
\begin{table}[H]
	\begin{tabular}{|c|p{11cm}|}
		\hline
		\textbf{Einbindungspunkt} & inc.php \\ \hline
		\textbf{Einbindungspunkt} & inc-sub.php \\ \hline
	\end{tabular}
\end{table}
Die Datei ist nicht direkt durch den Nutzer aufrufbar, dies wird durch folgenden Code-Ausschnitt sichergestellt:
\begin{lstlisting}[language=php]
if (!defined('NICE_PROJECT')) {
	die('Permission denied.');
}
\end{lstlisting}
Der Globale Wert {\glqq NICE\_PROJECT\grqq} wird durch für den Nutzer valide Aufrufpunkte festgelegt, z.B. {\glqq api.php\grqq}.
\newpage
\subsection{Funktionen}
\subsubsection{getUserdataApi}
\paragraph{Parameter} Die Funktion besitzt folgende Parameter:
\begin{table}[H]
	\begin{tabular}{|c|p{11cm}|}
		\hline
		\textbf{Parametername} & \textbf{Parameterbeschreibung} \\ \hline
		\$username            & Nutzername \\ \hline
		\$ignoreDeaktiviation & Wenn wahr, wird der Aktivierungsstatus ignoriert \\ \hline
		\$apitoken            & Identifikator des Moduls \\ \hline
	\end{tabular}
\end{table}
\paragraph{Beschreibung} Die Funktion ermittelt alle Daten, welche zum Login des Nutzers und alle Funktionen eines Moduls notwendig sind. Die Funktion nutzt folgende Quellen:
\begin{itemize}
	\item Nutzerdaten-Tabelle
\end{itemize}
Die Antwort wird als strukturiertes Array an den Aufrufer zurückgegeben.
\subsubsection{getRoledataApi}
\paragraph{Parameter} Die Funktion besitzt folgende Parameter:
\begin{table}[H]
	\begin{tabular}{|c|p{11cm}|}
		\hline
		\textbf{Parametername} & \textbf{Parameterbeschreibung} \\ \hline
		\$username & Nutzername \\ \hline
	\end{tabular}
\end{table}
\paragraph{Beschreibung} Die Funktion ermittelt die Rollendaten des gegebenen Benutzers. Die Funktion nutzt folgende Quellen:
\begin{itemize}
	\item Nutzerdaten-Tabelle
\end{itemize}
Die Antwort wird als strukturiertes Array an den Aufrufer zurückgegeben.
\subsubsection{generateJson}
\paragraph{Parameter} Die Funktion besitzt folgende Parameter:
\begin{table}[H]
	\begin{tabular}{|c|p{11cm}|}
		\hline
		\textbf{Parametername} & \textbf{Parameterbeschreibung} \\ \hline
		\$array & Array mit strukturierten Eingabedaten \\ \hline
	\end{tabular}
\end{table}
\paragraph{Beschreibung} Die Funktion generiert aus einem Array ein Daten im JSON-Format. Die Antwort wird als String an den Aufrufer zurückgegeben.
\subsubsection{generateError}
\paragraph{Parameter} Die Funktion besitzt folgende Parameter:
\begin{table}[H]
	\begin{tabular}{|c|p{11cm}|}
		\hline
		\textbf{Parametername} & \textbf{Parameterbeschreibung} \\ \hline
		\$input & Optionale Fehlermeldung oder andere Daten \\ \hline
	\end{tabular}
\end{table}
\paragraph{Beschreibung} Die Funktion generiert ein Array, welches eine Fehlermeldung darstellt. Die Antwort wird als strukturiertes Array an den Aufrufer zurückgegeben.
\subsubsection{generateSuccess}
\paragraph{Parameter} Die Funktion besitzt keine Parameter.
\paragraph{Beschreibung} Die Funktion generiert ein Array, welches eine Antwort auf einen erfolgreichen API-Aufruf darstellt. Die Antwort wird als strukturiertes Array an den Aufrufer zurückgegeben.
\subsubsection{addRankPoints}
\paragraph{Parameter} Die Funktion besitzt folgende Parameter:
\begin{table}[H]
	\begin{tabular}{|c|p{11cm}|}
		\hline
		\textbf{Parametername} & \textbf{Parameterbeschreibung} \\ \hline
		\$username & Nutzername \\ \hline
		\$token    & Authentifizierungstoken eines Moduls \\ \hline
		\$reason   & Begründung der Rangpunkte \\ \hline
		\$points   & Anzahl der Rangpunkte \\ \hline
	\end{tabular}
\end{table}
\paragraph{Beschreibung} Die Funktion fügt dem angegebenen Benutzer die angegebenen Rangpunkte hinzu. Die Funktion hat Auswirkungen auf folgende Quellen:
\begin{itemize}
	\item Tabelle mit Rangpunkten der Nutzer
\end{itemize}
Die Antwort wird als strukturiertes Array an den Aufrufer zurückgegeben.
\subsubsection{getSourceTypesAPI}
\paragraph{Parameter} Die Funktion besitzt keine Parameter.
\paragraph{Beschreibung} Die Funktion alle Typen von Quellen ab. Die Funktion hat Auswirkungen auf folgende Quellen:
\begin{itemize}
	\item Tabelle mit Typen von Quellen
\end{itemize}
Die Antwort wird als strukturiertes Array an den Aufrufer zurückgegeben.
\subsubsection{checkMailAddressExistentMoudleAPI}
\paragraph{Parameter} Die Funktion besitzt folgende Parameter:
\begin{table}[H]
	\begin{tabular}{|c|p{11cm}|}
		\hline
		\textbf{Parametername} & \textbf{Parameterbeschreibung} \\ \hline
		\$email & Nutzername \\ \hline
	\end{tabular}
\end{table}
\paragraph{Beschreibung} Die Funktion fügt dem angegebenen Benutzer die angegebenen Rangpunkte hinzu. Die Funktion nutzt folgende Quellen:
\begin{itemize}
	\item Tabelle mit Nutzerdaten
\end{itemize}
Die Antwort wird als strukturiertes Array an den Aufrufer zurückgegeben.
\newpage
\section{authSystem}
\subsection{Allgemeines} Diese Datei enthält alle Funktionen um Nutzer zu Verwalten, Authentifizieren und Anzulegen.
\begin{table}[H]
	\begin{tabular}{|c|p{11cm}|}
		\hline
		\textbf{Einbindungspunkt} & inc.php \\ \hline
		\textbf{Einbindungspunkt} & inc-sub.php \\ \hline
	\end{tabular}
\end{table}
Die Datei ist nicht direkt durch den Nutzer aufrufbar, dies wird durch folgenden Code-Ausschnitt sichergestellt:
\begin{lstlisting}[language=php]
if (!defined('NICE_PROJECT')) {
	die('Permission denied.');
}
\end{lstlisting}
Der Globale Wert {\glqq NICE\_PROJECT\grqq} wird durch für den Nutzer valide Aufrufpunkte festgelegt, z.B. {\glqq api.php\grqq}.
\newpage
\subsection{Funktionen}
\subsubsection{createNewUser}
\paragraph{Parameter} Die Funktion besitzt folgende Parameter:
\begin{table}[H]
	\begin{tabular}{|c|p{11cm}|}
		\hline
		\textbf{Parametername} & \textbf{Parameterbeschreibung} \\ \hline
		\$name      & Nutzername \\ \hline
		\$passwd    & Passwort des neuen Nutzers als Klartext \\ \hline
		\$email     & Nutzername \\ \hline
		\$firstname & Nutzername \\ \hline
		\$lastname  & Nutzername \\ \hline
	\end{tabular}
\end{table}
\paragraph{Beschreibung} Die Funktion fügt einen neuen Benutzer in das System ein. Die Funktion hat Auswirkungen auf folgende Quellen:
\begin{itemize}
	\item Nutzerdaten-Tabelle
\end{itemize}
Die Funktion hat keinen Rückgabewert.
\subsubsection{updateUserPassword}
\paragraph{Parameter} Die Funktion besitzt folgende Parameter:
\begin{table}[H]
	\begin{tabular}{|c|p{11cm}|}
		\hline
		\textbf{Parametername} & \textbf{Parameterbeschreibung} \\ \hline
		\$uid      & Identifikator des Nutzers \\ \hline
		\$password & Passwort des neuen Nutzers als Klartext \\ \hline
	\end{tabular}
\end{table}
\paragraph{Beschreibung} Die Funktion ändert das Passwort eines Nutzers. Die Funktion hat Auswirkungen auf folgende Quellen:
\begin{itemize}
	\item Nutzerdaten-Tabelle
\end{itemize}
Die Funktion hat keinen Rückgabewert.
\subsubsection{checkPassword}
\paragraph{Parameter} Die Funktion besitzt folgende Parameter:
\begin{table}[H]
	\begin{tabular}{|c|p{11cm}|}
		\hline
		\textbf{Parametername} & \textbf{Parameterbeschreibung} \\ \hline
		\$password & eingegebenes Passwort als Klartext \\ \hline
		\$username & Nutzername \\ \hline
	\end{tabular}
\end{table}
\paragraph{Beschreibung} Die Funktion prüft ein durch den Nutzer zur Authentifizierung angegebenes Passwort auf Korrektheit. Des Weiteren werden auch alle Session-Daten gesetzt. Die Funktion nutzt folgende Quellen:
\begin{itemize}
	\item Nutzerdaten-Tabelle
\end{itemize}
Die Antwort ist ein Boolean.
\subsubsection{checkPasswordOnly}
\paragraph{Parameter} Die Funktion besitzt folgende Parameter:
\begin{table}[H]
	\begin{tabular}{|c|p{11cm}|}
		\hline
		\textbf{Parametername} & \textbf{Parameterbeschreibung} \\ \hline
		\$password & eingegebenes Passwort als Klartext \\ \hline
		\$username & Nutzername \\ \hline
	\end{tabular}
\end{table}
\paragraph{Beschreibung} Die Funktion prüft ein durch den Nutzer zur Authentifizierung angegebenes Passwort auf Korrektheit. Es werden keine Session-Daten gesetzt. Die Funktion nutzt folgende Quellen:
\begin{itemize}
	\item Nutzerdaten-Tabelle
\end{itemize}
Die Antwort ist ein Boolean.
\subsubsection{inspectPassword}
\paragraph{Parameter} Die Funktion besitzt folgende Parameter:
\begin{table}[H]
	\begin{tabular}{|c|p{11cm}|}
		\hline
		\textbf{Parametername} & \textbf{Parameterbeschreibung} \\ \hline
		\$PasswordField1Val & Passwort Feld 1 \\ \hline
		\$PasswordField2Val & Passwort Feld 2 \\ \hline
	\end{tabular}
\end{table}
\paragraph{Beschreibung} Die Funktion prüft, ob ein Passwort die gewünschten Anforderungen erfüllt. Die Funktion nutzt folgende Quellen:
\begin{itemize}
	\item Nutzerdaten-Tabelle
\end{itemize}
Die Antwort ist ein Boolean.
\subsubsection{updateUser}
\paragraph{Parameter} Die Funktion besitzt folgende Parameter:
\begin{table}[H]
	\begin{tabular}{|c|p{11cm}|}
		\hline
		\textbf{Parametername} & \textbf{Parameterbeschreibung} \\ \hline
		\$firstname & Vorname \\ \hline
		\$lastname  & Nachname \\ \hline
		\$EMail     & E-Mailadresse \\ \hline
	\end{tabular}
\end{table}
\paragraph{Beschreibung} Die Funktion aktualisiert die Session-Daten eines Nutzers. Die Funktion hat keinen Rückgabewert.
\subsubsection{logLogin}
\paragraph{Parameter} Die Funktion besitzt folgende Parameter:
\begin{table}[H]
	\begin{tabular}{|c|p{11cm}|}
		\hline
		\textbf{Parametername} & \textbf{Parameterbeschreibung} \\ \hline
		\$type & Typ des Logins ({\glqq Gast\grqq} oder {\glqq user\grqq}) \\ \hline
	\end{tabular}
\end{table}
\paragraph{Beschreibung} Die Funktion loggt den Typ des Logins für statistische Zwecke. Die Funktion hat Auswirkungen auf folgende Quellen:
\begin{itemize}
	\item Tabelle mit statistischen Login-Daten
\end{itemize}
Die Funktion hat keinen Rückgabewert.
\newpage
\section{captcha}
\subsection{Allgemeines} Diese Datei enthält alle Funktionen um ein Captcha zu generieren.
\begin{table}[H]
	\begin{tabular}{|c|p{11cm}|}
		\hline
		\textbf{Einbindungspunkt} & inc.php \\ \hline
		\textbf{Einbindungspunkt} & inc-sub.php \\ \hline
	\end{tabular}
\end{table}
Die Datei ist nicht direkt durch den Nutzer aufrufbar, dies wird durch folgenden Code-Ausschnitt sichergestellt:
\begin{lstlisting}[language=php]
if (!defined('NICE_PROJECT')) {
	die('Permission denied.');
}
\end{lstlisting}
Der Globale Wert {\glqq NICE\_PROJECT\grqq} wird durch für den Nutzer valide Aufrufpunkte festgelegt, z.B. {\glqq api.php\grqq}.
\subsection{Funktionen}
\subsubsection{generateCaptcha}
\paragraph{Parameter} Die Funktion besitzt folgende Parameter:
\begin{table}[H]
	\begin{tabular}{|c|p{11cm}|}
		\hline
		\textbf{Parametername} & \textbf{Parameterbeschreibung} \\ \hline
		\$special & Flag zum Nutzen von Sonderzeichen \\ \hline
		\$ext     & Flag für externen Aufruf durch Modul \\ \hline
	\end{tabular}
\end{table}
\paragraph{Beschreibung} Die Funktion generiert ein Captcha. Die Antwort wird als strukturiertes Array an den Aufrufer zurückgegeben. Das Captcha wird Base64 codiert.

\newpage
\section{config-sample / config}
\subsection{Allgemeines} Diese Datei enthält die Beispielkonfiguration beziehungsweise die Konfiguration.
\begin{table}[H]
	\begin{tabular}{|c|p{11cm}|}
		\hline
		\textbf{Einbindungspunkt} & keiner \\ \hline
	\end{tabular}
\end{table}
Die Datei ist nicht direkt durch den Nutzer aufrufbar, dies wird durch folgenden Code-Ausschnitt sichergestellt:
\begin{lstlisting}[language=php]
if (!defined('NICE_PROJECT')) {
	die('Permission denied.');
}
\end{lstlisting}
Der Globale Wert {\glqq NICE\_PROJECT\grqq} wird durch für den Nutzer valide Aufrufpunkte festgelegt, z.B. {\glqq api.php\grqq}. Alle Nachfolgenden {\glqq Funktionen\grqq} sind statische Werte der Klasse {\glqq configsample\grqq}. Für Detailbeschreibungen und Standardwerte siehe \autoref{chapter:config}.
\subsection{Installationsanweisungen} Für einen produktiven Einsatz der Konfigurationsdatei muss sie von {\glqq config-sample.php\grqq} in {\glqq config.php\grqq} kopiert werden. Anschließend muss die Klasse {\glqq configsample\grqq} in {\glqq config\grqq} umbenannt werden.
\subsection{Funktionen}
\subsubsection{\$SQL\_SERVER} Setzt den zu verwendenden SQL-Server. Dieser sollte im Optimalfall ein MariaDB-Server sein.
\subsubsection{\$SQL\_USER} Setzt den Nutzernamen am SQL-Server.
\subsubsection{\$SQL\_PASSWORD} Setzt das Passwort des Nutzers am SQL-Server.
\subsubsection{\$SQL\_SCHEMA} Setzt das Schema am SQL-Server.
\subsubsection{\$SQL\_PREFIX} Setzt das zu nutzende Präfix für die SQL-Tabellen.
\subsubsection{\$SQL\_Connector} Bestimmt den SQL-Connector. Momentan ist nur der PDO-Connector implementiert.
\subsubsection{\$BASE\_DMN} Setzt die Basis-Domain der Anwendung.
\subsubsection{\$SELF\_REGISTRATION} Konfigurationsflag für Selbstregistrierung.
\subsubsection{\$MAIN\_CAPTION} Setzt den Hauptname der Anwendung.
\subsubsection{\$TAGLINE\_CAPTION} Setzt den Langname der Anwendung.
\subsubsection{\$DEBUG} Schaltet Debug-Funktionen frei.
\subsubsection{\$DEBUG\_LEVEL} Setzt das Debug-Level.
\subsubsection{\$PWD\_LENGTH} Setzt die mindestens benötigte Passwortlänge.
\subsubsection{\$PWD\_ALGORITHM} Setzt den Hash-Algorithmus zur Passwortspeicherung.
\subsubsection{\$RANDOM\_STRING\_LENGTH} Setzt die Länge von zufällig generierten Zeichenketten.
\subsubsection{\$DOMAIN} Setzt die Domain der Anwendung.
\subsubsection{\$SENDER\_ADDRESS} Setzt die Adresse, von welcher aus E-Mails versendet werden.
\subsubsection{\$HMAC\_SECRET} Setzt das Geheimnis für die Generierung von Hashed-Message-Authentikation-Codes.
\subsubsection{\$UPLOAD\_DIR} Setzt das Verzeichnis, in welchem Hochgeladene Bilder gespeichert werden.
\subsubsection{\$ZENTRAL\_MAIL} Setzt die Mailadresse des Administrators.
\subsubsection{\$ROLE\_GUEST} Setzt den Mindestwert der Rolle {\glqq Gast\grqq}.
\subsubsection{\$ROLE\_UNAUTH\_USER} Setzt den Mindestwert der Rolle {\glqq nicht authentifizierter Nutzer\grqq}.
\subsubsection{\$ROLE\_AUTH\_USER} Setzt den Mindestwert der Rolle {\glqq Nutzer\grqq}.
\subsubsection{\$ROLE\_EMPLOYEE} Setzt den Mindestwert der Rolle {\glqq Mitarbeiter\grqq}.
\subsubsection{\$ROLE\_ADMIN} Setzt den Mindestwert der Rolle {\glqq Administrator\grqq}.
\subsubsection{\$BETA} Schaltet den Beta-Modus an.
\subsubsection{\$MAINTENANCE} Schaltet den Wartungs-Modus an.
\subsubsection{\$IMPRESSUM\_NAME} Setzt den Namen des Verantwortlichen im Impressum.
\subsubsection{\$IMPRESSUM\_STREET} Setzt den Straßennamen und die Hausnummer des Verantwortlichen im Impressum.
\subsubsection{\$IMPRESSUM\_CITY} Setzt den Ortsnamen und die Postleitzahl des Verantwortlichen im Impressum.
\subsubsection{\$SPECIAL\_CHARS\_CAPTCHA} Schaltet Sonderzeichen in Captchas ein.
\subsubsection{\$PRIVACY\_COMPANY\_NAME} Setzt den Namen der Firma in der Datenschutzerklärung.
\subsubsection{\$PRIVACY\_COMPANY\_STREET} Setzt den Straßennamen und die Hausnummer der Firma in der Datenschutzerklärung.
\subsubsection{\$PRIVACY\_COMPANY\_CITY} Setzt den Ortsnamen und die Postleitzahl der Firma in der Datenschutzerklärung.
\subsubsection{\$PRIVACY\_COMPANY\_FON} Setzt die Telefonnummer der Firma in der Datenschutzerklärung.
\subsubsection{\$PRIVACY\_COMPANY\_FAX} Setzt die Faxnummer der Firma in der Datenschutzerklärung.
\subsubsection{\$PRIVACY\_COMPANY\_MAIL} Setzt die E-Mailadresse der Firma in der Datenschutzerklärung.
\subsubsection{\$PRIVACY\_REP\_NAME} Setzt den Namen des Datenschutzbeauftragten.
\subsubsection{\$PRIVACY\_REP\_POS} Setzt die Positionsbezeichnung des Datenschutzbeauftragten.
\subsubsection{\$PRIVACY\_REP\_STREET} Setzt den Straßennamen und die Hausnummer des Datenschutzbeauftragten.
\subsubsection{\$PRIVACY\_REP\_CITY} Setzt den Ortsnamen und die Postleitzahl des Datenschutzbeauftragten.
\subsubsection{\$PRIVACY\_REP\_FON} Setzt die Telefonnummer des Datenschutzbeauftragten.
\subsubsection{\$PRIVACY\_REP\_FAX} Setzt die Faxnummer des Datenschutzbeauftragten.
\subsubsection{\$PRIVACY\_REP\_MAIL} Setzt die E-Mailadresse des Datenschutzbeauftragten.
\subsubsection{\$DIRECT\_DELETE} Schaltet direktes Löschen frei.

\newpage
\section{api-db}
\subsection{Allgemeines} Diese Datei enthält alle Funktionen um die Tabelle mit API's zu verwalten.
\begin{table}[H]
	\begin{tabular}{|c|p{11cm}|}
		\hline
		\textbf{Einbindungspunkt} & inc-db.php \\ \hline
		\textbf{Einbindungspunkt} & inc-db-sub.php \\ \hline
	\end{tabular}
\end{table}
Die Datei ist nicht direkt durch den Nutzer aufrufbar, dies wird durch folgenden Code-Ausschnitt sichergestellt:
\begin{lstlisting}[language=php]
if (!defined('NICE_PROJECT')) {
	die('Permission denied.');
}
\end{lstlisting}
Der Globale Wert {\glqq NICE\_PROJECT\grqq} wird durch für den Nutzer valide Aufrufpunkte festgelegt, z.B. {\glqq api.php\grqq}.
\newpage
\subsection{Funktionen}
\subsubsection{getAllApiTokens}
\paragraph{Parameter} Die Funktion besitzt kene Parameter.
\paragraph{Beschreibung} Die Funktion ruft eine Liste aller API-Token aus der Datenbank ab. Die Funktion nutzt folgende Quellen:
\begin{itemize}
	\item Tabelle mit API-Berechtigungsschlüsseln
\end{itemize}
Die Antwort wird als strukturiertes Array an den Aufrufer zurückgegeben.
\subsubsection{getAllApiTokensOrderedByToken}
\paragraph{Parameter} Die Funktion besitzt kene Parameter.
\paragraph{Beschreibung} Die Funktion ruft eine Liste aller API-Token aus der Datenbank ab. Die Antwort wird hat als Array-Schlüssel die Token der einzelnen Module. Die Funktion nutzt folgende Quellen:
\begin{itemize}
	\item Tabelle mit API-Berechtigungsschlüsseln
\end{itemize}
Die Antwort wird als strukturiertes Array an den Aufrufer zurückgegeben.
\subsubsection{addApiToken}
\paragraph{Parameter} Die Funktion besitzt folgende Parameter:
\begin{table}[H]
	\begin{tabular}{|c|p{11cm}|}
		\hline
		\textbf{Parametername} & \textbf{Parameterbeschreibung} \\ \hline
		\$result & Array mit benötigten Informationen \\ \hline
	\end{tabular}
\end{table}
\subparagraph{\$result}Das Array enthält folgende Elemente:
\begin{table}[H]
	\begin{tabular}{|c|p{11cm}|}
		\hline
		\textbf{Parametername} & \textbf{Parameterbeschreibung} \\ \hline
		name & Name des neuen Moduls \\ \hline
		rapi & Pfad zur Reverse-API des Moduls \\ \hline
	\end{tabular}
\end{table}
\paragraph{Beschreibung} Die Funktion fügt {\glqq COSP\grqq} ein neues Modul hinzu. Die Funktion hat Auswirkungen auf folgende Quellen:
\begin{itemize}
	\item Tabelle mit API-Berechtigungsschlüsseln
\end{itemize}
Die Antwort wird als strukturiertes Array an den Aufrufer zurückgegeben.
\subsubsection{deleteApiToken}
\paragraph{Parameter} Die Funktion besitzt folgende Parameter:
\begin{table}[H]
	\begin{tabular}{|c|p{11cm}|}
		\hline
		\textbf{Parametername} & \textbf{Parameterbeschreibung} \\ \hline
		\$token & Token eines Moduls \\ \hline
	\end{tabular}
\end{table}
\paragraph{Beschreibung} Die Funktion löscht einen Modul Zugangstoken aus {\glqq COSP\grqq}. Die Funktion hat Auswirkungen auf folgende Quellen:
\begin{itemize}
	\item Tabelle mit API-Berechtigungsschlüsseln
\end{itemize}
Die Funktion besitzt keinen Rückgabewert.
\subsubsection{getApiTokenById}
\paragraph{Parameter} Die Funktion besitzt folgende Parameter:
\begin{table}[H]
	\begin{tabular}{|c|p{11cm}|}
		\hline
		\textbf{Parametername} & \textbf{Parameterbeschreibung} \\ \hline
		\$id & Identifikator eines Moduls \\ \hline
	\end{tabular}
\end{table}
\paragraph{Beschreibung} Die Funktion ruft die Daten einer API aus der Datenbank ab. Die Funktion nutzt folgende Quellen:
\begin{itemize}
	\item Tabelle mit API-Berechtigungsschlüsseln
\end{itemize}
Die Antwort wird als strukturiertes Array an den Aufrufer zurückgegeben.
\subsubsection{updateApiData}
\paragraph{Parameter} Die Funktion besitzt folgende Parameter:
\begin{table}[H]
	\begin{tabular}{|c|p{11cm}|}
		\hline
		\textbf{Parametername} & \textbf{Parameterbeschreibung} \\ \hline
		\$id   & Identifikator eines Moduls \\ \hline
		\$name & Name des Moduls \\ \hline
		\$url  & Revers-API-Url des Moduls \\ \hline
	\end{tabular}
\end{table}
\paragraph{Beschreibung} Die Funktion ändert Daten einer API aus der Datenbank ab. Die Funktion hat Auswirkungen auf folgende Quellen:
\begin{itemize}
	\item Tabelle mit API-Berechtigungsschlüsseln
\end{itemize}
\newpage
\section{basic-db}
\subsection{Allgemeines} Diese Datei enthält alle grundlegenden Funktionen für den Datenbankzugriff.
\begin{table}[H]
	\begin{tabular}{|c|p{11cm}|}
		\hline
		\textbf{Einbindungspunkt} & inc.php \\ \hline
		\textbf{Einbindungspunkt} & inc-sub.php \\ \hline
	\end{tabular}
\end{table}
Die Datei ist nicht direkt durch den Nutzer aufrufbar, dies wird durch folgenden Code-Ausschnitt sichergestellt:
\begin{lstlisting}[language=php]
if (!defined('NICE_PROJECT')) {
	die('Permission denied.');
}
\end{lstlisting}
Der Globale Wert {\glqq NICE\_PROJECT\grqq} wird durch für den Nutzer valide Aufrufpunkte festgelegt, z.B. {\glqq api.php\grqq}.
\newpage
\subsection{Funktionen}
\subsubsection{getPdo}
\paragraph{Parameter} Die Funktion besitzt keine Parameter.
\paragraph{Beschreibung} Die Funktion instanziiert die PDO-Klasse mit allen notwendigen Informationen. Die Funktion nutzt folgende Quellen:
\begin{itemize}
	\item Konfiguration
\end{itemize}
Es findet bei dieser Funktion kein Abruf von Daten aus {\glqq COSP\grqq} statt. Es wird eine PDO-Instanz zurück gegeben.
\subsubsection{ExecuteStatementWOR}
\paragraph{Parameter} Die Funktion besitzt folgende Parameter:
\begin{table}[H]
	\begin{tabular}{|c|p{11cm}|}
		\hline
		\textbf{Parametername} & \textbf{Parameterbeschreibung} \\ \hline
		\$prep\_stmt & Vorbereitete SQL-Abfrage \\ \hline
		\$params     & Array mit Parametern der Abfrage \\ \hline
	\end{tabular}
\end{table}
\subparagraph{\$params}Das Array enthält Elemente mit folgenden Elementen:
\begin{table}[H]
	\begin{tabular}{|c|p{11cm}|}
		\hline
		\textbf{Parametername} & \textbf{Parameterbeschreibung} \\ \hline
		val & Wert des Parameters \\ \hline
		typ & Typ des Parameters \\ \hline
	\end{tabular}
\end{table}
\paragraph{Beschreibung} Die Funktion dient dem ermitteln aller für die Anzeige des Persönlichen Bereiches benötigten Daten aus folgenden Quellen:
Es findet bei dieser Funktion kein Abruf von Daten aus {\glqq COSP\grqq} statt. Es wird eine Antwort zurück gegeben.
\paragraph{Vorgehensweise} Es werden die Parameter in die vorbereitete Abfrage in der Reihenfolge des Arrays eingebunden. Anschließend wird die Abfrage auf der Datenbank ausgeführt.
\subsubsection{ExecuteStatementWOR}
\paragraph{Parameter} Die Funktion besitzt folgende Parameter:
\begin{table}[H]
	\begin{tabular}{|c|p{11cm}|}
		\hline
		\textbf{Parametername} & \textbf{Parameterbeschreibung} \\ \hline
		\$prep\_stmt  & Vorbereitete SQL-Abfrage \\ \hline
		\$params      & Array mit Parametern der Abfrage \\ \hline
		\$read        & Schaltet das Lesen von Daten ab \\ \hline
		\$disableNull & Schaltet das setzten des Wertes {\glqq null\grqq} ab \\ \hline
	\end{tabular}
\end{table}
\subparagraph{\$params}Das Array enthält Elemente mit folgenden Elementen:
\begin{table}[H]
	\begin{tabular}{|c|p{11cm}|}
		\hline
		\textbf{Parametername} & \textbf{Parameterbeschreibung} \\ \hline
		val & Wert des Parameters \\ \hline
		typ & Typ des Parameters \\ \hline
		nam & Name des Parameters in der vorbereiteten Abfrage \\ \hline
	\end{tabular}
\end{table}
\paragraph{Beschreibung} Die Funktion dient dem ermitteln aller für die Anzeige des Persönlichen Bereiches benötigten Daten aus folgenden Quellen:
Es findet bei dieser Funktion kein Abruf von Daten aus {\glqq COSP\grqq} statt. Es wird eine Antwort zurück gegeben.
\paragraph{Vorgehensweise} Es werden die Parameter in die vorbereitete Abfrage in der Anhand der vergebenen Platzhalter eingebunden. Die Einbindung geschieht in der Reihenfolge der im Array enthaltenen Elemente. Anschließend wird die Abfrage auf der Datenbank ausgeführt. Sofern das Lesen aktiviert ist, werden die Ergebnisse als strukturiertes Array zurück gegeben. Bei Auftreten eines Fehlers ist dieser Rückgabewert.

\newpage
\section{inc-db-sub}
\input{Kapitel/Files/inc-db-sub}
\newpage
\section{inc-db}
\subsection{Allgemeines} Diese Datei Einbindungen aller benötigten Dateien für den Datenbankzugriff.
Die Datei ist nicht direkt durch den Nutzer aufrufbar, dies wird durch folgenden Code-Ausschnitt sichergestellt:
\begin{lstlisting}[language=php]
if (!defined('NICE_PROJECT')) {
	die('Permission denied.');
}
\end{lstlisting}
Der Globale Wert {\glqq NICE\_PROJECT\grqq} wird durch für den Nutzer valide Aufrufpunkte festgelegt, z.B. {\glqq api.php\grqq}.
\newpage
\subsection{Einbindungen}
\subsubsection{Grundlegendes}
Nachfolgend zu sehender Code-Block bindet alle benötigten Dateien in korrekter Reihenfolge ein. Beim Einbinden neuer Dateien, sind diese stets an das Ende zu schreiben, außer die Dateien sind Umstrukturierungen bereits existenten Dateien.
\begin{lstlisting}[language=php]
require_once 'bin/database/basic-db.php';
require_once 'bin/database/statistics-basic-dbfunctions.php';
require_once 'bin/database/user-db.php';
require_once 'bin/database/role-db.php';
require_once 'bin/database/session-db.php';
require_once 'bin/database/api-db.php';
require_once 'bin/database/point-reason-db.php';
require_once 'bin/database/rank-point-db.php';
require_once 'bin/database/ranks-db.php';
require_once 'bin/database/picture-db.php';
require_once 'bin/database/story-db.php';
require_once 'bin/database/validate-story-db.php';
require_once 'bin/database/validate-picture-db.php';
require_once 'bin/database/logging.php';
require_once 'bin/database/statistics-user.php';
require_once 'bin/database/statistics-pictures-db.php';
require_once 'bin/database/statistics-story-db.php';
require_once 'bin/database/module-rank-db.php';
require_once 'bin/database/source-type-db.php';
require_once 'bin/database/module-rights-db.php';
\end{lstlisting}
\subsubsection{Besonderheit}
Die Einbindungen sind immer vom Hauptordner aus zu erreichen und auch relativ zu diesem anzugeben.

\newpage
\section{logging}
\subsection{Allgemeines} Diese Datei enthält alle Funktionen zum Zugriff auf die Tabelle mit statistischen Daten zur Nutzungsanalyse.
\begin{table}[H]
	\begin{tabular}{|c|p{11cm}|}
		\hline
		\textbf{Einbindungspunkt} & inc-db.php \\ \hline
		\textbf{Einbindungspunkt} & inc-db-sub.php \\ \hline
	\end{tabular}
\end{table}
Die Datei ist nicht direkt durch den Nutzer aufrufbar, dies wird durch folgenden Code-Ausschnitt sichergestellt:
\begin{lstlisting}[language=php]
if (!defined('NICE_PROJECT')) {
	die('Permission denied.');
}
\end{lstlisting}
Der Globale Wert {\glqq NICE\_PROJECT\grqq} wird durch für den Nutzer valide Aufrufpunkte festgelegt, z.B. {\glqq api.php\grqq}.
\newpage
\subsection{Funktionen}
\subsubsection{insertLogUniqueVisitors}
\paragraph{Parameter} Die Funktion besitzt folgende Parameter:
\begin{table}[H]
	\begin{tabular}{|c|p{11cm}|}
		\hline
		\textbf{Parametername} & \textbf{Parameterbeschreibung} \\ \hline
		\$ip   & IP-Adresse des Aufrufers \\ \hline
		\$type & Typ des Aufrufs ({\glqq guest\grqq} oder {\glqq user\grqq}) \\ \hline
	\end{tabular}
\end{table}
\paragraph{Beschreibung} Die Funktion fügt einen statistischen Eintrag zum aktuellen Aufruf hinzu. Die Funktion hat Auswirkungen auf folgende Quellen:
\begin{itemize}
	\item Tabelle mit statistischen Nutzungsdaten
\end{itemize}
Die Antwort wird als strukturiertes Array an den Aufrufer zurückgegeben.
\subsubsection{getStatisticalDataLastWeeks}
\paragraph{Parameter} Die Funktion besitzt folgende Parameter:
\begin{table}[H]
	\begin{tabular}{|c|p{11cm}|}
		\hline
		\textbf{Parametername} & \textbf{Parameterbeschreibung} \\ \hline
		\$number & Anzahl an Zeiteinheiten \\ \hline
	\end{tabular}
\end{table}
\paragraph{Beschreibung} Die Funktion ruft alle statistischen Daten der angegeben letzten Wochen ab. Die Funktion nutzt folgende Quellen:
\begin{itemize}
	\item Tabelle mit statistischen Nutzungsdaten
\end{itemize}
Die Antwort wird als strukturiertes Array an den Aufrufer zurückgegeben.
\subsubsection{getStatisticalDataLastMonth}
\paragraph{Parameter} Die Funktion besitzt folgende Parameter:
\begin{table}[H]
	\begin{tabular}{|c|p{11cm}|}
		\hline
		\textbf{Parametername} & \textbf{Parameterbeschreibung} \\ \hline
		\$number & Anzahl an Zeiteinheiten \\ \hline
	\end{tabular}
\end{table}
\paragraph{Beschreibung} Die Funktion ruft alle statistischen Daten der angegeben letzten Monate ab. Die Funktion nutzt folgende Quellen:
\begin{itemize}
	\item Tabelle mit statistischen Nutzungsdaten
\end{itemize}
Die Antwort wird als strukturiertes Array an den Aufrufer zurückgegeben.
\subsubsection{getStatisticalDataLastYear}
\paragraph{Parameter} Die Funktion besitzt folgende Parameter:
\begin{table}[H]
	\begin{tabular}{|c|p{11cm}|}
		\hline
		\textbf{Parametername} & \textbf{Parameterbeschreibung} \\ \hline
		\$number & Anzahl an Zeiteinheiten \\ \hline
	\end{tabular}
\end{table}
\paragraph{Beschreibung} Die Funktion ruft alle statistischen Daten der angegeben letzten Jahre ab. Die Funktion nutzt folgende Quellen:
\begin{itemize}
	\item Tabelle mit statistischen Nutzungsdaten
\end{itemize}
Die Antwort wird als strukturiertes Array an den Aufrufer zurückgegeben.
\subsubsection{getStatisticalDataLastDays}
\paragraph{Parameter} Die Funktion besitzt folgende Parameter:
\begin{table}[H]
	\begin{tabular}{|c|p{11cm}|}
		\hline
		\textbf{Parametername} & \textbf{Parameterbeschreibung} \\ \hline
		\$number & Anzahl an Zeiteinheiten \\ \hline
	\end{tabular}
\end{table}
\paragraph{Beschreibung} Die Funktion ruft alle statistischen Daten der angegeben letzten Tage ab. Die Funktion nutzt folgende Quellen:
\begin{itemize}
	\item Tabelle mit statistischen Nutzungsdaten
\end{itemize}
Die Antwort wird als strukturiertes Array an den Aufrufer zurückgegeben.
\newpage
\section{module-rank-db}
\subsection{Allgemeines} Diese Datei enthält alle Funktionen zum Zugriff auf die Tabelle mit modulbasierten Rangnamen.
\begin{table}[H]
	\begin{tabular}{|c|p{11cm}|}
		\hline
		\textbf{Einbindungspunkt} & inc-db.php \\ \hline
		\textbf{Einbindungspunkt} & inc-db-sub.php \\ \hline
	\end{tabular}
\end{table}
Die Datei ist nicht direkt durch den Nutzer aufrufbar, dies wird durch folgenden Code-Ausschnitt sichergestellt:
\begin{lstlisting}[language=php]
if (!defined('NICE_PROJECT')) {
	die('Permission denied.');
}
\end{lstlisting}
Der Globale Wert {\glqq NICE\_PROJECT\grqq} wird durch für den Nutzer valide Aufrufpunkte festgelegt, z.B. {\glqq api.php\grqq}.
\newpage
\subsection{Funktionen}
\subsubsection{getNamesByRankId}
\paragraph{Parameter} Die Funktion besitzt folgende Parameter:
\begin{table}[H]
	\begin{tabular}{|c|p{11cm}|}
		\hline
		\textbf{Parametername} & \textbf{Parameterbeschreibung} \\ \hline
		\$rid & Identifikator eines Ranges \\ \hline
	\end{tabular}
\end{table}
\paragraph{Beschreibung} Die Funktion ruft eine Liste aller modulbasierten Rangnamen eines Ranges ab. Die Funktion nutzt folgende Quellen:
\begin{itemize}
	\item Tabelle mit modulbasierten Rangnamen
\end{itemize}
Die Antwort wird als strukturiertes Array an den Aufrufer zurückgegeben.
\subsubsection{insertModuleBasedRankName}
\paragraph{Parameter} Die Funktion besitzt folgende Parameter:
\begin{table}[H]
	\begin{tabular}{|c|p{11cm}|}
		\hline
		\textbf{Parametername} & \textbf{Parameterbeschreibung} \\ \hline
		\$rid  & Identifikator eines Ranges \\ \hline
		\$aid  & Identifikator eines Moduls \\ \hline
		\$name & modulbasierter Name \\ \hline
	\end{tabular}
\end{table}
\paragraph{Beschreibung} Die Funktion fügt einem Rang einen neuen modulbasierten Namen hinzu. Die Funktion hat Auswirkungen auf folgende Quellen:
\begin{itemize}
	\item Tabelle mit modulbasierten Rangnamen
\end{itemize}
Die Antwort wird als strukturiertes Array an den Aufrufer zurückgegeben.
\subsubsection{deleteModuleBasedRankName}
\paragraph{Parameter} Die Funktion besitzt folgende Parameter:
\begin{table}[H]
	\begin{tabular}{|c|p{11cm}|}
		\hline
		\textbf{Parametername} & \textbf{Parameterbeschreibung} \\ \hline
		\$id  & Identifikator eines modulbasierten Rangnamens \\ \hline
	\end{tabular}
\end{table}
\paragraph{Beschreibung} Die Funktion löscht eine modulbasierten Rangnamen. Die Funktion hat Auswirkungen auf folgende Quellen:
\begin{itemize}
	\item Tabelle mit modulbasierten Rangnamen
\end{itemize}
Die Antwort wird als strukturiertes Array an den Aufrufer zurückgegeben.
\subsubsection{getRankNamesByModule}
\paragraph{Parameter} Die Funktion besitzt folgende Parameter:
\begin{table}[H]
	\begin{tabular}{|c|p{11cm}|}
		\hline
		\textbf{Parametername} & \textbf{Parameterbeschreibung} \\ \hline
		\$aid & numerischer Identifikator eines Moduls \\ \hline
	\end{tabular}
\end{table}
\paragraph{Beschreibung} Die Funktion ruft eine Liste aller modulbasierten Rangnamen eines Moduls ab. Die Funktion nutzt folgende Quellen:
\begin{itemize}
	\item Tabelle mit modulbasierten Rangnamen
\end{itemize}
Die Antwort wird als strukturiertes Array an den Aufrufer zurückgegeben.
\newpage
\section{module-rights-db}
\subsection{Allgemeines} Diese Datei enthält alle Funktionen zum Zugriff auf die Tabelle mit modulbasierten Rechten.
\begin{table}[H]
	\begin{tabular}{|c|p{11cm}|}
		\hline
		\textbf{Einbindungspunkt} & inc-db.php \\ \hline
		\textbf{Einbindungspunkt} & inc-db-sub.php \\ \hline
	\end{tabular}
\end{table}
Die Datei ist nicht direkt durch den Nutzer aufrufbar, dies wird durch folgenden Code-Ausschnitt sichergestellt:
\begin{lstlisting}[language=php]
	if (!defined('NICE_PROJECT')) {
		die('Permission denied.');
	}
\end{lstlisting}
Der Globale Wert {\glqq NICE\_PROJECT\grqq} wird durch für den Nutzer valide Aufrufpunkte festgelegt, z.B. {\glqq api.php\grqq}.
\newpage
\subsection{Funktionen}
\subsubsection{getAllModuleBasedRightsByUserID}
\paragraph{Parameter} Die Funktion besitzt folgende Parameter:
\begin{table}[H]
	\begin{tabular}{|c|p{11cm}|}
		\hline
		\textbf{Parametername} & \textbf{Parameterbeschreibung} \\ \hline
		\$id & Identifikator eines Nutzers \\ \hline
	\end{tabular}
\end{table}
\paragraph{Beschreibung} Die Funktion ruft eine Liste aller modulbasierten Rechte eines Nutzers ab. Die Funktion nutzt folgende Quellen:
\begin{itemize}
	\item Tabelle mit modulbasierten Rechten
	\item Tabelle mit Modulen
	\item Tabelle mit Rollen
\end{itemize}
Die Antwort wird als strukturiertes Array an den Aufrufer zurückgegeben.
\subsubsection{addModuleBasedRole}
\paragraph{Parameter} Die Funktion besitzt folgende Parameter:
\begin{table}[H]
	\begin{tabular}{|c|p{11cm}|}
		\hline
		\textbf{Parametername} & \textbf{Parameterbeschreibung} \\ \hline
		\$nameId & Identifikator eines Nutzers \\ \hline
		\$roleId & Identifikator einer Rolle \\ \hline
		\$appID  & Identifikator einer Api \\ \hline
	\end{tabular}
\end{table}
\paragraph{Beschreibung} Die Funktion fügt modulbasierte Rechte für einen Nutzer ein. Die Funktion hat Auswirkungen auf folgende Quellen:
\begin{itemize}
	\item Tabelle mit modulbasierten Rechten
\end{itemize}
Die Antwort wird als strukturiertes Array an den Aufrufer zurückgegeben.
\subsubsection{getModuleBasedRightsByID}
\paragraph{Parameter} Die Funktion besitzt folgende Parameter:
\begin{table}[H]
	\begin{tabular}{|c|p{11cm}|}
		\hline
		\textbf{Parametername} & \textbf{Parameterbeschreibung} \\ \hline
		\$id & Identifikator eines Modulrechtes \\ \hline
	\end{tabular}
\end{table}
\paragraph{Beschreibung} Die Funktion ruft ein modulbasiertes Rechte ab. Die Funktion nutzt folgende Quellen:
\begin{itemize}
	\item Tabelle mit modulbasierten Rechten
	\item Tabelle mit Modulen
	\item Tabelle mit Rollen
\end{itemize}
Die Antwort wird als strukturiertes Array an den Aufrufer zurückgegeben.
\subsubsection{deleteModuleBasedRole}
\paragraph{Parameter} Die Funktion besitzt folgende Parameter:
\begin{table}[H]
	\begin{tabular}{|c|p{11cm}|}
		\hline
		\textbf{Parametername} & \textbf{Parameterbeschreibung} \\ \hline
		\$rid & Identifikator eines Modulrechtes \\ \hline
		\$uid & Identifikator eines Nutzers \\ \hline
	\end{tabular}
\end{table}
\paragraph{Beschreibung} Die Funktion löscht ein modulbasiertes Recht ab. Die Funktion hat Auswirkung folgende Quellen:
\begin{itemize}
	\item Tabelle mit modulbasierten Rechten
\end{itemize}
Die Antwort wird als strukturiertes Array an den Aufrufer zurückgegeben.
\subsubsection{getAllModuleBasedRightsByModulID}
\paragraph{Parameter} Die Funktion besitzt folgende Parameter:
\begin{table}[H]
	\begin{tabular}{|c|p{11cm}|}
		\hline
		\textbf{Parametername} & \textbf{Parameterbeschreibung} \\ \hline
		\$id & Identifikator eines Moduls \\ \hline
	\end{tabular}
\end{table}
\paragraph{Beschreibung} Die Funktion ruft eine Liste aller modulbasierten Rechte eines Moduls ab. Die Funktion nutzt folgende Quellen:
\begin{itemize}
	\item Tabelle mit modulbasierten Rechten
	\item Tabelle mit Modulen
	\item Tabelle mit Rollen
	\item Tabelle mit Nutzerinformationen
\end{itemize}
Die Antwort wird als strukturiertes Array an den Aufrufer zurückgegeben.
\subsubsection{updateModulRights}
\paragraph{Parameter} Die Funktion besitzt folgende Parameter:
\begin{table}[H]
	\begin{tabular}{|c|p{11cm}|}
		\hline
		\textbf{Parametername} & \textbf{Parameterbeschreibung} \\ \hline
		\$rightID & Identifikator eines Modulrechtes \\ \hline
		\$roleID  & Identifikator einer Rolle \\ \hline
	\end{tabular}
\end{table}
\paragraph{Beschreibung} Die Funktion aktualisiert ein modulbasiertes Recht. Die Funktion hat Auswirkung folgende Quellen:
\begin{itemize}
	\item Tabelle mit modulbasierten Rechten
\end{itemize}
Die Antwort wird als strukturiertes Array an den Aufrufer zurückgegeben.
\subsubsection{getPermissionUsersOfModule}
\paragraph{Parameter} Die Funktion besitzt folgende Parameter:
\begin{table}[H]
	\begin{tabular}{|c|p{11cm}|}
		\hline
		\textbf{Parametername} & \textbf{Parameterbeschreibung} \\ \hline
		\$aid                & Identifikator eines Moduls \\ \hline
		\$ReqPermissionValue & Benötigter Berechtigungswert \\ \hline
	\end{tabular}
\end{table}
\paragraph{Beschreibung} Die Funktion ruft eine Liste aller Nutzeridentifikatoren eines Moduls ab einer bestimmten Berechtigung ab. Die Funktion nutzt folgende Quellen:
\begin{itemize}
	\item Tabelle mit modulbasierten Rechten
	\item Tabelle mit Rollen
\end{itemize}
Die Antwort wird als strukturiertes Array an den Aufrufer zurückgegeben.
\subsubsection{updateDisableStateModuleRight}
\paragraph{Parameter} Die Funktion besitzt folgende Parameter:
\begin{table}[H]
	\begin{tabular}{|c|p{11cm}|}
		\hline
		\textbf{Parametername} & \textbf{Parameterbeschreibung} \\ \hline
		\$rightID & Identifikator eines Rechtes \\ \hline
		\$state   & Status der Deaktivierung \\ \hline
	\end{tabular}
\end{table}
\paragraph{Beschreibung} Die Funktion ändert den Deaktivierungsstatus einer Modulberechtigung. Die Funktion hat Auswirkungen auf folgende Quellen:
\begin{itemize}
	\item Tabelle mit modulbasierten Rechten
\end{itemize}
Die Antwort wird als strukturiertes Array an den Aufrufer zurückgegeben.
\subsubsection{getModuleRightByUsernameApiToken}
\paragraph{Parameter} Die Funktion besitzt folgende Parameter:
\begin{table}[H]
	\begin{tabular}{|c|p{11cm}|}
		\hline
		\textbf{Parametername} & \textbf{Parameterbeschreibung} \\ \hline
		\$username & Nutzername \\ \hline
		\$token    & alphanumerischer Identifikator eines Moduls \\ \hline
	\end{tabular}
\end{table}
\paragraph{Beschreibung} Die Funktion prüft die modulbasierte Berechtigung eines Nutzers für ein Modul. Die Funktion nutzt folgende Quellen:
\begin{itemize}
	\item Tabelle mit modulbasierten Rechten
	\item Tabelle mit Moduldaten
	\item Tabelle mit Rollen
	\item Tabelle mit Nutzerinformationen
\end{itemize}
Die Antwort wird als strukturiertes Array an den Aufrufer zurückgegeben.
\newpage
\section{picture-db}
\subsection{Allgemeines} Diese Datei enthält alle Funktionen zum Zugriff auf die Tabelle mit Bildern.
\begin{table}[H]
	\begin{tabular}{|c|p{11cm}|}
		\hline
		\textbf{Einbindungspunkt} & inc-db.php \\ \hline
		\textbf{Einbindungspunkt} & inc-db-sub.php \\ \hline
	\end{tabular}
\end{table}
Die Datei ist nicht direkt durch den Nutzer aufrufbar, dies wird durch folgenden Code-Ausschnitt sichergestellt:
\begin{lstlisting}[language=php]
if (!defined('NICE_PROJECT')) {
	die('Permission denied.');
}
\end{lstlisting}
Der Globale Wert {\glqq NICE\_PROJECT\grqq} wird durch für den Nutzer valide Aufrufpunkte festgelegt, z.B. {\glqq api.php\grqq}.
\newpage
\subsection{Funktionen}
\subsubsection{insertPic}
\paragraph{Parameter} Die Funktion besitzt folgende Parameter:
\begin{table}[H]
	\begin{tabular}{|c|p{11cm}|}
		\hline
		\textbf{Parametername} & \textbf{Parameterbeschreibung} \\ \hline
		\$title       & Titel des Bildes \\ \hline
		\$description & Beschreibung des Bildes \\ \hline
		\$picurl      & Dateiname des Bildes \\ \hline
		\$preview     & Base64 encodiertes Vorschaubild \\ \hline
		\$token       & alphanumerischer Identifikator des Bildes \\ \hline
		\$uid         & Identifikator des hochladenden Nutzers \\ \hline
		\$aid         & Identifikator des hochladenden Moduls \\ \hline
	\end{tabular}
\end{table}
\paragraph{Beschreibung} Die Funktion fügt ein neues Bild der Datenbank hinzu. Die Funktion hat Auswirkungen auf folgende Quellen:
\begin{itemize}
	\item Tabelle mit Bildern
\end{itemize}
Die Antwort wird als strukturiertes Array an den Aufrufer zurückgegeben.
\subsubsection{deletePicture}
\paragraph{Parameter} Die Funktion besitzt folgende Parameter:
\begin{table}[H]
	\begin{tabular}{|c|p{11cm}|}
		\hline
		\textbf{Parametername} & \textbf{Parameterbeschreibung} \\ \hline
		\$username & Nutzername \\ \hline
	\end{tabular}
\end{table}
\paragraph{Beschreibung} Die Funktion löscht ein Bild aus der Datenbank. Die Funktion hat Auswirkungen auf folgende Quellen:
\begin{itemize}
	\item Tabelle mit Bildern
\end{itemize}
Die Antwort wird als strukturiertes Array an den Aufrufer zurückgegeben.
\subsubsection{getPictureListDB}
\paragraph{Parameter} Die Funktion besitzt folgende Parameter:
\begin{table}[H]
	\begin{tabular}{|c|p{11cm}|}
		\hline
		\textbf{Parametername} & \textbf{Parameterbeschreibung} \\ \hline
		\$apptoken & Identifikator eines Moduls \\ \hline
	\end{tabular}
\end{table}
\paragraph{Beschreibung} Die Funktion ruft eine Liste aller Bilder eines Moduls ab. Die Funktion nutzt folgende Quellen:
\begin{itemize}
	\item Tabelle mit Bildern
\end{itemize}
Die Antwort wird als strukturiertes Array an den Aufrufer zurückgegeben.
\subsubsection{getSinglePictureFromDB}
\paragraph{Parameter} Die Funktion besitzt folgende Parameter:
\begin{table}[H]
	\begin{tabular}{|c|p{11cm}|}
		\hline
		\textbf{Parametername} & \textbf{Parameterbeschreibung} \\ \hline
		\$apptoken     & Identifikator eines Moduls \\ \hline
		\$pictureToken & alphanumerischer Identifikator eines Bildes \\ \hline
	\end{tabular}
\end{table}
\paragraph{Beschreibung} Die Funktion ruft alle Daten zu einem bestimmten Bild ab. Die Funktion nutzt folgende Quellen:
\begin{itemize}
	\item Tabelle mit Bildern
\end{itemize}
Die Antwort wird als strukturiertes Array an den Aufrufer zurückgegeben.
\subsubsection{getPreviewPicture}
\paragraph{Parameter} Die Funktion besitzt folgende Parameter:
\begin{table}[H]
	\begin{tabular}{|c|p{11cm}|}
		\hline
		\textbf{Parametername} & \textbf{Parameterbeschreibung} \\ \hline
		\$token & alphanumerischer Identifikator eines Bildes \\ \hline
	\end{tabular}
\end{table}
\paragraph{Beschreibung} Die Funktion ruft das Vorschaubild eines bestimmten Bildes ab. Die Funktion nutzt folgende Quellen:
\begin{itemize}
	\item Tabelle mit Bildern
\end{itemize}
Die Antwort wird als strukturiertes Array an den Aufrufer zurückgegeben.
\subsubsection{getFullsizePicture}
\paragraph{Parameter} Die Funktion besitzt folgende Parameter:
\begin{table}[H]
	\begin{tabular}{|c|p{11cm}|}
		\hline
		\textbf{Parametername} & \textbf{Parameterbeschreibung} \\ \hline
		\$token & alphanumerischer Identifikator eines Bildes \\ \hline
	\end{tabular}
\end{table}
\paragraph{Beschreibung} Die Funktion ruft die Daten zum Laden des Vollbildes eines bestimmten Bildes ab. Die Funktion nutzt folgende Quellen:
\begin{itemize}
	\item Tabelle mit Bildern
\end{itemize}
Die Antwort wird als strukturiertes Array an den Aufrufer zurückgegeben.
\subsubsection{getPictureIdByToken}
\paragraph{Parameter} Die Funktion besitzt folgende Parameter:
\begin{table}[H]
	\begin{tabular}{|c|p{11cm}|}
		\hline
		\textbf{Parametername} & \textbf{Parameterbeschreibung} \\ \hline
		\$token & alphanumerischer Identifikator eines Bildes \\ \hline
	\end{tabular}
\end{table}
\paragraph{Beschreibung} Die Funktion ruft den numerischen Identifikator eines bestimmten Bildes anhand des alphanumerischen Identifikator ab. Die Funktion nutzt folgende Quellen:
\begin{itemize}
	\item Tabelle mit Bildern
\end{itemize}
Die Antwort wird als strukturiertes Array an den Aufrufer zurückgegeben.
\subsubsection{getPictureUploadingUserByToken}
\paragraph{Parameter} Die Funktion besitzt folgende Parameter:
\begin{table}[H]
	\begin{tabular}{|c|p{11cm}|}
		\hline
		\textbf{Parametername} & \textbf{Parameterbeschreibung} \\ \hline
		\$token & alphanumerischer Identifikator eines Bildes \\ \hline
	\end{tabular}
\end{table}
\paragraph{Beschreibung} Die Funktion ruft den hochladenden Nutzer eines Bildes ab. Die Funktion nutzt folgende Quellen
\begin{itemize}
	\item Tabelle mit Bildern
	\item Tabelle mit Nutzerdaten
\end{itemize}
Die Antwort wird als strukturiertes Array an den Aufrufer zurückgegeben.
\subsubsection{updateMaterialByToken}
\paragraph{Parameter} Die Funktion besitzt folgende Parameter:
\begin{table}[H]
	\begin{tabular}{|c|p{11cm}|}
		\hline
		\textbf{Parametername} & \textbf{Parameterbeschreibung} \\ \hline
		\$token       & alphanumerischer Identifikator eines Bildes \\ \hline
		\$title       & Titel des Bildes \\ \hline
		\$description & Beschreibung des Bildes \\ \hline
	\end{tabular}
\end{table}
\paragraph{Beschreibung} Die Funktion aktualisiert die Metadaten eines Bildes. Die Funktion hat Auswirkungen auf folgende Quellen:
\begin{itemize}
	\item Tabelle mit Bildern
\end{itemize}
Die Antwort wird als strukturiertes Array an den Aufrufer zurückgegeben.
\subsubsection{updateDeletionStatePictureByToken}
\paragraph{Parameter} Die Funktion besitzt folgende Parameter:
\begin{table}[H]
	\begin{tabular}{|c|p{11cm}|}
		\hline
		\textbf{Parametername} & \textbf{Parameterbeschreibung} \\ \hline
		\$id    & numerischer Identifikator eines Bildes \\ \hline
		\$state & Status der Freischaltung \\ \hline
	\end{tabular}
\end{table}
\paragraph{Beschreibung} Die Funktion setzt den Status der Freischaltung eines Bildes. Die Funktion hat Auswirkungen auf folgende Quellen:
\begin{itemize}
	\item Tabelle mit Bildern
\end{itemize}
Die Antwort wird als strukturiertes Array an den Aufrufer zurückgegeben.
\subsubsection{updateMaterialSourceByToken}
\paragraph{Parameter} Die Funktion besitzt folgende Parameter:
\begin{table}[H]
	\begin{tabular}{|c|p{11cm}|}
		\hline
		\textbf{Parametername} & \textbf{Parameterbeschreibung} \\ \hline
		\$token       & alphanumerischer Identifikator eines Bildes \\ \hline
		\$source      & Quellenangabe des Bildes \\ \hline
		\$sourceType  & Identifikator des Typs der Quelle des Bildes \\ \hline
	\end{tabular}
\end{table}
\paragraph{Beschreibung} Die Funktion aktualisiert die Metadaten zu Quellen eines Bildes. Die Funktion hat Auswirkungen auf folgende Quellen:
\begin{itemize}
	\item Tabelle mit Bildern
\end{itemize}
Die Antwort wird als strukturiertes Array an den Aufrufer zurückgegeben.
\newpage
\section{point-reason-db}
\subsection{Allgemeines} Diese Datei enthält alle Funktionen zum Zugriff auf die Tabelle mit Gründen von Rangpunkten.
\begin{table}[H]
	\begin{tabular}{|c|p{11cm}|}
		\hline
		\textbf{Einbindungspunkt} & inc-db.php \\ \hline
		\textbf{Einbindungspunkt} & inc-db-sub.php \\ \hline
	\end{tabular}
\end{table}
Die Datei ist nicht direkt durch den Nutzer aufrufbar, dies wird durch folgenden Code-Ausschnitt sichergestellt:
\begin{lstlisting}[language=php]
if (!defined('NICE_PROJECT')) {
	die('Permission denied.');
}
\end{lstlisting}
Der Globale Wert {\glqq NICE\_PROJECT\grqq} wird durch für den Nutzer valide Aufrufpunkte festgelegt, z.B. {\glqq api.php\grqq}.
\newpage
\subsection{Funktionen}
\subsubsection{getAllReasons}
\paragraph{Parameter} Die Funktion besitzt keine Parameter.
\paragraph{Beschreibung} Die Funktion ruft eine Liste aller Begründungen von Rangpunkten ab. Die Funktion nutzt folgende Quellen:
\begin{itemize}
	\item Tabelle mit Gründen von Rangpunkten
\end{itemize}
Die Antwort wird als strukturiertes Array an den Aufrufer zurückgegeben.
\subsubsection{addReason}
\paragraph{Parameter} Die Funktion besitzt folgende Parameter:
\begin{table}[H]
	\begin{tabular}{|c|p{11cm}|}
		\hline
		\textbf{Parametername} & \textbf{Parameterbeschreibung} \\ \hline
		\$name & Begründung von Punkten \\ \hline
	\end{tabular}
\end{table}
\paragraph{Beschreibung} Die Funktion fügt eine Begründung von Rangpunkten hinzu. Die Funktion hat Auswirkungen auf folgende Quellen:
\begin{itemize}
	\item Tabelle mit Gründen von Rangpunkten
\end{itemize}
Die Antwort wird als strukturiertes Array an den Aufrufer zurückgegeben.

\newpage
\section{rank-point-db}
\subsection{Allgemeines} Diese Datei enthält alle Funktionen zum Zugriff auf die Tabelle mit Rangpunkten der Nutzer.
\begin{table}[H]
	\begin{tabular}{|c|p{11cm}|}
		\hline
		\textbf{Einbindungspunkt} & inc-db.php \\ \hline
		\textbf{Einbindungspunkt} & inc-db-sub.php \\ \hline
	\end{tabular}
\end{table}
Die Datei ist nicht direkt durch den Nutzer aufrufbar, dies wird durch folgenden Code-Ausschnitt sichergestellt:
\begin{lstlisting}[language=php]
if (!defined('NICE_PROJECT')) {
	die('Permission denied.');
}
\end{lstlisting}
Der Globale Wert {\glqq NICE\_PROJECT\grqq} wird durch für den Nutzer valide Aufrufpunkte festgelegt, z.B. {\glqq api.php\grqq}.
\newpage
\subsection{Funktionen}
\subsubsection{PersonalAreaCollection}
\paragraph{Parameter} Die Funktion besitzt folgende Parameter:
\begin{table}[H]
	\begin{tabular}{|c|p{11cm}|}
		\hline
		\textbf{Parametername} & \textbf{Parameterbeschreibung} \\ \hline
		\$uid   & Identifikator des Nutzers \\ \hline
		\$aid   & Identifikator des Moduls \\ \hline
		\$value & Anzahl der Rangpunkte \\ \hline
		\$oid   & Identifikator der Begründung \\ \hline
	\end{tabular}
\end{table}
\paragraph{Beschreibung} Die Funktion fügt einem Nutzer Rangpunkte hinzu. Die Funktion hat Auswirkungen auf folgende Quellen:
\begin{itemize}
	\item Tabelle mit Rangpunkten der Nutzer
\end{itemize}
Die Antwort wird als strukturiertes Array an den Aufrufer zurückgegeben.
\subsubsection{getPointsOfUser}
\paragraph{Parameter} Die Funktion besitzt folgende Parameter:
\begin{table}[H]
	\begin{tabular}{|c|p{11cm}|}
		\hline
		\textbf{Parametername} & \textbf{Parameterbeschreibung} \\ \hline
		\$username & Nutzername \\ \hline
	\end{tabular}
\end{table}
\subparagraph{\$json}Das Array enthält folgende Elemente:
\begin{table}[H]
	\begin{tabular}{|c|p{11cm}|}
		\hline
		\textbf{Parametername} & \textbf{Parameterbeschreibung} \\ \hline
		\$uid   & Identifikator des Nutzers \\ \hline
		\$aid   & Identifikator des Moduls \\ \hline
	\end{tabular}
\end{table}
\paragraph{Beschreibung} Die Funktion ruft die Anzahl der Rangpunkte eines Nutzers in einem Modul ab. Die Funktion nutzt folgende Quellen:
\begin{itemize}
	\item Tabelle mit Rangpunkten der Nutzer
\end{itemize}
Die Antwort wird als strukturiertes Array an den Aufrufer zurückgegeben.
\subsubsection{getPointsOfUserLastYear}
\paragraph{Parameter} Die Funktion besitzt folgende Parameter:
\begin{table}[H]
	\begin{tabular}{|c|p{11cm}|}
		\hline
		\textbf{Parametername} & \textbf{Parameterbeschreibung} \\ \hline
		\$username & Nutzername \\ \hline
	\end{tabular}
\end{table}
\subparagraph{\$json}Das Array enthält folgende Elemente:
\begin{table}[H]
	\begin{tabular}{|c|p{11cm}|}
		\hline
		\textbf{Parametername} & \textbf{Parameterbeschreibung} \\ \hline
		\$uid   & Identifikator des Nutzers \\ \hline
		\$aid   & Identifikator des Moduls \\ \hline
	\end{tabular}
\end{table}
\paragraph{Beschreibung} Die Funktion ruft die Anzahl der Rangpunkte innerhalb des letzten Jahres eines Nutzers in einem Modul ab. Die Funktion nutzt folgende Quellen:
\begin{itemize}
	\item Tabelle mit Rangpunkten der Nutzer
\end{itemize}
Die Antwort wird als strukturiertes Array an den Aufrufer zurückgegeben.
\subsubsection{getRanktableForApplication}
\paragraph{Parameter} Die Funktion besitzt folgende Parameter:
\begin{table}[H]
	\begin{tabular}{|c|p{11cm}|}
		\hline
		\textbf{Parametername} & \textbf{Parameterbeschreibung} \\ \hline
		\$aid   & Identifikator des Moduls \\ \hline
	\end{tabular}
\end{table}
\paragraph{Beschreibung} Die Funktion ruft die Rangliste der Nutzer eines Moduls ab. Die Funktion nutzt folgende Quellen:
\begin{itemize}
	\item Tabelle mit Rangpunkten der Nutzer
\end{itemize}
Die Antwort wird als strukturiertes Array an den Aufrufer zurückgegeben.
\subsubsection{getRankDataForAllUsers}
\paragraph{Parameter} Die Funktion besitzt keine Parameter.
\paragraph{Beschreibung} Die Funktion ruft alle Rangpunktevergaben ab. Die Funktion nutzt folgende Quellen:
\begin{itemize}
	\item Tabelle mit Rangpunkten der Nutzer
	\item Tabelle mit Nutzerdaten
\end{itemize}
Die Antwort wird als strukturiertes Array an den Aufrufer zurückgegeben.

\newpage
\section{ranks-db}
\subsection{Allgemeines} Diese Datei enthält alle Funktionen zum Zugriff auf die Tabelle mit Rängen.
\begin{table}[H]
	\begin{tabular}{|c|p{11cm}|}
		\hline
		\textbf{Einbindungspunkt} & inc-db.php \\ \hline
		\textbf{Einbindungspunkt} & inc-db-sub.php \\ \hline
	\end{tabular}
\end{table}
Die Datei ist nicht direkt durch den Nutzer aufrufbar, dies wird durch folgenden Code-Ausschnitt sichergestellt:
\begin{lstlisting}[language=php]
if (!defined('NICE_PROJECT')) {
	die('Permission denied.');
}
\end{lstlisting}
Der Globale Wert {\glqq NICE\_PROJECT\grqq} wird durch für den Nutzer valide Aufrufpunkte festgelegt, z.B. {\glqq api.php\grqq}.
\newpage
\subsection{Funktionen}
\subsubsection{getRanks}
\paragraph{Parameter} Die Funktion besitzt keine Parameter.
\paragraph{Beschreibung} Die Funktion gibt eine Liste von Rängen zurück. Die Funktion nutzt folgende Quellen:
\begin{itemize}
	\item Tabelle mit Rangdaten
\end{itemize}
Die Antwort wird als strukturiertes Array an den Aufrufer zurückgegeben.
\subsubsection{addRank}
\paragraph{Parameter} Die Funktion besitzt folgende Parameter:
\begin{table}[H]
	\begin{tabular}{|c|p{11cm}|}
		\hline
		\textbf{Parametername} & \textbf{Parameterbeschreibung} \\ \hline
		\$value & Rangwert \\ \hline
		\$name  & Rangname \\ \hline
	\end{tabular}
\end{table}
\paragraph{Beschreibung} Die Funktion fügt einen neuen Rang in das System ein. Die Funktion hat Auswirkungen auf folgende Quellen:
\begin{itemize}
	\item Tabelle mit Rangdaten
\end{itemize}
Die Antwort wird als strukturiertes Array an den Aufrufer zurückgegeben.
\subsubsection{deleteRank}
\paragraph{Parameter} Die Funktion besitzt folgende Parameter:
\begin{table}[H]
	\begin{tabular}{|c|p{11cm}|}
		\hline
		\textbf{Parametername} & \textbf{Parameterbeschreibung} \\ \hline
		\$rid & Identifikator eines Ranges \\ \hline
	\end{tabular}
\end{table}
\paragraph{Beschreibung} Die Funktion löscht einen Rang. Die Funktion hat Auswirkungen auf folgende Quellen:
\begin{itemize}
	\item Tabelle mit Rangdaten
\end{itemize}
Die Funktion besitzt keinen Rückgabewert.
\subsubsection{updateRank}
\paragraph{Parameter} Die Funktion besitzt folgende Parameter:
\begin{table}[H]
	\begin{tabular}{|c|p{11cm}|}
		\hline
		\textbf{Parametername} & \textbf{Parameterbeschreibung} \\ \hline
		\$id    & Identifikator des Ranges \\ \hline
		\$value & Rangwert \\ \hline
		\$name  & Rangname \\ \hline
	\end{tabular}
\end{table}
\paragraph{Beschreibung} Die Funktion ändert die Daten eines Ranges. Die Funktion hat Auswirkungen auf folgende Quellen:
\begin{itemize}
	\item Tabelle mit Rangdaten
\end{itemize}
Die Funktion besitzt keinen Rückgabewert.
\newpage
\section{role-db}
\subsection{Allgemeines} Diese Datei enthält alle Funktionen zum Zugriff auf die Tabelle mit Rollen.
\begin{table}[H]
	\begin{tabular}{|c|p{11cm}|}
		\hline
		\textbf{Einbindungspunkt} & inc-db.php \\ \hline
		\textbf{Einbindungspunkt} & inc-db-sub.php \\ \hline
	\end{tabular}
\end{table}
Die Datei ist nicht direkt durch den Nutzer aufrufbar, dies wird durch folgenden Code-Ausschnitt sichergestellt:
\begin{lstlisting}[language=php]
if (!defined('NICE_PROJECT')) {
	die('Permission denied.');
}
\end{lstlisting}
Der Globale Wert {\glqq NICE\_PROJECT\grqq} wird durch für den Nutzer valide Aufrufpunkte festgelegt, z.B. {\glqq api.php\grqq}.
\newpage
\subsection{Funktionen}
\subsubsection{getAllRolles}
\paragraph{Parameter} Die Funktion besitzt keine Parameter.
\paragraph{Beschreibung} Die Funktion gibt eine Liste von Rollen zurück. Die Funktion nutzt folgende Quellen:
\begin{itemize}
	\item Tabelle mit Rollen
\end{itemize}
Die Antwort wird als strukturiertes Array an den Aufrufer zurückgegeben.
\subsubsection{getRoleByID}
\paragraph{Parameter} Die Funktion besitzt folgende Parameter:
\begin{table}[H]
	\begin{tabular}{|c|p{11cm}|}
		\hline
		\textbf{Parametername} & \textbf{Parameterbeschreibung} \\ \hline
		\$id & Identifikator einer Rolle \\ \hline
	\end{tabular}
\end{table}
\paragraph{Beschreibung} Die Funktion ruft alle Rollendaten anhand des Identifikators einer Rolle ab. Die Funktion nutzt folgende Quellen:
\begin{itemize}
	\item Tabelle mit Rollen
\end{itemize}
Die Antwort wird als strukturiertes Array an den Aufrufer zurückgegeben.
\subsubsection{deleteRole}
\paragraph{Parameter} Die Funktion besitzt folgende Parameter:
\begin{table}[H]
	\begin{tabular}{|c|p{11cm}|}
		\hline
		\textbf{Parametername} & \textbf{Parameterbeschreibung} \\ \hline
		\$rid & Identifikator einer Rolle \\ \hline
	\end{tabular}
\end{table}
\paragraph{Beschreibung} Die Funktion löscht eine Rolle. Die Funktion hat Auswirkungen auf folgende Quellen:
\begin{itemize}
	\item Tabelle mit Rollen
\end{itemize}
Die Funktion besitzt keinen Rückgabewert.
\subsubsection{addRole}
\paragraph{Parameter} Die Funktion besitzt folgende Parameter:
\begin{table}[H]
	\begin{tabular}{|c|p{11cm}|}
		\hline
		\textbf{Parametername} & \textbf{Parameterbeschreibung} \\ \hline
		\$value & Rollenwert \\ \hline
		\$name  & Rollenname \\ \hline
	\end{tabular}
\end{table}
\paragraph{Beschreibung} Die Funktion fügt eine neue Rolle hinzu. Die Funktion hat Auswirkungen auf folgende Quellen:
\begin{itemize}
	\item Tabelle mit Rollen
\end{itemize}
Die Antwort wird als strukturiertes Array an den Aufrufer zurückgegeben.
\subsubsection{updateRole}
\paragraph{Parameter} Die Funktion besitzt folgende Parameter:
\begin{table}[H]
	\begin{tabular}{|c|p{11cm}|}
		\hline
		\textbf{Parametername} & \textbf{Parameterbeschreibung} \\ \hline
		\$id    & Identifikator einer Rolle \\ \hline
		\$value & Rollenwert \\ \hline
		\$name  & Rollenname \\ \hline
	\end{tabular}
\end{table}
\paragraph{Beschreibung} Die Funktion aktualisiert eine Rolle. Die Funktion hat Auswirkungen auf folgende Quellen:
\begin{itemize}
	\item Tabelle mit Rollen
\end{itemize}
Die Antwort wird als strukturiertes Array an den Aufrufer zurückgegeben.
\newpage
\section{session-db}
\subsection{Allgemeines} Diese Datei enthält alle Funktionen zum Zugriff auf die Tabelle mit Session-Daten.
\begin{table}[H]
	\begin{tabular}{|c|p{11cm}|}
		\hline
		\textbf{Einbindungspunkt} & inc-db.php \\ \hline
		\textbf{Einbindungspunkt} & inc-db-sub.php \\ \hline
	\end{tabular}
\end{table}
Die Datei ist nicht direkt durch den Nutzer aufrufbar, dies wird durch folgenden Code-Ausschnitt sichergestellt:
\begin{lstlisting}[language=php]
if (!defined('NICE_PROJECT')) {
	die('Permission denied.');
}
\end{lstlisting}
Der Globale Wert {\glqq NICE\_PROJECT\grqq} wird durch für den Nutzer valide Aufrufpunkte festgelegt, z.B. {\glqq api.php\grqq}.
\newpage
\subsection{Funktionen}
\subsubsection{replaceSessionData}
\paragraph{Parameter} Die Funktion besitzt folgende Parameter:
\begin{table}[H]
	\begin{tabular}{|c|p{11cm}|}
		\hline
		\textbf{Parametername} & \textbf{Parameterbeschreibung} \\ \hline
		\$sessionID   & Identifikator der Session \\ \hline
		\$sessionData & Session-Daten \\ \hline
	\end{tabular}
\end{table}
\paragraph{Beschreibung} Die Funktion ändert oder fügt Session-Daten in die Datenbank ein. Die Funktion hat Auswirkungen auf folgende Quellen:
\begin{itemize}
	\item Tabelle mit Session-Daten
\end{itemize}
Die Funktion gibt stets {\glqq true\grqq} zurück.
\subsubsection{readSessionData}
\paragraph{Parameter} Die Funktion besitzt folgende Parameter:
\begin{table}[H]
	\begin{tabular}{|c|p{11cm}|}
		\hline
		\textbf{Parametername} & \textbf{Parameterbeschreibung} \\ \hline
		\$SessionID & Identifikator der Session \\ \hline
	\end{tabular}
\end{table}
\paragraph{Beschreibung} Die Funktion ruft Session-Daten aus der Datenbank ab. Die Funktion nutzt folgende Quellen:
\begin{itemize}
	\item Tabelle mit Session-Daten
\end{itemize}
Die Funktion gibt eine Zeichenkette zurück.
\subsubsection{deleteSessionData}
\paragraph{Parameter} Die Funktion besitzt folgende Parameter:
\begin{table}[H]
	\begin{tabular}{|c|p{11cm}|}
		\hline
		\textbf{Parametername} & \textbf{Parameterbeschreibung} \\ \hline
		\$SessionID & Identifikator der Session \\ \hline
	\end{tabular}
\end{table}
\paragraph{Beschreibung} Die Funktion löscht bestimmte Session-Daten. Die Funktion hat Auswirkungen auf folgende Quellen:
\begin{itemize}
	\item Tabelle mit Rollen
\end{itemize}
Die Funktion besitzt keinen Rückgabewert.
\subsubsection{deleteSessionDataGC}
\paragraph{Parameter} Die Funktion besitzt folgende Parameter:
\begin{table}[H]
	\begin{tabular}{|c|p{11cm}|}
		\hline
		\textbf{Parametername} & \textbf{Parameterbeschreibung} \\ \hline
		\$life & Lebenszeit in Sekunden von Session-Daten \\ \hline
	\end{tabular}
\end{table}
\paragraph{Beschreibung} Die Funktion löscht Session-Daten, wenn der Timestamp zu alt ist. Die Funktion hat Auswirkungen auf folgende Quellen:
\begin{itemize}
	\item Tabelle mit Rollen
\end{itemize}
Die Funktion besitzt keinen Rückgabewert.
\newpage
\section{source-type-db}
\subsection{Allgemeines} Diese Datei enthält alle Funktionen, welche die Tabelle mit Typen von Quellen benutzen.
\begin{table}[H]
	\begin{tabular}{|c|p{11cm}|}
		\hline
		\textbf{Einbindungspunkt} & inc-db.php \\ \hline
		\textbf{Einbindungspunkt} & inc-db-sub.php \\ \hline
	\end{tabular}
\end{table}
Die Datei ist nicht direkt durch den Nutzer aufrufbar, dies wird durch folgenden Code-Ausschnitt sichergestellt:
\begin{lstlisting}[language=php]
	if (!defined('NICE_PROJECT')) {
		die('Permission denied.');
	}
\end{lstlisting}
Der Globale Wert {\glqq NICE\_PROJECT\grqq} wird durch für den Nutzer valide Aufrufpunkte festgelegt, z.B. {\glqq api.php\grqq}.
\newpage
\subsection{Funktionen}
\subsubsection{getAllSourceRelations}
\paragraph{Parameter} Die Funktion besitzt keine Parameter.
\paragraph{Beschreibung} Die Funktion ruft alle Typen von Quellen ab. Die Funktion nutzt folgende Quellen:
\begin{itemize}
	\item Tabelle mit Typen von Quellen
\end{itemize}
Es findet bei dieser Funktion kein Abruf von Daten aus {\glqq COSP\grqq} statt. Es gibt einen Rückgabewert.
\newpage
\section{statistics-basic-dbfunctions}
\subsection{Allgemeines} Diese Datei enthält alle grundlegenden Funktionen um statistische Daten ab zu rufen.
\begin{table}[H]
	\begin{tabular}{|c|p{11cm}|}
		\hline
		\textbf{Einbindungspunkt} & inc-db.php \\ \hline
		\textbf{Einbindungspunkt} & inc-db-sub.php \\ \hline
	\end{tabular}
\end{table}
Die Datei ist nicht direkt durch den Nutzer aufrufbar, dies wird durch folgenden Code-Ausschnitt sichergestellt:
\begin{lstlisting}[language=php]
if (!defined('NICE_PROJECT')) {
	die('Permission denied.');
}
\end{lstlisting}
Der Globale Wert {\glqq NICE\_PROJECT\grqq} wird durch für den Nutzer valide Aufrufpunkte festgelegt, z.B. {\glqq api.php\grqq}.
\subsection{Funktionen}
\subsubsection{ExecuteStatisticStatement}
\paragraph{Parameter} Die Funktion besitzt folgende Parameter:
\begin{table}[H]
	\begin{tabular}{|c|p{11cm}|}
		\hline
		\textbf{Parametername} & \textbf{Parameterbeschreibung} \\ \hline
		\$statement & Vorbereitetes Statement ohne Werte \\ \hline
		\$amount    & Anzahl an Zeiteinheiten \\ \hline
	\end{tabular}
\end{table}
\paragraph{Beschreibung} Die Funktion führt das angegebene Statement mit dem angegebenen Parameter aus. Die Antwort wird als strukturiertes Array an den Aufrufer zurückgegeben.
\newpage
\section{statistics-contact}
\subsection{Allgemeines} Diese Datei enthält alle Funktionen zum Zugriff auf die Tabelle mit statistischen Daten zur Nutzung der Kontaktformulare.
\begin{table}[H]
	\begin{tabular}{|c|p{11cm}|}
		\hline
		\textbf{Einbindungspunkt} & inc-db.php \\ \hline
		\textbf{Einbindungspunkt} & inc-db-sub.php \\ \hline
	\end{tabular}
\end{table}
Die Datei ist nicht direkt durch den Nutzer aufrufbar, dies wird durch folgenden Code-Ausschnitt sichergestellt:
\begin{lstlisting}[language=php]
	if (!defined('NICE_PROJECT')) {
		die('Permission denied.');
	}
\end{lstlisting}
Der Globale Wert {\glqq NICE\_PROJECT\grqq} wird durch für den Nutzer valide Aufrufpunkte festgelegt, z.B. {\glqq api.php\grqq}.
\subsection{Funktionen}
\subsubsection{insertLogContactMail}
\paragraph{Parameter} Die Funktion besitzt folgende Parameter:
\begin{table}[H]
	\begin{tabular}{|c|p{11cm}|}
		\hline
		\textbf{Parametername} & \textbf{Parameterbeschreibung} \\ \hline
		\$ip   & IP-Adresse des Aufrufers \\ \hline
		\$aid  & Identifikator eines Moduls \\ \hline
	\end{tabular}
\end{table}
\paragraph{Beschreibung} Die Funktion fügt einen statistischen Eintrag zum aktuellen Aufruf hinzu. Die Funktion hat Auswirkungen auf folgende Quellen:
\begin{itemize}
	\item Tabelle mit statistischen Daten zur Kontaktformularnutzung
\end{itemize}
Die Antwort wird als strukturiertes Array an den Aufrufer zurückgegeben.
\subsubsection{getContactStatisticalDataLastWeeks}
\paragraph{Parameter} Die Funktion besitzt folgende Parameter:
\begin{table}[H]
	\begin{tabular}{|c|p{11cm}|}
		\hline
		\textbf{Parametername} & \textbf{Parameterbeschreibung} \\ \hline
		\$number & Anzahl an Wochen \\ \hline
	\end{tabular}
\end{table}
\paragraph{Beschreibung} Die Funktion ruft statistische Daten zur Nutzung des Kontaktformulars der angegebenen letzten Wochen ab. Die Funktion nutzt folgende Quellen:
\begin{itemize}
	\item Tabelle mit Bildern
\end{itemize}
Die Antwort wird als strukturiertes Array an den Aufrufer zurückgegeben.
\subsubsection{getContactStatisticalDataLastMonth}
\paragraph{Parameter} Die Funktion besitzt folgende Parameter:
\begin{table}[H]
	\begin{tabular}{|c|p{11cm}|}
		\hline
		\textbf{Parametername} & \textbf{Parameterbeschreibung} \\ \hline
		\$number & Anzahl an Monaten \\ \hline
	\end{tabular}
\end{table}
\paragraph{Beschreibung} Die Funktion ruft statistische Daten zur Nutzung des Kontaktformulars der angegebenen letzten Monate ab. Die Funktion nutzt folgende Quellen:
\begin{itemize}
	\item Tabelle mit Bildern
\end{itemize}
Die Antwort wird als strukturiertes Array an den Aufrufer zurückgegeben.
\subsubsection{getContactStatisticalDataLastYear}
\paragraph{Parameter} Die Funktion besitzt folgende Parameter:
\begin{table}[H]
	\begin{tabular}{|c|p{11cm}|}
		\hline
		\textbf{Parametername} & \textbf{Parameterbeschreibung} \\ \hline
		\$number & Anzahl an Jahren \\ \hline
	\end{tabular}
\end{table}
\paragraph{Beschreibung} Die Funktion ruft statistische Daten zur Nutzung des Kontaktformulars der angegebenen letzten Jahren ab. Die Funktion nutzt folgende Quellen:
\begin{itemize}
	\item Tabelle mit Bildern
\end{itemize}
Die Antwort wird als strukturiertes Array an den Aufrufer zurückgegeben.
\subsubsection{getContactStatisticalDataLastDays}
\paragraph{Parameter} Die Funktion besitzt folgende Parameter:
\begin{table}[H]
	\begin{tabular}{|c|p{11cm}|}
		\hline
		\textbf{Parametername} & \textbf{Parameterbeschreibung} \\ \hline
		\$number & Anzahl an Tagen \\ \hline
	\end{tabular}
\end{table}
\paragraph{Beschreibung} Die Funktion ruft statistische Daten zur Nutzung des Kontaktformulars der angegebenen letzten Tage ab. Die Funktion nutzt folgende Quellen:
\begin{itemize}
	\item Tabelle mit Bildern
\end{itemize}
Die Antwort wird als strukturiertes Array an den Aufrufer zurückgegeben.
\newpage
\section{statistics-pictures-db}
\input{Kapitel/Files/statistics-pictures-db}
\newpage
\section{statistics-story-db}
\subsection{Allgemeines} Diese Datei enthält alle Funktionen zum Abrufen von statistischen Daten der Geschichten-Datenbank.
\begin{table}[H]
	\begin{tabular}{|c|p{11cm}|}
		\hline
		\textbf{Einbindungspunkt} & inc-db.php \\ \hline
		\textbf{Einbindungspunkt} & inc-db-sub.php \\ \hline
	\end{tabular}
\end{table}
Die Datei ist nicht direkt durch den Nutzer aufrufbar, dies wird durch folgenden Code-Ausschnitt sichergestellt:
\begin{lstlisting}[language=php]
if (!defined('NICE_PROJECT')) {
	die('Permission denied.');
}
\end{lstlisting}
Der Globale Wert {\glqq NICE\_PROJECT\grqq} wird durch für den Nutzer valide Aufrufpunkte festgelegt, z.B. {\glqq api.php\grqq}.
\newpage
\subsection{Funktionen}
\subsubsection{getNewStoryStatisticalDataLastWeeks}
\paragraph{Parameter} Die Funktion besitzt folgende Parameter:
\begin{table}[H]
	\begin{tabular}{|c|p{11cm}|}
		\hline
		\textbf{Parametername} & \textbf{Parameterbeschreibung} \\ \hline
		\$number & Anzahl an Wochen \\ \hline
	\end{tabular}
\end{table}
\paragraph{Beschreibung} Die Funktion ruft statistische Daten zu Geschichten der angegebenen letzten Wochen ab. Die Funktion nutzt folgende Quellen:
\begin{itemize}
	\item Tabelle mit Geschichten
\end{itemize}
Die Antwort wird als strukturiertes Array an den Aufrufer zurückgegeben.
\subsubsection{getNewStoryStatisticalDataLastMonth}
\paragraph{Parameter} Die Funktion besitzt folgende Parameter:
\begin{table}[H]
	\begin{tabular}{|c|p{11cm}|}
		\hline
		\textbf{Parametername} & \textbf{Parameterbeschreibung} \\ \hline
		\$number & Anzahl an Monaten \\ \hline
	\end{tabular}
\end{table}
\paragraph{Beschreibung} Die Funktion ruft statistische Daten zu Geschichten der angegebenen letzten Monate ab. Die Funktion nutzt folgende Quellen:
\begin{itemize}
	\item Tabelle mit Geschichten
\end{itemize}
Die Antwort wird als strukturiertes Array an den Aufrufer zurückgegeben.
\subsubsection{getNewStoryStatisticalDataLastYear}
\paragraph{Parameter} Die Funktion besitzt folgende Parameter:
\begin{table}[H]
	\begin{tabular}{|c|p{11cm}|}
		\hline
		\textbf{Parametername} & \textbf{Parameterbeschreibung} \\ \hline
		\$number & Anzahl an Jahren \\ \hline
	\end{tabular}
\end{table}
\paragraph{Beschreibung} Die Funktion ruft statistische Daten zu Geschichten der angegebenen letzten Jahren ab. Die Funktion nutzt folgende Quellen:
\begin{itemize}
	\item Tabelle mit Geschichten
\end{itemize}
Die Antwort wird als strukturiertes Array an den Aufrufer zurückgegeben.
\subsubsection{getNewStoryStatisticalDataLastDays}
\paragraph{Parameter} Die Funktion besitzt folgende Parameter:
\begin{table}[H]
	\begin{tabular}{|c|p{11cm}|}
		\hline
		\textbf{Parametername} & \textbf{Parameterbeschreibung} \\ \hline
		\$number & Anzahl an Tagen \\ \hline
	\end{tabular}
\end{table}
\paragraph{Beschreibung} Die Funktion ruft statistische Daten zu Geschichten der angegebenen letzten Tage ab. Die Funktion nutzt folgende Quellen:
\begin{itemize}
	\item Tabelle mit Geschichten
\end{itemize}
Die Antwort wird als strukturiertes Array an den Aufrufer zurückgegeben.
\newpage
\section{statistics-user}
\subsection{Allgemeines} Diese Datei enthält alle Funktionen zum Abrufen von statistischen Daten der Nutzer-Datenbank.
\begin{table}[H]
	\begin{tabular}{|c|p{11cm}|}
		\hline
		\textbf{Einbindungspunkt} & inc-db.php \\ \hline
		\textbf{Einbindungspunkt} & inc-db-sub.php \\ \hline
	\end{tabular}
\end{table}
Die Datei ist nicht direkt durch den Nutzer aufrufbar, dies wird durch folgenden Code-Ausschnitt sichergestellt:
\begin{lstlisting}[language=php]
if (!defined('NICE_PROJECT')) {
	die('Permission denied.');
}
\end{lstlisting}
Der Globale Wert {\glqq NICE\_PROJECT\grqq} wird durch für den Nutzer valide Aufrufpunkte festgelegt, z.B. {\glqq api.php\grqq}.
\newpage
\subsection{Funktionen}
\subsubsection{getNewUsersStatisticalDataLastWeeks}
\paragraph{Parameter} Die Funktion besitzt folgende Parameter:
\begin{table}[H]
	\begin{tabular}{|c|p{11cm}|}
		\hline
		\textbf{Parametername} & \textbf{Parameterbeschreibung} \\ \hline
		\$number & Anzahl an Wochen \\ \hline
	\end{tabular}
\end{table}
\paragraph{Beschreibung} Die Funktion ruft statistische Daten zu Nutzern der angegebenen letzten Wochen ab. Die Funktion nutzt folgende Quellen:
\begin{itemize}
	\item Tabelle mit Nutzerdaten
\end{itemize}
Die Antwort wird als strukturiertes Array an den Aufrufer zurückgegeben.
\subsubsection{getNewUsersStatisticalDataLastMonth}
\paragraph{Parameter} Die Funktion besitzt folgende Parameter:
\begin{table}[H]
	\begin{tabular}{|c|p{11cm}|}
		\hline
		\textbf{Parametername} & \textbf{Parameterbeschreibung} \\ \hline
		\$number & Anzahl an Monaten \\ \hline
	\end{tabular}
\end{table}
\paragraph{Beschreibung} Die Funktion ruft statistische Daten zu Nutzern der angegebenen letzten Monate ab. Die Funktion nutzt folgende Quellen:
\begin{itemize}
	\item Tabelle mit Nutzerdaten
\end{itemize}
Die Antwort wird als strukturiertes Array an den Aufrufer zurückgegeben.
\subsubsection{getNewUsersStatisticalDataLastYear}
\paragraph{Parameter} Die Funktion besitzt folgende Parameter:
\begin{table}[H]
	\begin{tabular}{|c|p{11cm}|}
		\hline
		\textbf{Parametername} & \textbf{Parameterbeschreibung} \\ \hline
		\$number & Anzahl an Jahren \\ \hline
	\end{tabular}
\end{table}
\paragraph{Beschreibung} Die Funktion ruft statistische Daten zu Nutzern der angegebenen letzten Jahren ab. Die Funktion nutzt folgende Quellen:
\begin{itemize}
	\item Tabelle mit Nutzerdaten
\end{itemize}
Die Antwort wird als strukturiertes Array an den Aufrufer zurückgegeben.
\subsubsection{getNewUsersStatisticalDataLastDays}
\paragraph{Parameter} Die Funktion besitzt folgende Parameter:
\begin{table}[H]
	\begin{tabular}{|c|p{11cm}|}
		\hline
		\textbf{Parametername} & \textbf{Parameterbeschreibung} \\ \hline
		\$number & Anzahl an Tagen \\ \hline
	\end{tabular}
\end{table}
\paragraph{Beschreibung} Die Funktion ruft statistische Daten zu Nutzern der angegebenen letzten Tage ab. Die Funktion nutzt folgende Quellen:
\begin{itemize}
	\item Tabelle mit Nutzerdaten
\end{itemize}
Die Antwort wird als strukturiertes Array an den Aufrufer zurückgegeben.
\newpage
\section{story-db}
\subsection{Allgemeines} Diese Datei enthält alle Funktionen zum Zugriff auf die Tabelle mit Geschichten.
\begin{table}[H]
	\begin{tabular}{|c|p{11cm}|}
		\hline
		\textbf{Einbindungspunkt} & inc-db.php \\ \hline
		\textbf{Einbindungspunkt} & inc-db-sub.php \\ \hline
	\end{tabular}
\end{table}
Die Datei ist nicht direkt durch den Nutzer aufrufbar, dies wird durch folgenden Code-Ausschnitt sichergestellt:
\begin{lstlisting}[language=php]
if (!defined('NICE_PROJECT')) {
	die('Permission denied.');
}
\end{lstlisting}
Der Globale Wert {\glqq NICE\_PROJECT\grqq} wird durch für den Nutzer valide Aufrufpunkte festgelegt, z.B. {\glqq api.php\grqq}.
\newpage
\subsection{Funktionen}
\subsubsection{insertStory}
\paragraph{Parameter} Die Funktion besitzt folgende Parameter:
\begin{table}[H]
	\begin{tabular}{|c|p{11cm}|}
		\hline
		\textbf{Parametername} & \textbf{Parameterbeschreibung} \\ \hline
		\$data & Array mit Daten \\ \hline
	\end{tabular}
\end{table}
\subparagraph{\$data}Das Array enthält folgende Elemente:
\begin{table}[H]
	\begin{tabular}{|c|p{11cm}|}
		\hline
		\textbf{Parametername} & \textbf{Parameterbeschreibung} \\ \hline
		uid   & Identifikator des hochladenden Nutzers \\ \hline
		hash  & SHA512 der Geschichte (Inhalt) \\ \hline
		title & Titel der Geschichte \\ \hline
		story & Inhalt der Geschichte \\ \hline
		aid   & numerischer Identifikator des hochladenden Moduls \\ \hline
	\end{tabular}
\end{table}
\paragraph{Beschreibung} Die Funktion fügt eine neue Geschichte hinzu. Die Funktion hat Auswirkungen auf folgende Quellen:
\begin{itemize}
	\item Tabelle mit Geschichten
\end{itemize}
Die Antwort wird als strukturiertes Array an den Aufrufer zurückgegeben.
\subsubsection{getUserStory}
\paragraph{Parameter} Die Funktion besitzt folgende Parameter:
\begin{table}[H]
	\begin{tabular}{|c|p{11cm}|}
		\hline
		\textbf{Parametername} & \textbf{Parameterbeschreibung} \\ \hline
		\$token & alphanumerischer Identifikator einer Geschichte \\ \hline
	\end{tabular}
\end{table}
\paragraph{Beschreibung} Die Funktion ruft die Daten einer Geschichte anhand des alphanumerischen Identifikators ab. Die Funktion nutzt folgende Quellen:
\begin{itemize}
	\item Tabelle mit Geschichten
	\item Tabelle mit Nutzerdaten
\end{itemize}
Die Antwort wird als strukturiertes Array an den Aufrufer zurückgegeben.
\subsubsection{getStoryIdByToken}
\paragraph{Parameter} Die Funktion besitzt folgende Parameter:
\begin{table}[H]
	\begin{tabular}{|c|p{11cm}|}
		\hline
		\textbf{Parametername} & \textbf{Parameterbeschreibung} \\ \hline
		\$token & alphanumerischer Identifikator einer Geschichte \\ \hline
	\end{tabular}
\end{table}
\paragraph{Beschreibung} Die Funktion ruft den numerischen Identifikator einer Geschichte anhand des alphanumerischen Identifikators ab. Die Funktion nutzt folgende Quellen:
\begin{itemize}
	\item Tabelle mit Geschichten
\end{itemize}
Die Antwort wird als strukturiertes Array an den Aufrufer zurückgegeben.
\subsubsection{getStoriesByApp}
\paragraph{Parameter} Die Funktion besitzt folgende Parameter:
\begin{table}[H]
	\begin{tabular}{|c|p{11cm}|}
		\hline
		\textbf{Parametername} & \textbf{Parameterbeschreibung} \\ \hline
		\$aid & numerischer Identifikator eines Moduls \\ \hline
	\end{tabular}
\end{table}
\paragraph{Beschreibung} Die Funktion ruft alle Geschichten eines Moduls ab. Die Funktion nutzt folgende Quellen:
\begin{itemize}
	\item Tabelle mit Geschichten
\end{itemize}
Die Antwort wird als strukturiertes Array an den Aufrufer zurückgegeben.
\subsubsection{updateStorieByToken}
\paragraph{Parameter} Die Funktion besitzt folgende Parameter:
\begin{table}[H]
	\begin{tabular}{|c|p{11cm}|}
		\hline
		\textbf{Parametername} & \textbf{Parameterbeschreibung} \\ \hline
		\$token & alphanumerischer Identifikator einer Geschichte \\ \hline
		\$title & Titel einer Geschichte \\ \hline
		\$story & Inhalt einer Geschichte \\ \hline
	\end{tabular}
\end{table}
\paragraph{Beschreibung} Die Funktion aktualisiert eine bestehende Geschichte. Die Funktion hat Auswirkungen auf folgende Quellen:
\begin{itemize}
	\item Tabelle mit Geschichten
\end{itemize}
Die Antwort wird als strukturiertes Array an den Aufrufer zurückgegeben.
\subsubsection{getModuleTokenByStoryToken}
\paragraph{Parameter} Die Funktion besitzt folgende Parameter:
\begin{table}[H]
	\begin{tabular}{|c|p{11cm}|}
		\hline
		\textbf{Parametername} & \textbf{Parameterbeschreibung} \\ \hline
		\$token & alphanumerischer Identifikator einer Geschichte \\ \hline
	\end{tabular}
\end{table}
\paragraph{Beschreibung} Die Funktion ruft den numerischen Identifikator des hochladenden Moduls einer Geschichte ab. Die Funktion nutzt folgende Quellen:
\begin{itemize}
	\item Tabelle mit Geschichten
\end{itemize}
Die Antwort wird als Zeichenkette an den Aufrufer zurückgegeben.
\subsubsection{getStoryIdByStoryToken}
\paragraph{Parameter} Die Funktion besitzt folgende Parameter:
\begin{table}[H]
	\begin{tabular}{|c|p{11cm}|}
		\hline
		\textbf{Parametername} & \textbf{Parameterbeschreibung} \\ \hline
		\$token & alphanumerischer Identifikator einer Geschichte \\ \hline
	\end{tabular}
\end{table}
\paragraph{Beschreibung} Die Funktion ruft den numerischen Identifikator der einer Geschichte anhand des alphanumerischen Identifikators ab. Die Funktion nutzt folgende Quellen:
\begin{itemize}
	\item Tabelle mit Geschichten
\end{itemize}
Die Antwort wird als Zeichenkette an den Aufrufer zurückgegeben.
\subsubsection{getCreatorByStoryToken}
\paragraph{Parameter} Die Funktion besitzt folgende Parameter:
\begin{table}[H]
	\begin{tabular}{|c|p{11cm}|}
		\hline
		\textbf{Parametername} & \textbf{Parameterbeschreibung} \\ \hline
		\$token & alphanumerischer Identifikator einer Geschichte \\ \hline
	\end{tabular}
\end{table}
\paragraph{Beschreibung} Die Funktion ruft den Ersteller der einer Geschichte anhand des alphanumerischen Identifikators ab. Die Funktion nutzt folgende Quellen:
\begin{itemize}
	\item Tabelle mit Geschichten
	\item Tabelle mit Nutzerdaten
\end{itemize}
Die Antwort wird als Zeichenkette an den Aufrufer zurückgegeben.
\subsubsection{deleteStory}
\paragraph{Parameter} Die Funktion besitzt folgende Parameter:
\begin{table}[H]
	\begin{tabular}{|c|p{11cm}|}
		\hline
		\textbf{Parametername} & \textbf{Parameterbeschreibung} \\ \hline
		\$sid & numerischer Identifikator einer Geschichte \\ \hline
	\end{tabular}
\end{table}
\paragraph{Beschreibung} Die Funktion löscht eine Geschichte aus der Datenbank. Die Funktion hat Auswirkungen auf folgende Quellen:
\begin{itemize}
	\item Tabelle mit Geschichten
\end{itemize}
Die Antwort wird als strukturiertes Array an den Aufrufer zurückgegeben.
\subsubsection{changeStoryApproval}
\paragraph{Parameter} Die Funktion besitzt folgende Parameter:
\begin{table}[H]
	\begin{tabular}{|c|p{11cm}|}
		\hline
		\textbf{Parametername} & \textbf{Parameterbeschreibung} \\ \hline
		\$token & alphanumerischer Identifikator einer Geschichte \\ \hline
		\$value & Freischaltungstatus ({\glqq 1\grqq}: Freigeschaltet, {\glqq 0\grqq}: Gesperrt) \\ \hline
	\end{tabular}
\end{table}
\paragraph{Beschreibung} Die Funktion setzt den Status der Freischaltung einer Geschichte. Die Funktion hat Auswirkungen auf folgende Quellen:
\begin{itemize}
	\item Tabelle mit Geschichten
\end{itemize}
Die Antwort wird als strukturiertes Array an den Aufrufer zurückgegeben.
\subsubsection{getStoryApprovalByStoryToken}
\paragraph{Parameter} Die Funktion besitzt folgende Parameter:
\begin{table}[H]
	\begin{tabular}{|c|p{11cm}|}
		\hline
		\textbf{Parametername} & \textbf{Parameterbeschreibung} \\ \hline
		\$token & alphanumerischer Identifikator einer Geschichte \\ \hline
	\end{tabular}
\end{table}
\paragraph{Beschreibung} Die Funktion ruft den Status der Freischaltung einer Geschichte anhand ihres alphanumerischen Identifikators ab. Die Funktion nutzt folgende Quellen:
\begin{itemize}
	\item Tabelle mit Geschichten
\end{itemize}
Die Antwort wird als strukturiertes Array an den Aufrufer zurückgegeben.
\subsubsection{updateDeletionStateStoryByToken}
\paragraph{Parameter} Die Funktion besitzt folgende Parameter:
\begin{table}[H]
	\begin{tabular}{|c|p{11cm}|}
		\hline
		\textbf{Parametername} & \textbf{Parameterbeschreibung} \\ \hline
		\$token & alphanumerischer Identifikator einer Geschichte \\ \hline
		\$value & Status der Löschung ({\glqq 1\grqq}: Gelöscht, {\glqq 0\grqq}: Ungelöscht) \\ \hline
	\end{tabular}
\end{table}
\paragraph{Beschreibung} Die Funktion markiert eine Geschichte als gelöscht oder ungelöscht. Die Funktion hat Auswirkungen auf folgende Quellen:
\begin{itemize}
	\item Tabelle mit Geschichten
\end{itemize}
Die Antwort wird als strukturiertes Array an den Aufrufer zurückgegeben.

\newpage
\section{user-db}
\subsection{Allgemeines} Diese Datei enthält alle Funktionen zum Zugriff auf die Tabelle mit Nutzerdaten.
\begin{table}[H]
	\begin{tabular}{|c|p{11cm}|}
		\hline
		\textbf{Einbindungspunkt} & inc-db.php \\ \hline
		\textbf{Einbindungspunkt} & inc-db-sub.php \\ \hline
	\end{tabular}
\end{table}
Die Datei ist nicht direkt durch den Nutzer aufrufbar, dies wird durch folgenden Code-Ausschnitt sichergestellt:
\begin{lstlisting}[language=php]
if (!defined('NICE_PROJECT')) {
	die('Permission denied.');
}
\end{lstlisting}
Der Globale Wert {\glqq NICE\_PROJECT\grqq} wird durch für den Nutzer valide Aufrufpunkte festgelegt, z.B. {\glqq api.php\grqq}.
\newpage
\subsection{Funktionen}
\subsubsection{addUser}
\paragraph{Parameter} Die Funktion besitzt folgende Parameter:
\begin{table}[H]
	\begin{tabular}{|c|p{11cm}|}
		\hline
		\textbf{Parametername} & \textbf{Parameterbeschreibung} \\ \hline
		\$name      & Nutzername \\ \hline
		\$pwd\_hash & gehashtes Passwort \\ \hline
		\$email     & E-Mailadresse des neuen Nutzers \\ \hline
		\$firstname & Vorname (optional) \\ \hline
		\$lastname  & Nachname (optional) \\ \hline
	\end{tabular}
\end{table}
\paragraph{Beschreibung} Die Funktion fügt einen neuen Nutzer der Datenbank hinzu. Die Funktion hat Auswirkungen auf folgende Quellen:
\begin{itemize}
	\item Tabelle mit Nutzerdaten
\end{itemize}
Die Antwort wird als strukturiertes Array an den Aufrufer zurückgegeben.
\subsubsection{getUserDataByID}
\paragraph{Parameter} Die Funktion besitzt folgende Parameter:
\begin{table}[H]
	\begin{tabular}{|c|p{11cm}|}
		\hline
		\textbf{Parametername} & \textbf{Parameterbeschreibung} \\ \hline
		\$ID & Identifikator eines Nutzers \\ \hline
	\end{tabular}
\end{table}
\paragraph{Beschreibung} Die Funktion ruft alle Nutzerdaten des gegebenen Nutzers anhand seines numerischen Identifikators ab. Die Funktion nutzt folgende Quellen:
\begin{itemize}
	\item Tabelle mit Nutzerdaten
\end{itemize}
Die Antwort wird als strukturiertes Array an den Aufrufer zurückgegeben.
\subsubsection{getUserData}
\paragraph{Parameter} Die Funktion besitzt folgende Parameter:
\begin{table}[H]
	\begin{tabular}{|c|p{11cm}|}
		\hline
		\textbf{Parametername} & \textbf{Parameterbeschreibung} \\ \hline
		\$name & Nutzername \\ \hline
	\end{tabular}
\end{table}
\paragraph{Beschreibung} Die Funktion ruft alle Nutzerdaten des gegebenen Nutzers anhand seines Nutzernamen ab. Die Funktion nutzt folgende Quellen:
\begin{itemize}
	\item Tabelle mit Nutzerdaten
\end{itemize}
Die Antwort wird als strukturiertes Array an den Aufrufer zurückgegeben.
\subsubsection{getAllUsernames}
\paragraph{Parameter} Die Funktion besitzt keine Parameter.
\paragraph{Beschreibung} Die Funktion ruft eine Liste aller bekannten Nutzernamen ab. Die Funktion nutzt folgende Quellen:
\begin{itemize}
	\item Tabelle mit Nutzerdaten
\end{itemize}
Die Antwort wird als strukturiertes Array an den Aufrufer zurückgegeben.
\subsubsection{getAllUserIds}
\paragraph{Parameter} Die Funktion besitzt keine Parameter.
\paragraph{Beschreibung} Die Funktion ruft eine Liste aller bekannten numerischen Nutzeridentifikatoren ab. Die Funktion nutzt folgende Quellen:
\begin{itemize}
	\item Tabelle mit Nutzerdaten
\end{itemize}
Die Antwort wird als strukturiertes Array an den Aufrufer zurückgegeben.
\subsubsection{updateMailValidated}
\paragraph{Parameter} Die Funktion besitzt folgende Parameter:
\begin{table}[H]
	\begin{tabular}{|c|p{11cm}|}
		\hline
		\textbf{Parametername} & \textbf{Parameterbeschreibung} \\ \hline
		\$name   & Nutzername \\ \hline
		\$status & Status der Bestätigung der Mailadresse ({\glqq 1\grqq}: Bestätigt, {\glqq 0 \grqq}: Unbestätigt)\\ \hline
	\end{tabular}
\end{table}
\paragraph{Beschreibung} Die Funktion setzt den Bestätigungsstatus einer Mailadresse. Die Funktion hat Auswirkungen auf folgende Quellen:
\begin{itemize}
	\item Tabelle mit Nutzerdaten
\end{itemize}
Die Funktion besitzt keinen Rückgabewert.
\subsubsection{updateEnableUser}
\paragraph{Parameter} Die Funktion besitzt folgende Parameter:
\begin{table}[H]
	\begin{tabular}{|c|p{11cm}|}
		\hline
		\textbf{Parametername} & \textbf{Parameterbeschreibung} \\ \hline
		\$name   & Nutzername \\ \hline
		\$status & Status der Freischaltung eines Nutzers ({\glqq 1\grqq}: Freigeschaltet, {\glqq 0 \grqq}: Gesperrt)\\ \hline
	\end{tabular}
\end{table}
\paragraph{Beschreibung} Die Funktion setzt den Status der Freischaltung eines Nutzers. Die Funktion hat Auswirkungen auf folgende Quellen:
\begin{itemize}
	\item Tabelle mit Nutzerdaten
\end{itemize}
Die Funktion besitzt keinen Rückgabewert.
\subsubsection{getAllUsers}
\paragraph{Parameter} Die Funktion besitzt keine Parameter.
\paragraph{Beschreibung} Die Funktion ruft alle Daten aller Nutzer ab. Die Funktion nutzt folgende Quellen:
\begin{itemize}
	\item Tabelle mit Nutzerdaten
\end{itemize}
Die Antwort wird als strukturiertes Array an den Aufrufer zurückgegeben.
\subsubsection{updatePassword}
\paragraph{Parameter} Die Funktion besitzt folgende Parameter:
\begin{table}[H]
	\begin{tabular}{|c|p{11cm}|}
		\hline
		\textbf{Parametername} & \textbf{Parameterbeschreibung} \\ \hline
		\$uid          & numerischer Identifikator eines Nutzers \\ \hline
		\$passwordHash & gehashtes neues Passwort \\ \hline
	\end{tabular}
\end{table}
\paragraph{Beschreibung} Die Funktion aktualisiert das Passwort eines Nutzers. Die Funktion hat Auswirkungen auf folgende Quellen:
\begin{itemize}
	\item Tabelle mit Nutzerdaten
\end{itemize}
Die Funktion besitzt keinen Rückgabewert.
\subsubsection{updateRoleUser}
\paragraph{Parameter} Die Funktion besitzt folgende Parameter:
\begin{table}[H]
	\begin{tabular}{|c|p{11cm}|}
		\hline
		\textbf{Parametername} & \textbf{Parameterbeschreibung} \\ \hline
		\$uid  & numerischer Identifikator eines Nutzers \\ \hline
		\$role & Identifikator einer Rolle \\ \hline
	\end{tabular}
\end{table}
\paragraph{Beschreibung} Die Funktion aktualisiert die Rolle eines Nutzers. Die Funktion hat Auswirkungen auf folgende Quellen:
\begin{itemize}
	\item Tabelle mit Nutzerdaten
\end{itemize}
Die Funktion besitzt keinen Rückgabewert.
\subsubsection{updateUserDB}
\paragraph{Parameter} Die Funktion besitzt folgende Parameter:
\begin{table}[H]
	\begin{tabular}{|c|p{11cm}|}
		\hline
		\textbf{Parametername} & \textbf{Parameterbeschreibung} \\ \hline
		\$firstname & Vorname \\ \hline
		\$lastname  & Nachname \\ \hline
		\$email     & E-Mailadresse \\ \hline
		\$username  & Nutzername \\ \hline
	\end{tabular}
\end{table}
\paragraph{Beschreibung} Die Funktion aktualisiert die Daten eines Nutzers. Die Funktion hat Auswirkungen auf folgende Quellen:
\begin{itemize}
	\item Tabelle mit Nutzerdaten
\end{itemize}
Die Funktion besitzt keinen Rückgabewert.
\newpage
\section{validate-picture-db}
\subsection{Allgemeines} Diese Datei enthält alle Funktionen zum Zugriff auf die Tabelle mit Validierungsdaten zu Bildern.
\begin{table}[H]
	\begin{tabular}{|c|p{11cm}|}
		\hline
		\textbf{Einbindungspunkt} & inc-db.php \\ \hline
		\textbf{Einbindungspunkt} & inc-db-sub.php \\ \hline
	\end{tabular}
\end{table}
Die Datei ist nicht direkt durch den Nutzer aufrufbar, dies wird durch folgenden Code-Ausschnitt sichergestellt:
\begin{lstlisting}[language=php]
if (!defined('NICE_PROJECT')) {
	die('Permission denied.');
}
\end{lstlisting}
Der Globale Wert {\glqq NICE\_PROJECT\grqq} wird durch für den Nutzer valide Aufrufpunkte festgelegt, z.B. {\glqq api.php\grqq}.
\newpage
\subsection{Funktionen}
\subsubsection{insertValidatePictures}
\paragraph{Parameter} Die Funktion besitzt folgende Parameter:
\begin{table}[H]
	\begin{tabular}{|c|p{11cm}|}
		\hline
		\textbf{Parametername} & \textbf{Parameterbeschreibung} \\ \hline
		\$picture\_id & numerischer Identifikator eines Bildes \\ \hline
		\$value       & Wert der Validierung \\ \hline
		\$username    & Nutzername des Validerenden \\ \hline
	\end{tabular}
\end{table}
\paragraph{Beschreibung} Die Funktion fügt einem Bild eine Validierung hinzu. Die Funktion hat Auswirkungen auf folgende Quellen:
\begin{itemize}
	\item Tabelle mit Validierungsdaten zu Bildern
\end{itemize}
Die Antwort wird als strukturiertes Array an den Aufrufer zurückgegeben.
\subsubsection{deleteValidatePictures}
\paragraph{Parameter} Die Funktion besitzt folgende Parameter:
\begin{table}[H]
	\begin{tabular}{|c|p{11cm}|}
		\hline
		\textbf{Parametername} & \textbf{Parameterbeschreibung} \\ \hline
		\$pid & numerischer Identifikator eines Bildes \\ \hline
	\end{tabular}
\end{table}
\paragraph{Beschreibung} Die Funktion löscht alle Validierungen eines Bildes. Die Funktion hat Auswirkungen auf folgende Quellen: 
\begin{itemize}
	\item Tabelle mit Validierungsdaten zu Bildern
\end{itemize}
Die Antwort wird als strukturiertes Array an den Aufrufer zurückgegeben.
\subsubsection{getUservalidationsPictures}
\paragraph{Parameter} Die Funktion besitzt folgende Parameter:
\begin{table}[H]
	\begin{tabular}{|c|p{11cm}|}
		\hline
		\textbf{Parametername} & \textbf{Parameterbeschreibung} \\ \hline
		\$pid & numerischer Identifikator eines Bildes \\ \hline
	\end{tabular}
\end{table}
\paragraph{Beschreibung} Die Funktion ruft alle Validierenden eines Bildes ab. Die Funktion nutzt folgende Quellen:
\begin{itemize}
	\item Tabelle mit Validierungsdaten zu Bildern
	\item Tabelle mit Nutzerdaten
\end{itemize}
Die Antwort wird als strukturiertes Array an den Aufrufer zurückgegeben.
\subsubsection{getValidateSumPictures}
\paragraph{Parameter} Die Funktion besitzt folgende Parameter:
\begin{table}[H]
	\begin{tabular}{|c|p{11cm}|}
		\hline
		\textbf{Parametername} & \textbf{Parameterbeschreibung} \\ \hline
		\$pid & numerischer Identifikator eines Bildes \\ \hline
	\end{tabular}
\end{table}
\paragraph{Beschreibung} Die Funktion ruft die Summe aller Validierungen eines Bildes ab. Die Funktion nutzt folgende Quellen:
\begin{itemize}
	\item Tabelle mit Validierungsdaten zu Bildern
\end{itemize}
Die Antwort wird als Zeichenkette an den Aufrufer zurückgegeben.
\newpage
\section{validate-story-db}
\subsection{Allgemeines} Diese Datei enthält alle Funktionen zum Zugriff auf die Tabelle mit Validierungsdaten zu Geschichten.
\begin{table}[H]
	\begin{tabular}{|c|p{11cm}|}
		\hline
		\textbf{Einbindungspunkt} & inc-db.php \\ \hline
		\textbf{Einbindungspunkt} & inc-db-sub.php \\ \hline
	\end{tabular}
\end{table}
Die Datei ist nicht direkt durch den Nutzer aufrufbar, dies wird durch folgenden Code-Ausschnitt sichergestellt:
\begin{lstlisting}[language=php]
if (!defined('NICE_PROJECT')) {
	die('Permission denied.');
}
\end{lstlisting}
Der Globale Wert {\glqq NICE\_PROJECT\grqq} wird durch für den Nutzer valide Aufrufpunkte festgelegt, z.B. {\glqq api.php\grqq}.
\newpage
\subsection{Funktionen}
\subsubsection{insertValidateStories}
\paragraph{Parameter} Die Funktion besitzt folgende Parameter:
\begin{table}[H]
	\begin{tabular}{|c|p{11cm}|}
		\hline
		\textbf{Parametername} & \textbf{Parameterbeschreibung} \\ \hline
		\$story\_id & numerischer Identifikator einer Geschichte \\ \hline
		\$value     & Wert der Validierung \\ \hline
		\$username  & Nutzername \\ \hline
	\end{tabular}
\end{table}
\paragraph{Beschreibung} Die Funktion fügt einer Geschichte eine Validierung hinzu. Die Funktion hat Auswirkungen auf folgende Quellen:
\begin{itemize}
	\item Tabelle mit Validierungsdaten zu Geschichten
\end{itemize}
Die Antwort wird als strukturiertes Array an den Aufrufer zurückgegeben.
\subsubsection{getUservalidationsStory}
\paragraph{Parameter} Die Funktion besitzt folgende Parameter:
\begin{table}[H]
	\begin{tabular}{|c|p{11cm}|}
		\hline
		\textbf{Parametername} & \textbf{Parameterbeschreibung} \\ \hline
		\$sid & numerischer Identifikator einer Geschichte \\ \hline
	\end{tabular}
\end{table}
\paragraph{Beschreibung} Die Funktion ruft alle Validatoren einer Geschichte ab. Die Funktion nutzt folgende Quellen:
\begin{itemize}
	\item Tabelle mit Validierungsdaten zu Geschichten
	\item Tabelle mit Nutzerdaten
\end{itemize}
Die Antwort wird als strukturiertes Array an den Aufrufer zurückgegeben.
\subsubsection{getValidateSumStories}
\paragraph{Parameter} Die Funktion besitzt folgende Parameter:
\begin{table}[H]
	\begin{tabular}{|c|p{11cm}|}
		\hline
		\textbf{Parametername} & \textbf{Parameterbeschreibung} \\ \hline
		\$sid & numerischer Identifikator einer Geschichte \\ \hline
	\end{tabular}
\end{table}
\paragraph{Beschreibung} Die Funktion ruft die Summe aller Validierungen einer Geschichte ab. Die Funktion nutzt folgende Quellen:
\begin{itemize}
	\item Tabelle mit Validierungsdaten zu Geschichten
\end{itemize}
Die Antwort wird als strukturiertes Array an den Aufrufer zurückgegeben.
\subsubsection{deleteStoryValidates}
\paragraph{Parameter} Die Funktion besitzt folgende Parameter:
\begin{table}[H]
	\begin{tabular}{|c|p{11cm}|}
		\hline
		\textbf{Parametername} & \textbf{Parameterbeschreibung} \\ \hline
		\$sid & numerischer Identifikator einer Geschichte \\ \hline
	\end{tabular}
\end{table}
\paragraph{Beschreibung} Die Funktion löscht alle Validierungen einer Geschichte. Die Funktion hat Auswirkungen auf folgende Quellen:
\begin{itemize}
	\item Tabelle mit Validierungsdaten zu Geschichten
\end{itemize}
Die Antwort wird als strukturiertes Array an den Aufrufer zurückgegeben.

\newpage
\section{deletions}
\subsection{Allgemeines} Diese Datei enthält alle Funktionen zum löschen von Daten beziehungsweise um Daten als gelöscht zu markieren.
\begin{table}[H]
	\begin{tabular}{|c|p{11cm}|}
		\hline
		\textbf{Einbindungspunkt} & inc.php \\ \hline
		\textbf{Einbindungspunkt} & inc-sub.php \\ \hline
	\end{tabular}
\end{table}
Die Datei ist nicht direkt durch den Nutzer aufrufbar, dies wird durch folgenden Code-Ausschnitt sichergestellt:
\begin{lstlisting}[language=php]
if (!defined('NICE_PROJECT')) {
	die('Permission denied.');
}
\end{lstlisting}
Der Globale Wert {\glqq NICE\_PROJECT\grqq} wird durch für den Nutzer valide Aufrufpunkte festgelegt, z.B. {\glqq api.php\grqq}.
\newpage
\subsection{Funktionen}
\subsubsection{deletePicByTokenDBWrap}
\paragraph{Parameter} Die Funktion besitzt folgende Parameter:
\begin{table}[H]
	\begin{tabular}{|c|p{11cm}|}
		\hline
		\textbf{Parametername} & \textbf{Parameterbeschreibung} \\ \hline
		\$apiToken  & alphanumerischer Identifikator eines Moduls \\ \hline
		\$token     & alphanumerischer Identifikator eines Bildes \\ \hline
		\$overwrite & Überschreibt den Konfigurationsparameter (siehe \autoref{config:direct-delete}) für direktes Löschen  \\ \hline
	\end{tabular}
\end{table}
\paragraph{Beschreibung} Die Funktion löscht ein Bild oder markiert dieses als gelöscht. Die Funktion hat Auswirkungen auf folgende Quellen:
\begin{itemize}
	\item Tabelle mit Bildern
	\item Tabelle mit Validierungsdaten zu Bildern
\end{itemize}
Die Funktion besitzt keinen Rückgabewert.
\subsubsection{deleteStoryByTokenDBWrap}
\paragraph{Parameter} Die Funktion besitzt folgende Parameter:
\begin{table}[H]
	\begin{tabular}{|c|p{11cm}|}
		\hline
		\textbf{Parametername} & \textbf{Parameterbeschreibung} \\ \hline
		\$story\_token & alphanumerischer Identifikator einer Geschichte \\ \hline
		\$overwrite    & Überschreibt den Konfigurationsparameter (siehe \autoref{config:direct-delete}) für direktes Löschen \\ \hline
	\end{tabular}
\end{table}
\paragraph{Beschreibung} Die Funktion löscht eine Geschichte oder markiert diese als gelöscht. Die Funktion hat Auswirkungen auf folgende Quellen:
\begin{itemize}
	\item Tabelle mit Geschichten
	\item Tabelle mit Validierungsdaten zu Geschichten
\end{itemize}
Die Funktion besitzt keinen Rückgabewert.
\newpage
\section{functionLib}
\subsection{Allgemeines} Diese Datei enthält alle Funktionen welche an mehreren Stellen des Projektes verwendet werden.
\begin{table}[H]
	\begin{tabular}{|c|p{11cm}|}
		\hline
		\textbf{Einbindungspunkt} & inc.php \\ \hline
		\textbf{Einbindungspunkt} & inc-sub.php \\ \hline
	\end{tabular}
\end{table}
Die Datei ist nicht direkt durch den Nutzer aufrufbar, dies wird durch folgenden Code-Ausschnitt sichergestellt:
\begin{lstlisting}[language=php]
if (!defined('NICE_PROJECT')) {
	die('Permission denied.');
}
\end{lstlisting}
Der Globale Wert {\glqq NICE\_PROJECT\grqq} wird durch für den Nutzer valide Aufrufpunkte festgelegt, z.B. {\glqq api.php\grqq}.
\newpage
\subsection{Funktionen}
\subsubsection{generateHeader}
\paragraph{Parameter} Die Funktion besitzt folgende Parameter:
\begin{table}[H]
	\begin{tabular}{|c|p{11cm}|}
		\hline
		\textbf{Parametername} & \textbf{Parameterbeschreibung} \\ \hline
		\$LOGIN      & Status des Logins \\ \hline
		\$loginpage  & Gibt an, ob Aufrufer die Login-Seite ist \\ \hline
		\$rightpages & Gibt an, ob Aufrufer die Datenschutz- oder Impressums-Seite ist \\ \hline
	\end{tabular}
\end{table}
\paragraph{Beschreibung} Die Funktion generiert die Navbar. Die Antwort wird direkt Ausgegeben.
\subsubsection{dump}
\paragraph{Parameter} Die Funktion besitzt folgende Parameter:
\begin{table}[H]
	\begin{tabular}{|c|p{11cm}|}
		\hline
		\textbf{Parametername} & \textbf{Parameterbeschreibung} \\ \hline
		\$data  & Daten für var\_dump \\ \hline
		\$level & Optionale Angabe des Debuglevels, wird mit Angabe aus Konfiguration verglichen , siehe \autoref{config:debug-level} \\ \hline
		\$dark  & Gibt an, ob Dump für dunklen Hintergrund optimiert werden soll. \\ \hline
	\end{tabular}
\end{table}
\paragraph{Beschreibung} Die Funktion dient der Ausgabe von Ergebnissen von Funktionen zu Entwicklungszwecken. Die Funktion nutzt folgende Quellen:
\begin{itemize}
	\item Konfigurationsdatei
\end{itemize}
Die Antwort wird direkt Ausgegeben.
\subsubsection{Redirect}
\paragraph{Parameter} Die Funktion besitzt folgende Parameter:
\begin{table}[H]
	\begin{tabular}{|c|p{11cm}|}
		\hline
		\textbf{Parametername} & \textbf{Parameterbeschreibung} \\ \hline
		\$url       & Leitet die Anfrage auf eine andere Seite weiter \\ \hline
		\$permanent & Legt fest, ob Weiterleitung permanent ist. \\ \hline
	\end{tabular}
\end{table}
\paragraph{Beschreibung} Die Funktion leitet den Aufrufer auf eine andere Seite weiter und beendet die Ausführung des aktuellen PHP-Scriptes. Die Antwort wird direkt Ausgegeben.
\subsubsection{generateStringHmac}
\paragraph{Parameter} Die Funktion besitzt folgende Parameter:
\begin{table}[H]
	\begin{tabular}{|c|p{11cm}|}
		\hline
		\textbf{Parametername} & \textbf{Parameterbeschreibung} \\ \hline
		\$string & zu sichernde Zeichenkette \\ \hline
	\end{tabular}
\end{table}
\paragraph{Beschreibung} Die Funktion erzeugt einen HMAC zu einer gegebenen Zeichenkette mithilfe eines Geheimnisses (\autoref{config:hmac-secret}). Die Funktion nutzt folgende Quellen:
\begin{itemize}
	\item Konfigurationsdatei
\end{itemize}
Die Antwort wird als Zeichenkette an den Aufrufer zurückgegeben.
\subsubsection{generateValidatableDataMaterial}
\paragraph{Parameter} Die Funktion besitzt folgende Parameter:
\begin{table}[H]
	\begin{tabular}{|c|p{11cm}|}
		\hline
		\textbf{Parametername} & \textbf{Parameterbeschreibung} \\ \hline
		\$tokenstring & Zeichenkette mit allen benötigten Informationen \\ \hline
		\$time        & Zeitstempel (optional) \\ \hline
	\end{tabular}
\end{table}
\paragraph{Beschreibung} Die Funktion erzeugt eine Array mit allen benötigten Daten um die daraus bestehende Anfrage zu validieren. Die Antwort wird als strukturiertes Array an den Aufrufer zurückgegeben.
\subsubsection{checkValidatableMaterial}
\paragraph{Parameter} Die Funktion besitzt folgende Parameter:
\begin{table}[H]
	\begin{tabular}{|c|p{11cm}|}
		\hline
		\textbf{Parametername} & \textbf{Parameterbeschreibung} \\ \hline
		\$data & Array mit benötigten Informationen \\ \hline
	\end{tabular}
\end{table}
\subparagraph{\$data} Das Array enthält folgende Elemente:
\begin{table}[H]
	\begin{tabular}{|c|p{11cm}|}
		\hline
		\textbf{Parametername} & \textbf{Parameterbeschreibung} \\ \hline
		token   & gesicherte Zeichenkette \\ \hline
		seccode & Sicherheitscode \\ \hline
		time    & gesicherter Zeitstempel \\ \hline
	\end{tabular}
\end{table}
\paragraph{Beschreibung} Die Funktion prüft, ob eine abgesicherte Anfrage auch durch das System erstellt wurde. Auch eine Ablaufüberprüfung anhand des übermittelten Datums erfolgt. Die Antwort wird als strukturiertes Array an den Aufrufer zurückgegeben.
\subsubsection{generateValidatableData}
\paragraph{Parameter} Die Funktion besitzt folgende Parameter:
\begin{table}[H]
	\begin{tabular}{|c|p{11cm}|}
		\hline
		\textbf{Parametername} & \textbf{Parameterbeschreibung} \\ \hline
		\$username    & Nutzername \\ \hline
		\$tokenstring & Zeichenkette mit Informationen (optional) \\ \hline
	\end{tabular}
\end{table}
\paragraph{Beschreibung} Die Funktion erzeugt eine überprüfbare Zeichenkette mit einem Nutzernamen. Die Antwort wird als strukturiertes Array an den Aufrufer zurückgegeben.
\subsubsection{generateValidatableDataTimed}
\paragraph{Parameter} Die Funktion besitzt folgende Parameter:
\begin{table}[H]
	\begin{tabular}{|c|p{11cm}|}
		\hline
		\textbf{Parametername} & \textbf{Parameterbeschreibung} \\ \hline
		\$username    & Nutzername \\ \hline
		\$tokenstring & Zeichenkette mit Informationen (optional) \\ \hline
		\$time        & Zeitstempel (optional) \\ \hline
	\end{tabular}
\end{table}
\paragraph{Beschreibung} Die Funktion erzeugt eine überprüfbare Zeichenkette mit einem Nutzernamen und einem prüfbaren Zeitstempel. Die Antwort wird als strukturiertes Array an den Aufrufer zurückgegeben.\textbf{}
\subsubsection{generateValidateableLink}
\paragraph{Parameter} Die Funktion besitzt folgende Parameter:
\begin{table}[H]
	\begin{tabular}{|c|p{11cm}|}
		\hline
		\textbf{Parametername} & \textbf{Parameterbeschreibung} \\ \hline
		\$username    & Nutzername \\ \hline
		\$targetside  & Ziel des Links auf aktueller Website \\ \hline
		\$tokenstring & Zeichenkette mit Informationen \\ \hline
	\end{tabular}
\end{table}
\paragraph{Beschreibung} Die Funktion erzeugt einen validierbaren Link. Die Antwort wird als Zeichenkette an den Aufrufer zurückgegeben.
\subsubsection{generateValidateableLinkTimed}
\paragraph{Parameter} Die Funktion besitzt folgende Parameter:
\begin{table}[H]
	\begin{tabular}{|c|p{11cm}|}
		\hline
		\textbf{Parametername} & \textbf{Parameterbeschreibung} \\ \hline
		\$username    & Nutzername \\ \hline
		\$targetside  & Ziel des Links auf aktueller Website \\ \hline
		\$tokenstring & Zeichenkette mit Informationen \\ \hline
	\end{tabular}
\end{table}
\paragraph{Beschreibung} Die Funktion erzeugt einen validierbaren Link mit Ablaufdatum. Die Antwort wird als Zeichenkette an den Aufrufer zurückgegeben.
\subsubsection{checkValidatableLink}
\paragraph{Parameter} Die Funktion besitzt folgende Parameter:
\begin{table}[H]
	\begin{tabular}{|c|p{11cm}|}
		\hline
		\textbf{Parametername} & \textbf{Parameterbeschreibung} \\ \hline
		\$data & Array mit benötigten Informationen \\ \hline
	\end{tabular}
\end{table}
\subparagraph{\$json}Das Array enthält folgende Elemente:
\begin{table}[H]
	\begin{tabular}{|c|p{11cm}|}
		\hline
		\textbf{Parametername} & \textbf{Parameterbeschreibung} \\ \hline
		token    & gesicherte Zeichenkette \\ \hline
		username & gesicherter Nutzername \\ \hline
		seccode  & Sicherheitscode \\ \hline
		time     & gesicherter Zeitstempel \\ \hline
	\end{tabular}
\end{table}
\paragraph{Beschreibung} Die Funktion prüft die Daten eines gesicherten Links. Die Antwort wird als Boolean an den Aufrufer zurückgegeben.
\subsubsection{generateRandomString}
\paragraph{Parameter} Die Funktion besitzt folgende Parameter:
\begin{table}[H]
	\begin{tabular}{|c|p{11cm}|}
		\hline
		\textbf{Parametername} & \textbf{Parameterbeschreibung} \\ \hline
		\$length  & Länge der durch Zufall generierten Zeichenkette \\ \hline
		\$special & Schaltet Sonderzeichen frei \\ \hline
	\end{tabular}
\end{table}
\paragraph{Beschreibung} Die Funktion erzeugt eine durch Zufall generierte Zeichenkette. Die Antwort wird als Zeichenkette an den Aufrufer zurückgegeben.
\subsubsection{permissionDenied}
\paragraph{Parameter} Die Funktion besitzt folgende Parameter:
\begin{table}[H]
	\begin{tabular}{|c|p{11cm}|}
		\hline
		\textbf{Parametername} & \textbf{Parameterbeschreibung} \\ \hline
		\$string & Angabe eines Grundes (optional) \\ \hline
	\end{tabular}
\end{table}
\paragraph{Beschreibung} Die Funktion verhindert eine weitere Ausführung des Codes und bricht die Ausführung ab. Es kann eine Begründung für den Abbruch der Ausführung angegeben werden. Die Antwort wird direkt Ausgegeben.
\subsubsection{grantPermission}
\paragraph{Parameter} Die Funktion besitzt folgende Parameter:
\begin{table}[H]
	\begin{tabular}{|c|p{11cm}|}
		\hline
		\textbf{Parametername} & \textbf{Parameterbeschreibung} \\ \hline
		\$string & Zeichenkette mit Begründung (optional) \\ \hline
	\end{tabular}
\end{table}
\paragraph{Beschreibung} Die Funktion erlaubt einen Zugriff durch eine API und sendet den entsprechenden HTTP-Statuscode für Cross-Site XHR-Requests zurück. Die Antwort wird direkt Ausgegeben.
\subsubsection{ServerError}
\paragraph{Parameter} Die Funktion besitzt keine Parameter.
\paragraph{Beschreibung} Die Funktion gibt den entsprechenden HTTP-Statuscode zurück, wenn ein API-Request falsche Parameter nutzt. Die Antwort wird direkt Ausgegeben.
\subsubsection{generateHeaderTags}
\paragraph{Parameter} Die Funktion besitzt folgende Parameter:
\begin{table}[H]
	\begin{tabular}{|c|p{11cm}|}
		\hline
		\textbf{Parametername} & \textbf{Parameterbeschreibung} \\ \hline
		\$additional & Optional ein zu bindende Daten \\ \hline
	\end{tabular}
\end{table}
\subparagraph{\$additional}Das Array enthält Einträge mit folgenden Elementen:
\begin{table}[H]
	\begin{tabular}{|c|p{11cm}|}
		\hline
		\textbf{Parametername} & \textbf{Parameterbeschreibung} \\ \hline
		type    & Typ der Datei ({\glqq link\grqq} für zum Beispiel CSS-Files oder {\glqq script\grqq} für zum Beispiel javaScript-Files ) \\ \hline
		rel     & Gibt den Rel-Tag eines HTML-Link Elements an \\ \hline
		href    & Gibt die Position der Datei an \\ \hline
		hrefmin & Gibt die minimierte Datei an \\ \hline
		typeval & Gibt den Typ der Datei an (zum Beispiel: {\glqq text/javascript\grqq}) \\ \hline
	\end{tabular}
\end{table}
\paragraph{Beschreibung} Die Funktion dient der Generierung von HTML-Head Elementen und der zentralen Pflege der eingebundenen Dateien. Die Antwort wird direkt Ausgegeben.
\paragraph{Besonderheiten} Die Einbindung zusätzlicher Dateien erfolgt in der im Array angegebenen Reihenfolge mittels:
\begin{lstlisting}[language=php]
foreach ($additional as $line) {
	switch ($line['type']) {
		case 'link':
			if (isset($line['typeval']) === false || $line['typeval'] === "") {
				echo '<link rel="' . $line['rel'] . '" href="' . $line['href'] . '" >';
			} else {
				echo '<link rel="' . $line['rel'] . '" type="' . $line['typeval'] . '" href="' . $line['href'] . '" >';
			}
			break;
		case 'script':
			echo '<script type="' . $line['typeval'] . '" src="' . $line['href'] . '" ></script>';
			break;
	}
}
\end{lstlisting}
\subsubsection{checkRoleID}
\paragraph{Parameter} Die Funktion besitzt folgende Parameter:
\begin{table}[H]
	\begin{tabular}{|c|p{11cm}|}
		\hline
		\textbf{Parametername} & \textbf{Parameterbeschreibung} \\ \hline
		\$rid & Identifikator einer Rolle \\ \hline
	\end{tabular}
\end{table}
\paragraph{Beschreibung} Die Funktion prüft die Existenz eines Identifikators einer Rolle. Die Funktion nutzt folgende Quellen:
\begin{itemize}
	\item Tabelle mit Rollen
\end{itemize}
Die Antwort wird als Boolean an den Aufrufer zurückgegeben.
\subsubsection{checkApiTokenExists}
\paragraph{Parameter} Die Funktion besitzt folgende Parameter:
\begin{table}[H]
	\begin{tabular}{|c|p{11cm}|}
		\hline
		\textbf{Parametername} & \textbf{Parameterbeschreibung} \\ \hline
		\$token & alphanumerischer Identifikator eines Moduls \\ \hline
	\end{tabular}
\end{table}
\paragraph{Beschreibung} Die Funktion prüft die Existenz eines Moduls. Die Funktion nutzt folgende Quellen:
\begin{itemize}
	\item Tabelle mit API-Token
\end{itemize}
Die Antwort wird als Boolean an den Aufrufer zurückgegeben.
\subsubsection{decode\_json}
\paragraph{Parameter} Die Funktion besitzt folgende Parameter:
\begin{table}[H]
	\begin{tabular}{|c|p{11cm}|}
		\hline
		\textbf{Parametername} & \textbf{Parameterbeschreibung} \\ \hline
		\$string & Eingabe des JSON als Zeichenkette \\ \hline
	\end{tabular}
\end{table}
\paragraph{Beschreibung} Die Funktion decodiert Daten im JSON-Format in PHP-Arrays.Die Antwort wird als strukturiertes Array an den Aufrufer zurückgegeben.
\subsubsection{checkMailAddress}
\paragraph{Parameter} Die Funktion besitzt folgende Parameter:
\begin{table}[H]
	\begin{tabular}{|c|p{11cm}|}
		\hline
		\textbf{Parametername} & \textbf{Parameterbeschreibung} \\ \hline
		\$email & Mailadresse als Zeichenkette \\ \hline
	\end{tabular}
\end{table}
\paragraph{Beschreibung} Die Funktion prüft, ob eine gegebene Zeichenkette eine E-Mailadresse ist. Die Antwort wird als Zeichenkette an den Aufrufer zurückgegeben (sofern es eine E-Mailadresse ist).
\subsubsection{checkReasonExists}
\paragraph{Parameter} Die Funktion besitzt folgende Parameter:
\begin{table}[H]
	\begin{tabular}{|c|p{11cm}|}
		\hline
		\textbf{Parametername} & \textbf{Parameterbeschreibung} \\ \hline
		\$reason & Begründung für Rangpunkte \\ \hline
	\end{tabular}
\end{table}
\paragraph{Beschreibung} Die Funktion prüft ob eine Begründung für Rangpunkte bereits existiert.
\begin{itemize}
	\item Tabelle mit Begründungen für Rangpunkte.
\end{itemize}
Es wird eine Antwort an den Aufrufer zurückgegeben.
\subsubsection{createThumbnail}
\paragraph{Parameter} Die Funktion besitzt folgende Parameter:
\begin{table}[H]
	\begin{tabular}{|c|p{11cm}|}
		\hline
		\textbf{Parametername} & \textbf{Parameterbeschreibung} \\ \hline
		\$picture & Pfad zu einem Bild \\ \hline
	\end{tabular}
\end{table}
\paragraph{Beschreibung} Die Funktion erstellt aus einem Bild ein Base64-Codiertes Vorschaubild. Die Antwort wird als Zeichenkette an den Aufrufer zurückgegeben.
\subsubsection{checkPermission}
\paragraph{Parameter} Die Funktion besitzt folgende Parameter:
\begin{table}[H]
	\begin{tabular}{|c|p{11cm}|}
		\hline
		\textbf{Parametername} & \textbf{Parameterbeschreibung} \\ \hline
		\$requiredPermission & benötigte Berechtigung \\ \hline
	\end{tabular}
\end{table}
\paragraph{Beschreibung} Die Funktion prüft, ob der aktuelle Benutzer die benötigte Berechtigung besitzt.
\subsubsection{deletePictureReferences}
\paragraph{Parameter} Die Funktion besitzt folgende Parameter:
\begin{table}[H]
	\begin{tabular}{|c|p{11cm}|}
		\hline
		\textbf{Parametername} & \textbf{Parameterbeschreibung} \\ \hline
		\$picToken  & alphanumerischer Identifikator eines Bildes \\ \hline
		\$overwrite & Überschreibt den Direkt-Löschen-Konfigurationsparameter (siehe \autoref{config:direct-delete}) des Zielmoduls \\ \hline
	\end{tabular}
\end{table}
\paragraph{Beschreibung} Die Funktion löscht alle Referenzen auf ein Bild oder markiert diese als gelöscht in allen Modulen. 
\subsubsection{restorePictureReferences}
\paragraph{Parameter} Die Funktion besitzt folgende Parameter:
\begin{table}[H]
	\begin{tabular}{|c|p{11cm}|}
		\hline
		\textbf{Parametername} & \textbf{Parameterbeschreibung} \\ \hline
		\$picToken & alphanumerischer Identifikator eines Bildes \\ \hline
	\end{tabular}
\end{table}
\paragraph{Beschreibung} Die Funktion stellt alle als gelöscht markierten Referenzen eines Bildes in allen Modulen wieder her.
\subsubsection{ApiCall}
\paragraph{Parameter} Die Funktion besitzt folgende Parameter:
\begin{table}[H]
	\begin{tabular}{|c|p{11cm}|}
		\hline
		\textbf{Parametername} & \textbf{Parameterbeschreibung} \\ \hline
		\$params       & Array mit Aufrufparametern (Key-Value-Paare) \\ \hline
		\$type         & Type des Aufrufs (im Allgemeinen dreistellige Zeichenkette) \\ \hline
		\$token        & alphanumerischer Identifikationstoken der fremden API \\ \hline
		\$url          & URI der fremden API \\ \hline
		\$file\_upload & Schaltet den Upload von Daten frei \\ \hline
	\end{tabular}
\end{table}
\paragraph{Beschreibung} Die Funktion ruft eine API eines Moduls auf und übergibt dieser eine Anfrage. Die Antwort der API wird im Allgemeinen ignoriert.
\subsubsection{AktivationRemoteUser}
\paragraph{Parameter} Die Funktion besitzt folgende Parameter:
\begin{table}[H]
	\begin{tabular}{|c|p{11cm}|}
		\hline
		\textbf{Parametername} & \textbf{Parameterbeschreibung} \\ \hline
		\$uri            & URI der fremden API \\ \hline
		\$token          & alphanumerischer Identifikator der fremden API \\ \hline
		\$username       & Nutzername \\ \hline
		\$AktivationSate & Status der Freischaltung des Nutzers \\ \hline
	\end{tabular}
\end{table}
\paragraph{Beschreibung} Die Funktion schaltet einen Nutzer in einem Modul frei oder sperrt diesen.
\subsubsection{hashString}
\paragraph{Parameter} Die Funktion besitzt folgende Parameter:
\begin{table}[H]
	\begin{tabular}{|c|p{11cm}|}
		\hline
		\textbf{Parametername} & \textbf{Parameterbeschreibung} \\ \hline
		\$string & Zeichenkette \\ \hline
	\end{tabular}
\end{table}
\paragraph{Beschreibung} Die Funktion erstellt einen Hash einer Zeichenkette. Die Antwort wird als Zeichenkette an den Aufrufer zurückgegeben.
\subsubsection{PasswordResetViaMail}
\paragraph{Parameter} Die Funktion besitzt folgende Parameter:
\begin{table}[H]
	\begin{tabular}{|c|p{11cm}|}
		\hline
		\textbf{Parametername} & \textbf{Parameterbeschreibung} \\ \hline
		\$username & Nutzername \\ \hline
		\$email    & E-Mailadresse \\ \hline
	\end{tabular}
\end{table}
\paragraph{Beschreibung} Die Funktion sendet einem Nutzer eine E-Mail zum zurücksetzen seines Passwortes.
\subsubsection{checkRankID}
\paragraph{Parameter} Die Funktion besitzt folgende Parameter:
\begin{table}[H]
	\begin{tabular}{|c|p{11cm}|}
		\hline
		\textbf{Parametername} & \textbf{Parameterbeschreibung} \\ \hline
		\$rid & Identifikator eines Ranges \\ \hline
	\end{tabular}
\end{table}
\paragraph{Beschreibung} Die Funktion prüft, ob eine gegebener Identifikator eines Ranges existiert. Die Funktion nutzt folgende Quellen:
\begin{itemize}
	\item Tabelle mit Rängen
\end{itemize}
Es wird eine Antwort an den Aufrufer zurückgegeben.
\subsubsection{deleteStoryReference}
\paragraph{Parameter} Die Funktion besitzt folgende Parameter:
\begin{table}[H]
	\begin{tabular}{|c|p{11cm}|}
		\hline
		\textbf{Parametername} & \textbf{Parameterbeschreibung} \\ \hline
		\$storyToken & alphanumerischer Identifikator einer Geschichte \\ \hline
		\$overwrite  & Überschreibt den Direkt-Löschen-Konfigurationsparameter (siehe \autoref{config:direct-delete}) \\ \hline
	\end{tabular}
\end{table}
\paragraph{Beschreibung} Die Funktion löscht alle Referenzen einer Geschichte in allen Modulen oder markiert diese als gelöscht.
\subsubsection{restoreStoryReference}
\paragraph{Parameter} Die Funktion besitzt folgende Parameter:
\begin{table}[H]
	\begin{tabular}{|c|p{11cm}|}
		\hline
		\textbf{Parametername} & \textbf{Parameterbeschreibung} \\ \hline
		\$storyToken & alphanumerischer Identifikator einer Geschichte \\ \hline
	\end{tabular}
\end{table}
\paragraph{Beschreibung} Die Funktion stellt alle Referenzen einer Geschichte in allen Modulen wieder her.
\subsubsection{isStaff}
\paragraph{Parameter} Die Funktion besitzt folgende Parameter:
\begin{table}[H]
	\begin{tabular}{|c|p{11cm}|}
		\hline
		\textbf{Parametername} & \textbf{Parameterbeschreibung} \\ \hline
		\$name & Nutzername des zu prüfenden Nutzers \\ \hline
	\end{tabular}
\end{table}
\paragraph{Beschreibung} Die Funktion prüft, ob eine gegebener Nutzer Mitarbeiter ist. Die Funktion nutzt folgende Quellen:
\begin{itemize}
	\item Tabelle mit Nutzerdaten
	\item Tabelle mit Rollen
\end{itemize}
Der Rückgabewert ist ein Boolean.
\subsubsection{checkModulModuleRights}
\paragraph{Parameter} Die Funktion besitzt folgende Parameter:
\begin{table}[H]
	\begin{tabular}{|c|p{11cm}|}
		\hline
		\textbf{Parametername} & \textbf{Parameterbeschreibung} \\ \hline
		\$requiredPermission & benötigte Berechtigung \\ \hline
		\$return             & Legt fest ob ein Rückgabewert existiert \\ \hline
	\end{tabular}
\end{table}
\paragraph{Beschreibung} Die Funktion prüft, ob der aktuelle Benutzer die benötigte Berechtigung auf mindestens einem Modul besitzt.
\subsubsection{countRightsOnModules}
\paragraph{Parameter} Die Funktion besitzt folgende Parameter:
\begin{table}[H]
	\begin{tabular}{|c|p{11cm}|}
		\hline
		\textbf{Parametername} & \textbf{Parameterbeschreibung} \\ \hline
		\$user & Nutzername (optional) \\ \hline
	\end{tabular}
\end{table}
\paragraph{Beschreibung} Die Funktion zählt alle Module auf welchen der Benutzer mindestens Mitarbeiterrechte besitzt.
\subsubsection{getUserIp}
\paragraph{Parameter} Die Funktion besitzt keine Parameter.
\paragraph{Beschreibung} Die Funktion bestimmt die IP-Adresse des Aufrufenden Nutzers. Es findet bei dieser Funktion ein Abruf von Daten aus {\glqq COSP\grqq} statt. Die Antwort wird als Zeichenkette an den Aufrufer zurückgegeben.
\subsubsection{checkMailAddressExistent}
\paragraph{Parameter} Die Funktion besitzt folgende Parameter:
\begin{table}[H]
	\begin{tabular}{|c|p{11cm}|}
		\hline
		\textbf{Parametername} & \textbf{Parameterbeschreibung} \\ \hline
		\$email & Mailadresse \\ \hline
	\end{tabular}
\end{table}
\paragraph{Beschreibung} Die Funktion prüft, ob eine Mailadresse bereits verwendet wird.
\newpage
\section{inc-sub}
\subsection{Allgemeines} Diese Datei Einbindungen aller benötigten Dateien für die Ausführung der Anwendung d.
Die Datei ist nicht direkt durch den Nutzer aufrufbar, dies wird durch folgenden Code-Ausschnitt sichergestellt:
\begin{lstlisting}[language=php]
if (!defined('NICE_PROJECT')) {
	die('Permission denied.');
}
\end{lstlisting}
Der Globale Wert {\glqq NICE\_PROJECT\grqq} wird durch für den Nutzer valide Aufrufpunkte festgelegt, z.B. {\glqq api.php\grqq}.
\newpage
\subsection{Einbindungen}
\subsubsection{Grundlegendes}
Zu Anfangs wird zunächst der HTTP-Content-Typ festgelegt:
\begin{lstlisting}[language=php]
header('Content-Type: text/html; charset=utf-8');
\end{lstlisting}
Nachfolgend zu sehender Code-Block bindet alle benötigten Dateien in korrekter Reihenfolge ein. Beim Einbinden neuer Dateien, sind diese stets an das Ende zu schreiben, außer die Dateien sind Umstrukturierungen bereits existenten Dateien.
\begin{lstlisting}[language=php]
require_once '../bin/config.php';
require_once '../bin/MailTemplates.php';
require_once '../bin/settings.php';
require_once '../bin/database/inc-db-sub.php';
require_once '../bin/deletions.php';
require_once '../bin/functionLib.php';
require_once '../bin/authSystem.php';
require_once '../bin/SessionValues.php';
require_once '../bin/api-functions.php';
require_once '../bin/uapi-functions.php';
require_once '../bin/mapi-functions.php';
require_once '../bin/mailer.php';
require_once '../bin/statistic-calc.php';
require_once '../bin/captcha.php';
\end{lstlisting}
Des weiteren wird hier die für den Nutzer sichtbare Fehlerausgabe anhand des Debug-Konfigurationsparameters festgelegt:
\begin{lstlisting}[language=php]
if (config::$DEBUG === true) {
	error_reporting(E_ALL);
	ini_set('display_errors', 1);
	ini_set('display_startup_errors', 1);
}
\end{lstlisting}
\subsubsection{Besonderheit}
Die Einbindungen sind immer mit {\glqq ../\grqq} anzufangen, da Sie für Subordner des Hauptordners gedacht sind.

\newpage
\section{inc}
\input{Kapitel/Files/inc}
\newpage
\section{mailer}
\subsection{Allgemeines} Diese Datei enthält alle Funktionen zum Senden einer E-Mail.
\begin{table}[H]
	\begin{tabular}{|c|p{11cm}|}
		\hline
		\textbf{Einbindungspunkt} & inc.php \\ \hline
		\textbf{Einbindungspunkt} & inc-sub.php \\ \hline
	\end{tabular}
\end{table}
Die Datei ist nicht direkt durch den Nutzer aufrufbar, dies wird durch folgenden Code-Ausschnitt sichergestellt:
\begin{lstlisting}[language=php]
if (!defined('NICE_PROJECT')) {
	die('Permission denied.');
}
\end{lstlisting}
Der Globale Wert {\glqq NICE\_PROJECT\grqq} wird durch für den Nutzer valide Aufrufpunkte festgelegt, z.B. {\glqq api.php\grqq}.
\subsection{Funktionen}
\subsubsection{sendMail}
\paragraph{Parameter} Die Funktion besitzt folgende Parameter:
\begin{table}[H]
	\begin{tabular}{|c|p{11cm}|}
		\hline
		\textbf{Parametername} & \textbf{Parameterbeschreibung} \\ \hline
		\$receiver        & E-Mailadresse des Empfängers \\ \hline
		\$title           & Betreff der E-Mail \\ \hline
		\$msg             & Inhalt der E-Mail \\ \hline
		\$html\_flag      & Legt fest, ob E-Mail HTML Inhalt besitzt \\ \hline
		\$reply           & Antwortadresse der E-Mail \\ \hline
		\$additionalParam & Legt fest, ob zusätzliche Mail-Header aus der Konfigurationsdatei (siehe: \autoref{config:additional-mail-header}) verwendet werden sollen \\ \hline
	\end{tabular}
\end{table}
\paragraph{Beschreibung} Die Funktion sendet eine E-Mail an die entsprechende Adresse.

\newpage
\section{MailTemplates}
\subsection{Allgemeines} Diese Datei enthält eine Klasse mit allen E-Mail-Templates.
\begin{table}[H]
	\begin{tabular}{|c|p{11cm}|}
		\hline
		\textbf{Einbindungspunkt} & inc.php \\ \hline
		\textbf{Einbindungspunkt} & inc-sub.php \\ \hline
	\end{tabular}
\end{table}
Die Datei ist nicht direkt durch den Nutzer aufrufbar, dies wird durch folgenden Code-Ausschnitt sichergestellt:
\begin{lstlisting}[language=php]
if (!defined('NICE_PROJECT')) {
	die('Permission denied.');
}
\end{lstlisting}
Der Globale Wert {\glqq NICE\_PROJECT\grqq} wird durch für den Nutzer valide Aufrufpunkte festgelegt, z.B. {\glqq api.php\grqq}.
\newpage
\subsection{Klasse}
\subsubsection{MailTemplates} Diese Klasse enthält alle im Projekt verwendeten E-Mail-Templates. Alle folgenden Funktionen geben den Inhalt der Nachricht als Zeichenkette zurück:
\paragraph{RegisterMail}
\subparagraph{Parameter} Die Funktion besitzt folgende Parameter:
\begin{table}[H]
	\begin{tabular}{|c|p{11cm}|}
		\hline
		\textbf{Parametername} & \textbf{Parameterbeschreibung} \\ \hline
		\$Link     & Link zum Validieren der Mailadresse \\ \hline
		\$Username & Nutzername \\ \hline
	\end{tabular}
\end{table}
\subparagraph{Beschreibung} Die Funktion gibt ein instanziiertes Template zur Validierung der bei Registrierung gegebenen Mailadresse zurück.
\paragraph{ZentralRegisterMail}
\subparagraph{Parameter} Die Funktion besitzt folgende Parameter:
\begin{table}[H]
	\begin{tabular}{|c|p{11cm}|}
		\hline
		\textbf{Parametername} & \textbf{Parameterbeschreibung} \\ \hline
		\$Username & Nutzername \\ \hline
	\end{tabular}
\end{table}
\subparagraph{Beschreibung} Die Funktion gibt ein instanziiertes Template zur Benachrichtigung der Administrators und Mitarbeiter über einen neuen Nutzer zurück.
\paragraph{ResetPasswordByMail}
\subparagraph{Parameter} Die Funktion besitzt folgende Parameter:
\begin{table}[H]
	\begin{tabular}{|c|p{11cm}|}
		\hline
		\textbf{Parametername} & \textbf{Parameterbeschreibung} \\ \hline
		\$Link     & Link zum Zurücksetzen des Passwortes \\ \hline
		\$Username & Nutzername \\ \hline
	\end{tabular}
\end{table}
\subparagraph{Beschreibung} Die Funktion gibt ein instanziiertes Template zum Passwort ändern durch den Nutzer bei vergessen Passwort zurück.
\paragraph{MailChangedOldAddress}
\subparagraph{Parameter} Die Funktion besitzt folgende Parameter:
\begin{table}[H]
	\begin{tabular}{|c|p{11cm}|}
		\hline
		\textbf{Parametername} & \textbf{Parameterbeschreibung} \\ \hline
		\$newmail  & Neue E-Mailadresse des Nutzers \\ \hline
		\$Username & Nutzername \\ \hline
	\end{tabular}
\end{table}
\subparagraph{Beschreibung} Die Funktion gibt ein instanziiertes Template zur Information des Nutzers über eine geänderte E-Mailadresse zurück
\paragraph{MailChangedNewAddress}
\subparagraph{Parameter} Die Funktion besitzt folgende Parameter:
\begin{table}[H]
	\begin{tabular}{|c|p{11cm}|}
		\hline
		\textbf{Parametername} & \textbf{Parameterbeschreibung} \\ \hline
		\$Link     & Link zum Validieren der neuen Mailadresse \\ \hline
		\$Username & Nutzername \\ \hline
	\end{tabular}
\end{table}
\subparagraph{Beschreibung} Die Funktion gibt ein instanziiertes Template zur Validierung der neuen E-Mailadresse des Nutzers zurück.
\paragraph{ZentralMailChanged}
\subparagraph{Parameter} Die Funktion besitzt folgende Parameter:
\begin{table}[H]
	\begin{tabular}{|c|p{11cm}|}
		\hline
		\textbf{Parametername} & \textbf{Parameterbeschreibung} \\ \hline
		\$Username & Nutzername \\ \hline
	\end{tabular}
\end{table}
\subparagraph{Beschreibung} Die Funktion gibt ein instanziiertes Template zur Benachrichtigung der Administrators und Mitarbeiter über eine geänderte Mailadresse eines Nutzers zurück.
\paragraph{ZentralMailContact}
\subparagraph{Parameter} Die Funktion besitzt folgende Parameter:
\begin{table}[H]
	\begin{tabular}{|c|p{11cm}|}
		\hline
		\textbf{Parametername} & \textbf{Parameterbeschreibung} \\ \hline
		\$msg      & Nachricht des Nutzers \\ \hline
		\$Username & Nutzername \\ \hline
	\end{tabular}
\end{table}
\subparagraph{Beschreibung} Die Funktion gibt ein instanziiertes Template zur Benachrichtigung eines Mitarbeiters oder Administrators über eine Nachricht eines Nutzers.
\newpage
\section{mapi-functions}
\subsection{Allgemeines} Diese Datei enthält alle Funktionen, welche durch die Management-API aufgerufen werden.
\begin{table}[H]
	\begin{tabular}{|c|p{11cm}|}
		\hline
		\textbf{Einbindungspunkt} & inc.php \\ \hline
		\textbf{Einbindungspunkt} & inc-sub.php \\ \hline
	\end{tabular}
\end{table}
Die Datei ist nicht direkt durch den Nutzer aufrufbar, dies wird durch folgenden Code-Ausschnitt sichergestellt:
\begin{lstlisting}[language=php]
if (!defined('NICE_PROJECT')) {
	die('Permission denied.');
}
\end{lstlisting}
Der Globale Wert {\glqq NICE\_PROJECT\grqq} wird durch für den Nutzer valide Aufrufpunkte festgelegt, z.B. {\glqq api.php\grqq}.
\newpage
\subsection{Funktionen}
\subsubsection{changeRole}
\paragraph{Parameter} Die Funktion besitzt folgende Parameter:
\begin{table}[H]
	\begin{tabular}{|c|p{11cm}|}
		\hline
		\textbf{Parametername} & \textbf{Parameterbeschreibung} \\ \hline
		\$json & Array mit Informationen \\ \hline
	\end{tabular}
\end{table}
\subparagraph{\$json}Das Array enthält folgende Elemente:
\begin{table}[H]
	\begin{tabular}{|c|p{11cm}|}
		\hline
		\textbf{Parametername} & \textbf{Parameterbeschreibung} \\ \hline
		username & Nutzername \\ \hline
		role     & Identifikator einer Rolle \\ \hline
	\end{tabular}
\end{table}
\paragraph{Beschreibung} Die Funktion ändert die Rolle eines Nutzers. Die Funktion hat Auswirkungen auf folgende Quellen:
\begin{itemize}
	\item Tabelle mit Nutzerdaten
\end{itemize}
Die Antwort wird als strukturiertes Array an den Aufrufer zurückgegeben.
\subsubsection{generateFalse}
\paragraph{Parameter} Die Funktion besitzt folgende Parameter:
\begin{table}[H]
	\begin{tabular}{|c|p{11cm}|}
		\hline
		\textbf{Parametername} & \textbf{Parameterbeschreibung} \\ \hline
		\$json & Array oder Zeichenkette als Nutzlast der Antwort \\ \hline
	\end{tabular}
\end{table}
\paragraph{Beschreibung} Die Funktion erstellt eine generische Nachricht, welche einen nicht erfolgreichen API-Request signalisiert. Die Antwort wird als strukturiertes Array an den Aufrufer zurückgegeben.
\subsubsection{addNewRole}
\paragraph{Parameter} Die Funktion besitzt folgende Parameter:
\begin{table}[H]
	\begin{tabular}{|c|p{11cm}|}
		\hline
		\textbf{Parametername} & \textbf{Parameterbeschreibung} \\ \hline
		\$json & Array mit Informationen \\ \hline
	\end{tabular}
\end{table}
\subparagraph{\$json}Das Array enthält folgende Elemente:
\begin{table}[H]
	\begin{tabular}{|c|p{11cm}|}
		\hline
		\textbf{Parametername} & \textbf{Parameterbeschreibung} \\ \hline
		name  & Name der neuen Rolle \\ \hline
		value & Wert der neuen Rolle \\ \hline
	\end{tabular}
\end{table}
\paragraph{Beschreibung} Die Funktion fügt eine neue Rolle dem System hinzu. Die Funktion hat Auswirkungen auf folgende Quellen:
\begin{itemize}
	\item Tabelle mit Rollen
\end{itemize}
Die Antwort wird als strukturiertes Array an den Aufrufer zurückgegeben.
\subsubsection{generateSuccessMAPI}
\paragraph{Parameter} Die Funktion besitzt folgende Parameter:
\begin{table}[H]
	\begin{tabular}{|c|p{11cm}|}
		\hline
		\textbf{Parametername} & \textbf{Parameterbeschreibung} \\ \hline
		\$json & Array oder Zeichenkette als Nutzlast der Antwort \\ \hline
	\end{tabular}
\end{table}
\paragraph{Beschreibung} Die Funktion erstellt eine generische Nachricht, welche einen erfolgreichen API-Request signalisiert. Die Antwort wird als strukturiertes Array an den Aufrufer zurückgegeben.
\subsubsection{saveEditRoleMapi}
\paragraph{Parameter} Die Funktion besitzt folgende Parameter:
\begin{table}[H]
	\begin{tabular}{|c|p{11cm}|}
		\hline
		\textbf{Parametername} & \textbf{Parameterbeschreibung} \\ \hline
		\$json & Array mit Informationen \\ \hline
	\end{tabular}
\end{table}
\subparagraph{\$json}Das Array enthält folgende Elemente:
\begin{table}[H]
	\begin{tabular}{|c|p{11cm}|}
		\hline
		\textbf{Parametername} & \textbf{Parameterbeschreibung} \\ \hline
		id    & Identifikator einer Rolle \\ \hline
		value & Wert der Rolle \\ \hline
		name  & Name der Rolle \\ \hline
	\end{tabular}
\end{table}
\paragraph{Beschreibung} Die Funktion aktualisiert eine Rolle. Die Funktion hat Auswirkungen auf folgende Quellen:
\begin{itemize}
	\item Tabelle mit Rollen
\end{itemize}
Die Antwort wird als strukturiertes Array an den Aufrufer zurückgegeben.
\subsubsection{deleteRoleMapi}
\paragraph{Parameter} Die Funktion besitzt folgende Parameter:
\begin{table}[H]
	\begin{tabular}{|c|p{11cm}|}
		\hline
		\textbf{Parametername} & \textbf{Parameterbeschreibung} \\ \hline
		\$id & Identifikator einer Rolle \\ \hline
	\end{tabular}
\end{table}
\paragraph{Beschreibung} Die Funktion löscht eine Rolle. Die Funktion hat Auswirkungen auf folgende Quellen:
\begin{itemize}
	\item Tabelle mit Rollen
\end{itemize}
Die Antwort wird als strukturiertes Array an den Aufrufer zurückgegeben.
\subsubsection{changeUserPasswordMapi}
\paragraph{Parameter} Die Funktion besitzt folgende Parameter:
\begin{table}[H]
	\begin{tabular}{|c|p{11cm}|}
		\hline
		\textbf{Parametername} & \textbf{Parameterbeschreibung} \\ \hline
		\$json & Array mit Informationen \\ \hline
	\end{tabular}
\end{table}
\subparagraph{\$json}Das Array enthält folgende Elemente:
\begin{table}[H]
	\begin{tabular}{|c|p{11cm}|}
		\hline
		\textbf{Parametername} & \textbf{Parameterbeschreibung} \\ \hline
		id   & Identifikator eines Nutzers \\ \hline
		pwd1 & Inhalt des Passwortfeldes 1 \\ \hline
		pwd2 & Inhalt des Passwortfeldes 2 \\ \hline
	\end{tabular}
\end{table}
\paragraph{Beschreibung} Die Funktion ändert das Passwort eines Nutzers. Die Funktion hat Auswirkungen auf folgende Quellen:
\begin{itemize}
	\item Tabelle mit Nutzerdaten
\end{itemize}
Die Antwort wird als strukturiertes Array an den Aufrufer zurückgegeben.
\subsubsection{enableUserMAPI}
\paragraph{Parameter} Die Funktion besitzt folgende Parameter:
\begin{table}[H]
	\begin{tabular}{|c|p{11cm}|}
		\hline
		\textbf{Parametername} & \textbf{Parameterbeschreibung} \\ \hline
		\$uid & Identifikator eines Nutzers \\ \hline
	\end{tabular}
\end{table}
\paragraph{Beschreibung} Die Funktion schaltet einen Nutzer frei oder sperrt diesen (anhängig vom aktuellen Status). Die Funktion hat Auswirkungen auf folgende Quellen:
\begin{itemize}
	\item Tabelle mit Nutzerdaten
\end{itemize}
Die Antwort wird als strukturiertes Array an den Aufrufer zurückgegeben.
\subsubsection{resetUserPasswordMAPI}
\paragraph{Parameter} Die Funktion besitzt folgende Parameter:
\begin{table}[H]
	\begin{tabular}{|c|p{11cm}|}
		\hline
		\textbf{Parametername} & \textbf{Parameterbeschreibung} \\ \hline
		\$uid & Identifikator eines Nutzers \\ \hline
	\end{tabular}
\end{table}
\paragraph{Beschreibung} Die Funktion sendet eine Mail zum zurücksetzen des Passwortes. Die Funktion nutzt folgende Quellen:
\begin{itemize}
	\item Tabelle mit Nutzerdaten
\end{itemize}
Die Antwort wird als strukturiertes Array an den Aufrufer zurückgegeben.
\subsubsection{addNewRank}
\paragraph{Parameter} Die Funktion besitzt folgende Parameter:
\begin{table}[H]
	\begin{tabular}{|c|p{11cm}|}
		\hline
		\textbf{Parametername} & \textbf{Parameterbeschreibung} \\ \hline
		\$json & Array mit Informationen \\ \hline
	\end{tabular}
\end{table}
\subparagraph{\$json}Das Array enthält folgende Elemente:
\begin{table}[H]
	\begin{tabular}{|c|p{11cm}|}
		\hline
		\textbf{Parametername} & \textbf{Parameterbeschreibung} \\ \hline
		value & Wert eines Ranges \\ \hline
		name  & Name eines Ranges \\ \hline
	\end{tabular}
\end{table}
\paragraph{Beschreibung} Die Funktion fügt der Anwendung einen neuen Rang hinzu. Die Funktion hat Auswirkungen auf folgende Quellen:
\begin{itemize}
	\item Tabelle mit Rängen
\end{itemize}
Die Antwort wird als strukturiertes Array an den Aufrufer zurückgegeben.
\subsubsection{deleteRankMapi}
\paragraph{Parameter} Die Funktion besitzt folgende Parameter:
\begin{table}[H]
	\begin{tabular}{|c|p{11cm}|}
		\hline
		\textbf{Parametername} & \textbf{Parameterbeschreibung} \\ \hline
		\$id & Identifikator eines Ranges \\ \hline
	\end{tabular}
\end{table}
\paragraph{Beschreibung} Die Funktion löscht einen Rang. Die Funktion hat Auswirkungen auf folgende Quellen:
\begin{itemize}
	\item Tabelle mit Rängen
\end{itemize}
Die Antwort wird als strukturiertes Array an den Aufrufer zurückgegeben.
\subsubsection{saveEditRankMapi}
\paragraph{Parameter} Die Funktion besitzt folgende Parameter:
\begin{table}[H]
	\begin{tabular}{|c|p{11cm}|}
		\hline
		\textbf{Parametername} & \textbf{Parameterbeschreibung} \\ \hline
		\$json & Array mit Informationen \\ \hline
	\end{tabular}
\end{table}
\subparagraph{\$json}Das Array enthält folgende Elemente:
\begin{table}[H]
	\begin{tabular}{|c|p{11cm}|}
		\hline
		\textbf{Parametername} & \textbf{Parameterbeschreibung} \\ \hline
		id    & Identifikator eines Ranges \\ \hline
		value & Wert des Ranges \\ \hline
		name  & Name des Ranges \\ \hline
	\end{tabular}
\end{table}
\paragraph{Beschreibung} Die Funktion fügt aktualisiert einen Rang. Die Funktion hat Auswirkungen auf folgende Quellen:
\begin{itemize}
	\item Tabelle mit Rängen
\end{itemize}
Die Antwort wird als strukturiertes Array an den Aufrufer zurückgegeben.
\subsubsection{getStatisticalDataAPI}
\paragraph{Parameter} Die Funktion besitzt folgende Parameter:
\begin{table}[H]
	\begin{tabular}{|c|p{11cm}|}
		\hline
		\textbf{Parametername} & \textbf{Parameterbeschreibung} \\ \hline
		\$json & Array mit Informationen \\ \hline
	\end{tabular}
\end{table}
\subparagraph{\$json}Das Array enthält folgende Elemente:
\begin{table}[H]
	\begin{tabular}{|c|p{11cm}|}
		\hline
		\textbf{Parametername} & \textbf{Parameterbeschreibung} \\ \hline
		data & Daten zum Abfragen der Statistiken \\ \hline
	\end{tabular}
\end{table}
\paragraph{Beschreibung} Die Funktion ruft statistische Daten für Chart.js ab. Die Funktion nutzt folgende Quellen:
\begin{itemize}
	\item Tabelle mit statistischen Nutzungsdaten
	\item Tabelle mit Nutzerdaten
	\item Tabelle mit Bildern
	\item Tabelle mit Geschichten
\end{itemize}
Die Antwort wird als strukturiertes Array an den Aufrufer zurückgegeben.
\subsubsection{getAllRoleNamesAPI}
\paragraph{Parameter} Die Funktion besitzt keine Parameter.
\paragraph{Beschreibung} Die Funktion ruft eine Liste aller Rollennamen ab. Die Funktion nutzt folgende Quellen:
\begin{itemize}
	\item Tabelle mit Rollen
\end{itemize}
Die Antwort wird als strukturiertes Array an den Aufrufer zurückgegeben.
\subsubsection{getAllRankNamesAPI}
\paragraph{Parameter} Die Funktion besitzt keine Parameter.
\paragraph{Beschreibung} Die Funktion ruft eine Liste aller Rangnamen ab. Die Funktion nutzt folgende Quellen:
\begin{itemize}
	\item Tabelle mit Rängen
\end{itemize}
Die Antwort wird als strukturiertes Array an den Aufrufer zurückgegeben.
\subsubsection{generateCaptchaAPI}
\paragraph{Parameter} Die Funktion besitzt keine Parameter.
\paragraph{Beschreibung} Die Funktion ruft generiert ein Captcha und sendet das Bild an das Frontend, der Inhalt des Captchas wird ind der Session-Variable des Servers gespeichert. Die Antwort wird als strukturiertes Array an den Aufrufer zurückgegeben.
\subsubsection{sendContactMessage}
\paragraph{Parameter} Die Funktion besitzt folgende Parameter:
\begin{table}[H]
	\begin{tabular}{|c|p{11cm}|}
		\hline
		\textbf{Parametername} & \textbf{Parameterbeschreibung} \\ \hline
		\$json & Array mit Informationen \\ \hline
	\end{tabular}
\end{table}
\subparagraph{\$json}Das Array enthält folgende Elemente:
\begin{table}[H]
	\begin{tabular}{|c|p{11cm}|}
		\hline
		\textbf{Parametername} & \textbf{Parameterbeschreibung} \\ \hline
		cap   & Durch Nutzer eingegebenen Inhalt des Captchas \\ \hline
		msg   & Inhalt der Nachricht \\ \hline
		title & Betreff der Nachricht  \\ \hline
	\end{tabular}
\end{table}
\paragraph{Beschreibung} Die Funktion sendet eine Kontakt-E-Mail an eine in der Konfigurationsdatei (siehe \autoref{config:zentral-mail}) hinterlegten Mailadresse. Die Antwort wird als strukturiertes Array an den Aufrufer zurückgegeben.
\subsubsection{getUnusedApisForRank}
\paragraph{Parameter} Die Funktion besitzt folgende Parameter:
\begin{table}[H]
	\begin{tabular}{|c|p{11cm}|}
		\hline
		\textbf{Parametername} & \textbf{Parameterbeschreibung} \\ \hline
		\$json & Array mit Informationen \\ \hline
	\end{tabular}
\end{table}
\subparagraph{\$json}Das Array enthält folgende Elemente:
\begin{table}[H]
	\begin{tabular}{|c|p{11cm}|}
		\hline
		\textbf{Parametername} & \textbf{Parameterbeschreibung} \\ \hline
		id    & Identifikator eines Ranges \\ \hline
	\end{tabular}
\end{table}
\paragraph{Beschreibung} Die Funktion ruft eine Liste aller Module ab, welche keinen modulbasierten Rangnamen eines konkreten Ranges haben. Die Antwort wird als strukturiertes Array an den Aufrufer zurückgegeben.
\subsubsection{getModuleRankNames}
\paragraph{Parameter} Die Funktion besitzt folgende Parameter:
\begin{table}[H]
	\begin{tabular}{|c|p{11cm}|}
		\hline
		\textbf{Parametername} & \textbf{Parameterbeschreibung} \\ \hline
		\$json & Array mit Informationen \\ \hline
	\end{tabular}
\end{table}
\subparagraph{\$json}Das Array enthält folgende Elemente:
\begin{table}[H]
	\begin{tabular}{|c|p{11cm}|}
		\hline
		\textbf{Parametername} & \textbf{Parameterbeschreibung} \\ \hline
		id    & Identifikator eines Ranges \\ \hline
	\end{tabular}
\end{table}
\paragraph{Beschreibung} Die Funktion ruft eine Liste aller modulbasierten Namen eines Ranges ab. Die Antwort wird als strukturiertes Array an den Aufrufer zurückgegeben.
\subsubsection{insertModuleBasedRankNameMapi}
\paragraph{Parameter} Die Funktion besitzt folgende Parameter:
\begin{table}[H]
	\begin{tabular}{|c|p{11cm}|}
		\hline
		\textbf{Parametername} & \textbf{Parameterbeschreibung} \\ \hline
		\$json & Array mit Informationen \\ \hline
	\end{tabular}
\end{table}
\subparagraph{\$json}Das Array enthält folgende Elemente:
\begin{table}[H]
	\begin{tabular}{|c|p{11cm}|}
		\hline
		\textbf{Parametername} & \textbf{Parameterbeschreibung} \\ \hline
		rid    & Identifikator eines Ranges \\ \hline
		aid    & Identifikator eines Moduls \\ \hline
		name   & modulbasierter Name eines Ranges \\ \hline
	\end{tabular}
\end{table}
\paragraph{Beschreibung} Die Funktion fügt einen neuen modulbasierten Rangnamen hinzu. Die Antwort wird als strukturiertes Array an den Aufrufer zurückgegeben.
\subsubsection{getAllApiData}
\paragraph{Parameter} Die Funktion besitzt folgende Parameter:
\begin{table}[H]
	\begin{tabular}{|c|p{11cm}|}
		\hline
		\textbf{Parametername} & \textbf{Parameterbeschreibung} \\ \hline
		\$json & Array mit Informationen \\ \hline
	\end{tabular}
\end{table}
\subparagraph{\$json}Das Array enthält folgende Elemente:
\begin{table}[H]
	\begin{tabular}{|c|p{11cm}|}
		\hline
		\textbf{Parametername} & \textbf{Parameterbeschreibung} \\ \hline
		id     & Identifikator einer API \\ \hline
	\end{tabular}
\end{table}
\paragraph{Beschreibung} Die Funktion ruft alle Daten der angegebenen API ab.
\subsubsection{saveApiDataAPI}
\paragraph{Parameter} Die Funktion besitzt folgende Parameter:
\begin{table}[H]
	\begin{tabular}{|c|p{11cm}|}
		\hline
		\textbf{Parametername} & \textbf{Parameterbeschreibung} \\ \hline
		\$json & Array mit Informationen \\ \hline
	\end{tabular}
\end{table}
\subparagraph{\$json}Das Array enthält folgende Elemente:
\begin{table}[H]
	\begin{tabular}{|c|p{11cm}|}
		\hline
		\textbf{Parametername} & \textbf{Parameterbeschreibung} \\ \hline
		id     & Identifikator einer API \\ \hline
		name   & Name einer API \\ \hline
		url    & Reverse-API-URL einer API \\ \hline
	\end{tabular}
\end{table}
\paragraph{Beschreibung} Die Funktion speichert geänderte Daten eines Moduls.
\subsubsection{getModulBasedRights}
\paragraph{Parameter} Die Funktion besitzt folgende Parameter:
\begin{table}[H]
	\begin{tabular}{|c|p{11cm}|}
		\hline
		\textbf{Parametername} & \textbf{Parameterbeschreibung} \\ \hline
		\$json & Array mit Informationen \\ \hline
	\end{tabular}
\end{table}
\subparagraph{\$json}Das Array enthält folgende Elemente:
\begin{table}[H]
	\begin{tabular}{|c|p{11cm}|}
		\hline
		\textbf{Parametername} & \textbf{Parameterbeschreibung} \\ \hline
		id     & Identifikator eines Nutzers \\ \hline
	\end{tabular}
\end{table}
\paragraph{Beschreibung} Die Funktion ruft die Module ab, für welche der Nutzer Rechte hat.
\subsubsection{getAllModules}
\paragraph{Parameter} Die Funktion besitzt keine Parameter.
\paragraph{Beschreibung} Die Funktion ruft den Namen und die Id aller Module ab.
\subsubsection{addModuleRoleMapi}
\paragraph{Parameter} Die Funktion besitzt folgende Parameter:
\begin{table}[H]
	\begin{tabular}{|c|p{11cm}|}
		\hline
		\textbf{Parametername} & \textbf{Parameterbeschreibung} \\ \hline
		\$json & Array mit Informationen \\ \hline
	\end{tabular}
\end{table}
\subparagraph{\$json}Das Array enthält folgende Elemente:
\begin{table}[H]
	\begin{tabular}{|c|p{11cm}|}
		\hline
		\textbf{Parametername} & \textbf{Parameterbeschreibung} \\ \hline
		role     & Identifikator eines Rolle \\ \hline
		user     & Identifikator eines Nutzers \\ \hline
		module   & Identifikator eines Moduls \\ \hline
	\end{tabular}
\end{table}
\paragraph{Beschreibung} Die Funktion fügt einem Nutzer eine neue Modul-Rolle hinzu.
\subsubsection{generateErrorMAPI}
\paragraph{Parameter} Die Funktion besitzt folgende Parameter:
\begin{table}[H]
	\begin{tabular}{|c|p{11cm}|}
		\hline
		\textbf{Parametername} & \textbf{Parameterbeschreibung} \\ \hline
		\$json & Array oder Zeichenkette als Nutzlast der Antwort \\ \hline
	\end{tabular}
\end{table}
\paragraph{Beschreibung} Die Funktion erstellt eine generische Nachricht, welche einen nicht erfolgreichen API-Request signalisiert. Die Antwort wird als strukturiertes Array an den Aufrufer zurückgegeben.
\subsubsection{deleteModuleBasedRoleMapi}
\paragraph{Parameter} Die Funktion besitzt folgende Parameter:
\begin{table}[H]
	\begin{tabular}{|c|p{11cm}|}
		\hline
		\textbf{Parametername} & \textbf{Parameterbeschreibung} \\ \hline
		\$json & Array mit Informationen \\ \hline
	\end{tabular}
\end{table}
\subparagraph{\$json}Das Array enthält folgende Elemente:
\begin{table}[H]
	\begin{tabular}{|c|p{11cm}|}
		\hline
		\textbf{Parametername} & \textbf{Parameterbeschreibung} \\ \hline
		id     & Identifikator eines modulbasierten Rechtes \\ \hline
	\end{tabular}
\end{table}
\paragraph{Beschreibung} Die Funktion löscht ein modulbasiertes Recht.
\subsubsection{setMailValidationValue}
\paragraph{Parameter} Die Funktion besitzt folgende Parameter:
\begin{table}[H]
	\begin{tabular}{|c|p{11cm}|}
		\hline
		\textbf{Parametername} & \textbf{Parameterbeschreibung} \\ \hline
		\$json & Array mit Informationen \\ \hline
	\end{tabular}
\end{table}
\subparagraph{\$json}Das Array enthält folgende Elemente:
\begin{table}[H]
	\begin{tabular}{|c|p{11cm}|}
		\hline
		\textbf{Parametername} & \textbf{Parameterbeschreibung} \\ \hline
		name     & Nutzername \\ \hline
		state    & Status der Mailvalidierung \\ \hline
	\end{tabular}
\end{table}
\paragraph{Beschreibung} Die Funktion setzt den Validierungsstatus einer Mailadresse.
\subsubsection{updateModuleRightsMapi}
\paragraph{Parameter} Die Funktion besitzt folgende Parameter:
\begin{table}[H]
	\begin{tabular}{|c|p{11cm}|}
		\hline
		\textbf{Parametername} & \textbf{Parameterbeschreibung} \\ \hline
		\$json & Array mit Informationen \\ \hline
	\end{tabular}
\end{table}
\subparagraph{\$json}Das Array enthält folgende Elemente:
\begin{table}[H]
	\begin{tabular}{|c|p{11cm}|}
		\hline
		\textbf{Parametername} & \textbf{Parameterbeschreibung} \\ \hline
		roleid   & Identifikator einer Rolle \\ \hline
		rightid  & Identifikator eines modulbasierten Rechtes \\ \hline
	\end{tabular}
\end{table}
\paragraph{Beschreibung} Die Funktion aktualisiert die modulbasierten Berechtigungen.
\subsubsection{setDisableModuleRightStateMapi}
\paragraph{Parameter} Die Funktion besitzt folgende Parameter:
\begin{table}[H]
	\begin{tabular}{|c|p{11cm}|}
		\hline
		\textbf{Parametername} & \textbf{Parameterbeschreibung} \\ \hline
		\$json & Array mit Informationen \\ \hline
	\end{tabular}
\end{table}
\subparagraph{\$json}Das Array enthält folgende Elemente:
\begin{table}[H]
	\begin{tabular}{|c|p{11cm}|}
		\hline
		\textbf{Parametername} & \textbf{Parameterbeschreibung} \\ \hline
		state    & Status der Deaktivierung \\ \hline
		rightid  & Identifikator eines modulbasierten Rechtes \\ \hline
	\end{tabular}
\end{table}
\paragraph{Beschreibung} Die Funktion aktualisiert den Deaktivierungsstatus eines Modulrechtes.
\subsubsection{getUnsubscribedUserForModule}
\paragraph{Parameter} Die Funktion besitzt folgende Parameter:
\begin{table}[H]
	\begin{tabular}{|c|p{11cm}|}
		\hline
		\textbf{Parametername} & \textbf{Parameterbeschreibung} \\ \hline
		\$json & Array mit Informationen \\ \hline
	\end{tabular}
\end{table}
\subparagraph{\$json}Das Array enthält folgende Elemente:
\begin{table}[H]
	\begin{tabular}{|c|p{11cm}|}
		\hline
		\textbf{Parametername} & \textbf{Parameterbeschreibung} \\ \hline
		state    & Status der Deaktivierung \\ \hline
		module   & Identifikator eines Moduls \\ \hline
	\end{tabular}
\end{table}
\paragraph{Beschreibung} Die Funktion ruft eine Liste aller Benutzer ohne eine Berechtigung für das gegebene Modul ab.
\subsubsection{addNewModuleApi}
\paragraph{Parameter} Die Funktion besitzt folgende Parameter:
\begin{table}[H]
	\begin{tabular}{|c|p{11cm}|}
		\hline
		\textbf{Parametername} & \textbf{Parameterbeschreibung} \\ \hline
		\$json & Array mit Informationen \\ \hline
	\end{tabular}
\end{table}
\subparagraph{\$json}Das Array enthält folgende Elemente:
\begin{table}[H]
	\begin{tabular}{|c|p{11cm}|}
		\hline
		\textbf{Parametername} & \textbf{Parameterbeschreibung} \\ \hline
		name    & Name des neuen Moduls \\ \hline
		url     & Url der Reverse-Api des Moduls \\ \hline
	\end{tabular}
\end{table}
\paragraph{Beschreibung} Die Funktion fügt COSP ein neues Modul hinzu.
\subsubsection{checkMailAddressExistentAPI}
\paragraph{Parameter} Die Funktion besitzt folgende Parameter:
\begin{table}[H]
	\begin{tabular}{|c|p{11cm}|}
		\hline
		\textbf{Parametername} & \textbf{Parameterbeschreibung} \\ \hline
		\$json & Array mit Informationen \\ \hline
	\end{tabular}
\end{table}
\subparagraph{\$json}Das Array enthält folgende Elemente:
\begin{table}[H]
	\begin{tabular}{|c|p{11cm}|}
		\hline
		\textbf{Parametername} & \textbf{Parameterbeschreibung} \\ \hline
		email    & Emailadresse \\ \hline
	\end{tabular}
\end{table}
\paragraph{Beschreibung} Die Funktion prüft ob eine Mailadresse bereits verwendet wird.
\newpage
\section{SessionValues}
\subsection{Allgemeines} Diese Datei ermöglicht es, Session-Daten innerhalb einer Datenbank zu speichern und initialisiert die Session.
\begin{table}[H]
	\begin{tabular}{|c|p{11cm}|}
		\hline
		\textbf{Einbindungspunkt} & inc.php \\ \hline
		\textbf{Einbindungspunkt} & inc-sub.php \\ \hline
	\end{tabular}
\end{table}
Die Datei ist nicht direkt durch den Nutzer aufrufbar, dies wird durch folgenden Code-Ausschnitt sichergestellt:
\begin{lstlisting}[language=php]
if (!defined('NICE_PROJECT')) {
	die('Permission denied.');
}
\end{lstlisting}
Der Globale Wert {\glqq NICE\_PROJECT\grqq} wird durch für den Nutzer valide Aufrufpunkte festgelegt, z.B. {\glqq api.php\grqq}.
\newpage
\subsection{Allgemeines}
Der Sessionstart erfolgt mittels nachfolgendem Code:
\begin{lstlisting}[language=php]
session_set_save_handler('ses_open', 'ses_close', 'ses_read', 'ses_write', 'ses_destroy', 'ses_gc');
register_shutdown_function('session_write_close');
session_start();
\end{lstlisting}
Desweiteren wird auch die Variable \$LOGIN gesetzt:
\begin{lstlisting}[language=php]
$LOGIN = false;
if (isset($_SESSION)) {
	if (isset($_SESSION['name'])) {
		$LOGIN = true;
	}
}
\end{lstlisting}
\subsection{Besonderheiten}
\newpage
\subsection{Funktionen}
\subsubsection{ses\_open}
\paragraph{Parameter} Die Funktion besitzt folgende Parameter:
\begin{table}[H]
	\begin{tabular}{|c|p{11cm}|}
		\hline
		\textbf{Parametername} & \textbf{Parameterbeschreibung} \\ \hline
		\$path & irgendein Pfad (wird nicht genutzt, aber benötigt) \\ \hline
		\$name & Irgendein Name (wird nicht genutzt, aber benötigt) \\ \hline
	\end{tabular}
\end{table}
\paragraph{Beschreibung} Die Funktion eröffnet eine Session und gibt stets {\glqq true\grqq} zurück.
\subsubsection{ses\_close}
\paragraph{Parameter} Die Funktion besitzt keine Parameter.
\paragraph{Beschreibung} Die Funktion schließt eine Session und gibt stets {\glqq true\grqq} zurück.
\subsubsection{ses\_read}
\paragraph{Parameter} Die Funktion besitzt folgende Parameter:
\begin{table}[H]
	\begin{tabular}{|c|p{11cm}|}
		\hline
		\textbf{Parametername} & \textbf{Parameterbeschreibung} \\ \hline
		\$ses\_id & Identifikator einer Session \\ \hline
	\end{tabular}
\end{table}
\paragraph{Beschreibung} Die Funktion liest eine Session aus der Datenbank. Die Funktion nutzt folgende Quellen:
\begin{itemize}
	\item Tabelle mit Sessiondaten
\end{itemize}
Die Antwort wird als Zeichenkette an den Aufrufer zurückgegeben.
\subsubsection{ses\_write}
\paragraph{Parameter} Die Funktion besitzt folgende Parameter:
\begin{table}[H]
	\begin{tabular}{|c|p{11cm}|}
		\hline
		\textbf{Parametername} & \textbf{Parameterbeschreibung} \\ \hline
		\$ses\_id & Identifikator der Session \\ \hline
		\$data    & Daten der Session \\ \hline
	\end{tabular}
\end{table}
\paragraph{Beschreibung} Die Funktion schreibt die Sessiondaten in die Datenbank. Die Funktion hat Auswirkungen auf folgende Quellen:
\begin{itemize}
	\item Tabelle mit Sessiondaten.
\end{itemize}
Der Rückgabewert ist stets {\glqq true\grqq}.
\subsubsection{ses\_destroy}
\paragraph{Parameter} Die Funktion besitzt folgende Parameter:
\begin{table}[H]
	\begin{tabular}{|c|p{11cm}|}
		\hline
		\textbf{Parametername} & \textbf{Parameterbeschreibung} \\ \hline
		\$ses\_id & Identifikator der Session \\ \hline
	\end{tabular}
\end{table}
\paragraph{Beschreibung} Die Funktion löscht die Sessiondaten aus der Datenbank. Die Funktion hat Auswirkungen auf folgende Quellen:
\begin{itemize}
	\item Tabelle mit Sessiondaten.
\end{itemize}
Der Rückgabewert ist stets {\glqq true\grqq}.
\subsubsection{ses\_gc}
\paragraph{Parameter} Die Funktion besitzt folgende Parameter:
\begin{table}[H]
	\begin{tabular}{|c|p{11cm}|}
		\hline
		\textbf{Parametername} & \textbf{Parameterbeschreibung} \\ \hline
		\$life & Lebenszeit der Sessions in Sekunden \\ \hline
	\end{tabular}
\end{table}
\paragraph{Beschreibung} Die Funktion löscht alle Sessiondaten aus der Datenbank, welche älter das jetzt minus die angegebenen Sekunden sind. Die Funktion hat Auswirkungen auf folgende Quellen:
\begin{itemize}
	\item Tabelle mit Sessiondaten.
\end{itemize}
Der Rückgabewert ist stets {\glqq true\grqq}.
\subsubsection{checkLoginDeny}
\paragraph{Parameter} Die Funktion besitzt folgende Parameter:
\begin{table}[H]
	\begin{tabular}{|c|p{11cm}|}
		\hline
		\textbf{Parametername} & \textbf{Parameterbeschreibung} \\ \hline
		\$login & Status des Login eines Nutzers \\ \hline
	\end{tabular}
\end{table}
\paragraph{Beschreibung} Die Funktion beendet die Ausführung des PHP, wenn der Loginstatus nicht korrekt gesetzt ist. Die Antwort wird als Boolean an den Aufrufer zurückgegeben.

\newpage
\section{settings}
\subsection{Allgemeines} Diese Datei stellt entsprechende PHp-Initialwerte ein.
\begin{table}[H]
	\begin{tabular}{|c|p{11cm}|}
		\hline
		\textbf{Einbindungspunkt} & inc.php \\ \hline
		\textbf{Einbindungspunkt} & inc-sub.php \\ \hline
	\end{tabular}
\end{table}
Die Datei ist nicht direkt durch den Nutzer aufrufbar, dies wird durch folgenden Code-Ausschnitt sichergestellt:
\begin{lstlisting}[language=php]
if (!defined('NICE_PROJECT')) {
	die('Permission denied.');
}
\end{lstlisting}
Der Globale Wert {\glqq NICE\_PROJECT\grqq} wird durch für den Nutzer valide Aufrufpunkte festgelegt, z.B. {\glqq api.php\grqq}.
\subsection{Allgemeines}
Momentan wird die Lebenszeit der Session auf 24h gesetzt. Es erfolgt bei fast jedem Zugriff eine Bereinigung alter Sessions.
\begin{lstlisting}[language=php]
ini_set('session.gc_maxlifetime', 86400);
ini_set('session.gc_probability', 1);
ini_set('session.gc_divisor', 100);
\end{lstlisting}
\subsection{Besonderheiten}
Hier sind alle Projektweiten PHP-Einstellungen vorzunehmen.

\newpage
\section{statistic-calc}
\subsection{Allgemeines} Diese Datei enthält Funktionen, welche zur Darstellung statistischer Daten verwendet werden.
\begin{table}[H]
	\begin{tabular}{|c|p{11cm}|}
		\hline
		\textbf{Einbindungspunkt} & inc.php \\ \hline
		\textbf{Einbindungspunkt} & inc-sub.php \\ \hline
	\end{tabular}
\end{table}
Die Datei ist nicht direkt durch den Nutzer aufrufbar, dies wird durch folgenden Code-Ausschnitt sichergestellt:
\begin{lstlisting}[language=php]
if (!defined('NICE_PROJECT')) {
	die('Permission denied.');
}
\end{lstlisting}
Der Globale Wert {\glqq NICE\_PROJECT\grqq} wird durch für den Nutzer valide Aufrufpunkte festgelegt, z.B. {\glqq api.php\grqq}.
\newpage
\subsection{Funktionen}
\subsubsection{colorCounter}
\paragraph{Parameter} Die Funktion besitzt keine Parameter.
\paragraph{Beschreibung} Die Funktion gibt ein Array mit möglichen Farben für Graphen zurück. Die Antwort wird als strukturiertes Array an den Aufrufer zurückgegeben.
\subsubsection{loginStatistics}
\paragraph{Parameter} Die Funktion besitzt folgende Parameter:
\begin{table}[H]
	\begin{tabular}{|c|p{11cm}|}
		\hline
		\textbf{Parametername} & \textbf{Parameterbeschreibung} \\ \hline
		\$Input & Eingabedaten als Array \\ \hline
	\end{tabular}
\end{table}
\subparagraph{\$Input}Das Array enthält Einträge mit folgenden Elemente:
\begin{table}[H]
	\begin{tabular}{|c|p{11cm}|}
		\hline
		\textbf{Parametername} & \textbf{Parameterbeschreibung} \\ \hline
		Amount & Anzahl an Zeiteinheiten \\ \hline
		type   & Typ der Zeiteinheit \\ \hline
	\end{tabular}
\end{table}
\paragraph{Beschreibung} Die Funktion erstellt ein Datenarray, welches zur Anzeige der Nutzungsstatistiken verwendet wird. Die Funktion nutzt folgende Quellen:
\begin{itemize}
	\item Tabelle mit statistischen Nutzungsdaten
\end{itemize}
Die Antwort wird als strukturiertes Array an den Aufrufer zurückgegeben.
\subsubsection{createGraph}
\paragraph{Parameter} Die Funktion besitzt folgende Parameter:
\begin{table}[H]
	\begin{tabular}{|c|p{11cm}|}
		\hline
		\textbf{Parametername} & \textbf{Parameterbeschreibung} \\ \hline
		\$data  & statistische Daten in Array-Form \\ \hline
		\$color & Farbe \\ \hline
		\$label & Name des Datensatzes \\ \hline
		\$fill  & Gibt an, ob Graph gefüllt werden soll \\ \hline
	\end{tabular}
\end{table}
\paragraph{Beschreibung} Die Funktion generiert einen durch das Chart.js anzeigbaren Datensatz. Die Antwort wird als strukturiertes Array an den Aufrufer zurückgegeben.
\subsubsection{fillUnknownData}
\paragraph{Parameter} Die Funktion besitzt folgende Parameter:
\begin{table}[H]
	\begin{tabular}{|c|p{11cm}|}
		\hline
		\textbf{Parametername} & \textbf{Parameterbeschreibung} \\ \hline
		\$statisticalData & Array mit statistischen Daten \\ \hline
		\$periodeAmount   & Anzahl an Zeiteinheiten \\ \hline
		\$type            & Typ der Zeiteinheit \\ \hline
	\end{tabular}
\end{table}
\paragraph{Beschreibung} Die Funktion füllt fehlende Einträge in den statistischen Daten mit dem Wert {\glqq 0\grqq} auf. Die Antwort wird als strukturiertes Array an den Aufrufer zurückgegeben.
\subsubsection{NewUsersStatistics}
\paragraph{Parameter} Die Funktion besitzt folgende Parameter:
\begin{table}[H]
	\begin{tabular}{|c|p{11cm}|}
		\hline
		\textbf{Parametername} & \textbf{Parameterbeschreibung} \\ \hline
		\$Input & Eingabedaten als Array \\ \hline
	\end{tabular}
\end{table}
\subparagraph{\$Input}Das Array enthält Einträge mit folgenden Elemente:
\begin{table}[H]
	\begin{tabular}{|c|p{11cm}|}
		\hline
		\textbf{Parametername} & \textbf{Parameterbeschreibung} \\ \hline
		Amount & Anzahl an Zeiteinheiten \\ \hline
		type   & Typ der Zeiteinheit \\ \hline
	\end{tabular}
\end{table}
\paragraph{Beschreibung} Die Funktion erstellt ein Datenarray, welches zur Anzeige der Statistik zu neuen Nutzern verwendet wird. Die Funktion nutzt folgende Quellen:
\begin{itemize}
	\item Tabelle mit Nutzerdaten
\end{itemize}
Die Antwort wird als strukturiertes Array an den Aufrufer zurückgegeben.
\subsubsection{getStatistics}
\paragraph{Parameter} Die Funktion besitzt folgende Parameter:
\begin{table}[H]
	\begin{tabular}{|c|p{11cm}|}
		\hline
		\textbf{Parametername} & \textbf{Parameterbeschreibung} \\ \hline
		\$Input  & Eingabedaten als Array \\ \hline
		\$Source & Quelle der Statistischen Daten \\ \hline
	\end{tabular}
\end{table}
\subparagraph{\$Input}Das Array enthält Einträge mit folgenden Elemente:
\begin{table}[H]
	\begin{tabular}{|c|p{11cm}|}
		\hline
		\textbf{Parametername} & \textbf{Parameterbeschreibung} \\ \hline
		Amount & Anzahl an Zeiteinheiten \\ \hline
		type   & Typ der Zeiteinheit \\ \hline
	\end{tabular}
\end{table}
\paragraph{Beschreibung} Die Funktion erstellt ein Datenarray, welches zur Anzeige der Statistik zu neuen/geänderten Bildern oder Geschichten verwendet wird. Die Funktion nutzt folgende Quellen:
\begin{itemize}
	\item Tabelle mit Nutzerdaten
\end{itemize}
Die Antwort wird als strukturiertes Array an den Aufrufer zurückgegeben.
\subsubsection{correctData}
\paragraph{Parameter} Die Funktion besitzt folgende Parameter:
\begin{table}[H]
	\begin{tabular}{|c|p{11cm}|}
		\hline
		\textbf{Parametername} & \textbf{Parameterbeschreibung} \\ \hline
		\$inputArray  & Eingabedaten als Array \\ \hline
	\end{tabular}
\end{table}
\paragraph{Beschreibung} Die Funktion korrigiert fehlerhafte statistische Daten. Die Antwort wird als strukturiertes Array an den Aufrufer zurückgegeben.
\newpage
\section{uapi-functions}
\subsection{Allgemeines} Diese Datei enthält alle Funktionen, welche durch die Nutzer-API aufgerufen werden.
\begin{table}[H]
	\begin{tabular}{|c|p{11cm}|}
		\hline
		\textbf{Einbindungspunkt} & inc.php \\ \hline
		\textbf{Einbindungspunkt} & inc-sub.php \\ \hline
	\end{tabular}
\end{table}
Die Datei ist nicht direkt durch den Nutzer aufrufbar, dies wird durch folgenden Code-Ausschnitt sichergestellt:
\begin{lstlisting}[language=php]
if (!defined('NICE_PROJECT')) {
	die('Permission denied.');
}
\end{lstlisting}
Der Globale Wert {\glqq NICE\_PROJECT\grqq} wird durch für den Nutzer valide Aufrufpunkte festgelegt, z.B. {\glqq api.php\grqq}.
\newpage
\subsection{Funktionen}
\subsubsection{getPreviewPictureAPI}
\paragraph{Parameter} Die Funktion besitzt folgende Parameter:
\begin{table}[H]
	\begin{tabular}{|c|p{11cm}|}
		\hline
		\textbf{Parametername} & \textbf{Parameterbeschreibung} \\ \hline
		\$token & alphanumerischer Identifikator eines Bildes \\ \hline
	\end{tabular}
\end{table}
\paragraph{Beschreibung} Die Funktion liefert einen Base64-Codiertes Vorschaubild zurück. Die Funktion nutzt folgende Quellen:
\begin{itemize}
	\item Tabelle mit Bildern
\end{itemize}
Die Antwort wird als Zeichenkette an den Aufrufer zurückgegeben.
\subsubsection{getPictureFullsizeAPI}
\paragraph{Parameter} Die Funktion besitzt folgende Parameter:
\begin{table}[H]
	\begin{tabular}{|c|p{11cm}|}
		\hline
		\textbf{Parametername} & \textbf{Parameterbeschreibung} \\ \hline
		\$token & alphanumerischer Identifikator eines Bildes \\ \hline
	\end{tabular}
\end{table}
\paragraph{Beschreibung} Die Funktion liefert das Ursprungsbild als Binary zurück. Die Funktion nutzt folgende Quellen:
\begin{itemize}
	\item Tabelle mit Bildern
	\item Dateisystem
\end{itemize}
Die Antwort wird als Binary an den Aufrufer zurückgegeben.
\subsubsection{generateErrorUAPI}
\paragraph{Parameter} Die Funktion besitzt keine Parameter.
\paragraph{Beschreibung} Die Funktion liefert einen generischen Fehler auf eine API-Anfrage. Die Antwort wird als strukturiertes Array an den Aufrufer zurückgegeben.
\subsubsection{generateSuccessUAPI}
\paragraph{Parameter} Die Funktion besitzt folgende Parameter:
\begin{table}[H]
	\begin{tabular}{|c|p{11cm}|}
		\hline
		\textbf{Parametername} & \textbf{Parameterbeschreibung} \\ \hline
		\$message & Nachricht (optional) \\ \hline
	\end{tabular}
\end{table}
\paragraph{Beschreibung} Die Funktion liefert einen generischen Erfolg auf eine API-Anfrage. Die Antwort wird als strukturiertes Array an den Aufrufer zurückgegeben.
\subsubsection{generateJsonUAPI}
\paragraph{Parameter} Die Funktion besitzt folgende Parameter:
\begin{table}[H]
	\begin{tabular}{|c|p{11cm}|}
		\hline
		\textbf{Parametername} & \textbf{Parameterbeschreibung} \\ \hline
		\$array & Array mit Daten \\ \hline
	\end{tabular}
\end{table}
\paragraph{Beschreibung} Die Funktion generiert das JSON für Antwort auf einen API-Anfrage. Die Antwort wird direkt Ausgegeben.
\subsubsection{getStoryDataUAPI}
\paragraph{Parameter} Die Funktion besitzt folgende Parameter:
\begin{table}[H]
	\begin{tabular}{|c|p{11cm}|}
		\hline
		\textbf{Parametername} & \textbf{Parameterbeschreibung} \\ \hline
		\$token & alphanumerischer Token einer Geschichte \\ \hline
	\end{tabular}
\end{table}
\paragraph{Beschreibung} Die Funktion ruft alle Daten einer Geschichte ab und sendet diese zurück an den Aufrufer. Die Funktion nutzt folgende Quellen:
\begin{itemize}
	\item Tabelle mit Geschichten
\end{itemize}
Die Antwort wird als strukturiertes Array an den Aufrufer zurückgegeben.
\subsubsection{getStoriesDataUAPI}
\paragraph{Parameter} Die Funktion besitzt folgende Parameter:
\begin{table}[H]
	\begin{tabular}{|c|p{11cm}|}
		\hline
		\textbf{Parametername} & \textbf{Parameterbeschreibung} \\ \hline
		\$tokens & alphanumerische Identifikatoren von Geschichten \\ \hline
	\end{tabular}
\end{table}
\paragraph{Beschreibung} Die Funktion ruft die Daten zu allen angegeben Geschichten ab. Die Funktion nutzt folgende Quellen:
\begin{itemize}
	\item Tabelle mit Geschichten
\end{itemize}
Die Antwort wird als strukturiertes Array an den Aufrufer zurückgegeben.
\subsubsection{buildUserStoryArray}
\paragraph{Parameter} Die Funktion besitzt folgende Parameter:
\begin{table}[H]
	\begin{tabular}{|c|p{11cm}|}
		\hline
		\textbf{Parametername} & \textbf{Parameterbeschreibung} \\ \hline
		\$token           & alphanumerischer Identifikator einer Geschichte \\ \hline
		\$username        & Nutzername des Abfragenden \\ \hline
		\$ValidationValue & Validierungswert des Abfragenden \\ \hline
		\$apphashtoken    & Hash des alphanumerischen Modulidentifikators \\ \hline
	\end{tabular}
\end{table}
\paragraph{Beschreibung} Die Funktion generiert ein Array mit allen Daten einer Geschichte. Die Funktion nutzt folgende Quellen:
\begin{itemize}
	\item Tabelle mit Geschichten
	\item Tabelle mit Nutzerdaten
\end{itemize}
Die Antwort wird als strukturiertes Array an den Aufrufer zurückgegeben.
\subsubsection{validateStory}
\paragraph{Parameter} Die Funktion besitzt folgende Parameter:
\begin{table}[H]
	\begin{tabular}{|c|p{11cm}|}
		\hline
		\textbf{Parametername} & \textbf{Parameterbeschreibung} \\ \hline
		\$data & Zeichenkette mit Informationen \\ \hline
	\end{tabular}
\end{table}
\paragraph{Beschreibung} Die Funktion validiert die angegebene Geschichte. Die Funktion hat Auswirkungen auf folgende Quellen:
\begin{itemize}
	\item Tabelle mit Validierungsinformationen zu Geschichten
\end{itemize}
Die Antwort wird als strukturiertes Array an den Aufrufer zurückgegeben.
\subsubsection{validatePicture}
\paragraph{Parameter} Die Funktion besitzt folgende Parameter:
\begin{table}[H]
	\begin{tabular}{|c|p{11cm}|}
		\hline
		\textbf{Parametername} & \textbf{Parameterbeschreibung} \\ \hline
		\$data & Zeichenkette mit Informationen \\ \hline
	\end{tabular}
\end{table}
\paragraph{Beschreibung} Die Funktion validiert die angegebene Bilder. Die Funktion hat Auswirkungen auf folgende Quellen:
\begin{itemize}
	\item Tabelle mit Validierungsinformationen zu Bildern
\end{itemize}
Die Antwort wird als strukturiertes Array an den Aufrufer zurückgegeben.
\newpage
\section{changePwd}
\subsection{Allgemeines} Diese Datei zeigt ein Formular zum ändern des Passwortes, der Nutzer muss angemeldet sein.
Die Datei ist direkt durch den Nutzer aufrufbar. Sie setzt auch die entsprechende Konstante und bindet alle notwendigen Dateien ein:
\begin{lstlisting}[language=php]
define('NICE_PROJECT', true);
require_once "bin/inc.php";
\end{lstlisting}
\subsection{Allgemeines}
Auf dieser Seite kann ein angemeldeter Nutzer sein Passwort ändern.
\subsection{Besonderheiten}
Der Nutzer muss auch sein altes Passwort wissen.
\newpage
\section{contact}
\begin{usecase}

\addtitle{Kontaktanfrage absenden} 

%Primary Actor: Calls on the system to deliver its services.
\addfield{Benutzer:}{Endnutzer, Mitarbeiter, Administratoren}
\addfield{Endbenutzergruppen:}{nicht Authentifiziert, Authentifiziert, Freigeschaltet}

%Preconditions: What must be true on start and worth telling the reader?
\addfield{Vorbedingungen:}{Login erfolgreich, Kontakt aufrufen}
%when multiple
%\additemizedfield{Preconditions:}{} 

%Main Success Scenario: A typical, unconditional happy path scenario of success.
\addscenario{Szenario:}{
	\checkeditem Captcha-Code lesen und eingeben
	\checkeditem Nachricht und Betreff eingeben
	\checkeditem {\glqq Absenden\grqq}-Button betätigen
}

%Extensions: Alternate scenarios of success or failure.
\addscenario{Erweiterung:}{
	\checkeditem[1.a] {\glqq Neues Captcha\grqq}-Button betätigen, Captcha lesen und eingeben
	\checkeditem[1.b] Falschen Captcha-Code eingeben
	\checkeditem[2.a] Nachricht, aber keinen Betreff eingeben
	\checkeditem[2.b] Betreff, aber keine Nachricht eingeben
	\checkeditem[2.c] keine Nachricht und keinen Betreff eingeben
}


\end{usecase}
\newpage
\section{hub}
\subsection{Allgemeines} Diese Datei stellt eine Funktionsübersichtsseite dar und leitet die Nutzer zu den entsprechenden Seiten.
Die Datei ist direkt durch den Nutzer aufrufbar. Sie setzt auch die entsprechende Konstante und bindet alle notwendigen Dateien ein:
\begin{lstlisting}[language=php]
define('NICE_PROJECT', true);
require_once "bin/inc.php";
\end{lstlisting}
\subsection{Allgemeines}
Es werden die Hauptfunktionen als Kacheln in der Mitte des Browserfensters dargestellt. Diese Ansicht wird automatisch aus einem Array generiert. Dieses Array hat folgende Form:
\subsubsection{Array mit Links} Das Array enthält Eintrage mit folgenden Elementen:
\begin{table}[H]
	\begin{tabular}{|c|p{11cm}|}
		\hline
		\textbf{Name} & \textbf{Beschreibung} \\ \hline
		href           & relativer Link zur entsprechenden Seite \\ \hline
		title          & Titel des Links \\ \hline
		image          & Bild des Links \\ \hline
		needeRoleValue & benötigte Rolle um Link zu nutzen \\ \hline
		text           & Beschreibung der entsprechenden Funktion \\ \hline
	\end{tabular}
\end{table}
\subsection{Besonderheiten}
Diese Seite soll den Nutzer ansprechend auf seine Möglichkeiten aufmerksam machen.

\newpage
\section{impressum}
\subsection{Allgemeines} Diese Datei zeigt das Impressum an.
Die Datei ist direkt durch den Nutzer aufrufbar. Sie setzt auch die entsprechende Konstante und bindet alle notwendigen Dateien ein:
\begin{lstlisting}[language=php]
define('NICE_PROJECT', true);
require_once "bin/inc.php";
\end{lstlisting}
\subsection{Allgemeines}
Hier steht für jeden sichtbar das Impressum. Die Seite unterstützt keine Funktionen.
\subsection{Besonderheiten}
Die Seite ist im Debug-Modus nicht für alle Sichtbar. Es soll hier auch die Adresse, des Verantwortlichen mittels Konfigurationsdatei geändert werden können. Hierfür sind die Konfigurationsflags \autoref{config:impressum-name}, \autoref{config:impressum-street} und \autoref{config:impressum-city} gedacht.

\newpage
\section{index}
\subsection{Allgemeines} Diese Datei dient dem Login und ist die erste Seite auf welcher der Nutzer landet.
Die Datei ist direkt durch den Nutzer aufrufbar. Sie setzt auch die entsprechende Konstante und bindet alle notwendigen Dateien ein:
\begin{lstlisting}[language=php]
define('NICE_PROJECT', true);
require_once "bin/inc.php";
\end{lstlisting}
\subsection{Allgemeines}
Auf dieser Seite kann sich der Nutzer authentifizieren oder sich als Gast anmelden. Alternativ kann hat er die Option über den Registrieren-Button zur Selbstregistrierung zu gelangen. Des Weiteren wird der Nutzer auch auf die Verwendung von Cookies aufmerksam gemacht.
\subsection{Besonderheiten}
Die Seite besitzt eine vereinfachte Navbar, da zu diesem Zeitpunkt nicht alle Funktionen der Navbar zur Verfügung stehen. Diese Seite dient auch dem Logout.

\newpage
\section{mapi}
\newpage
\section{Steuerunsg API-Spezifikation}\label{mapi}
\subsection{Beschreibung}Diese API dient der Kommunikation zwischen dem Frontend und dem Backend von {\glqq COSP\grqq}. Sie unterstützt Funktionen zur Nutzer-, Rollen-, Ränge- und Modul-Verwaltung. Des weiteren unterstützt sie auch das Abfragen einiger statistischer Daten. Diese API ist nur für angemeldete und authorisierte Nutzer verfügbar.
\subsection{Befehlsübersicht}
\begin{longtable}[H]{|c|p{12cm}|}
		\hline
		\textbf{Api-Befehl} & \textbf{Kurzbeschreibung}              \\ \hline
		cur                 & Nutzerrolle ändern          \\ \hline
		anr                 & Neue Rolle hinzufügen            \\ \hline
		eer                 & Bestehende Rolle ändern \\ \hline
		der                 & Rolle löschen \\ \hline
		cup                 & Nutzerpasswort ändern \\ \hline
		teu                 & Nutzer Aktivieren oder Deaktivieren (Toggle) \\ \hline
		rup                 & Reset Nutzerpasswort \\ \hline
		adr                 & Neuen Rang anlegen \\ \hline
		dra                 & Rang löschen \\ \hline
		era                 & Bestehenden Rang ändern \\ \hline
		gsd                 & Statistische Daten abfragen \\ \hline
		gar                 & Alle Rollennamen abfragen \\ \hline
		grn                 & Alle Rangnamen abfragen \\ \hline
		cpa                 & Anfordern eines Captcha-Codes als Base64 endocdiertes Bild \\ \hline
        cmg                 & Kontaktnachricht absenden \\ \hline
        rns                 & Alle Module, welche keinen spezialisierten Namen bei einem Rang haben \\ \hline
        rna                 & Modulbasierte Namen eines Ranges abfragen \\ \hline
        imr                 & Modulbasierte Namen eines Ranges hinzufügen\\ \hline
        dmr                 & Modulbasierte Namen eines Ranges löschen\\ \hline
        gap                 & Abfrage aller Daten einer API\\ \hline
        sap                 & Speichert Daten einer API\\ \hline
        gmr                 & Fragt alle Module ab, für welche ein Nutzer Rechte hat\\ \hline
        gam                 & Name und ID aller APIs abfragen\\ \hline
        sar                 & Gibt alle für den Nutzer vergebbaren Rollen zurück\\ \hline
        smr                 & Speichert eine Modul-Rolle eines Benutzers ab\\ \hline
        drm                 & Löscht eine Modul-Rolle\\ \hline
        smv                 & Mailvalidierungsstatus setzen\\ \hline
        umr					& Aktualisiere Modulberechtigungen \\ \hline
        dsr					& Deaktivierungsstatus eines Nutzers für ein Modul setzen \\ \hline
        cna					& Anlegen eines neuen Moduls beziehungsweise einer neuen API. \\ \hline
        cma					& Prüft ob eine Mailadresse bereits verwendet wird. \\ \hline
\end{longtable}
\newpage
\subsection{Befehle}
\subsubsection{Nutzerrolle ändern}
\paragraph{Kurzbeschreibung}Dieser API-Request wird dazu genutzt um die Rolle eines Nutzers zu ändern.
\paragraph{Anfrage}Folgende Daten werden zu Anfrage benötigt:
\begin{table}[H]
	\begin{tabular}{|c|c|c|p{6.5cm}|}
		\hline
		\textbf{Paramtername} & \textbf{Datentyp} & \textbf{Konstante} & \textbf{Kurzbeschreibung}                                                                                               \\ \hline
		type                & string            & cur                & Nutzerrolle ändern \\ \hline
		username            & string            &                    & Nutzername \\ \hline
		role                & int               &                    & Identifikator der Rolle \\ \hline
	\end{tabular}
\end{table}
\paragraph{Antwort}Die Antwort ist wie folgt aufgebaut:
\begin{table}[H]
	\begin{tabular}{|c|c|c|p{6.5cm}|}
		\hline
		\textbf{Paramtername} & \textbf{Datentyp} & \textbf{Konstante} & \textbf{Kurzbeschreibung}            \\ \hline                
		success             & bool             &                 & Erfolgreich wenn Wert {\glqq true\grqq} ist \\ \hline
		type                & string           & cur             & Nutzerrolle ändern \\ \hline
		username            & string           &                 & Nutzername \\ \hline
		role                & int              &                 & Identifikator der Rolle \\ \hline
	\end{tabular}
\end{table}
\subsubsection{Neue Rolle anlegen}
\paragraph{Kurzbeschreibung}Dieser API-Request wird dazu genutzt um eine neue Rolle hinzuzufügen.
\paragraph{Anfrage}Folgende Daten werden zu Anfrage benötigt:
\begin{table}[H]
	\begin{tabular}{|c|c|c|p{6.5cm}|}
		\hline
		\textbf{Paramtername} & \textbf{Datentyp} & \textbf{Konstante} & \textbf{Kurzbeschreibung}                                                                                               \\ \hline
		type                & string            & anr                & Neue Rolle anlegen \\ \hline
		name                & string            &                    & Rollenname \\ \hline
		value               & int               &                    & Rollenwert \\ \hline
	\end{tabular}
\end{table}
\paragraph{Antwort}Die Antwort ist wie folgt aufgebaut:
\begin{table}[H]
	\begin{tabular}{|c|c|c|p{6.5cm}|}
		\hline
		\textbf{Paramtername} & \textbf{Datentyp} & \textbf{Konstante} & \textbf{Kurzbeschreibung}            \\ \hline                
		success             & bool             &                 & Erfolgreich wenn Wert {\glqq true\grqq} ist \\ \hline
		payload             & bool             &                 & Wahr, wenn Erfolgreich \\ \hline
	\end{tabular}
\end{table}
\subsubsection{Bestehende Rolle ändern}
\paragraph{Kurzbeschreibung}Dieser API-Request wird dazu genutzt um eine einzelne bestehende Rolle zu ändern.
\paragraph{Anfrage}Folgende Daten werden zu Anfrage benötigt:
\begin{table}[H]
	\begin{tabular}{|c|c|c|p{6.5cm}|}
		\hline
		\textbf{Paramtername} & \textbf{Datentyp} & \textbf{Konstante} & \textbf{Kurzbeschreibung}                                                                                               \\ \hline
		type                & string            & eer                & Rolle ändern \\ \hline
		name                & string            &                    & Rollenname \\ \hline
		value               & int               &                    & Rollenwert \\ \hline
		id                  & int               &                    & Identifikator der Rolle \\ \hline
	\end{tabular}
\end{table}
\paragraph{Antwort}Die Antwort ist wie folgt aufgebaut:
\begin{table}[H]
	\begin{tabular}{|c|c|c|p{6.5cm}|}
		\hline
		\textbf{Paramtername} & \textbf{Datentyp} & \textbf{Konstante} & \textbf{Kurzbeschreibung}            \\ \hline                
		success             & bool             &                 & Erfolgreich wenn Wert {\glqq true\grqq} ist \\ \hline
		payload             & array            &                 & Leeres Array \\ \hline
	\end{tabular}
\end{table}
\subsubsection{Rolle löschen}
\paragraph{Kurzbeschreibung}Dieser API-Request wird dazu genutzt um eine einzelne Rolle zu löschen.
\paragraph{Anfrage}Folgende Daten werden zu Anfrage benötigt:
\begin{table}[H]
	\begin{tabular}{|c|c|c|p{6.5cm}|}
		\hline
		\textbf{Paramtername} & \textbf{Datentyp} & \textbf{Konstante} & \textbf{Kurzbeschreibung}                                                                                               \\ \hline
		type                & string            & der                & Rolle löschen \\ \hline
		id                  & int               &                    & Identifikator der Rolle \\ \hline
	\end{tabular}
\end{table}
\paragraph{Antwort}Die Antwort ist wie folgt aufgebaut:
\begin{table}[H]
	\begin{tabular}{|c|c|c|p{6.5cm}|}
		\hline
		\textbf{Paramtername} & \textbf{Datentyp} & \textbf{Konstante} & \textbf{Kurzbeschreibung}            \\ \hline                
		success             & bool             &                 & Erfolgreich wenn Wert {\glqq true\grqq} ist \\ \hline
		payload             & string           &                 & Bei Erfolg: {\glqq Role successfully deleted\grqq} \\ \hline
	\end{tabular}
\end{table}
\subsubsection{Nutzerpasswort ändern}
\paragraph{Kurzbeschreibung}Dieser API-Request wird dazu genutzt um ein Nutzerpasswort zu ändern.
\paragraph{Anfrage}Folgende Daten werden zu Anfrage benötigt:
\begin{table}[H]
	\begin{tabular}{|c|c|c|p{6.5cm}|}
		\hline
		\textbf{Paramtername} & \textbf{Datentyp} & \textbf{Konstante} & \textbf{Kurzbeschreibung}                                                                                               \\ \hline
		type                & string            & cup                & Passwort ändern \\ \hline
		id                  & int               &                    & Identifikator des Nutzers \\ \hline
		pwd1                & string            &                    & Inhalt Passwortbox 1 \\ \hline
		pwd2                & string            &                    & Inhalt Passwortbox 2 \\ \hline
	\end{tabular}
\end{table}
\paragraph{Antwort}Die Antwort ist wie folgt aufgebaut:
\begin{table}[H]
	\begin{tabular}{|c|c|c|p{6.5cm}|}
		\hline
		\textbf{Paramtername} & \textbf{Datentyp} & \textbf{Konstante} & \textbf{Kurzbeschreibung}            \\ \hline                
		success             & bool             &                 & Erfolgreich wenn Wert {\glqq true\grqq} ist \\ \hline
		payload             & string           &                 & Bei Erfolg: {\glqq Successfully updated Password!\grqq} \\ \hline
	\end{tabular}
\end{table}
\subsubsection{Nutzeraktivierung umschalten}
\paragraph{Kurzbeschreibung}Dieser API-Request wird dazu genutzt um das Nutzerpasswort eines Nutzers neu zu setzen.
\paragraph{Anfrage}Folgende Daten werden zu Anfrage benötigt:
\begin{table}[H]
	\begin{tabular}{|c|c|c|p{6.5cm}|}
		\hline
		\textbf{Paramtername} & \textbf{Datentyp} & \textbf{Konstante} & \textbf{Kurzbeschreibung}                                                                                               \\ \hline
		type                & string            & teu                & Nutzeraktivierung umschalten \\ \hline
		id                  & int               &                    & Identifikator des Nutzers \\ \hline
	\end{tabular}
\end{table}
\paragraph{Antwort}Die Antwort ist wie folgt aufgebaut:
\begin{table}[H]
	\begin{tabular}{|c|c|c|p{6.5cm}|}
		\hline
		\textbf{Paramtername} & \textbf{Datentyp} & \textbf{Konstante} & \textbf{Kurzbeschreibung}            \\ \hline                
		success             & bool             &                 & Erfolgreich wenn Wert {\glqq true\grqq} ist \\ \hline
		payload             & string           &                 & Bei Erfolg: {\glqq Successfully updated User!\grqq} \\ \hline
	\end{tabular}
\end{table}
\subsubsection{Nutzerpasswort Reset via Mail}
\paragraph{Kurzbeschreibung}Dieser API-Request wird dazu genutzt um eine ein Nutzerpasswort zu ändern.
\paragraph{Anfrage}Folgende Daten werden zu Anfrage benötigt:
\begin{table}[H]
	\begin{tabular}{|c|c|c|p{6.5cm}|}
		\hline
		\textbf{Paramtername} & \textbf{Datentyp} & \textbf{Konstante} & \textbf{Kurzbeschreibung}                                                                                               \\ \hline
		type                & string            & rup                & Nutzerpasswort Reset \\ \hline
		id                  & int               &                    & Identifikator des Nutzers \\ \hline
	\end{tabular}
\end{table}
\paragraph{Antwort}Die Antwort ist wie folgt aufgebaut:
\begin{table}[H]
	\begin{tabular}{|c|c|c|p{6.5cm}|}
		\hline
		\textbf{Paramtername} & \textbf{Datentyp} & \textbf{Konstante} & \textbf{Kurzbeschreibung}            \\ \hline                
		success             & bool             &                 & Erfolgreich wenn Wert {\glqq true\grqq} ist \\ \hline
		payload             & array            &                 & Bei Erfolg: Leeres Array \\ \hline
	\end{tabular}
\end{table}
\subsubsection{Neuen Rang hinzufügen}
\paragraph{Kurzbeschreibung}Dieser API-Request wird dazu genutzt um einen neuen Rang hinzuzufügen.
\paragraph{Anfrage}Folgende Daten werden zu Anfrage benötigt:
\begin{table}[H]
	\begin{tabular}{|c|c|c|p{6.5cm}|}
		\hline
		\textbf{Paramtername} & \textbf{Datentyp} & \textbf{Konstante} & \textbf{Kurzbeschreibung}                                                                                               \\ \hline
		type                & string            & adr                & Rang anlegen \\ \hline
		name                & string            &                    & Rangname \\ \hline
		value               & int               &                    & Rangwert \\ \hline
	\end{tabular}
\end{table}
\paragraph{Antwort}Die Antwort ist wie folgt aufgebaut:
\begin{table}[H]
	\begin{tabular}{|c|c|c|p{6.5cm}|}
		\hline
		\textbf{Paramtername} & \textbf{Datentyp} & \textbf{Konstante} & \textbf{Kurzbeschreibung}            \\ \hline                
		success             & bool             &                 & Erfolgreich wenn Wert {\glqq true\grqq} ist \\ \hline
		payload             & bool             &                 & Bei Erfolg: {\glqq true\grqq} \\ \hline
	\end{tabular}
\end{table}
\subsubsection{Rang löschen}
\paragraph{Kurzbeschreibung}Dieser API-Request wird dazu genutzt um einen Rang zu löschen.
\paragraph{Anfrage}Folgende Daten werden zu Anfrage benötigt:
\begin{table}[H]
	\begin{tabular}{|c|c|c|p{6.5cm}|}
		\hline
		\textbf{Paramtername} & \textbf{Datentyp} & \textbf{Konstante} & \textbf{Kurzbeschreibung}                                                                                               \\ \hline
		type                & string            & dra                & Rang löschen \\ \hline
		id                  & int               &                    & Identifikator des Rangs \\ \hline
	\end{tabular}
\end{table}
\paragraph{Antwort}Die Antwort ist wie folgt aufgebaut:
\begin{table}[H]
	\begin{tabular}{|c|c|c|p{6.5cm}|}
		\hline
		\textbf{Paramtername} & \textbf{Datentyp} & \textbf{Konstante} & \textbf{Kurzbeschreibung}            \\ \hline                
		success             & bool             &                 & Erfolgreich wenn Wert {\glqq true\grqq} ist \\ \hline
		payload             & bool             &                 & Bei Erfolg: {\glqq Rank successfully deleted\grqq} \\ \hline
	\end{tabular}
\end{table}
\subsubsection{Bestehenden Rang ändern}
\paragraph{Kurzbeschreibung}Dieser API-Request wird dazu genutzt um einen bestehenden Rang zu ändern.
\paragraph{Anfrage}Folgende Daten werden zu Anfrage benötigt:
\begin{table}[H]
	\begin{tabular}{|c|c|c|p{6.5cm}|}
		\hline
		\textbf{Paramtername} & \textbf{Datentyp} & \textbf{Konstante} & \textbf{Kurzbeschreibung}                                                                                               \\ \hline
		type                & string            & era                & Rang ändern \\ \hline
		id                  & int               &                    & Identifikator des Rangs \\ \hline
		name                & string            &                    & Rangname \\ \hline
		value               & int               &                    & Rangwert \\ \hline
	\end{tabular}
\end{table}
\paragraph{Antwort}Die Antwort ist wie folgt aufgebaut:
\begin{table}[H]
	\begin{tabular}{|c|c|c|p{6.5cm}|}
		\hline
		\textbf{Paramtername} & \textbf{Datentyp} & \textbf{Konstante} & \textbf{Kurzbeschreibung}            \\ \hline                
		success             & bool             &                 & Erfolgreich wenn Wert {\glqq true\grqq} ist \\ \hline
		payload             & array            &                 & Bei Erfolg: Leeres Array \\ \hline
	\end{tabular}
\end{table}
\subsubsection{Abrufen von Statistiken}
\paragraph{Kurzbeschreibung}Dieser API-Request wird dazu genutzt um Statistiken abzufragen.
\paragraph{Anfrage}Folgende Daten werden zu Anfrage benötigt:
\begin{table}[H]
	\begin{tabular}{|c|c|c|p{6.5cm}|}
		\hline
		\textbf{Paramtername} & \textbf{Datentyp} & \textbf{Konstante} & \textbf{Kurzbeschreibung}                                                                                               \\ \hline
		type                & string            & gsd                & Statistiken anfordern \\ \hline
		data                & array             &                    & Strukturierter Request \\ \hline
	\end{tabular}
\end{table}
\subparagraph{data}Dieses Array enthält Einträge in der nachstehend dargestellten Form haben:
\begin{table}[H]
	\begin{tabular}{|c|c|c|p{6.5cm}|}
		\hline
		\textbf{Paramtername} & \textbf{Datentyp} & \textbf{Konstante} & \textbf{Kurzbeschreibung}    \\ \hline
		data               & Array             &                 & Liste der angeforderten Daten \\ \hline
		src                & string            &                 & Quelle der Daten \\ \hline
	\end{tabular}
\end{table}
\subparagraph{data}Dieses Array enthält Elemente mit Einträgen in der nachstehend dargestellten Form haben:
\begin{table}[H]
	\begin{tabular}{|c|c|c|p{6.5cm}|}
		\hline
		\textbf{Paramtername} & \textbf{Datentyp} & \textbf{Konstante} & \textbf{Kurzbeschreibung}    \\ \hline
		type               & string            &                 & Zeiteinheit (D:Tage, W:Wochen, M:Monate, Y:Jahre) \\ \hline
		Amount             & int               &                 & Anzahl an Einheiten \\ \hline
		ID                 & int               &                 & Identifikator \\ \hline
	\end{tabular}
\end{table}
\paragraph{Antwort}Die Antwort ist wie folgt aufgebaut:
\begin{table}[H]
	\begin{tabular}{|c|c|c|p{6.5cm}|}
		\hline
		\textbf{Paramtername} & \textbf{Datentyp} & \textbf{Konstante} & \textbf{Kurzbeschreibung}            \\ \hline                
		code                & int              &                 & Erfolgreich wenn Wert {\glqq 0\grqq} ist \\ \hline
		result              & string           &                 & Bei Erfolg: {\glqq ack\grqq} \\ \hline
		data                & array            &                 & Strukturiertes Ergebnis \\ \hline
	\end{tabular}
\end{table}
\subparagraph{data}Dieses Array enthält Elemente mit Einträgen in der nachstehend dargestellten Form haben:
\begin{table}[H]
	\begin{tabular}{|c|c|c|p{6.5cm}|}
		\hline
		\textbf{Paramtername} & \textbf{Datentyp} & \textbf{Konstante} & \textbf{Kurzbeschreibung}    \\ \hline
		type               & string            &                 & Zeiteinheit (D:Tage, W:Wochen, M:Monate, Y:Jahre) \\ \hline
		Amount             & int               &                 & Anzahl an Einheiten \\ \hline
		ID                 & int               &                 & Identifikator \\ \hline
		data               & array             &                 & Eintrag entsprechend Doku zu Chart.js \\ \hline
	\end{tabular}
\end{table}
\subsubsection{Abfrage aller Rollennamen}
\paragraph{Kurzbeschreibung}Dieser API-Request wird dazu genutzt um eine Liste aller Rollennamen abzurufen.
\paragraph{Anfrage}Folgende Daten werden zu Anfrage benötigt:
\begin{table}[H]
	\begin{tabular}{|c|c|c|p{6.5cm}|}
		\hline
		\textbf{Paramtername} & \textbf{Datentyp} & \textbf{Konstante} & \textbf{Kurzbeschreibung}                                                                                               \\ \hline
		type                & string            & gar                & Rollennamen abfragen \\ \hline
	\end{tabular}
\end{table}
\paragraph{Antwort}Die Antwort ist wie folgt aufgebaut:
\begin{table}[H]
	\begin{tabular}{|c|c|c|p{6.5cm}|}
		\hline
		\textbf{Paramtername} & \textbf{Datentyp} & \textbf{Konstante} & \textbf{Kurzbeschreibung}            \\ \hline                
		success             & bool             &                 & Erfolgreich wenn Wert {\glqq true\grqq} ist \\ \hline
		payload             & array            &                 & Bei Erfolg: Leeres Array \\ \hline
		data                & array            &                 & Liste mit allen Rollennamen \\ \hline
	\end{tabular}
\end{table}
\subsubsection{Abfrage aller Rangnamen}
\paragraph{Kurzbeschreibung}Dieser API-Request wird dazu genutzt um eine Liste aller Rangnamen abzurufen.
\paragraph{Anfrage}Folgende Daten werden zu Anfrage benötigt:
\begin{table}[H]
	\begin{tabular}{|c|c|c|p{6.5cm}|}
		\hline
		\textbf{Paramtername} & \textbf{Datentyp} & \textbf{Konstante} & \textbf{Kurzbeschreibung}                                                                                               \\ \hline
		type                & string            & grn                & Rangnamen abfragen \\ \hline
	\end{tabular}
\end{table}
\paragraph{Antwort}Die Antwort ist wie folgt aufgebaut:
\begin{table}[H]
	\begin{tabular}{|c|c|c|p{6.5cm}|}
		\hline
		\textbf{Paramtername} & \textbf{Datentyp} & \textbf{Konstante} & \textbf{Kurzbeschreibung}            \\ \hline                
		success             & bool             &                 & Erfolgreich wenn Wert {\glqq true\grqq} ist \\ \hline
		payload             & array            &                 & Bei Erfolg: Leeres Array \\ \hline
		data                & array            &                 & Liste mit allen Rangnamen \\ \hline
	\end{tabular}
\end{table}
\subsubsection{Anfordern Captcha-Bild}
\paragraph{Kurzbeschreibung}Dieser API-Request wird dazu genutzt um ein Base64 encodiertes Captcha-Bild anzufordern.
\paragraph{Anfrage}Folgende Daten werden zu Anfrage benötigt:
\begin{table}[H]
	\begin{tabular}{|c|c|c|p{6.5cm}|}
		\hline
		\textbf{Paramtername} & \textbf{Datentyp} & \textbf{Konstante} & \textbf{Kurzbeschreibung}                                                                                               \\ \hline
		type                & string            & cpa                & Captcha-Bild anfordern \\ \hline
	\end{tabular}
\end{table}
\paragraph{Antwort}Die Antwort ist wie folgt aufgebaut:
\begin{table}[H]
	\begin{tabular}{|c|c|c|p{6.5cm}|}
		\hline
		\textbf{Paramtername} & \textbf{Datentyp} & \textbf{Konstante} & \textbf{Kurzbeschreibung}            \\ \hline                
		success             & bool             &                 & Erfolgreich wenn Wert {\glqq true\grqq} ist \\ \hline
		payload             & array            &                 & Bei Erfolg: Leeres Array \\ \hline
		data                & string           &                 & Base64 codiertes Captcha-JPEG \\ \hline
	\end{tabular}
\end{table}
\subsubsection{Kontaktnachricht senden}
\paragraph{Kurzbeschreibung}Dieser API-Request wird dazu genutzt um eine Kontaktnachricht zu versenden.
\paragraph{Anfrage}Folgende Daten werden zu Anfrage benötigt:
\begin{table}[H]
	\begin{tabular}{|c|c|c|p{6.5cm}|}
		\hline
		\textbf{Paramtername} & \textbf{Datentyp} & \textbf{Konstante} & \textbf{Kurzbeschreibung}                                                                                               \\ \hline
		type                & string            & cmg                & Captcha-Bild anfordern \\ \hline
		cap                 & string            &                    & Nutzereingabe des Captchas \\ \hline
		title               & string            &                    & Betreff der Nachricht \\ \hline
		msg                 & string            &                    & Nachricht \\ \hline
	\end{tabular}
\end{table}
\paragraph{Antwort}Die Antwort ist wie folgt aufgebaut:
\begin{table}[H]
	\begin{tabular}{|c|c|c|p{6.5cm}|}
		\hline
		\textbf{Paramtername} & \textbf{Datentyp} & \textbf{Konstante} & \textbf{Kurzbeschreibung}            \\ \hline                
		success             & bool             &                 & Erfolgreich wenn Wert {\glqq true\grqq} ist \\ \hline
		payload             & array            &                 & Bei Erfolg: Leeres Array \\ \hline
	\end{tabular}
\end{table}
\subsubsection{Liste mit Modulen ohne Rangnamenspezifikation}
\paragraph{Kurzbeschreibung}Dieser API-Request wird dazu genutzt um eine Liste von Modulen zu bekommen, welche keinen spezialisierten Namen eines Ranges haben.
\paragraph{Anfrage}Folgende Daten werden zu Anfrage benötigt:
\begin{table}[H]
	\begin{tabular}{|c|c|c|p{6.5cm}|}
		\hline
		\textbf{Paramtername} & \textbf{Datentyp} & \textbf{Konstante} & \textbf{Kurzbeschreibung}                                                                                               \\ \hline
		type                & string            & rns                & Modul-Liste anfordern \\ \hline
		id                  & int               &                    & Identifikator eines Ranges \\ \hline
	\end{tabular}
\end{table}
\paragraph{Antwort}Die Antwort ist wie folgt aufgebaut:
\begin{table}[H]
	\begin{tabular}{|c|c|c|p{6.5cm}|}
		\hline
		\textbf{Paramtername} & \textbf{Datentyp} & \textbf{Konstante} & \textbf{Kurzbeschreibung}            \\ \hline                
		success             & bool             &                 & Erfolgreich wenn Wert {\glqq true\grqq} ist \\ \hline
		payload             & array            &                 & Liste der Apis \\ \hline
	\end{tabular}
\end{table}
\subparagraph{payload}Dieses Array enthält Elemente mit Einträgen in der nachstehend dargestellten Form haben:
\begin{table}[H]
	\begin{tabular}{|c|c|c|p{6.5cm}|}
		\hline
		\textbf{Paramtername} & \textbf{Datentyp} & \textbf{Konstante} & \textbf{Kurzbeschreibung}    \\ \hline
		name               & string            &                 & Name des Moduls \\ \hline
		id                 & int               &                 & Identifikator des Moduls \\ \hline
	\end{tabular}
\end{table}
\subsubsection{Liste aller modulbasierten Rangnamen}
\paragraph{Kurzbeschreibung}Dieser API-Request wird dazu genutzt um eine Liste mit modulbasierten Namen eines Ranges ab zu fragen.
\paragraph{Anfrage}Folgende Daten werden zu Anfrage benötigt:
\begin{table}[H]
	\begin{tabular}{|c|c|c|p{6.5cm}|}
		\hline
		\textbf{Paramtername} & \textbf{Datentyp} & \textbf{Konstante} & \textbf{Kurzbeschreibung}                                                                                               \\ \hline
		type                & string            & rna                & Modul-Liste anfordern \\ \hline
		id                  & int               &                    & Identifikator eines Ranges \\ \hline
	\end{tabular}
\end{table}
\paragraph{Antwort}Die Antwort ist wie folgt aufgebaut:
\begin{table}[H]
	\begin{tabular}{|c|c|c|p{6.5cm}|}
		\hline
		\textbf{Paramtername} & \textbf{Datentyp} & \textbf{Konstante} & \textbf{Kurzbeschreibung}            \\ \hline                
		success             & bool             &                 & Erfolgreich wenn Wert {\glqq true\grqq} ist \\ \hline
		payload             & array            &                 & Liste der Apis \\ \hline
	\end{tabular}
\end{table}
\subparagraph{payload}Dieses Array enthält Elemente mit Einträgen in der nachstehend dargestellten Form haben:
\begin{table}[H]
	\begin{tabular}{|c|c|c|p{6.5cm}|}
		\hline
		\textbf{Paramtername} & \textbf{Datentyp} & \textbf{Konstante} & \textbf{Kurzbeschreibung}    \\ \hline
		id                      &                   &                 & Identifikator des modulbasierten Rangnamens \\ \hline
		rankname                & string            &                 & Name des Moduls \\ \hline
		modulename              & string            &                 & Name des Moduls \\ \hline
	\end{tabular}
\end{table}
\subsubsection{Modulbasierten Rangnamen hinzufügen}
\paragraph{Kurzbeschreibung}Dieser API-Request wird dazu genutzt um eine Liste mit modulbasierten Namen eines Ranges ab zu fragen.
\paragraph{Anfrage}Folgende Daten werden zu Anfrage benötigt:
\begin{table}[H]
	\begin{tabular}{|c|c|c|p{6.5cm}|}
		\hline
		\textbf{Paramtername} & \textbf{Datentyp} & \textbf{Konstante} & \textbf{Kurzbeschreibung}                                                                                               \\ \hline
		type                & string            & imr                & Modul-Liste anfordern \\ \hline
		rid                 & int               &                    & Identifikator eines Ranges \\ \hline
		aid                 & int               &                    & Identifikator eines Moduls \\ \hline
		name                & string            &                    & modulbasierter Name des Ranges  \\ \hline
	\end{tabular}
\end{table}
\paragraph{Antwort}Die Antwort ist wie folgt aufgebaut:
\begin{table}[H]
	\begin{tabular}{|c|c|c|p{6.5cm}|}
		\hline
		\textbf{Paramtername} & \textbf{Datentyp} & \textbf{Konstante} & \textbf{Kurzbeschreibung}            \\ \hline                
		success             & bool             &                 & Erfolgreich wenn Wert {\glqq true\grqq} ist \\ \hline
		payload             & array            &                 & Leeres Array \\ \hline
	\end{tabular}
\end{table}
\subsubsection{Modulbasierten Rangnamen löschen}
\paragraph{Kurzbeschreibung}Dieser API-Request wird dazu genutzt um eine Liste mit modulbasierten Namen eines Ranges ab zu fragen.
\paragraph{Anfrage}Folgende Daten werden zu Anfrage benötigt:
\begin{table}[H]
	\begin{tabular}{|c|c|c|p{6.5cm}|}
		\hline
		\textbf{Paramtername} & \textbf{Datentyp} & \textbf{Konstante} & \textbf{Kurzbeschreibung}                                                                                               \\ \hline
		type                & string            & imr                & Modul-Liste anfordern \\ \hline
		id                  & int               &                    & Identifikator eines modulbasierten Ranganmens \\ \hline
	\end{tabular}
\end{table}
\paragraph{Antwort}Die Antwort ist wie folgt aufgebaut:
\begin{table}[H]
	\begin{tabular}{|c|c|c|p{6.5cm}|}
		\hline
		\textbf{Paramtername} & \textbf{Datentyp} & \textbf{Konstante} & \textbf{Kurzbeschreibung}            \\ \hline                
		success             & bool             &                 & Erfolgreich wenn Wert {\glqq true\grqq} ist \\ \hline
		payload             & array            &                 & Leeres Array \\ \hline
	\end{tabular}
\end{table}
\subsubsection{Abfrage aller Daten einer API}
\paragraph{Kurzbeschreibung}Dieser API-Request wird dazu genutzt um alle Daten einer API abzufragen.
\paragraph{Anfrage}Folgende Daten werden zu Anfrage benötigt:
\begin{table}[H]
	\begin{tabular}{|c|c|c|p{6.5cm}|}
		\hline
		\textbf{Paramtername} & \textbf{Datentyp} & \textbf{Konstante} & \textbf{Kurzbeschreibung}                                                                                               \\ \hline
		type                & string            & gap                & API-Daten anfordern \\ \hline
		id                  & int               &                    & Identifikator einer API \\ \hline
	\end{tabular}
\end{table}
\paragraph{Antwort}Die Antwort ist wie folgt aufgebaut:
\begin{table}[H]
	\begin{tabular}{|c|c|c|p{6.5cm}|}
		\hline
		\textbf{Paramtername} & \textbf{Datentyp} & \textbf{Konstante} & \textbf{Kurzbeschreibung}            \\ \hline                
		success             & bool             &                 & Erfolgreich wenn Wert {\glqq true\grqq} ist \\ \hline
		payload             & array            &                 & Daten der API \\ \hline
	\end{tabular}
\end{table}
\subparagraph{payload}Dieses Array enthält Elemente mit Einträgen in der nachstehend dargestellten Form haben:
\begin{table}[H]
	\begin{tabular}{|c|c|c|p{6.5cm}|}
		\hline
		\textbf{Paramtername} & \textbf{Datentyp} & \textbf{Konstante} & \textbf{Kurzbeschreibung}    \\ \hline
		id                      &                   &                 & Identifikator der API \\ \hline
		name                    & string            &                 & Name der API \\ \hline
		url                     & string            &                 & Reverse-API-URL der API \\ \hline
	\end{tabular}
\end{table}
\subsubsection{Speichert Daten einer API}
\paragraph{Kurzbeschreibung}Dieser API-Request wird dazu genutzt um alle Daten einer API abzufragen.
\paragraph{Anfrage}Folgende Daten werden zu Anfrage benötigt:
\begin{table}[H]
	\begin{tabular}{|c|c|c|p{6.5cm}|}
		\hline
		\textbf{Paramtername} & \textbf{Datentyp} & \textbf{Konstante} & \textbf{Kurzbeschreibung}                                                                                               \\ \hline
		type                & string            & sap                & API-Daten speichern \\ \hline
		id                  & int               &                    & Identifikator einer API \\ \hline
		name                & string            &                    & Name einer API \\ \hline
		url                 & string            &                    & Url der Reverse-API des Moduls \\ \hline
	\end{tabular}
\end{table}
\paragraph{Antwort}Die Antwort ist wie folgt aufgebaut:
\begin{table}[H]
	\begin{tabular}{|c|c|c|p{6.5cm}|}
		\hline
		\textbf{Paramtername} & \textbf{Datentyp} & \textbf{Konstante} & \textbf{Kurzbeschreibung}            \\ \hline                
		success             & bool             &                 & Erfolgreich wenn Wert {\glqq true\grqq} ist \\ \hline
	\end{tabular}
\end{table}
\subsubsection{Modulbasierte Rechte abfragen}
\paragraph{Kurzbeschreibung}Dieser API-Request wird dazu genutzt um Module abzufragen, für die der Nutzer rechte besitzt.
\paragraph{Anfrage}Folgende Daten werden zu Anfrage benötigt:
\begin{table}[H]
	\begin{tabular}{|c|c|c|p{6.5cm}|}
		\hline
		\textbf{Paramtername} & \textbf{Datentyp} & \textbf{Konstante} & \textbf{Kurzbeschreibung}                                                                                               \\ \hline
		type                & string            & gmr                & API-Daten speichern \\ \hline
		id                  & int               &                    & Identifikator eines Nutzers\\ \hline
	\end{tabular}
\end{table}
\paragraph{Antwort}Die Antwort ist wie folgt aufgebaut:
\begin{table}[H]
	\begin{tabular}{|c|c|c|p{6.5cm}|}
		\hline
		\textbf{Paramtername} & \textbf{Datentyp} & \textbf{Konstante} & \textbf{Kurzbeschreibung}            \\ \hline                
		success             & bool             &                 & Erfolgreich wenn Wert {\glqq true\grqq} ist \\ \hline
		payload             & array            &                 & Daten der API \\ \hline
	\end{tabular}
\end{table}
\subparagraph{payload}Dieses Array enthält Elemente mit Einträgen in der nachstehend dargestellten Form haben:
\begin{table}[H]
	\begin{tabular}{|c|c|c|p{6.5cm}|}
		\hline
		\textbf{Paramtername} & \textbf{Datentyp} & \textbf{Konstante} & \textbf{Kurzbeschreibung}    \\ \hline
		id                      & int               &                 & Identifikator der Berechtigung \\ \hline
		api                     & string            &                 & Name der API \\ \hline
		name                    & string            &                 & Name der Rolle \\ \hline
		apiid                   & int               &                 & Identifikator eines Moduls\\ \hline
		enabled                 & boolean           &                 & Freischaltungsstatus auf Modul\\ \hline
	\end{tabular}
\end{table}
\subsubsection{Name und ID aller APIs abfragen}
\paragraph{Kurzbeschreibung}Dieser API-Request wird dazu genutzt um Module abzufragen, für die der Nutzer rechte besitzt.
\paragraph{Anfrage}Folgende Daten werden zu Anfrage benötigt:
\begin{table}[H]
	\begin{tabular}{|c|c|c|p{6.5cm}|}
		\hline
		\textbf{Paramtername} & \textbf{Datentyp} & \textbf{Konstante} & \textbf{Kurzbeschreibung}                                                                                               \\ \hline
		type                & string            & gam                & API-Daten speichern \\ \hline
	\end{tabular}
\end{table}
\paragraph{Antwort}Die Antwort ist wie folgt aufgebaut:
\begin{table}[H]
	\begin{tabular}{|c|c|c|p{6.5cm}|}
		\hline
		\textbf{Paramtername} & \textbf{Datentyp} & \textbf{Konstante} & \textbf{Kurzbeschreibung}            \\ \hline                
		success             & bool             &                 & Erfolgreich wenn Wert {\glqq true\grqq} ist \\ \hline
		payload             & array            &                 & Daten der API \\ \hline
	\end{tabular}
\end{table}
\subparagraph{payload}Dieses Array enthält Elemente mit Einträgen in der nachstehend dargestellten Form haben:
\begin{table}[H]
	\begin{tabular}{|c|c|c|p{6.5cm}|}
		\hline
		\textbf{Paramtername} & \textbf{Datentyp} & \textbf{Konstante} & \textbf{Kurzbeschreibung}    \\ \hline
		id                      & int               &                 & Identifikator der API \\ \hline
		name                    & string            &                 & Name der API \\ \hline
	\end{tabular}
\end{table}
\subsubsection{Alle erlaubten Rollen abfragen}
\paragraph{Kurzbeschreibung}Dieser API-Request wird dazu genutzt alle Rollen die der aktuelle Benutzer vergeben darf ab zu fragen.
\paragraph{Anfrage}Folgende Daten werden zu Anfrage benötigt:
\begin{table}[H]
	\begin{tabular}{|c|c|c|p{6.5cm}|}
		\hline
		\textbf{Paramtername} & \textbf{Datentyp} & \textbf{Konstante} & \textbf{Kurzbeschreibung}                                                                                               \\ \hline
		type                & string            & sar                & API-Daten speichern \\ \hline
	\end{tabular}
\end{table}
\paragraph{Antwort}Die Antwort ist wie folgt aufgebaut:
\begin{table}[H]
	\begin{tabular}{|c|c|c|p{6.5cm}|}
		\hline
		\textbf{Paramtername} & \textbf{Datentyp} & \textbf{Konstante} & \textbf{Kurzbeschreibung}            \\ \hline                
		success             & bool             &                 & Erfolgreich wenn Wert {\glqq true\grqq} ist \\ \hline
		payload             & array            &                 & Daten der API \\ \hline
	\end{tabular}
\end{table}
\subparagraph{payload}Dieses Array enthält Elemente mit Einträgen in der nachstehend dargestellten Form haben:
\begin{table}[H]
	\begin{tabular}{|c|c|c|p{6.5cm}|}
		\hline
		\textbf{Paramtername} & \textbf{Datentyp} & \textbf{Konstante} & \textbf{Kurzbeschreibung}    \\ \hline
		id                      & int               &                 & Identifikator der Rolle \\ \hline
		name                    & string            &                 & Name der Rolle \\ \hline
		value                   & int               &                 & Wert der Rolle \\ \hline
	\end{tabular}
\end{table}
\subsubsection{Modulbasierte Rolle speichern}
\paragraph{Kurzbeschreibung}Dieser API-Request wird dazu genutzt um Rechte für ein Modul der Datenbank hinzuzufügen.
\paragraph{Anfrage}Folgende Daten werden zu Anfrage benötigt:
\begin{table}[H]
	\begin{tabular}{|c|c|c|p{6.5cm}|}
		\hline
		\textbf{Paramtername} & \textbf{Datentyp} & \textbf{Konstante} & \textbf{Kurzbeschreibung}                                                                                               \\ \hline
		type                & string            & smr                & Modul-Rolle speichern \\ \hline
		module              & int               &                    & Identifikator eines Moduls \\ \hline
		role                & int               &                    & Identifikator einer Rolle \\ \hline
		user                & int               &                    & Identifikator eines Nutzers \\ \hline
	\end{tabular}
\end{table}
\paragraph{Antwort}Die Antwort ist wie folgt aufgebaut:
\begin{table}[H]
	\begin{tabular}{|c|c|c|p{6.5cm}|}
		\hline
		\textbf{Paramtername} & \textbf{Datentyp} & \textbf{Konstante} & \textbf{Kurzbeschreibung}            \\ \hline                
		success             & bool             &                 & Erfolgreich wenn Wert {\glqq true\grqq} ist \\ \hline
		payload             & array            &                 & Leeres Array \\ \hline
	\end{tabular}
\end{table}
\subsubsection{Modulbasierte Rolle löschen}
\paragraph{Kurzbeschreibung}Dieser API-Request wird dazu genutzt um Rechte für ein Modul der Datenbank hinzuzufügen.
\paragraph{Anfrage}Folgende Daten werden zu Anfrage benötigt:
\begin{table}[H]
	\begin{tabular}{|c|c|c|p{6.5cm}|}
		\hline
		\textbf{Paramtername} & \textbf{Datentyp} & \textbf{Konstante} & \textbf{Kurzbeschreibung}                                                                                               \\ \hline
		type                & string            & drm                & Modul-Rolle löschen \\ \hline
		id                  & int               &                    & Identifikator einer Berechtigung \\ \hline
	\end{tabular}
\end{table}
\paragraph{Antwort}Die Antwort ist wie folgt aufgebaut:
\begin{table}[H]
	\begin{tabular}{|c|c|c|p{6.5cm}|}
		\hline
		\textbf{Paramtername} & \textbf{Datentyp} & \textbf{Konstante} & \textbf{Kurzbeschreibung}            \\ \hline                
		success             & bool             &                 & Erfolgreich wenn Wert {\glqq true\grqq} ist \\ \hline
		payload             & array            &                 & Leeres Array \\ \hline
	\end{tabular}
\end{table}
\subsubsection{Mailvalidierungsstatus setzen}
\paragraph{Kurzbeschreibung}Dieser API-Request wird dazu genutzt um die Validierung einer Mailadresse manuell zu setzen.
\paragraph{Anfrage}Folgende Daten werden zu Anfrage benötigt:
\begin{table}[H]
	\begin{tabular}{|c|c|c|p{6.5cm}|}
		\hline
		\textbf{Paramtername} & \textbf{Datentyp} & \textbf{Konstante} & \textbf{Kurzbeschreibung}                                                                                               \\ \hline
		type                & string            & smv                & Mailvalidierungsstatus setzen \\ \hline
		name                & string            &                    & Nutzername \\ \hline
		state               & bool              &                    & Status der Validierung der Mailadresse \\ \hline
	\end{tabular}
\end{table}
\paragraph{Antwort}Die Antwort ist wie folgt aufgebaut:
\begin{table}[H]
	\begin{tabular}{|c|c|c|p{6.5cm}|}
		\hline
		\textbf{Paramtername} & \textbf{Datentyp} & \textbf{Konstante} & \textbf{Kurzbeschreibung}            \\ \hline                
		success             & bool             &                 & Erfolgreich wenn Wert {\glqq true\grqq} ist \\ \hline
		payload             & array            &                 & Leeres Array \\ \hline
	\end{tabular}
\end{table}
\subsubsection{Modulberechtigungen aktualisieren}
\paragraph{Kurzbeschreibung}Dieser API-Request wird dazu genutzt um Rechte für ein Modul in der Datenbank zu ändern.
\paragraph{Anfrage}Folgende Daten werden zu Anfrage benötigt:
\begin{table}[H]
	\begin{tabular}{|c|c|c|p{6.5cm}|}
		\hline
		\textbf{Paramtername} & \textbf{Datentyp} & \textbf{Konstante} & \textbf{Kurzbeschreibung}                                                                                               \\ \hline
		type                & string            & umr                & Modulberechtigungen aktualisieren \\ \hline
		roleid              & int               &                    & Identifikator eines Nutzers \\ \hline
		rightid             & int               &                    & Identifikator einer Rolle \\ \hline
	\end{tabular}
\end{table}
\paragraph{Antwort}Die Antwort ist wie folgt aufgebaut:
\begin{table}[H]
	\begin{tabular}{|c|c|c|p{6.5cm}|}
		\hline
		\textbf{Paramtername} & \textbf{Datentyp} & \textbf{Konstante} & \textbf{Kurzbeschreibung}            \\ \hline                
		success             & bool             &                 & Erfolgreich wenn Wert {\glqq true\grqq} ist \\ \hline
		payload             & string            &                 & Link zum Neuladen der Seite \\ \hline
	\end{tabular}
\end{table}
\subsubsection{Deaktivierungsstatus für Modulrechte eines Nutzer setzen}
\paragraph{Kurzbeschreibung}Dieser API-Request wird dazu genutzt um den Deaktivierungsstatus eines Nutzers für ein Modul in der Datenbank zu ändern.
\paragraph{Anfrage}Folgende Daten werden zu Anfrage benötigt:
\begin{table}[H]
	\begin{tabular}{|c|c|c|p{6.5cm}|}
		\hline
		\textbf{Paramtername} & \textbf{Datentyp} & \textbf{Konstante} & \textbf{Kurzbeschreibung}                                                                                               \\ \hline
		type                & string            & dsr                & Ändern Deaktivierungsstatus eines Nutzers für ein Modul \\ \hline
		state               & bool              &                    & Status der Deaktivierung \\ \hline
		rightid             & int               &                    & Identifikator eines Rechtes \\ \hline
	\end{tabular}
\end{table}
\paragraph{Antwort}Die Antwort ist wie folgt aufgebaut:
\begin{table}[H]
	\begin{tabular}{|c|c|c|p{6.5cm}|}
		\hline
		\textbf{Paramtername} & \textbf{Datentyp} & \textbf{Konstante} & \textbf{Kurzbeschreibung}            \\ \hline                
		success             & bool             &                 & Erfolgreich wenn Wert {\glqq true\grqq} ist \\ \hline
		payload             & string           &                 & Link zum Neuladen der Seite \\ \hline
	\end{tabular}
\end{table}
\subsubsection{Nutzer ohne Rechte für Modul suchen}
\paragraph{Kurzbeschreibung}Dieser API-Request wird dazu genutzt um Nutzer ohne Rechte für ein Modul ab zu fragen.
\paragraph{Anfrage}Folgende Daten werden zu Anfrage benötigt:
\begin{table}[H]
	\begin{tabular}{|c|c|c|p{6.5cm}|}
		\hline
		\textbf{Paramtername} & \textbf{Datentyp} & \textbf{Konstante} & \textbf{Kurzbeschreibung}                                                                                               \\ \hline
		type                & string            & gum                & Liste von Nutzern abfragen \\ \hline
		module              & int               &                    & Identifikator eines Moduls \\ \hline
	\end{tabular}
\end{table}
\paragraph{Antwort}Die Antwort ist wie folgt aufgebaut:
\begin{table}[H]
	\begin{tabular}{|c|c|c|p{6.5cm}|}
		\hline
		\textbf{Paramtername} & \textbf{Datentyp} & \textbf{Konstante} & \textbf{Kurzbeschreibung}            \\ \hline                
		success             & bool             &                 & Erfolgreich wenn Wert {\glqq true\grqq} ist \\ \hline
		payload             & string           &                 & Link zum Neuladen der Seite \\ \hline
	\end{tabular}
\end{table}
\subparagraph{payload}Dieses Array enthält Elemente mit Einträgen in der nachstehend dargestellten Form haben:
\begin{table}[H]
	\begin{tabular}{|c|c|c|p{6.5cm}|}
		\hline
		\textbf{Paramtername} & \textbf{Datentyp} & \textbf{Konstante} & \textbf{Kurzbeschreibung}    \\ \hline
		id                      & int               &                 & Identifikator der Rolle \\ \hline
		name                    & string            &                 & Name des Nutzers \\ \hline
		firstname               & string            &                 & Vorname \\ \hline
		lastname                & string            &                 & Nachname \\ \hline
	\end{tabular}
\end{table}
\subsubsection{Neues Modul anlegen}
\paragraph{Kurzbeschreibung}Dieser API-Request wird dazu genutzt um ein neues Modul an zu legen.
\paragraph{Anfrage}Folgende Daten werden zu Anfrage benötigt:
\begin{table}[H]
	\begin{tabular}{|c|c|c|p{6.5cm}|}
		\hline
		\textbf{Paramtername} & \textbf{Datentyp} & \textbf{Konstante} & \textbf{Kurzbeschreibung}                                                                                               \\ \hline
		type                & string            & cna                & Modul anlegen \\ \hline
		name                & string            &                    & Name des neuen Moduls \\ \hline
		url					& string			&					 & Url der Reverse-Api des Moduls \\ \hline
	\end{tabular}
\end{table}
\paragraph{Antwort}Die Antwort ist wie folgt aufgebaut:
\begin{table}[H]
	\begin{tabular}{|c|c|c|p{6.5cm}|}
		\hline
		\textbf{Paramtername} & \textbf{Datentyp} & \textbf{Konstante} & \textbf{Kurzbeschreibung}            \\ \hline                
		success             & bool             &                 & Erfolgreich wenn Wert {\glqq true\grqq} ist \\ \hline
		payload             & string           &                 & generierter Authentifizierungstoken \\ \hline
	\end{tabular}
\end{table}
\subsubsection{Existenz Mailadresse prüfen}
\paragraph{Kurzbeschreibung}Dieser API-Request wird dazu genutzt um ein neues Modul an zu legen.
\paragraph{Anfrage}Folgende Daten werden zu Anfrage benötigt:
\begin{table}[H]
	\begin{tabular}{|c|c|c|p{6.5cm}|}
		\hline
		\textbf{Paramtername} & \textbf{Datentyp} & \textbf{Konstante} & \textbf{Kurzbeschreibung}                                                                                               \\ \hline
		type                & string            & cma                & Mailadresse prüfen \\ \hline
		email               & string            &                    & Mailadresse \\ \hline
	\end{tabular}
\end{table}
\paragraph{Antwort}Die Antwort ist wie folgt aufgebaut:
\begin{table}[H]
	\begin{tabular}{|c|c|c|p{6.5cm}|}
		\hline
		\textbf{Paramtername} & \textbf{Datentyp} & \textbf{Konstante} & \textbf{Kurzbeschreibung}            \\ \hline                
		success             & bool             &                 & Erfolgreich wenn Wert {\glqq true\grqq} ist \\ \hline
		payload             & string           &                 & Wahr, wenn Mailadresse bereits verwendet wird \\ \hline
	\end{tabular}
\end{table}
\newpage
\section{myUser}
\subsection{Allgemeines} Diese Datei ist enthält ein Formular zum Ändern der persönlichen Daten.
Die Datei ist direkt durch den Nutzer aufrufbar. Sie setzt auch die entsprechende Konstante und bindet alle notwendigen Dateien ein:
\begin{lstlisting}[language=php]
define('NICE_PROJECT', true);
require_once "bin/inc.php";
\end{lstlisting}
\subsection{Allgemeines}
Auf dieser Seite kann der Nutzer seine persönlichen Daten wie Vorname, Nachname und E-Mailadresse ändern.
\subsection{Besonderheiten}
Eine Änderung des Nutzernamens ist nicht möglich.

\newpage
\section{privacy-policy}
\subsection{Allgemeines} Diese Datei zeigt die Datenschutzerklärung an.
Die Datei ist direkt durch den Nutzer aufrufbar. Sie setzt auch die entsprechende Konstante und bindet alle notwendigen Dateien ein:
\begin{lstlisting}[language=php]
	define('NICE_PROJECT', true);
	require_once "bin/inc.php";
\end{lstlisting}
\subsection{Allgemeines}
Auf dieser Seite kann der Nutzer Datenschutzerklärung lesen. Diese Seite ist für jeden Sichtbar.
\subsection{Besonderheiten}
Die Seite wird im Debug-Modus nicht für jeden sichtbar angezeigt. Der Name des Verantwortlichen kann mittels der Konfigurationsparameter \autoref{config:privacy-comp-name}, \autoref{config:privacy-comp-street}, \autoref{config:privacy-comp-city}, \autoref{config:privacy-comp-fon}, \autoref{config:privacy-comp-fax} und \autoref{config:privacy-comp-mail}. Auch kann der Datenschutzverantwortliche mittels der Konfigurationsparameter \autoref{config:privacy-rep-name}, \autoref{config:privacy-rep-pos}, \autoref{config:privacy-rep-street}, \autoref{config:privacy-rep-city}, \autoref{config:privacy-rep-fon}, \autoref{config:privacy-rep-fax} und \autoref{config:privacy-rep-mail} gesetzt werden.

\newpage
\section{rankMgmt}
\subsection{Allgemeines} Diese Datei zeigt eine Auflistung aller Ränge an.
Die Datei ist direkt durch den Nutzer aufrufbar. Sie setzt auch die entsprechende Konstante und bindet alle notwendigen Dateien ein:
\begin{lstlisting}[language=php]
define('NICE_PROJECT', true);
require_once "bin/inc.php";
\end{lstlisting}
\subsection{Allgemeines}
Auf dieser Seite kann ein Administrator eine Liste aller Ränge einsehen. Des Weiteren kann er auch die Ränge löschen und bearbeiten. Es können auch Ränge hinzugefügt werden.
\subsection{Besonderheiten}
Die Seite nutzt zum Ausführen der Funktionen teilweise JavaScript.

\newpage
\section{registration}
\subsection{Allgemeines} Diese Datei zeigt ein Formular zur Selbstregistrierung an.
Die Datei ist direkt durch den Nutzer aufrufbar. Sie setzt auch die entsprechende Konstante und bindet alle notwendigen Dateien ein:
\begin{lstlisting}[language=php]
define('NICE_PROJECT', true);
require_once "bin/inc.php";
\end{lstlisting}
\subsection{Allgemeines}
Auf dieser Seite kann sich Nutzer selbst registrieren.
\subsection{Besonderheiten}
Ein Administrator oder Mitarbeiter kann hier auch Accounts für Nutzer anlegen.

\newpage
\section{resetPwd}
\subsection{Allgemeines} Diese Datei zeigt ein Formular zum ändern des Passwortes, der Nutzer muss angemeldet sein.
Die Datei ist direkt durch den Nutzer aufrufbar. Sie setzt auch die entsprechende Konstante und bindet alle notwendigen Dateien ein:
\begin{lstlisting}[language=php]
define('NICE_PROJECT', true);
require_once "bin/inc.php";
\end{lstlisting}
\subsection{Allgemeines}
Auf dieser Seite kann ein angemeldeter Nutzer sein Passwort ändern.
\subsection{Besonderheiten}
Der Nutzer muss nicht sein altes Passwort wissen. Ein Link zum Aufruf der Seite wird an die Mailadresse des Nutzers versendet.
\newpage
\section{roleMgmt}
\subsection{Allgemeines} Diese Datei zeigt eine Auflistung aller Rollen an.
Die Datei ist direkt durch den Nutzer aufrufbar. Sie setzt auch die entsprechende Konstante und bindet alle notwendigen Dateien ein:
\begin{lstlisting}[language=php]
define('NICE_PROJECT', true);
require_once "bin/inc.php";
\end{lstlisting}
\subsection{Allgemeines}
Auf dieser Seite kann ein Administrator eine Liste aller Rollen einsehen. Des Weiteren kann er auch die Rollen löschen und bearbeiten. Es können auch Rollen hinzugefügt werden.
\subsection{Besonderheiten}
Die Seite nutzt zum Ausführen der Funktionen teilweise JavaScript.

\newpage
\section{statistics}
\subsection{Allgemeines} Diese Datei zeigt statistische Grafen an.
Die Datei ist direkt durch den Nutzer aufrufbar. Sie setzt auch die entsprechende Konstante und bindet alle notwendigen Dateien ein:
\begin{lstlisting}[language=php]
define('NICE_PROJECT', true);
require_once "bin/inc.php";
\end{lstlisting}
\subsection{Allgemeines}
Auf dieser Seite sind mittels {\glqq Chart.js\grqq} erzeugte Diagramme für statistische Daten wie neue oder geänderte Interessenpunkte beziehungsweise Kommentare zu sehen.
\subsection{Besonderheiten}
Diese Seite kann nur durch Mitarbeiter und Administratoren eingesehen werden.

\newpage
\section{uapi}
\newpage
\section{Nutzer API-Spezifikation}\label{uapi}
\subsection{Beschreibung}Diese API dient der Kommunikation zwischen einem Nutzer eines Moduls und den Inhalten die durch {\glqq COSP\grqq} bereit gestellt werden. Hauptsächlich unterstützt sie das Abrufen von Bildern und anderen Informationen. Durch diese API können keine Informationen geändert, gelöscht oder neu hinzugefügt werden. Jedoch können Geschichten und Bilder validiert werden. Diese API wird durch ein HMAC-Verfahren geschützt.
\subsection{Befehlsübersicht}
\begin{longtable}[H]{|c|p{12cm}|}
		\hline
		\textbf{Api-Befehl} & \textbf{Kurzbeschreibung}              \\ \hline
		gpp                 & Vollbild laden          \\ \hline
		gpf                 & Vorschaubild laden            \\ \hline
		gus                 & Geschichte laden \\ \hline
		gas                 & Alle Geschichten einer Liste laden \\ \hline
		vas                 & Geschichte Validieren \\ \hline
		vap                 & Bild validieren \\ \hline
\end{longtable}
\newpage
\subsection{Befehle}
\subsubsection{Vollbildladen}
\paragraph{Kurzbeschreibung}Dieser API-Request wird dazu genutzt um ein Vollbild zu laden.
\paragraph{Anfrage}Folgende Daten werden zu Anfrage benötigt:
\begin{table}[H]
	\begin{tabular}{|c|c|c|p{6.5cm}|}
		\hline
		\textbf{Paramtername} & \textbf{Datentyp} & \textbf{Konstante} & \textbf{Kurzbeschreibung}                                                                                               \\ \hline
		type                & string            & gpp                & Vollbild abrufen \\ \hline
		data                & string            &                    & Token \\ \hline
		seccode             & string            &                    & Security Code \\ \hline
		time                & int               &                    & Timestamp \\ \hline
	\end{tabular}
\end{table}
\paragraph{Antwort}Die Antwort ist das Bild mit einem entsprechendem Header.
\subsubsection{Vorschaubild laden}
\paragraph{Kurzbeschreibung}Dieser API-Request wird dazu genutzt um ein Vorschaubild zu laden.
\paragraph{Anfrage}Folgende Daten werden zu Anfrage benötigt:
\begin{table}[H]
	\begin{tabular}{|c|c|c|p{6.5cm}|}
		\hline
		\textbf{Paramtername} & \textbf{Datentyp} & \textbf{Konstante} & \textbf{Kurzbeschreibung}                                                                                               \\ \hline
		type                & string            & gpf                & Vorschaubild abrufen \\ \hline
		data                & string            &                    & Token \\ \hline
		seccode             & string            &                    & Security Code \\ \hline
		time                & int               &                    & Timestamp \\ \hline
	\end{tabular}
\end{table}
\paragraph{Antwort}Die Antwort ist das Bild mit einem entsprechendem Header.

\subsubsection{Geschichte laden}
\paragraph{Kurzbeschreibung}Dieser API-Request wird dazu genutzt um eine einzelne Geschichte zu laden.
\paragraph{Anfrage}Folgende Daten werden zu Anfrage benötigt:
\begin{table}[H]
	\begin{tabular}{|c|c|c|p{6.5cm}|}
		\hline
		\textbf{Paramtername} & \textbf{Datentyp} & \textbf{Konstante} & \textbf{Kurzbeschreibung}                                                                                               \\ \hline
		type                & string            & gus                & Geschichte abrufen \\ \hline
		data                & string            &                    & Token \\ \hline
		seccode             & string            &                    & Security Code \\ \hline
		time                & int               &                    & Timestamp \\ \hline
	\end{tabular}
\end{table}
\paragraph{Antwort}Die Antwort ist wie folgt aufgebaut:
\begin{table}[H]
	\begin{tabular}{|c|c|c|p{6.5cm}|}
		\hline
		\textbf{Paramtername} & \textbf{Datentyp} & \textbf{Konstante} & \textbf{Kurzbeschreibung}            \\ \hline                
		result              & string           &                 & Erfolgreich wenn Wert {\glqq ack\grqq} ist \\ \hline
		Code                & int              &                 & Erfolgreich wenn Wert {\glqq 0\grqq} ist \\ \hline
		data                & array            &                 & Abgefragter Inhalt \\ \hline
	\end{tabular}
\end{table}
\subparagraph{data}Dieses Array enthält Einträge in der nachstehend dargestellten Form haben:
\begin{table}[H]
	\begin{tabular}{|c|c|c|p{6.5cm}|}
		\hline
		\textbf{Paramtername} & \textbf{Datentyp} & \textbf{Konstante} & \textbf{Kurzbeschreibung}    \\ \hline
		token              & string            &                 & Identifikator der Geschichte \\ \hline
		story              & string            &                 & Inhalt der Geschichte \\ \hline
		title              & string            &                 & Titel der Geschichte \\ \hline
		name               & string            &                 & Nutzername des Erstellers \\ \hline
		date               & timestamp         &                 & Erstellungsdatum \\ \hline
		validatedByUser    & bool              &                 & Wahr, wenn Nutzer bereits Geschichte validiert hat \\ \hline
		validate           & bool              &                 & Wahr, Geschichte validiert ist \\ \hline
		valLink            & string            &                 & Validierungslink \\ \hline
		approval           & bool              &                 & Freischaltstatus \\ \hline
		editable           & bool              &                 & Wahr, wenn Geschichte änderbar \\ \hline
		deleted            & bool              &                 & Wahr, wenn Geschichte als gelöscht gilt \\ \hline
	\end{tabular}
\end{table}

\subsubsection{Alle Geschichten einer Liste laden}
\paragraph{Kurzbeschreibung}Dieser API-Request wird dazu genutzt um  alle Geschichten einer Liste zu laden.
\paragraph{Anfrage}Folgende Daten werden zu Anfrage benötigt:
\begin{table}[H]
	\begin{tabular}{|c|c|c|p{6.5cm}|}
		\hline
		\textbf{Paramtername} & \textbf{Datentyp} & \textbf{Konstante} & \textbf{Kurzbeschreibung}                                                                                               \\ \hline
		type                & string            & gas                & Geschichten abrufen \\ \hline
		data                & string            &                    & Token \\ \hline
		seccode             & string            &                    & Security Code \\ \hline
		time                & int               &                    & Timestamp \\ \hline
	\end{tabular}
\end{table}
\paragraph{Antwort}Die ist ein Array mit Elementen, welche folgende Einträge haben:
\begin{table}[H]
	\begin{tabular}{|c|c|c|p{6.5cm}|}
		\hline
		\textbf{Paramtername} & \textbf{Datentyp} & \textbf{Konstante} & \textbf{Kurzbeschreibung}            \\ \hline                
		token              & string            &                 & Identifikator der Geschichte \\ \hline
		story              & string            &                 & Inhalt der Geschichte \\ \hline
		title              & string            &                 & Titel der Geschichte \\ \hline
		name               & string            &                 & Nutzername des Erstellers \\ \hline
		date               & timestamp         &                 & Erstellungsdatum \\ \hline
		validatedByUser    & bool              &                 & Wahr, wenn Nutzer bereits Geschichte validiert hat \\ \hline
		validate           & bool              &                 & Wahr, Geschichte validiert ist \\ \hline
		valLink            & string            &                 & Validierungslink \\ \hline
		approval           & bool              &                 & Freischaltstatus \\ \hline
		editable           & bool              &                 & Wahr, wenn Geschichte änderbar \\ \hline
		deleted            & bool              &                 & Wahr, wenn Geschichte als gelöscht gilt \\ \hline
	\end{tabular}
\end{table}

\subsubsection{Geschichte validieren}
\paragraph{Kurzbeschreibung}Dieser API-Request wird dazu genutzt um eine einzelne Geschichte zu validieren.
\paragraph{Anfrage}Folgende Daten werden zu Anfrage benötigt:
\begin{table}[H]
	\begin{tabular}{|c|c|c|p{6.5cm}|}
		\hline
		\textbf{Paramtername} & \textbf{Datentyp} & \textbf{Konstante} & \textbf{Kurzbeschreibung}                                                                                               \\ \hline
		type                & string            & vas                & Geschichte validieren \\ \hline
		data                & string            &                    & Token \\ \hline
		seccode             & string            &                    & Security Code \\ \hline
		time                & int               &                    & Timestamp \\ \hline
	\end{tabular}
\end{table}
\paragraph{Antwort}Die Antwort ist wie folgt aufgebaut:
\begin{table}[H]
	\begin{tabular}{|c|c|c|p{6.5cm}|}
		\hline
		\textbf{Paramtername} & \textbf{Datentyp} & \textbf{Konstante} & \textbf{Kurzbeschreibung}            \\ \hline                
		result              & string           &                 & Erfolgreich wenn Wert {\glqq ack\grqq} ist \\ \hline
		Code                & int              &                 & Erfolgreich wenn Wert {\glqq 0\grqq} ist \\ \hline
	\end{tabular}
\end{table}

\subsubsection{Bild validieren}
\paragraph{Kurzbeschreibung}Dieser API-Request wird dazu genutzt um eine einzelnes Bild zu validieren.
\paragraph{Anfrage}Folgende Daten werden zu Anfrage benötigt:
\begin{table}[H]
	\begin{tabular}{|c|c|c|p{6.5cm}|}
		\hline
		\textbf{Paramtername} & \textbf{Datentyp} & \textbf{Konstante} & \textbf{Kurzbeschreibung}                                                                                               \\ \hline
		type                & string            & vap                & Bild validieren \\ \hline
		data                & string            &                    & Token \\ \hline
		seccode             & string            &                    & Security Code \\ \hline
		time                & int               &                    & Timestamp \\ \hline
	\end{tabular}
\end{table}
\paragraph{Antwort}Die Antwort ist wie folgt aufgebaut:
\begin{table}[H]
	\begin{tabular}{|c|c|c|p{6.5cm}|}
		\hline
		\textbf{Paramtername} & \textbf{Datentyp} & \textbf{Konstante} & \textbf{Kurzbeschreibung}            \\ \hline                
		result              & string           &                 & Erfolgreich wenn Wert {\glqq ack\grqq} ist \\ \hline
		Code                & int              &                 & Erfolgreich wenn Wert {\glqq 0\grqq} ist \\ \hline
	\end{tabular}
\end{table}
\newpage
\section{usermgmt}
\subsection{Allgemeines} Diese Datei zeigt eine Auflistung aller Nutzer an.
Die Datei ist direkt durch den Nutzer aufrufbar. Sie setzt auch die entsprechende Konstante und bindet alle notwendigen Dateien ein:
\begin{lstlisting}[language=php]
define('NICE_PROJECT', true);
require_once "bin/inc.php";
\end{lstlisting}
\subsection{Allgemeines}
Auf dieser Seite kann ein Administrator eine Liste aller Nutzer einsehen. Des Weiteren kann er auch die Rolle der Nutzer bearbeiten, das Passwort neu setzen, eine Mail zum Neusetzen des Passwortes versenden oder den Nutzer aktivieren/sperren.
\subsection{Besonderheiten}
Die Seite nutzt zum Ausführen der Funktionen teilweise JavaScript.

\newpage
\section{moduleRights}
\subsection{Allgemeines} Diese Datei zeigt eine Auflistung aller Nutzer an.
Die Datei ist direkt durch den Nutzer aufrufbar. Sie setzt auch die entsprechende Konstante und bindet alle notwendigen Dateien ein:
\begin{lstlisting}[language=php]
	define('NICE_PROJECT', true);
	require_once "bin/inc.php";
\end{lstlisting}
\subsection{Allgemeines}
Auf dieser Seite kann ein Administrator eines Moduls eine Liste aller Nutzer des Moduls einsehen. Des Weiteren kann er auch die Rolle der Nutzer für das Modul bearbeiten.
\subsection{Besonderheiten}
Die Seite nutzt zum Ausführen der Funktionen teilweise JavaScript.

\newpage
\section{validate}
\subsection{Allgemeines} Diese Datei dient dem Validieren der E-Mailadresse.
Die Datei ist direkt durch den Nutzer aufrufbar. Sie setzt auch die entsprechende Konstante und bindet alle notwendigen Dateien ein:
\begin{lstlisting}[language=php]
define('NICE_PROJECT', true);
require_once "bin/inc.php";
\end{lstlisting}
\subsection{Allgemeines}
Mit dem Aufruf dieser Seite über einen gesicherten Link, kann der Nutzer seine Mailadresse bestätigen.
\subsection{Besonderheiten}
Die Seite nutzt zum Ausführen der Funktionen kein JavaScript.

\newpage
\section{validations}
\subsection{Allgemeines} Diese Datei zeigt eine Auflistung aller Rangpunkte an.
Die Datei ist direkt durch den Nutzer aufrufbar. Sie setzt auch die entsprechende Konstante und bindet alle notwendigen Dateien ein:
\begin{lstlisting}[language=php]
define('NICE_PROJECT', true);
require_once "bin/inc.php";
\end{lstlisting}
\subsection{Allgemeines}
Auf dieser Seite kann ein Nutzer eine Liste all seiner Rangpunkte einsehen. Des Weiteren kann ein Administrator eine Liste aller vergebenen Rangpunkte einsehen.
\subsection{Besonderheiten}
Die Seite nutzt zum Ausführen der Funktionen teilweise JavaScript.

\newpage
\section{addRank}
\lstset{
	language=JavaScript,
	extendedchars=true,
	basicstyle= \small\ttfamily,
	showstringspaces=true,
	showspaces=false,
	tabsize=2,
	breaklines=true,
	showtabs=false,
	captionpos=b,
	showlines=true,
	xleftmargin=4.0ex,
	extendedchars=true,
	literate={ä}{{\"a}}1 {ö}{{\"o}}1 {ü}{{\"u}}1 {Ä}{{\"A}}1 {Ö}{{\"O}}1 {Ü}{{\"U}}1,
	breaklines=true,
	postbreak=\mbox{\textcolor{red}{$\hookrightarrow$}\space},
}
\subsection{Allgemeines} Diese Datei enthält alle Funktionen die zusätzlich auf der Seite des Rangmanagements benötigt werden.
Es wird auch die benötigte Variable {\glqq ranknames\grqq} mittels folgendem Code gesetzt:
\begin{lstlisting}[language=JavaScript]
var ranknames = sendApiRequest({type: "grn"}, false).data;
\end{lstlisting}
\subsection{Funktionen}
\subsubsection{checkRankName}
\paragraph{Parameter} Die Funktion besitzt folgende Parameter:
\begin{table}[H]
	\begin{tabular}{|c|p{11cm}|}
		\hline
		\textbf{Parametername} & \textbf{Parameterbeschreibung} \\ \hline
		modify & Wenn Wahr, existiert Rang bereits \\ \hline
	\end{tabular}
\end{table}
\paragraph{Beschreibung} Die Funktion prüft, ob ein Rangname bereits verwendet wird, sofern dieser geändert oder neu gesetzt wird. Die Funktion nutzt folgende Quellen:
\begin{itemize}
	\item Management-API
\end{itemize}
\subsubsection{openModuleBasedRankName}
\paragraph{Parameter} Die Funktion besitzt folgende Parameter:
\begin{table}[H]
	\begin{tabular}{|c|p{11cm}|}
		\hline
		\textbf{Parametername} & \textbf{Parameterbeschreibung} \\ \hline
		id & Identifikator eines Ranges \\ \hline
	\end{tabular}
\end{table}
\paragraph{Beschreibung} Die Funktion lädt alle Daten welche zum Öffnen des Modals zur Verwaltung von Modulbasierten Rangnamen benötigt wird. Die Funktion nutzt folgende Quellen:
\begin{itemize}
	\item Management-API
\end{itemize}
\subsubsection{saveModulBasedRankName}
\paragraph{Parameter} Die Funktion besitzt keine Parameter.
\paragraph{Beschreibung} Die Funktion speichert einen neuen modulbasierten Rangnamen. Die Funktion hat Auswirkungen auf folgende Quellen:
\begin{itemize}
	\item Management-API
\end{itemize}
\subsubsection{deleteModuleBasedRankName}
\paragraph{Parameter} Die Funktion besitzt folgende Parameter:
\begin{table}[H]
	\begin{tabular}{|c|p{11cm}|}
		\hline
		\textbf{Parametername} & \textbf{Parameterbeschreibung} \\ \hline
		id & Identifikator eines modulbasierten Rangnamens \\ \hline
	\end{tabular}
\end{table}
\paragraph{Beschreibung} Die Funktion löscht einen neuen modulbasierten Rangnamen. Die Funktion hat Auswirkungen auf folgende Quellen:
\begin{itemize}
	\item Management-API
\end{itemize}
\newpage
\section{addRole}
\subsection{Allgemeines} Diese Datei enthält alle Funktionen die zusätzlich auf der Seite des Rangmanagements benötigt werden.
Es wird auch die benötigte Variable {\glqq rolenames\grqq} mittels folgendem Code gesetzt:
\begin{lstlisting}[language=JavaScript]
var rolenames = sendApiRequest({type: "gar"}, false).data;
\end{lstlisting}
\subsection{Funktionen}
\subsubsection{checkRoleName}
\paragraph{Parameter} Die Funktion besitzt folgende Parameter:
\begin{table}[H]
	\begin{tabular}{|c|p{11cm}|}
		\hline
		\textbf{Parametername} & \textbf{Parameterbeschreibung} \\ \hline
		modify & Wenn Wahr, existiert Rolle bereits \\ \hline
	\end{tabular}
\end{table}
\paragraph{Beschreibung} Die Funktion prüft, ob ein Rollenname bereits verwendet wird, sofern dieser geändert oder neu gesetzt wird. Die Funktion nutzt folgende Quellen:
\begin{itemize}
	\item Management-API
\end{itemize}
\newpage
\section{coseMainLib}
\subsection{Allgemeines} Diese Datei enthält alle Funktionen, welche an diversen stellen im JavaScript dieser Seite verwendet werden.
Es wird auch die zum Bereitstellen von Tooltipps benötigten Befehle ausgeführt:
\begin{lstlisting}[language=JavaScript]
$(document).ready(function () {
	$('[data-toggle="tooltip"]').tooltip();
});
\end{lstlisting}
\subsection{Funktionen}
\subsubsection{AddNewRole}
\paragraph{Parameter} Die Funktion besitzt keine Parameter.
\paragraph{Beschreibung} Die Funktion sendet eine API-Anfrage zum hinzufügen einer neuen Rolle. Die Funktion hat Auswirkungen auf folgende Quellen:
\begin{itemize}
	\item Management-API
\end{itemize}
\subsubsection{sendApiRequest}
\paragraph{Parameter} Die Funktion besitzt folgende Parameter:
\begin{table}[H]
	\begin{tabular}{|c|p{11cm}|}
		\hline
		\textbf{Parametername} & \textbf{Parameterbeschreibung} \\ \hline
		json   & Daten der Anfrage (JSON-Format) \\ \hline
		reload & Legt fest, ob aktuelle Seite nach erfolgreicher Anfrage neu geladen werden soll \\ \hline
	\end{tabular}
\end{table}
\paragraph{Beschreibung} Die Funktion führt eine Anfrage mit den gegebenen Daten aus und lädt, wenn gewünscht, die Seite neu.
\subsubsection{openEditRoleModal}
\paragraph{Parameter} Die Funktion besitzt folgende Parameter:
\begin{table}[H]
	\begin{tabular}{|c|p{11cm}|}
		\hline
		\textbf{Parametername} & \textbf{Parameterbeschreibung} \\ \hline
		name  & Rollenname \\ \hline
		value & Rollenwert \\ \hline
		id    & Identifikator der Rolle \\ \hline
	\end{tabular}
\end{table}
\paragraph{Beschreibung} Die Funktion öffnet das Modal zum bearbeiten einer Rolle.
\subsubsection{saveEditRoleModal}
\paragraph{Parameter} Die Funktion besitzt keine Parameter.
\paragraph{Beschreibung} Die Funktion speichert die geänderten Werte einer Rolle. Die Funktion hat Auswirkungen auf folgende Quellen:
\begin{itemize}
	\item Management-API
\end{itemize}
\subsubsection{deleteRole}
\paragraph{Parameter} Die Funktion besitzt folgende Parameter:
\begin{table}[H]
	\begin{tabular}{|c|p{11cm}|}
		\hline
		\textbf{Parametername} & \textbf{Parameterbeschreibung} \\ \hline
		id & Identifikator einer Rolle \\ \hline
	\end{tabular}
\end{table}
\paragraph{Beschreibung} Die Funktion löscht eine Rolle. Die Funktion hat Auswirkungen auf folgende Quellen:
\begin{itemize}
	\item Management-API
\end{itemize}
\subsubsection{resetPasswordModalShow}
\paragraph{Parameter} Die Funktion besitzt folgende Parameter:
\begin{table}[H]
	\begin{tabular}{|c|p{11cm}|}
		\hline
		\textbf{Parametername} & \textbf{Parameterbeschreibung} \\ \hline
		username & Nutzername \\ \hline
		id       & Identifikator eines Nutzers \\ \hline
	\end{tabular}
\end{table}
\paragraph{Beschreibung} Die Funktion öffnet ein Modal um das Passwort des ausgewählten Nutzers zu ändern. 
\subsubsection{resetPasswordModalSave}
\paragraph{Parameter} Die Funktion besitzt keine Parameter.
\paragraph{Beschreibung} Die Funktion speichert das geänderte Passwort eines ausgewählten Nutzers. Die Funktion hat Auswirkungen auf folgende Quellen:
\begin{itemize}
	\item Management-API
	\item Tabelle mit Nutzerdaten
\end{itemize}
\subsubsection{enableDisableUser}
\paragraph{Parameter} Die Funktion besitzt folgende Parameter:
\begin{table}[H]
	\begin{tabular}{|c|p{11cm}|}
		\hline
		\textbf{Parametername} & \textbf{Parameterbeschreibung} \\ \hline
		id & Identifikator eines Nutzers \\ \hline
	\end{tabular}
\end{table}
\paragraph{Beschreibung} Die Funktion aktiviert oder deaktiviert einen Nutzer (je nach aktuellem Status der Aktivierung). Die Funktion hat Auswirkungen auf folgende Quellen:
\begin{itemize}
	\item Management-API
	\item Tabelle mit Nutzerdaten
\end{itemize}
\subsubsection{sendPasswordResetMail}
\paragraph{Parameter} Die Funktion besitzt folgende Parameter:
\begin{table}[H]
	\begin{tabular}{|c|p{11cm}|}
		\hline
		\textbf{Parametername} & \textbf{Parameterbeschreibung} \\ \hline
		id & Identifikator eines Nutzers \\ \hline
	\end{tabular}
\end{table}
\paragraph{Beschreibung} Die Funktion sendet dem ausgewählten Nutzer eine Mail zum zurücksetzen des Passwortes.
\subsubsection{AddNewRank}
\paragraph{Parameter} Die Funktion besitzt keine Parameter.
\paragraph{Beschreibung} Die Funktion fügt dem System einen neuen Rang hinzu. Die Funktion hat Auswirkungen auf folgende Quellen:
\begin{itemize}
	\item Management-API
\end{itemize}
\subsubsection{deleteRank}
\paragraph{Parameter} Die Funktion besitzt folgende Parameter:
\begin{table}[H]
	\begin{tabular}{|c|p{11cm}|}
		\hline
		\textbf{Parametername} & \textbf{Parameterbeschreibung} \\ \hline
		id & Identifikator eines Ranges \\ \hline
	\end{tabular}
\end{table}
\paragraph{Beschreibung} Die Funktion löscht den gegebenen Rang. Die Funktion hat Auswirkungen auf folgende Quellen:
\begin{itemize}
	\item Management-API
\end{itemize}
\subsubsection{openEditRankModal}
\paragraph{Parameter} Die Funktion besitzt folgende Parameter:
\begin{table}[H]
	\begin{tabular}{|c|p{11cm}|}
		\hline
		\textbf{Parametername} & \textbf{Parameterbeschreibung} \\ \hline
		name  & Rangname \\ \hline
		value & Rangwert \\ \hline
		id    & Identifikator eines Ranges \\ \hline
	\end{tabular}
\end{table}
\paragraph{Beschreibung} Die Funktion öffnet das Modal zum bearbeiten eines Ranges.
\subsubsection{saveEditRankModal}
\paragraph{Parameter} Die Funktion besitzt keine Parameter.
\paragraph{Beschreibung} Die Funktion speichert die geänderten Informationen eines Ranges. Die Funktion hat Auswirkungen auf folgende Quellen:
\begin{itemize}
	\item Management-API
\end{itemize}
\subsubsection{loadCaptchaContact}
\paragraph{Parameter} Die Funktion besitzt keine Parameter.
\paragraph{Beschreibung} Die Funktion lädt einen Captcha-Code für das Kontaktformular. Die Funktion nutzt folgende Quellen:
\begin{itemize}
	\item Management-API
\end{itemize}
\subsubsection{submitContact}
\paragraph{Parameter} Die Funktion besitzt keine Parameter.
\paragraph{Beschreibung} Die Funktion sendet eine API-Anfrage zum versenden einer Kontaktnachricht an die Mitarbeiter des Projekts.
\subsubsection{setErrorOnInputContact}
\paragraph{Parameter} Die Funktion besitzt folgende Parameter:
\begin{table}[H]
	\begin{tabular}{|c|p{11cm}|}
		\hline
		\textbf{Parametername} & \textbf{Parameterbeschreibung} \\ \hline
		elementid & Identifikator des HTML-Elementes\\ \hline
		state & Status der gesetzt werden soll\\ \hline
		tootip & Tooltipp der angezeigt werden soll\\ \hline
	\end{tabular}
\end{table}
\paragraph{Beschreibung} Die Funktion setzt die Darstellung eines Fehler-Status.
\subsubsection{getCookie}
\paragraph{Parameter} Die Funktion besitzt folgende Parameter:
\begin{table}[H]
	\begin{tabular}{|c|p{11cm}|}
		\hline
		\textbf{Parametername} & \textbf{Parameterbeschreibung} \\ \hline
		name & Name eines Cookies \\ \hline
	\end{tabular}
\end{table}
\paragraph{Beschreibung} Die Funktion fragt den Wert eines Cookies ab.
\subsubsection{testCookie}
\paragraph{Parameter} Die Funktion besitzt folgende Parameter:
\begin{table}[H]
	\begin{tabular}{|c|p{11cm}|}
		\hline
		\textbf{Parametername} & \textbf{Parameterbeschreibung} \\ \hline
		name & Name eines Cookies \\ \hline
	\end{tabular}
\end{table}
\paragraph{Beschreibung} Die Funktion prüft, ob ein bestimmter Cookie gesetzt ist.
\subsubsection{setCookie}
\paragraph{Parameter} Die Funktion besitzt folgende Parameter:
\begin{table}[H]
	\begin{tabular}{|c|p{11cm}|}
		\hline
		\textbf{Parametername} & \textbf{Parameterbeschreibung} \\ \hline
		name   & Name des Cookies \\ \hline
		value  & Wert beziehungsweise Inhalt des Cookies \\ \hline
		exdays & Ablaufdatum \\ \hline
	\end{tabular}
\end{table}
\paragraph{Beschreibung} Die Funktion setzt einen Cookie.
\subsubsection{deleteCookie}
\paragraph{Parameter} Die Funktion besitzt folgende Parameter:
\begin{table}[H]
	\begin{tabular}{|c|p{11cm}|}
		\hline
		\textbf{Parametername} & \textbf{Parameterbeschreibung} \\ \hline
		name & Name eines Cookies \\ \hline
	\end{tabular}
\end{table}
\paragraph{Beschreibung} Die Funktion löscht einen gesetzten Cookie.
\newpage
\section{coseMainLibNg}
\subsection{Allgemeines} Diese Datei enthält alle Funktionen, welche an diversen stellen im JavaScript dieser Seite verwendet werden.
\subsection{Funktionen}
\subsubsection{deleteModuleBasedRole}
\paragraph{Parameter} Die Funktion besitzt folgende Parameter:
\begin{table}[H]
	\begin{tabular}{|c|p{11cm}|}
		\hline
		\textbf{Parametername} & \textbf{Parameterbeschreibung} \\ \hline
		rightId & Identifikator des Modulbasierten Rechtes \\ \hline
	\end{tabular}
\end{table}
\paragraph{Beschreibung} Die Funktion löscht ein Modulbasiertes Recht.
\newpage
\section{DropDownMenuApi}
\subsection{Allgemeines} Diese Datei enthält alle Funktionen die zusätzlich auf der Seite des Nutzermanagements benötigt werden.
\subsection{Funktionen}
\subsubsection{PersonalAreaCollection}
\paragraph{Parameter} Die Funktion besitzt folgende Parameter:
\begin{table}[H]
	\begin{tabular}{|c|p{11cm}|}
		\hline
		\textbf{Parametername} & \textbf{Parameterbeschreibung} \\ \hline
		roleId   & Identifikator der Rolle \\ \hline
		roleName & Name der Rolle \\ \hline
		userId   & Identifikator der Rolle \\ \hline
		userName & Name des Nutzers \\ \hline
	\end{tabular}
\end{table}
\paragraph{Beschreibung} Die Funktion ändert die Rolle des angegebenen Nutzers. Die Funktion hat Auswirkungen auf folgende Quellen:
\begin{itemize}
	\item Management-API
\end{itemize}

\newpage
\section{loadCaptcha}
\subsection{Allgemeines} Diese Datei lädt den Captcha-Code auf der Kontakt-Seite.
Die Ausführung des Codes findet im Browser statt. Das Laden des Captchas wird mittels folgendem Code initialisiert:
\begin{lstlisting}[language=JavaScript]
window.onload = function () {
	loadCaptchaContact();
}
\end{lstlisting}

\newpage
\section{loadCookie}
\subsection{Allgemeines} Diese Datei öffnet das Cookie-Modal auf der Index-Seite, sollte dieses nicht bereits bestätigt worden sein.
Die Ausführung des Codes findet im Browser statt. Das öffnen des Modals erfolgt mittels folgendem Code:
\begin{lstlisting}[language=JavaScript]
window.onload = function (){
	if (testCookie('CookieAccept')){
		if (getCookie('CookieAccept') == 'true')
		{
			setCookie('CookieAccept', true, 3650);
			return;
		}
	}
	$('#CookieBannerModal').modal({
		keyboard: false,
		backdrop: 'static'
	});
}
\end{lstlisting}
Dabei wird auch das schließen des Modals mittels Tastendruck beziehungsweise daneben Clickens verhindert.
\subsection{Funktionen}
\subsubsection{AcceptCookies}
\paragraph{Parameter} Die Funktion besitzt keine Parameter.
\paragraph{Beschreibung} Die Funktion setzt einen Cookie, das der Nutzer Cookies akzeptiert und schließt das Modal.

\newpage
\section{statistics}
\subsection{Allgemeines} Diese Datei dient enthält alle Funktionen, welche für die Statistik-Seite zusätzlich benötigt werden.
Die Ausführung des Codes findet im Browser statt. Hier wird eine Variable für statistische Daten gesetzt:
\begin{lstlisting}[language=JavaScript]
var StatData = {};
\end{lstlisting} 
\subsection{Funktionen}
\subsubsection{generateConfig}
\paragraph{Parameter} Die Funktion besitzt folgende Parameter:
\begin{table}[H]
	\begin{tabular}{|c|p{11cm}|}
		\hline
		\textbf{Parametername} & \textbf{Parameterbeschreibung} \\ \hline
		ID   & Position der Daten im {\glqq StatData\grqq}-Array \\ \hline
		type & Typ der Zeiteinheit (Tag: D, Woche: W, Monat: M, Jahr; Y) \\ \hline
	\end{tabular}
\end{table}
\paragraph{Beschreibung} Die Funktion erzeugt die Konfiguration, welche für eine Anzeige der Daten mittels {\glqq Chart.js\grqq} benötigt wird.
\subsubsection{loadStatisticalData}
\paragraph{Parameter} Die Funktion besitzt keine Parameter.
\paragraph{Beschreibung} Die Funktion lädt alle benötigten Daten mittels API. Die Funktion nutzt folgende Quellen:
\begin{itemize}
	\item Management-API
\end{itemize}

\newpage
\section{ApiManagement}
\subsection{Allgemeines} Diese Datei enthält alle Funktionen, welche nur auf der API-Management-Seite verwendet werden.
\subsection{Funktionen}
\subsubsection{openEditApiModal}
\paragraph{Parameter} Die Funktion besitzt folgende Parameter:
\begin{table}[H]
	\begin{tabular}{|c|p{11cm}|}
		\hline
		\textbf{Parametername} & \textbf{Parameterbeschreibung} \\ \hline
		id & Identifikator einer Api \\ \hline
	\end{tabular}
\end{table}
\paragraph{Beschreibung} Die Funktion öffnet das Modal zum Bearbeiten eines API-Eintrags.
\subsubsection{saveEditedApi}
\paragraph{Parameter} Die Funktion besitzt keine Parameter.
\paragraph{Beschreibung} Die Funktion speichert die bearbeiteten Daten eines Moduls.
\subsubsection{CreateNewApi}
\paragraph{Parameter} Die Funktion besitzt keine Parameter.
\paragraph{Beschreibung} Die Funktion legt ein neues Modul mit den angegeben Daten an.
\newpage
\section{usermanagement}
\subsection{Allgemeines} Diese Datei enthält alle Funktionen, welche nur auf der Nutzer-Management-Seite verwendet werden.
\subsection{Funktionen}
\subsubsection{openEditApiModal}
\paragraph{Parameter} Die Funktion besitzt folgende Parameter:
\begin{table}[H]
	\begin{tabular}{|c|p{11cm}|}
		\hline
		\textbf{Parametername} & \textbf{Parameterbeschreibung} \\ \hline
		id & Identifikator eines Nutzers \\ \hline
	\end{tabular}
\end{table}
\paragraph{Beschreibung} Die Funktion öffnet das Modal zum Bearbeiten der modulbasierten Rechte eines Nutzers.
\subsubsection{saveModuleRoleUsermgmt}
\paragraph{Parameter} Die Funktion besitzt keine Parameter.
\paragraph{Beschreibung} Die Funktion speichert modulbasierte Rechte ab.
\subsubsection{deleteModuleBasedRole}
\paragraph{Parameter} Die Funktion besitzt folgende Parameter:
\begin{table}[H]
	\begin{tabular}{|c|p{11cm}|}
		\hline
		\textbf{Parametername} & \textbf{Parameterbeschreibung} \\ \hline
		rid & Identifikator eines modulbasierten Rechtes \\ \hline
		uid & Identifikator eines Nutzers \\ \hline
	\end{tabular}
\end{table}
\paragraph{Beschreibung} Die Funktion löscht modulbasierte Rechte.
\subsubsection{setMailValidation}
\paragraph{Parameter} Die Funktion besitzt folgende Parameter:
\begin{table}[H]
	\begin{tabular}{|c|p{11cm}|}
		\hline
		\textbf{Parametername} & \textbf{Parameterbeschreibung} \\ \hline
		uname & Nutzername \\ \hline
		value & Status der Validierung der Mailadresse \\ \hline
	\end{tabular}
\end{table}
\paragraph{Beschreibung} Die Funktion setzt den Status der Validierung der Mailadresse.
\newpage
\section{moduleRights (TypeScript)}
\subsection{Allgemeines} Diese Datei enthält alle Funktionen, welche nur auf der Modul-Rechte-Seite verwendet werden.
\subsection{Funktionen}
\subsubsection{changeModuleRightTableBody}
\paragraph{Parameter} Die Funktion besitzt folgende Parameter:
\begin{table}[H]
	\begin{tabular}{|c|p{11cm}|}
		\hline
		\textbf{Parametername} & \textbf{Parameterbeschreibung} \\ \hline
		apiid & Identifikator eines Moduls \\ \hline
		name  & Name des Moduls \\ \hline
	\end{tabular}
\end{table}
\paragraph{Beschreibung} Die Funktion ändert den angezeigten Tabellenbody.
\subsubsection{updateModulright}
\paragraph{Parameter} Die Funktion besitzt folgende Parameter:
\begin{table}[H]
	\begin{tabular}{|c|p{11cm}|}
		\hline
		\textbf{Parametername} & \textbf{Parameterbeschreibung} \\ \hline
		rightId & Identifikator eines Rechtes \\ \hline
		roleId  & Identifikator einer Rolle \\ \hline
	\end{tabular}
\end{table}
\paragraph{Beschreibung} Die Funktion ändert ein bestimmtes modulbasierte Rechte.
\subsubsection{deleteModuleBasedRightMgmt}
\paragraph{Parameter} Die Funktion besitzt folgende Parameter:
\begin{table}[H]
	\begin{tabular}{|c|p{11cm}|}
		\hline
		\textbf{Parametername} & \textbf{Parameterbeschreibung} \\ \hline
		rightId & Identifikator eines Rechtes \\ \hline
	\end{tabular}
\end{table}
\paragraph{Beschreibung} Die Funktion löscht ein bestimmtes modulbasierte Rechte.
\subsubsection{OpenMassAddRightsToModule}
\paragraph{Parameter} Die Funktion besitzt keine Parameter.
\paragraph{Beschreibung} Die Funktion öffnet ein Modal zum Hinzufügen von modulbasierten Rechten für mehrere Nutzer.
\subsubsection{SelectMassAddRoleUserRow}
\paragraph{Parameter} Die Funktion besitzt folgende Parameter:
\begin{table}[H]
	\begin{tabular}{|c|p{11cm}|}
		\hline
		\textbf{Parametername} & \textbf{Parameterbeschreibung} \\ \hline
		RowCounter & Identifikator einer Reihe der Tabelle \\ \hline
	\end{tabular}
\end{table}
\paragraph{Beschreibung} Die Funktion markiert eine Rolle in der Tabelle mit Nutzern als ausgewählt.
\subsubsection{saveAddMassUsers}
\paragraph{Parameter} Die Funktion besitzt keine Parameter.
\paragraph{Beschreibung} Die Funktion fügt modulbasierte Nutzerberechtigungen hinzu.
\newpage
\section{registration (TypeScript)}
\subsection{Allgemeines} Diese Datei enthält alle Funktionen, welche nur auf der Registrierungsseite verwendet werden.
\subsection{Funktionen}
\subsubsection{checkMailAdress}
\paragraph{Parameter} Die Funktion besitzt keine Parameter.
\paragraph{Beschreibung} Die Funktion prüft, ob eine Mailadresse schon verwendet wird.

\end{document}