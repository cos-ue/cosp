\subsection{Allgemeines} Diese Datei enthält alle Funktionen, welche durch die Management-API aufgerufen werden.
\begin{table}[H]
	\begin{tabular}{|c|p{11cm}|}
		\hline
		\textbf{Einbindungspunkt} & inc.php \\ \hline
		\textbf{Einbindungspunkt} & inc-sub.php \\ \hline
	\end{tabular}
\end{table}
Die Datei ist nicht direkt durch den Nutzer aufrufbar, dies wird durch folgenden Code-Ausschnitt sichergestellt:
\begin{lstlisting}[language=php]
if (!defined('NICE_PROJECT')) {
	die('Permission denied.');
}
\end{lstlisting}
Der Globale Wert {\glqq NICE\_PROJECT\grqq} wird durch für den Nutzer valide Aufrufpunkte festgelegt, z.B. {\glqq api.php\grqq}.
\newpage
\subsection{Funktionen}
\subsubsection{changeRole}
\paragraph{Parameter} Die Funktion besitzt folgende Parameter:
\begin{table}[H]
	\begin{tabular}{|c|p{11cm}|}
		\hline
		\textbf{Parametername} & \textbf{Parameterbeschreibung} \\ \hline
		\$json & Array mit Informationen \\ \hline
	\end{tabular}
\end{table}
\subparagraph{\$json}Das Array enthält folgende Elemente:
\begin{table}[H]
	\begin{tabular}{|c|p{11cm}|}
		\hline
		\textbf{Parametername} & \textbf{Parameterbeschreibung} \\ \hline
		username & Nutzername \\ \hline
		role     & Identifikator einer Rolle \\ \hline
	\end{tabular}
\end{table}
\paragraph{Beschreibung} Die Funktion ändert die Rolle eines Nutzers. Die Funktion hat Auswirkungen auf folgende Quellen:
\begin{itemize}
	\item Tabelle mit Nutzerdaten
\end{itemize}
Die Antwort wird als strukturiertes Array an den Aufrufer zurückgegeben.
\subsubsection{generateFalse}
\paragraph{Parameter} Die Funktion besitzt folgende Parameter:
\begin{table}[H]
	\begin{tabular}{|c|p{11cm}|}
		\hline
		\textbf{Parametername} & \textbf{Parameterbeschreibung} \\ \hline
		\$json & Array oder Zeichenkette als Nutzlast der Antwort \\ \hline
	\end{tabular}
\end{table}
\paragraph{Beschreibung} Die Funktion erstellt eine generische Nachricht, welche einen nicht erfolgreichen API-Request signalisiert. Die Antwort wird als strukturiertes Array an den Aufrufer zurückgegeben.
\subsubsection{addNewRole}
\paragraph{Parameter} Die Funktion besitzt folgende Parameter:
\begin{table}[H]
	\begin{tabular}{|c|p{11cm}|}
		\hline
		\textbf{Parametername} & \textbf{Parameterbeschreibung} \\ \hline
		\$json & Array mit Informationen \\ \hline
	\end{tabular}
\end{table}
\subparagraph{\$json}Das Array enthält folgende Elemente:
\begin{table}[H]
	\begin{tabular}{|c|p{11cm}|}
		\hline
		\textbf{Parametername} & \textbf{Parameterbeschreibung} \\ \hline
		name  & Name der neuen Rolle \\ \hline
		value & Wert der neuen Rolle \\ \hline
	\end{tabular}
\end{table}
\paragraph{Beschreibung} Die Funktion fügt eine neue Rolle dem System hinzu. Die Funktion hat Auswirkungen auf folgende Quellen:
\begin{itemize}
	\item Tabelle mit Rollen
\end{itemize}
Die Antwort wird als strukturiertes Array an den Aufrufer zurückgegeben.
\subsubsection{generateSuccessMAPI}
\paragraph{Parameter} Die Funktion besitzt folgende Parameter:
\begin{table}[H]
	\begin{tabular}{|c|p{11cm}|}
		\hline
		\textbf{Parametername} & \textbf{Parameterbeschreibung} \\ \hline
		\$json & Array oder Zeichenkette als Nutzlast der Antwort \\ \hline
	\end{tabular}
\end{table}
\paragraph{Beschreibung} Die Funktion erstellt eine generische Nachricht, welche einen erfolgreichen API-Request signalisiert. Die Antwort wird als strukturiertes Array an den Aufrufer zurückgegeben.
\subsubsection{saveEditRoleMapi}
\paragraph{Parameter} Die Funktion besitzt folgende Parameter:
\begin{table}[H]
	\begin{tabular}{|c|p{11cm}|}
		\hline
		\textbf{Parametername} & \textbf{Parameterbeschreibung} \\ \hline
		\$json & Array mit Informationen \\ \hline
	\end{tabular}
\end{table}
\subparagraph{\$json}Das Array enthält folgende Elemente:
\begin{table}[H]
	\begin{tabular}{|c|p{11cm}|}
		\hline
		\textbf{Parametername} & \textbf{Parameterbeschreibung} \\ \hline
		id    & Identifikator einer Rolle \\ \hline
		value & Wert der Rolle \\ \hline
		name  & Name der Rolle \\ \hline
	\end{tabular}
\end{table}
\paragraph{Beschreibung} Die Funktion aktualisiert eine Rolle. Die Funktion hat Auswirkungen auf folgende Quellen:
\begin{itemize}
	\item Tabelle mit Rollen
\end{itemize}
Die Antwort wird als strukturiertes Array an den Aufrufer zurückgegeben.
\subsubsection{deleteRoleMapi}
\paragraph{Parameter} Die Funktion besitzt folgende Parameter:
\begin{table}[H]
	\begin{tabular}{|c|p{11cm}|}
		\hline
		\textbf{Parametername} & \textbf{Parameterbeschreibung} \\ \hline
		\$id & Identifikator einer Rolle \\ \hline
	\end{tabular}
\end{table}
\paragraph{Beschreibung} Die Funktion löscht eine Rolle. Die Funktion hat Auswirkungen auf folgende Quellen:
\begin{itemize}
	\item Tabelle mit Rollen
\end{itemize}
Die Antwort wird als strukturiertes Array an den Aufrufer zurückgegeben.
\subsubsection{changeUserPasswordMapi}
\paragraph{Parameter} Die Funktion besitzt folgende Parameter:
\begin{table}[H]
	\begin{tabular}{|c|p{11cm}|}
		\hline
		\textbf{Parametername} & \textbf{Parameterbeschreibung} \\ \hline
		\$json & Array mit Informationen \\ \hline
	\end{tabular}
\end{table}
\subparagraph{\$json}Das Array enthält folgende Elemente:
\begin{table}[H]
	\begin{tabular}{|c|p{11cm}|}
		\hline
		\textbf{Parametername} & \textbf{Parameterbeschreibung} \\ \hline
		id   & Identifikator eines Nutzers \\ \hline
		pwd1 & Inhalt des Passwortfeldes 1 \\ \hline
		pwd2 & Inhalt des Passwortfeldes 2 \\ \hline
	\end{tabular}
\end{table}
\paragraph{Beschreibung} Die Funktion ändert das Passwort eines Nutzers. Die Funktion hat Auswirkungen auf folgende Quellen:
\begin{itemize}
	\item Tabelle mit Nutzerdaten
\end{itemize}
Die Antwort wird als strukturiertes Array an den Aufrufer zurückgegeben.
\subsubsection{enableUserMAPI}
\paragraph{Parameter} Die Funktion besitzt folgende Parameter:
\begin{table}[H]
	\begin{tabular}{|c|p{11cm}|}
		\hline
		\textbf{Parametername} & \textbf{Parameterbeschreibung} \\ \hline
		\$uid & Identifikator eines Nutzers \\ \hline
	\end{tabular}
\end{table}
\paragraph{Beschreibung} Die Funktion schaltet einen Nutzer frei oder sperrt diesen (anhängig vom aktuellen Status). Die Funktion hat Auswirkungen auf folgende Quellen:
\begin{itemize}
	\item Tabelle mit Nutzerdaten
\end{itemize}
Die Antwort wird als strukturiertes Array an den Aufrufer zurückgegeben.
\subsubsection{resetUserPasswordMAPI}
\paragraph{Parameter} Die Funktion besitzt folgende Parameter:
\begin{table}[H]
	\begin{tabular}{|c|p{11cm}|}
		\hline
		\textbf{Parametername} & \textbf{Parameterbeschreibung} \\ \hline
		\$uid & Identifikator eines Nutzers \\ \hline
	\end{tabular}
\end{table}
\paragraph{Beschreibung} Die Funktion sendet eine Mail zum zurücksetzen des Passwortes. Die Funktion nutzt folgende Quellen:
\begin{itemize}
	\item Tabelle mit Nutzerdaten
\end{itemize}
Die Antwort wird als strukturiertes Array an den Aufrufer zurückgegeben.
\subsubsection{addNewRank}
\paragraph{Parameter} Die Funktion besitzt folgende Parameter:
\begin{table}[H]
	\begin{tabular}{|c|p{11cm}|}
		\hline
		\textbf{Parametername} & \textbf{Parameterbeschreibung} \\ \hline
		\$json & Array mit Informationen \\ \hline
	\end{tabular}
\end{table}
\subparagraph{\$json}Das Array enthält folgende Elemente:
\begin{table}[H]
	\begin{tabular}{|c|p{11cm}|}
		\hline
		\textbf{Parametername} & \textbf{Parameterbeschreibung} \\ \hline
		value & Wert eines Ranges \\ \hline
		name  & Name eines Ranges \\ \hline
	\end{tabular}
\end{table}
\paragraph{Beschreibung} Die Funktion fügt der Anwendung einen neuen Rang hinzu. Die Funktion hat Auswirkungen auf folgende Quellen:
\begin{itemize}
	\item Tabelle mit Rängen
\end{itemize}
Die Antwort wird als strukturiertes Array an den Aufrufer zurückgegeben.
\subsubsection{deleteRankMapi}
\paragraph{Parameter} Die Funktion besitzt folgende Parameter:
\begin{table}[H]
	\begin{tabular}{|c|p{11cm}|}
		\hline
		\textbf{Parametername} & \textbf{Parameterbeschreibung} \\ \hline
		\$id & Identifikator eines Ranges \\ \hline
	\end{tabular}
\end{table}
\paragraph{Beschreibung} Die Funktion löscht einen Rang. Die Funktion hat Auswirkungen auf folgende Quellen:
\begin{itemize}
	\item Tabelle mit Rängen
\end{itemize}
Die Antwort wird als strukturiertes Array an den Aufrufer zurückgegeben.
\subsubsection{saveEditRankMapi}
\paragraph{Parameter} Die Funktion besitzt folgende Parameter:
\begin{table}[H]
	\begin{tabular}{|c|p{11cm}|}
		\hline
		\textbf{Parametername} & \textbf{Parameterbeschreibung} \\ \hline
		\$json & Array mit Informationen \\ \hline
	\end{tabular}
\end{table}
\subparagraph{\$json}Das Array enthält folgende Elemente:
\begin{table}[H]
	\begin{tabular}{|c|p{11cm}|}
		\hline
		\textbf{Parametername} & \textbf{Parameterbeschreibung} \\ \hline
		id    & Identifikator eines Ranges \\ \hline
		value & Wert des Ranges \\ \hline
		name  & Name des Ranges \\ \hline
	\end{tabular}
\end{table}
\paragraph{Beschreibung} Die Funktion fügt aktualisiert einen Rang. Die Funktion hat Auswirkungen auf folgende Quellen:
\begin{itemize}
	\item Tabelle mit Rängen
\end{itemize}
Die Antwort wird als strukturiertes Array an den Aufrufer zurückgegeben.
\subsubsection{getStatisticalDataAPI}
\paragraph{Parameter} Die Funktion besitzt folgende Parameter:
\begin{table}[H]
	\begin{tabular}{|c|p{11cm}|}
		\hline
		\textbf{Parametername} & \textbf{Parameterbeschreibung} \\ \hline
		\$json & Array mit Informationen \\ \hline
	\end{tabular}
\end{table}
\subparagraph{\$json}Das Array enthält folgende Elemente:
\begin{table}[H]
	\begin{tabular}{|c|p{11cm}|}
		\hline
		\textbf{Parametername} & \textbf{Parameterbeschreibung} \\ \hline
		data & Daten zum Abfragen der Statistiken \\ \hline
	\end{tabular}
\end{table}
\paragraph{Beschreibung} Die Funktion ruft statistische Daten für Chart.js ab. Die Funktion nutzt folgende Quellen:
\begin{itemize}
	\item Tabelle mit statistischen Nutzungsdaten
	\item Tabelle mit Nutzerdaten
	\item Tabelle mit Bildern
	\item Tabelle mit Geschichten
\end{itemize}
Die Antwort wird als strukturiertes Array an den Aufrufer zurückgegeben.
\subsubsection{getAllRoleNamesAPI}
\paragraph{Parameter} Die Funktion besitzt keine Parameter.
\paragraph{Beschreibung} Die Funktion ruft eine Liste aller Rollennamen ab. Die Funktion nutzt folgende Quellen:
\begin{itemize}
	\item Tabelle mit Rollen
\end{itemize}
Die Antwort wird als strukturiertes Array an den Aufrufer zurückgegeben.
\subsubsection{getAllRankNamesAPI}
\paragraph{Parameter} Die Funktion besitzt keine Parameter.
\paragraph{Beschreibung} Die Funktion ruft eine Liste aller Rangnamen ab. Die Funktion nutzt folgende Quellen:
\begin{itemize}
	\item Tabelle mit Rängen
\end{itemize}
Die Antwort wird als strukturiertes Array an den Aufrufer zurückgegeben.
\subsubsection{generateCaptchaAPI}
\paragraph{Parameter} Die Funktion besitzt keine Parameter.
\paragraph{Beschreibung} Die Funktion ruft generiert ein Captcha und sendet das Bild an das Frontend, der Inhalt des Captchas wird ind der Session-Variable des Servers gespeichert. Die Antwort wird als strukturiertes Array an den Aufrufer zurückgegeben.
\subsubsection{sendContactMessage}
\paragraph{Parameter} Die Funktion besitzt folgende Parameter:
\begin{table}[H]
	\begin{tabular}{|c|p{11cm}|}
		\hline
		\textbf{Parametername} & \textbf{Parameterbeschreibung} \\ \hline
		\$json & Array mit Informationen \\ \hline
	\end{tabular}
\end{table}
\subparagraph{\$json}Das Array enthält folgende Elemente:
\begin{table}[H]
	\begin{tabular}{|c|p{11cm}|}
		\hline
		\textbf{Parametername} & \textbf{Parameterbeschreibung} \\ \hline
		cap   & Durch Nutzer eingegebenen Inhalt des Captchas \\ \hline
		msg   & Inhalt der Nachricht \\ \hline
		title & Betreff der Nachricht  \\ \hline
	\end{tabular}
\end{table}
\paragraph{Beschreibung} Die Funktion sendet eine Kontakt-E-Mail an eine in der Konfigurationsdatei (siehe \autoref{config:zentral-mail}) hinterlegten Mailadresse. Die Antwort wird als strukturiertes Array an den Aufrufer zurückgegeben.
\subsubsection{getUnusedApisForRank}
\paragraph{Parameter} Die Funktion besitzt folgende Parameter:
\begin{table}[H]
	\begin{tabular}{|c|p{11cm}|}
		\hline
		\textbf{Parametername} & \textbf{Parameterbeschreibung} \\ \hline
		\$json & Array mit Informationen \\ \hline
	\end{tabular}
\end{table}
\subparagraph{\$json}Das Array enthält folgende Elemente:
\begin{table}[H]
	\begin{tabular}{|c|p{11cm}|}
		\hline
		\textbf{Parametername} & \textbf{Parameterbeschreibung} \\ \hline
		id    & Identifikator eines Ranges \\ \hline
	\end{tabular}
\end{table}
\paragraph{Beschreibung} Die Funktion ruft eine Liste aller Module ab, welche keinen modulbasierten Rangnamen eines konkreten Ranges haben. Die Antwort wird als strukturiertes Array an den Aufrufer zurückgegeben.
\subsubsection{getModuleRankNames}
\paragraph{Parameter} Die Funktion besitzt folgende Parameter:
\begin{table}[H]
	\begin{tabular}{|c|p{11cm}|}
		\hline
		\textbf{Parametername} & \textbf{Parameterbeschreibung} \\ \hline
		\$json & Array mit Informationen \\ \hline
	\end{tabular}
\end{table}
\subparagraph{\$json}Das Array enthält folgende Elemente:
\begin{table}[H]
	\begin{tabular}{|c|p{11cm}|}
		\hline
		\textbf{Parametername} & \textbf{Parameterbeschreibung} \\ \hline
		id    & Identifikator eines Ranges \\ \hline
	\end{tabular}
\end{table}
\paragraph{Beschreibung} Die Funktion ruft eine Liste aller modulbasierten Namen eines Ranges ab. Die Antwort wird als strukturiertes Array an den Aufrufer zurückgegeben.
\subsubsection{insertModuleBasedRankNameMapi}
\paragraph{Parameter} Die Funktion besitzt folgende Parameter:
\begin{table}[H]
	\begin{tabular}{|c|p{11cm}|}
		\hline
		\textbf{Parametername} & \textbf{Parameterbeschreibung} \\ \hline
		\$json & Array mit Informationen \\ \hline
	\end{tabular}
\end{table}
\subparagraph{\$json}Das Array enthält folgende Elemente:
\begin{table}[H]
	\begin{tabular}{|c|p{11cm}|}
		\hline
		\textbf{Parametername} & \textbf{Parameterbeschreibung} \\ \hline
		rid    & Identifikator eines Ranges \\ \hline
		aid    & Identifikator eines Moduls \\ \hline
		name   & modulbasierter Name eines Ranges \\ \hline
	\end{tabular}
\end{table}
\paragraph{Beschreibung} Die Funktion fügt einen neuen modulbasierten Rangnamen hinzu. Die Antwort wird als strukturiertes Array an den Aufrufer zurückgegeben.
\subsubsection{getAllApiData}
\paragraph{Parameter} Die Funktion besitzt folgende Parameter:
\begin{table}[H]
	\begin{tabular}{|c|p{11cm}|}
		\hline
		\textbf{Parametername} & \textbf{Parameterbeschreibung} \\ \hline
		\$json & Array mit Informationen \\ \hline
	\end{tabular}
\end{table}
\subparagraph{\$json}Das Array enthält folgende Elemente:
\begin{table}[H]
	\begin{tabular}{|c|p{11cm}|}
		\hline
		\textbf{Parametername} & \textbf{Parameterbeschreibung} \\ \hline
		id     & Identifikator einer API \\ \hline
	\end{tabular}
\end{table}
\paragraph{Beschreibung} Die Funktion ruft alle Daten der angegebenen API ab.
\subsubsection{saveApiDataAPI}
\paragraph{Parameter} Die Funktion besitzt folgende Parameter:
\begin{table}[H]
	\begin{tabular}{|c|p{11cm}|}
		\hline
		\textbf{Parametername} & \textbf{Parameterbeschreibung} \\ \hline
		\$json & Array mit Informationen \\ \hline
	\end{tabular}
\end{table}
\subparagraph{\$json}Das Array enthält folgende Elemente:
\begin{table}[H]
	\begin{tabular}{|c|p{11cm}|}
		\hline
		\textbf{Parametername} & \textbf{Parameterbeschreibung} \\ \hline
		id     & Identifikator einer API \\ \hline
		name   & Name einer API \\ \hline
		url    & Reverse-API-URL einer API \\ \hline
	\end{tabular}
\end{table}
\paragraph{Beschreibung} Die Funktion speichert geänderte Daten eines Moduls.
\subsubsection{getModulBasedRights}
\paragraph{Parameter} Die Funktion besitzt folgende Parameter:
\begin{table}[H]
	\begin{tabular}{|c|p{11cm}|}
		\hline
		\textbf{Parametername} & \textbf{Parameterbeschreibung} \\ \hline
		\$json & Array mit Informationen \\ \hline
	\end{tabular}
\end{table}
\subparagraph{\$json}Das Array enthält folgende Elemente:
\begin{table}[H]
	\begin{tabular}{|c|p{11cm}|}
		\hline
		\textbf{Parametername} & \textbf{Parameterbeschreibung} \\ \hline
		id     & Identifikator eines Nutzers \\ \hline
	\end{tabular}
\end{table}
\paragraph{Beschreibung} Die Funktion ruft die Module ab, für welche der Nutzer Rechte hat.
\subsubsection{getAllModules}
\paragraph{Parameter} Die Funktion besitzt keine Parameter.
\paragraph{Beschreibung} Die Funktion ruft den Namen und die Id aller Module ab.
\subsubsection{addModuleRoleMapi}
\paragraph{Parameter} Die Funktion besitzt folgende Parameter:
\begin{table}[H]
	\begin{tabular}{|c|p{11cm}|}
		\hline
		\textbf{Parametername} & \textbf{Parameterbeschreibung} \\ \hline
		\$json & Array mit Informationen \\ \hline
	\end{tabular}
\end{table}
\subparagraph{\$json}Das Array enthält folgende Elemente:
\begin{table}[H]
	\begin{tabular}{|c|p{11cm}|}
		\hline
		\textbf{Parametername} & \textbf{Parameterbeschreibung} \\ \hline
		role     & Identifikator eines Rolle \\ \hline
		user     & Identifikator eines Nutzers \\ \hline
		module   & Identifikator eines Moduls \\ \hline
	\end{tabular}
\end{table}
\paragraph{Beschreibung} Die Funktion fügt einem Nutzer eine neue Modul-Rolle hinzu.
\subsubsection{generateErrorMAPI}
\paragraph{Parameter} Die Funktion besitzt folgende Parameter:
\begin{table}[H]
	\begin{tabular}{|c|p{11cm}|}
		\hline
		\textbf{Parametername} & \textbf{Parameterbeschreibung} \\ \hline
		\$json & Array oder Zeichenkette als Nutzlast der Antwort \\ \hline
	\end{tabular}
\end{table}
\paragraph{Beschreibung} Die Funktion erstellt eine generische Nachricht, welche einen nicht erfolgreichen API-Request signalisiert. Die Antwort wird als strukturiertes Array an den Aufrufer zurückgegeben.
\subsubsection{deleteModuleBasedRoleMapi}
\paragraph{Parameter} Die Funktion besitzt folgende Parameter:
\begin{table}[H]
	\begin{tabular}{|c|p{11cm}|}
		\hline
		\textbf{Parametername} & \textbf{Parameterbeschreibung} \\ \hline
		\$json & Array mit Informationen \\ \hline
	\end{tabular}
\end{table}
\subparagraph{\$json}Das Array enthält folgende Elemente:
\begin{table}[H]
	\begin{tabular}{|c|p{11cm}|}
		\hline
		\textbf{Parametername} & \textbf{Parameterbeschreibung} \\ \hline
		id     & Identifikator eines modulbasierten Rechtes \\ \hline
	\end{tabular}
\end{table}
\paragraph{Beschreibung} Die Funktion löscht ein modulbasiertes Recht.
\subsubsection{setMailValidationValue}
\paragraph{Parameter} Die Funktion besitzt folgende Parameter:
\begin{table}[H]
	\begin{tabular}{|c|p{11cm}|}
		\hline
		\textbf{Parametername} & \textbf{Parameterbeschreibung} \\ \hline
		\$json & Array mit Informationen \\ \hline
	\end{tabular}
\end{table}
\subparagraph{\$json}Das Array enthält folgende Elemente:
\begin{table}[H]
	\begin{tabular}{|c|p{11cm}|}
		\hline
		\textbf{Parametername} & \textbf{Parameterbeschreibung} \\ \hline
		name     & Nutzername \\ \hline
		state    & Status der Mailvalidierung \\ \hline
	\end{tabular}
\end{table}
\paragraph{Beschreibung} Die Funktion setzt den Validierungsstatus einer Mailadresse.
\subsubsection{updateModuleRightsMapi}
\paragraph{Parameter} Die Funktion besitzt folgende Parameter:
\begin{table}[H]
	\begin{tabular}{|c|p{11cm}|}
		\hline
		\textbf{Parametername} & \textbf{Parameterbeschreibung} \\ \hline
		\$json & Array mit Informationen \\ \hline
	\end{tabular}
\end{table}
\subparagraph{\$json}Das Array enthält folgende Elemente:
\begin{table}[H]
	\begin{tabular}{|c|p{11cm}|}
		\hline
		\textbf{Parametername} & \textbf{Parameterbeschreibung} \\ \hline
		roleid   & Identifikator einer Rolle \\ \hline
		rightid  & Identifikator eines modulbasierten Rechtes \\ \hline
	\end{tabular}
\end{table}
\paragraph{Beschreibung} Die Funktion aktualisiert die modulbasierten Berechtigungen.
\subsubsection{setDisableModuleRightStateMapi}
\paragraph{Parameter} Die Funktion besitzt folgende Parameter:
\begin{table}[H]
	\begin{tabular}{|c|p{11cm}|}
		\hline
		\textbf{Parametername} & \textbf{Parameterbeschreibung} \\ \hline
		\$json & Array mit Informationen \\ \hline
	\end{tabular}
\end{table}
\subparagraph{\$json}Das Array enthält folgende Elemente:
\begin{table}[H]
	\begin{tabular}{|c|p{11cm}|}
		\hline
		\textbf{Parametername} & \textbf{Parameterbeschreibung} \\ \hline
		state    & Status der Deaktivierung \\ \hline
		rightid  & Identifikator eines modulbasierten Rechtes \\ \hline
	\end{tabular}
\end{table}
\paragraph{Beschreibung} Die Funktion aktualisiert den Deaktivierungsstatus eines Modulrechtes.
\subsubsection{getUnsubscribedUserForModule}
\paragraph{Parameter} Die Funktion besitzt folgende Parameter:
\begin{table}[H]
	\begin{tabular}{|c|p{11cm}|}
		\hline
		\textbf{Parametername} & \textbf{Parameterbeschreibung} \\ \hline
		\$json & Array mit Informationen \\ \hline
	\end{tabular}
\end{table}
\subparagraph{\$json}Das Array enthält folgende Elemente:
\begin{table}[H]
	\begin{tabular}{|c|p{11cm}|}
		\hline
		\textbf{Parametername} & \textbf{Parameterbeschreibung} \\ \hline
		state    & Status der Deaktivierung \\ \hline
		module   & Identifikator eines Moduls \\ \hline
	\end{tabular}
\end{table}
\paragraph{Beschreibung} Die Funktion ruft eine Liste aller Benutzer ohne eine Berechtigung für das gegebene Modul ab.
\subsubsection{addNewModuleApi}
\paragraph{Parameter} Die Funktion besitzt folgende Parameter:
\begin{table}[H]
	\begin{tabular}{|c|p{11cm}|}
		\hline
		\textbf{Parametername} & \textbf{Parameterbeschreibung} \\ \hline
		\$json & Array mit Informationen \\ \hline
	\end{tabular}
\end{table}
\subparagraph{\$json}Das Array enthält folgende Elemente:
\begin{table}[H]
	\begin{tabular}{|c|p{11cm}|}
		\hline
		\textbf{Parametername} & \textbf{Parameterbeschreibung} \\ \hline
		name    & Name des neuen Moduls \\ \hline
		url     & Url der Reverse-Api des Moduls \\ \hline
	\end{tabular}
\end{table}
\paragraph{Beschreibung} Die Funktion fügt COSP ein neues Modul hinzu.
\subsubsection{checkMailAddressExistentAPI}
\paragraph{Parameter} Die Funktion besitzt folgende Parameter:
\begin{table}[H]
	\begin{tabular}{|c|p{11cm}|}
		\hline
		\textbf{Parametername} & \textbf{Parameterbeschreibung} \\ \hline
		\$json & Array mit Informationen \\ \hline
	\end{tabular}
\end{table}
\subparagraph{\$json}Das Array enthält folgende Elemente:
\begin{table}[H]
	\begin{tabular}{|c|p{11cm}|}
		\hline
		\textbf{Parametername} & \textbf{Parameterbeschreibung} \\ \hline
		email    & Emailadresse \\ \hline
	\end{tabular}
\end{table}
\paragraph{Beschreibung} Die Funktion prüft ob eine Mailadresse bereits verwendet wird.