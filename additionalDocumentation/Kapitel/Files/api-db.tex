\subsection{Allgemeines} Diese Datei enthält alle Funktionen um die Tabelle mit API's zu verwalten.
\begin{table}[H]
	\begin{tabular}{|c|p{11cm}|}
		\hline
		\textbf{Einbindungspunkt} & inc-db.php \\ \hline
		\textbf{Einbindungspunkt} & inc-db-sub.php \\ \hline
	\end{tabular}
\end{table}
Die Datei ist nicht direkt durch den Nutzer aufrufbar, dies wird durch folgenden Code-Ausschnitt sichergestellt:
\begin{lstlisting}[language=php]
if (!defined('NICE_PROJECT')) {
	die('Permission denied.');
}
\end{lstlisting}
Der Globale Wert {\glqq NICE\_PROJECT\grqq} wird durch für den Nutzer valide Aufrufpunkte festgelegt, z.B. {\glqq api.php\grqq}.
\newpage
\subsection{Funktionen}
\subsubsection{getAllApiTokens}
\paragraph{Parameter} Die Funktion besitzt kene Parameter.
\paragraph{Beschreibung} Die Funktion ruft eine Liste aller API-Token aus der Datenbank ab. Die Funktion nutzt folgende Quellen:
\begin{itemize}
	\item Tabelle mit API-Berechtigungsschlüsseln
\end{itemize}
Die Antwort wird als strukturiertes Array an den Aufrufer zurückgegeben.
\subsubsection{getAllApiTokensOrderedByToken}
\paragraph{Parameter} Die Funktion besitzt kene Parameter.
\paragraph{Beschreibung} Die Funktion ruft eine Liste aller API-Token aus der Datenbank ab. Die Antwort wird hat als Array-Schlüssel die Token der einzelnen Module. Die Funktion nutzt folgende Quellen:
\begin{itemize}
	\item Tabelle mit API-Berechtigungsschlüsseln
\end{itemize}
Die Antwort wird als strukturiertes Array an den Aufrufer zurückgegeben.
\subsubsection{addApiToken}
\paragraph{Parameter} Die Funktion besitzt folgende Parameter:
\begin{table}[H]
	\begin{tabular}{|c|p{11cm}|}
		\hline
		\textbf{Parametername} & \textbf{Parameterbeschreibung} \\ \hline
		\$result & Array mit benötigten Informationen \\ \hline
	\end{tabular}
\end{table}
\subparagraph{\$result}Das Array enthält folgende Elemente:
\begin{table}[H]
	\begin{tabular}{|c|p{11cm}|}
		\hline
		\textbf{Parametername} & \textbf{Parameterbeschreibung} \\ \hline
		name & Name des neuen Moduls \\ \hline
		rapi & Pfad zur Reverse-API des Moduls \\ \hline
	\end{tabular}
\end{table}
\paragraph{Beschreibung} Die Funktion fügt {\glqq COSP\grqq} ein neues Modul hinzu. Die Funktion hat Auswirkungen auf folgende Quellen:
\begin{itemize}
	\item Tabelle mit API-Berechtigungsschlüsseln
\end{itemize}
Die Antwort wird als strukturiertes Array an den Aufrufer zurückgegeben.
\subsubsection{deleteApiToken}
\paragraph{Parameter} Die Funktion besitzt folgende Parameter:
\begin{table}[H]
	\begin{tabular}{|c|p{11cm}|}
		\hline
		\textbf{Parametername} & \textbf{Parameterbeschreibung} \\ \hline
		\$token & Token eines Moduls \\ \hline
	\end{tabular}
\end{table}
\paragraph{Beschreibung} Die Funktion löscht einen Modul Zugangstoken aus {\glqq COSP\grqq}. Die Funktion hat Auswirkungen auf folgende Quellen:
\begin{itemize}
	\item Tabelle mit API-Berechtigungsschlüsseln
\end{itemize}
Die Funktion besitzt keinen Rückgabewert.
\subsubsection{getApiTokenById}
\paragraph{Parameter} Die Funktion besitzt folgende Parameter:
\begin{table}[H]
	\begin{tabular}{|c|p{11cm}|}
		\hline
		\textbf{Parametername} & \textbf{Parameterbeschreibung} \\ \hline
		\$id & Identifikator eines Moduls \\ \hline
	\end{tabular}
\end{table}
\paragraph{Beschreibung} Die Funktion ruft die Daten einer API aus der Datenbank ab. Die Funktion nutzt folgende Quellen:
\begin{itemize}
	\item Tabelle mit API-Berechtigungsschlüsseln
\end{itemize}
Die Antwort wird als strukturiertes Array an den Aufrufer zurückgegeben.
\subsubsection{updateApiData}
\paragraph{Parameter} Die Funktion besitzt folgende Parameter:
\begin{table}[H]
	\begin{tabular}{|c|p{11cm}|}
		\hline
		\textbf{Parametername} & \textbf{Parameterbeschreibung} \\ \hline
		\$id   & Identifikator eines Moduls \\ \hline
		\$name & Name des Moduls \\ \hline
		\$url  & Revers-API-Url des Moduls \\ \hline
	\end{tabular}
\end{table}
\paragraph{Beschreibung} Die Funktion ändert Daten einer API aus der Datenbank ab. Die Funktion hat Auswirkungen auf folgende Quellen:
\begin{itemize}
	\item Tabelle mit API-Berechtigungsschlüsseln
\end{itemize}