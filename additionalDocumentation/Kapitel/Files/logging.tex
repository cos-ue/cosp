\subsection{Allgemeines} Diese Datei enthält alle Funktionen zum Zugriff auf die Tabelle mit statistischen Daten zur Nutzungsanalyse.
\begin{table}[H]
	\begin{tabular}{|c|p{11cm}|}
		\hline
		\textbf{Einbindungspunkt} & inc-db.php \\ \hline
		\textbf{Einbindungspunkt} & inc-db-sub.php \\ \hline
	\end{tabular}
\end{table}
Die Datei ist nicht direkt durch den Nutzer aufrufbar, dies wird durch folgenden Code-Ausschnitt sichergestellt:
\begin{lstlisting}[language=php]
if (!defined('NICE_PROJECT')) {
	die('Permission denied.');
}
\end{lstlisting}
Der Globale Wert {\glqq NICE\_PROJECT\grqq} wird durch für den Nutzer valide Aufrufpunkte festgelegt, z.B. {\glqq api.php\grqq}.
\newpage
\subsection{Funktionen}
\subsubsection{insertLogUniqueVisitors}
\paragraph{Parameter} Die Funktion besitzt folgende Parameter:
\begin{table}[H]
	\begin{tabular}{|c|p{11cm}|}
		\hline
		\textbf{Parametername} & \textbf{Parameterbeschreibung} \\ \hline
		\$ip   & IP-Adresse des Aufrufers \\ \hline
		\$type & Typ des Aufrufs ({\glqq guest\grqq} oder {\glqq user\grqq}) \\ \hline
	\end{tabular}
\end{table}
\paragraph{Beschreibung} Die Funktion fügt einen statistischen Eintrag zum aktuellen Aufruf hinzu. Die Funktion hat Auswirkungen auf folgende Quellen:
\begin{itemize}
	\item Tabelle mit statistischen Nutzungsdaten
\end{itemize}
Die Antwort wird als strukturiertes Array an den Aufrufer zurückgegeben.
\subsubsection{getStatisticalDataLastWeeks}
\paragraph{Parameter} Die Funktion besitzt folgende Parameter:
\begin{table}[H]
	\begin{tabular}{|c|p{11cm}|}
		\hline
		\textbf{Parametername} & \textbf{Parameterbeschreibung} \\ \hline
		\$number & Anzahl an Zeiteinheiten \\ \hline
	\end{tabular}
\end{table}
\paragraph{Beschreibung} Die Funktion ruft alle statistischen Daten der angegeben letzten Wochen ab. Die Funktion nutzt folgende Quellen:
\begin{itemize}
	\item Tabelle mit statistischen Nutzungsdaten
\end{itemize}
Die Antwort wird als strukturiertes Array an den Aufrufer zurückgegeben.
\subsubsection{getStatisticalDataLastMonth}
\paragraph{Parameter} Die Funktion besitzt folgende Parameter:
\begin{table}[H]
	\begin{tabular}{|c|p{11cm}|}
		\hline
		\textbf{Parametername} & \textbf{Parameterbeschreibung} \\ \hline
		\$number & Anzahl an Zeiteinheiten \\ \hline
	\end{tabular}
\end{table}
\paragraph{Beschreibung} Die Funktion ruft alle statistischen Daten der angegeben letzten Monate ab. Die Funktion nutzt folgende Quellen:
\begin{itemize}
	\item Tabelle mit statistischen Nutzungsdaten
\end{itemize}
Die Antwort wird als strukturiertes Array an den Aufrufer zurückgegeben.
\subsubsection{getStatisticalDataLastYear}
\paragraph{Parameter} Die Funktion besitzt folgende Parameter:
\begin{table}[H]
	\begin{tabular}{|c|p{11cm}|}
		\hline
		\textbf{Parametername} & \textbf{Parameterbeschreibung} \\ \hline
		\$number & Anzahl an Zeiteinheiten \\ \hline
	\end{tabular}
\end{table}
\paragraph{Beschreibung} Die Funktion ruft alle statistischen Daten der angegeben letzten Jahre ab. Die Funktion nutzt folgende Quellen:
\begin{itemize}
	\item Tabelle mit statistischen Nutzungsdaten
\end{itemize}
Die Antwort wird als strukturiertes Array an den Aufrufer zurückgegeben.
\subsubsection{getStatisticalDataLastDays}
\paragraph{Parameter} Die Funktion besitzt folgende Parameter:
\begin{table}[H]
	\begin{tabular}{|c|p{11cm}|}
		\hline
		\textbf{Parametername} & \textbf{Parameterbeschreibung} \\ \hline
		\$number & Anzahl an Zeiteinheiten \\ \hline
	\end{tabular}
\end{table}
\paragraph{Beschreibung} Die Funktion ruft alle statistischen Daten der angegeben letzten Tage ab. Die Funktion nutzt folgende Quellen:
\begin{itemize}
	\item Tabelle mit statistischen Nutzungsdaten
\end{itemize}
Die Antwort wird als strukturiertes Array an den Aufrufer zurückgegeben.