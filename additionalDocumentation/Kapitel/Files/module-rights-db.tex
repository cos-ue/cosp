\subsection{Allgemeines} Diese Datei enthält alle Funktionen zum Zugriff auf die Tabelle mit modulbasierten Rechten.
\begin{table}[H]
	\begin{tabular}{|c|p{11cm}|}
		\hline
		\textbf{Einbindungspunkt} & inc-db.php \\ \hline
		\textbf{Einbindungspunkt} & inc-db-sub.php \\ \hline
	\end{tabular}
\end{table}
Die Datei ist nicht direkt durch den Nutzer aufrufbar, dies wird durch folgenden Code-Ausschnitt sichergestellt:
\begin{lstlisting}[language=php]
	if (!defined('NICE_PROJECT')) {
		die('Permission denied.');
	}
\end{lstlisting}
Der Globale Wert {\glqq NICE\_PROJECT\grqq} wird durch für den Nutzer valide Aufrufpunkte festgelegt, z.B. {\glqq api.php\grqq}.
\newpage
\subsection{Funktionen}
\subsubsection{getAllModuleBasedRightsByUserID}
\paragraph{Parameter} Die Funktion besitzt folgende Parameter:
\begin{table}[H]
	\begin{tabular}{|c|p{11cm}|}
		\hline
		\textbf{Parametername} & \textbf{Parameterbeschreibung} \\ \hline
		\$id & Identifikator eines Nutzers \\ \hline
	\end{tabular}
\end{table}
\paragraph{Beschreibung} Die Funktion ruft eine Liste aller modulbasierten Rechte eines Nutzers ab. Die Funktion nutzt folgende Quellen:
\begin{itemize}
	\item Tabelle mit modulbasierten Rechten
	\item Tabelle mit Modulen
	\item Tabelle mit Rollen
\end{itemize}
Die Antwort wird als strukturiertes Array an den Aufrufer zurückgegeben.
\subsubsection{addModuleBasedRole}
\paragraph{Parameter} Die Funktion besitzt folgende Parameter:
\begin{table}[H]
	\begin{tabular}{|c|p{11cm}|}
		\hline
		\textbf{Parametername} & \textbf{Parameterbeschreibung} \\ \hline
		\$nameId & Identifikator eines Nutzers \\ \hline
		\$roleId & Identifikator einer Rolle \\ \hline
		\$appID  & Identifikator einer Api \\ \hline
	\end{tabular}
\end{table}
\paragraph{Beschreibung} Die Funktion fügt modulbasierte Rechte für einen Nutzer ein. Die Funktion hat Auswirkungen auf folgende Quellen:
\begin{itemize}
	\item Tabelle mit modulbasierten Rechten
\end{itemize}
Die Antwort wird als strukturiertes Array an den Aufrufer zurückgegeben.
\subsubsection{getModuleBasedRightsByID}
\paragraph{Parameter} Die Funktion besitzt folgende Parameter:
\begin{table}[H]
	\begin{tabular}{|c|p{11cm}|}
		\hline
		\textbf{Parametername} & \textbf{Parameterbeschreibung} \\ \hline
		\$id & Identifikator eines Modulrechtes \\ \hline
	\end{tabular}
\end{table}
\paragraph{Beschreibung} Die Funktion ruft ein modulbasiertes Rechte ab. Die Funktion nutzt folgende Quellen:
\begin{itemize}
	\item Tabelle mit modulbasierten Rechten
	\item Tabelle mit Modulen
	\item Tabelle mit Rollen
\end{itemize}
Die Antwort wird als strukturiertes Array an den Aufrufer zurückgegeben.
\subsubsection{deleteModuleBasedRole}
\paragraph{Parameter} Die Funktion besitzt folgende Parameter:
\begin{table}[H]
	\begin{tabular}{|c|p{11cm}|}
		\hline
		\textbf{Parametername} & \textbf{Parameterbeschreibung} \\ \hline
		\$rid & Identifikator eines Modulrechtes \\ \hline
		\$uid & Identifikator eines Nutzers \\ \hline
	\end{tabular}
\end{table}
\paragraph{Beschreibung} Die Funktion löscht ein modulbasiertes Recht ab. Die Funktion hat Auswirkung folgende Quellen:
\begin{itemize}
	\item Tabelle mit modulbasierten Rechten
\end{itemize}
Die Antwort wird als strukturiertes Array an den Aufrufer zurückgegeben.
\subsubsection{getAllModuleBasedRightsByModulID}
\paragraph{Parameter} Die Funktion besitzt folgende Parameter:
\begin{table}[H]
	\begin{tabular}{|c|p{11cm}|}
		\hline
		\textbf{Parametername} & \textbf{Parameterbeschreibung} \\ \hline
		\$id & Identifikator eines Moduls \\ \hline
	\end{tabular}
\end{table}
\paragraph{Beschreibung} Die Funktion ruft eine Liste aller modulbasierten Rechte eines Moduls ab. Die Funktion nutzt folgende Quellen:
\begin{itemize}
	\item Tabelle mit modulbasierten Rechten
	\item Tabelle mit Modulen
	\item Tabelle mit Rollen
	\item Tabelle mit Nutzerinformationen
\end{itemize}
Die Antwort wird als strukturiertes Array an den Aufrufer zurückgegeben.
\subsubsection{updateModulRights}
\paragraph{Parameter} Die Funktion besitzt folgende Parameter:
\begin{table}[H]
	\begin{tabular}{|c|p{11cm}|}
		\hline
		\textbf{Parametername} & \textbf{Parameterbeschreibung} \\ \hline
		\$rightID & Identifikator eines Modulrechtes \\ \hline
		\$roleID  & Identifikator einer Rolle \\ \hline
	\end{tabular}
\end{table}
\paragraph{Beschreibung} Die Funktion aktualisiert ein modulbasiertes Recht. Die Funktion hat Auswirkung folgende Quellen:
\begin{itemize}
	\item Tabelle mit modulbasierten Rechten
\end{itemize}
Die Antwort wird als strukturiertes Array an den Aufrufer zurückgegeben.
\subsubsection{getPermissionUsersOfModule}
\paragraph{Parameter} Die Funktion besitzt folgende Parameter:
\begin{table}[H]
	\begin{tabular}{|c|p{11cm}|}
		\hline
		\textbf{Parametername} & \textbf{Parameterbeschreibung} \\ \hline
		\$aid                & Identifikator eines Moduls \\ \hline
		\$ReqPermissionValue & Benötigter Berechtigungswert \\ \hline
	\end{tabular}
\end{table}
\paragraph{Beschreibung} Die Funktion ruft eine Liste aller Nutzeridentifikatoren eines Moduls ab einer bestimmten Berechtigung ab. Die Funktion nutzt folgende Quellen:
\begin{itemize}
	\item Tabelle mit modulbasierten Rechten
	\item Tabelle mit Rollen
\end{itemize}
Die Antwort wird als strukturiertes Array an den Aufrufer zurückgegeben.
\subsubsection{updateDisableStateModuleRight}
\paragraph{Parameter} Die Funktion besitzt folgende Parameter:
\begin{table}[H]
	\begin{tabular}{|c|p{11cm}|}
		\hline
		\textbf{Parametername} & \textbf{Parameterbeschreibung} \\ \hline
		\$rightID & Identifikator eines Rechtes \\ \hline
		\$state   & Status der Deaktivierung \\ \hline
	\end{tabular}
\end{table}
\paragraph{Beschreibung} Die Funktion ändert den Deaktivierungsstatus einer Modulberechtigung. Die Funktion hat Auswirkungen auf folgende Quellen:
\begin{itemize}
	\item Tabelle mit modulbasierten Rechten
\end{itemize}
Die Antwort wird als strukturiertes Array an den Aufrufer zurückgegeben.
\subsubsection{getModuleRightByUsernameApiToken}
\paragraph{Parameter} Die Funktion besitzt folgende Parameter:
\begin{table}[H]
	\begin{tabular}{|c|p{11cm}|}
		\hline
		\textbf{Parametername} & \textbf{Parameterbeschreibung} \\ \hline
		\$username & Nutzername \\ \hline
		\$token    & alphanumerischer Identifikator eines Moduls \\ \hline
	\end{tabular}
\end{table}
\paragraph{Beschreibung} Die Funktion prüft die modulbasierte Berechtigung eines Nutzers für ein Modul. Die Funktion nutzt folgende Quellen:
\begin{itemize}
	\item Tabelle mit modulbasierten Rechten
	\item Tabelle mit Moduldaten
	\item Tabelle mit Rollen
	\item Tabelle mit Nutzerinformationen
\end{itemize}
Die Antwort wird als strukturiertes Array an den Aufrufer zurückgegeben.