\subsection{Allgemeines} Diese Datei enthält alle Funktionen zum Zugriff auf die Tabelle mit Rangpunkten der Nutzer.
\begin{table}[H]
	\begin{tabular}{|c|p{11cm}|}
		\hline
		\textbf{Einbindungspunkt} & inc-db.php \\ \hline
		\textbf{Einbindungspunkt} & inc-db-sub.php \\ \hline
	\end{tabular}
\end{table}
Die Datei ist nicht direkt durch den Nutzer aufrufbar, dies wird durch folgenden Code-Ausschnitt sichergestellt:
\begin{lstlisting}[language=php]
if (!defined('NICE_PROJECT')) {
	die('Permission denied.');
}
\end{lstlisting}
Der Globale Wert {\glqq NICE\_PROJECT\grqq} wird durch für den Nutzer valide Aufrufpunkte festgelegt, z.B. {\glqq api.php\grqq}.
\newpage
\subsection{Funktionen}
\subsubsection{PersonalAreaCollection}
\paragraph{Parameter} Die Funktion besitzt folgende Parameter:
\begin{table}[H]
	\begin{tabular}{|c|p{11cm}|}
		\hline
		\textbf{Parametername} & \textbf{Parameterbeschreibung} \\ \hline
		\$uid   & Identifikator des Nutzers \\ \hline
		\$aid   & Identifikator des Moduls \\ \hline
		\$value & Anzahl der Rangpunkte \\ \hline
		\$oid   & Identifikator der Begründung \\ \hline
	\end{tabular}
\end{table}
\paragraph{Beschreibung} Die Funktion fügt einem Nutzer Rangpunkte hinzu. Die Funktion hat Auswirkungen auf folgende Quellen:
\begin{itemize}
	\item Tabelle mit Rangpunkten der Nutzer
\end{itemize}
Die Antwort wird als strukturiertes Array an den Aufrufer zurückgegeben.
\subsubsection{getPointsOfUser}
\paragraph{Parameter} Die Funktion besitzt folgende Parameter:
\begin{table}[H]
	\begin{tabular}{|c|p{11cm}|}
		\hline
		\textbf{Parametername} & \textbf{Parameterbeschreibung} \\ \hline
		\$username & Nutzername \\ \hline
	\end{tabular}
\end{table}
\subparagraph{\$json}Das Array enthält folgende Elemente:
\begin{table}[H]
	\begin{tabular}{|c|p{11cm}|}
		\hline
		\textbf{Parametername} & \textbf{Parameterbeschreibung} \\ \hline
		\$uid   & Identifikator des Nutzers \\ \hline
		\$aid   & Identifikator des Moduls \\ \hline
	\end{tabular}
\end{table}
\paragraph{Beschreibung} Die Funktion ruft die Anzahl der Rangpunkte eines Nutzers in einem Modul ab. Die Funktion nutzt folgende Quellen:
\begin{itemize}
	\item Tabelle mit Rangpunkten der Nutzer
\end{itemize}
Die Antwort wird als strukturiertes Array an den Aufrufer zurückgegeben.
\subsubsection{getPointsOfUserLastYear}
\paragraph{Parameter} Die Funktion besitzt folgende Parameter:
\begin{table}[H]
	\begin{tabular}{|c|p{11cm}|}
		\hline
		\textbf{Parametername} & \textbf{Parameterbeschreibung} \\ \hline
		\$username & Nutzername \\ \hline
	\end{tabular}
\end{table}
\subparagraph{\$json}Das Array enthält folgende Elemente:
\begin{table}[H]
	\begin{tabular}{|c|p{11cm}|}
		\hline
		\textbf{Parametername} & \textbf{Parameterbeschreibung} \\ \hline
		\$uid   & Identifikator des Nutzers \\ \hline
		\$aid   & Identifikator des Moduls \\ \hline
	\end{tabular}
\end{table}
\paragraph{Beschreibung} Die Funktion ruft die Anzahl der Rangpunkte innerhalb des letzten Jahres eines Nutzers in einem Modul ab. Die Funktion nutzt folgende Quellen:
\begin{itemize}
	\item Tabelle mit Rangpunkten der Nutzer
\end{itemize}
Die Antwort wird als strukturiertes Array an den Aufrufer zurückgegeben.
\subsubsection{getRanktableForApplication}
\paragraph{Parameter} Die Funktion besitzt folgende Parameter:
\begin{table}[H]
	\begin{tabular}{|c|p{11cm}|}
		\hline
		\textbf{Parametername} & \textbf{Parameterbeschreibung} \\ \hline
		\$aid   & Identifikator des Moduls \\ \hline
	\end{tabular}
\end{table}
\paragraph{Beschreibung} Die Funktion ruft die Rangliste der Nutzer eines Moduls ab. Die Funktion nutzt folgende Quellen:
\begin{itemize}
	\item Tabelle mit Rangpunkten der Nutzer
\end{itemize}
Die Antwort wird als strukturiertes Array an den Aufrufer zurückgegeben.
\subsubsection{getRankDataForAllUsers}
\paragraph{Parameter} Die Funktion besitzt keine Parameter.
\paragraph{Beschreibung} Die Funktion ruft alle Rangpunktevergaben ab. Die Funktion nutzt folgende Quellen:
\begin{itemize}
	\item Tabelle mit Rangpunkten der Nutzer
	\item Tabelle mit Nutzerdaten
\end{itemize}
Die Antwort wird als strukturiertes Array an den Aufrufer zurückgegeben.
