\subsection{Allgemeines} Diese Datei enthält alle Funktionen zum Zugriff auf die Tabelle mit Bildern.
\begin{table}[H]
	\begin{tabular}{|c|p{11cm}|}
		\hline
		\textbf{Einbindungspunkt} & inc-db.php \\ \hline
		\textbf{Einbindungspunkt} & inc-db-sub.php \\ \hline
	\end{tabular}
\end{table}
Die Datei ist nicht direkt durch den Nutzer aufrufbar, dies wird durch folgenden Code-Ausschnitt sichergestellt:
\begin{lstlisting}[language=php]
if (!defined('NICE_PROJECT')) {
	die('Permission denied.');
}
\end{lstlisting}
Der Globale Wert {\glqq NICE\_PROJECT\grqq} wird durch für den Nutzer valide Aufrufpunkte festgelegt, z.B. {\glqq api.php\grqq}.
\newpage
\subsection{Funktionen}
\subsubsection{insertPic}
\paragraph{Parameter} Die Funktion besitzt folgende Parameter:
\begin{table}[H]
	\begin{tabular}{|c|p{11cm}|}
		\hline
		\textbf{Parametername} & \textbf{Parameterbeschreibung} \\ \hline
		\$title       & Titel des Bildes \\ \hline
		\$description & Beschreibung des Bildes \\ \hline
		\$picurl      & Dateiname des Bildes \\ \hline
		\$preview     & Base64 encodiertes Vorschaubild \\ \hline
		\$token       & alphanumerischer Identifikator des Bildes \\ \hline
		\$uid         & Identifikator des hochladenden Nutzers \\ \hline
		\$aid         & Identifikator des hochladenden Moduls \\ \hline
	\end{tabular}
\end{table}
\paragraph{Beschreibung} Die Funktion fügt ein neues Bild der Datenbank hinzu. Die Funktion hat Auswirkungen auf folgende Quellen:
\begin{itemize}
	\item Tabelle mit Bildern
\end{itemize}
Die Antwort wird als strukturiertes Array an den Aufrufer zurückgegeben.
\subsubsection{deletePicture}
\paragraph{Parameter} Die Funktion besitzt folgende Parameter:
\begin{table}[H]
	\begin{tabular}{|c|p{11cm}|}
		\hline
		\textbf{Parametername} & \textbf{Parameterbeschreibung} \\ \hline
		\$username & Nutzername \\ \hline
	\end{tabular}
\end{table}
\paragraph{Beschreibung} Die Funktion löscht ein Bild aus der Datenbank. Die Funktion hat Auswirkungen auf folgende Quellen:
\begin{itemize}
	\item Tabelle mit Bildern
\end{itemize}
Die Antwort wird als strukturiertes Array an den Aufrufer zurückgegeben.
\subsubsection{getPictureListDB}
\paragraph{Parameter} Die Funktion besitzt folgende Parameter:
\begin{table}[H]
	\begin{tabular}{|c|p{11cm}|}
		\hline
		\textbf{Parametername} & \textbf{Parameterbeschreibung} \\ \hline
		\$apptoken & Identifikator eines Moduls \\ \hline
	\end{tabular}
\end{table}
\paragraph{Beschreibung} Die Funktion ruft eine Liste aller Bilder eines Moduls ab. Die Funktion nutzt folgende Quellen:
\begin{itemize}
	\item Tabelle mit Bildern
\end{itemize}
Die Antwort wird als strukturiertes Array an den Aufrufer zurückgegeben.
\subsubsection{getSinglePictureFromDB}
\paragraph{Parameter} Die Funktion besitzt folgende Parameter:
\begin{table}[H]
	\begin{tabular}{|c|p{11cm}|}
		\hline
		\textbf{Parametername} & \textbf{Parameterbeschreibung} \\ \hline
		\$apptoken     & Identifikator eines Moduls \\ \hline
		\$pictureToken & alphanumerischer Identifikator eines Bildes \\ \hline
	\end{tabular}
\end{table}
\paragraph{Beschreibung} Die Funktion ruft alle Daten zu einem bestimmten Bild ab. Die Funktion nutzt folgende Quellen:
\begin{itemize}
	\item Tabelle mit Bildern
\end{itemize}
Die Antwort wird als strukturiertes Array an den Aufrufer zurückgegeben.
\subsubsection{getPreviewPicture}
\paragraph{Parameter} Die Funktion besitzt folgende Parameter:
\begin{table}[H]
	\begin{tabular}{|c|p{11cm}|}
		\hline
		\textbf{Parametername} & \textbf{Parameterbeschreibung} \\ \hline
		\$token & alphanumerischer Identifikator eines Bildes \\ \hline
	\end{tabular}
\end{table}
\paragraph{Beschreibung} Die Funktion ruft das Vorschaubild eines bestimmten Bildes ab. Die Funktion nutzt folgende Quellen:
\begin{itemize}
	\item Tabelle mit Bildern
\end{itemize}
Die Antwort wird als strukturiertes Array an den Aufrufer zurückgegeben.
\subsubsection{getFullsizePicture}
\paragraph{Parameter} Die Funktion besitzt folgende Parameter:
\begin{table}[H]
	\begin{tabular}{|c|p{11cm}|}
		\hline
		\textbf{Parametername} & \textbf{Parameterbeschreibung} \\ \hline
		\$token & alphanumerischer Identifikator eines Bildes \\ \hline
	\end{tabular}
\end{table}
\paragraph{Beschreibung} Die Funktion ruft die Daten zum Laden des Vollbildes eines bestimmten Bildes ab. Die Funktion nutzt folgende Quellen:
\begin{itemize}
	\item Tabelle mit Bildern
\end{itemize}
Die Antwort wird als strukturiertes Array an den Aufrufer zurückgegeben.
\subsubsection{getPictureIdByToken}
\paragraph{Parameter} Die Funktion besitzt folgende Parameter:
\begin{table}[H]
	\begin{tabular}{|c|p{11cm}|}
		\hline
		\textbf{Parametername} & \textbf{Parameterbeschreibung} \\ \hline
		\$token & alphanumerischer Identifikator eines Bildes \\ \hline
	\end{tabular}
\end{table}
\paragraph{Beschreibung} Die Funktion ruft den numerischen Identifikator eines bestimmten Bildes anhand des alphanumerischen Identifikator ab. Die Funktion nutzt folgende Quellen:
\begin{itemize}
	\item Tabelle mit Bildern
\end{itemize}
Die Antwort wird als strukturiertes Array an den Aufrufer zurückgegeben.
\subsubsection{getPictureUploadingUserByToken}
\paragraph{Parameter} Die Funktion besitzt folgende Parameter:
\begin{table}[H]
	\begin{tabular}{|c|p{11cm}|}
		\hline
		\textbf{Parametername} & \textbf{Parameterbeschreibung} \\ \hline
		\$token & alphanumerischer Identifikator eines Bildes \\ \hline
	\end{tabular}
\end{table}
\paragraph{Beschreibung} Die Funktion ruft den hochladenden Nutzer eines Bildes ab. Die Funktion nutzt folgende Quellen
\begin{itemize}
	\item Tabelle mit Bildern
	\item Tabelle mit Nutzerdaten
\end{itemize}
Die Antwort wird als strukturiertes Array an den Aufrufer zurückgegeben.
\subsubsection{updateMaterialByToken}
\paragraph{Parameter} Die Funktion besitzt folgende Parameter:
\begin{table}[H]
	\begin{tabular}{|c|p{11cm}|}
		\hline
		\textbf{Parametername} & \textbf{Parameterbeschreibung} \\ \hline
		\$token       & alphanumerischer Identifikator eines Bildes \\ \hline
		\$title       & Titel des Bildes \\ \hline
		\$description & Beschreibung des Bildes \\ \hline
	\end{tabular}
\end{table}
\paragraph{Beschreibung} Die Funktion aktualisiert die Metadaten eines Bildes. Die Funktion hat Auswirkungen auf folgende Quellen:
\begin{itemize}
	\item Tabelle mit Bildern
\end{itemize}
Die Antwort wird als strukturiertes Array an den Aufrufer zurückgegeben.
\subsubsection{updateDeletionStatePictureByToken}
\paragraph{Parameter} Die Funktion besitzt folgende Parameter:
\begin{table}[H]
	\begin{tabular}{|c|p{11cm}|}
		\hline
		\textbf{Parametername} & \textbf{Parameterbeschreibung} \\ \hline
		\$id    & numerischer Identifikator eines Bildes \\ \hline
		\$state & Status der Freischaltung \\ \hline
	\end{tabular}
\end{table}
\paragraph{Beschreibung} Die Funktion setzt den Status der Freischaltung eines Bildes. Die Funktion hat Auswirkungen auf folgende Quellen:
\begin{itemize}
	\item Tabelle mit Bildern
\end{itemize}
Die Antwort wird als strukturiertes Array an den Aufrufer zurückgegeben.
\subsubsection{updateMaterialSourceByToken}
\paragraph{Parameter} Die Funktion besitzt folgende Parameter:
\begin{table}[H]
	\begin{tabular}{|c|p{11cm}|}
		\hline
		\textbf{Parametername} & \textbf{Parameterbeschreibung} \\ \hline
		\$token       & alphanumerischer Identifikator eines Bildes \\ \hline
		\$source      & Quellenangabe des Bildes \\ \hline
		\$sourceType  & Identifikator des Typs der Quelle des Bildes \\ \hline
	\end{tabular}
\end{table}
\paragraph{Beschreibung} Die Funktion aktualisiert die Metadaten zu Quellen eines Bildes. Die Funktion hat Auswirkungen auf folgende Quellen:
\begin{itemize}
	\item Tabelle mit Bildern
\end{itemize}
Die Antwort wird als strukturiertes Array an den Aufrufer zurückgegeben.