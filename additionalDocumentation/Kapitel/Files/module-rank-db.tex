\subsection{Allgemeines} Diese Datei enthält alle Funktionen zum Zugriff auf die Tabelle mit modulbasierten Rangnamen.
\begin{table}[H]
	\begin{tabular}{|c|p{11cm}|}
		\hline
		\textbf{Einbindungspunkt} & inc-db.php \\ \hline
		\textbf{Einbindungspunkt} & inc-db-sub.php \\ \hline
	\end{tabular}
\end{table}
Die Datei ist nicht direkt durch den Nutzer aufrufbar, dies wird durch folgenden Code-Ausschnitt sichergestellt:
\begin{lstlisting}[language=php]
if (!defined('NICE_PROJECT')) {
	die('Permission denied.');
}
\end{lstlisting}
Der Globale Wert {\glqq NICE\_PROJECT\grqq} wird durch für den Nutzer valide Aufrufpunkte festgelegt, z.B. {\glqq api.php\grqq}.
\newpage
\subsection{Funktionen}
\subsubsection{getNamesByRankId}
\paragraph{Parameter} Die Funktion besitzt folgende Parameter:
\begin{table}[H]
	\begin{tabular}{|c|p{11cm}|}
		\hline
		\textbf{Parametername} & \textbf{Parameterbeschreibung} \\ \hline
		\$rid & Identifikator eines Ranges \\ \hline
	\end{tabular}
\end{table}
\paragraph{Beschreibung} Die Funktion ruft eine Liste aller modulbasierten Rangnamen eines Ranges ab. Die Funktion nutzt folgende Quellen:
\begin{itemize}
	\item Tabelle mit modulbasierten Rangnamen
\end{itemize}
Die Antwort wird als strukturiertes Array an den Aufrufer zurückgegeben.
\subsubsection{insertModuleBasedRankName}
\paragraph{Parameter} Die Funktion besitzt folgende Parameter:
\begin{table}[H]
	\begin{tabular}{|c|p{11cm}|}
		\hline
		\textbf{Parametername} & \textbf{Parameterbeschreibung} \\ \hline
		\$rid  & Identifikator eines Ranges \\ \hline
		\$aid  & Identifikator eines Moduls \\ \hline
		\$name & modulbasierter Name \\ \hline
	\end{tabular}
\end{table}
\paragraph{Beschreibung} Die Funktion fügt einem Rang einen neuen modulbasierten Namen hinzu. Die Funktion hat Auswirkungen auf folgende Quellen:
\begin{itemize}
	\item Tabelle mit modulbasierten Rangnamen
\end{itemize}
Die Antwort wird als strukturiertes Array an den Aufrufer zurückgegeben.
\subsubsection{deleteModuleBasedRankName}
\paragraph{Parameter} Die Funktion besitzt folgende Parameter:
\begin{table}[H]
	\begin{tabular}{|c|p{11cm}|}
		\hline
		\textbf{Parametername} & \textbf{Parameterbeschreibung} \\ \hline
		\$id  & Identifikator eines modulbasierten Rangnamens \\ \hline
	\end{tabular}
\end{table}
\paragraph{Beschreibung} Die Funktion löscht eine modulbasierten Rangnamen. Die Funktion hat Auswirkungen auf folgende Quellen:
\begin{itemize}
	\item Tabelle mit modulbasierten Rangnamen
\end{itemize}
Die Antwort wird als strukturiertes Array an den Aufrufer zurückgegeben.
\subsubsection{getRankNamesByModule}
\paragraph{Parameter} Die Funktion besitzt folgende Parameter:
\begin{table}[H]
	\begin{tabular}{|c|p{11cm}|}
		\hline
		\textbf{Parametername} & \textbf{Parameterbeschreibung} \\ \hline
		\$aid & numerischer Identifikator eines Moduls \\ \hline
	\end{tabular}
\end{table}
\paragraph{Beschreibung} Die Funktion ruft eine Liste aller modulbasierten Rangnamen eines Moduls ab. Die Funktion nutzt folgende Quellen:
\begin{itemize}
	\item Tabelle mit modulbasierten Rangnamen
\end{itemize}
Die Antwort wird als strukturiertes Array an den Aufrufer zurückgegeben.