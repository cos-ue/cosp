\subsection{Allgemeines} Diese Datei enthält alle durch die Modul-API aufgerufene Funktionen sowie zusätzliche Funktionen, um die Datenstrukturierung der Antworten zu vereinheitlichen.
\begin{table}[H]
	\begin{tabular}{|c|p{11cm}|}
		\hline
		\textbf{Einbindungspunkt} & inc.php \\ \hline
		\textbf{Einbindungspunkt} & inc-sub.php \\ \hline
	\end{tabular}
\end{table}
Die Datei ist nicht direkt durch den Nutzer aufrufbar, dies wird durch folgenden Code-Ausschnitt sichergestellt:
\begin{lstlisting}[language=php]
if (!defined('NICE_PROJECT')) {
	die('Permission denied.');
}
\end{lstlisting}
Der Globale Wert {\glqq NICE\_PROJECT\grqq} wird durch für den Nutzer valide Aufrufpunkte festgelegt, z.B. {\glqq api.php\grqq}.
\newpage
\subsection{Funktionen}
\subsubsection{getUserdataApi}
\paragraph{Parameter} Die Funktion besitzt folgende Parameter:
\begin{table}[H]
	\begin{tabular}{|c|p{11cm}|}
		\hline
		\textbf{Parametername} & \textbf{Parameterbeschreibung} \\ \hline
		\$username            & Nutzername \\ \hline
		\$ignoreDeaktiviation & Wenn wahr, wird der Aktivierungsstatus ignoriert \\ \hline
		\$apitoken            & Identifikator des Moduls \\ \hline
	\end{tabular}
\end{table}
\paragraph{Beschreibung} Die Funktion ermittelt alle Daten, welche zum Login des Nutzers und alle Funktionen eines Moduls notwendig sind. Die Funktion nutzt folgende Quellen:
\begin{itemize}
	\item Nutzerdaten-Tabelle
\end{itemize}
Die Antwort wird als strukturiertes Array an den Aufrufer zurückgegeben.
\subsubsection{getRoledataApi}
\paragraph{Parameter} Die Funktion besitzt folgende Parameter:
\begin{table}[H]
	\begin{tabular}{|c|p{11cm}|}
		\hline
		\textbf{Parametername} & \textbf{Parameterbeschreibung} \\ \hline
		\$username & Nutzername \\ \hline
	\end{tabular}
\end{table}
\paragraph{Beschreibung} Die Funktion ermittelt die Rollendaten des gegebenen Benutzers. Die Funktion nutzt folgende Quellen:
\begin{itemize}
	\item Nutzerdaten-Tabelle
\end{itemize}
Die Antwort wird als strukturiertes Array an den Aufrufer zurückgegeben.
\subsubsection{generateJson}
\paragraph{Parameter} Die Funktion besitzt folgende Parameter:
\begin{table}[H]
	\begin{tabular}{|c|p{11cm}|}
		\hline
		\textbf{Parametername} & \textbf{Parameterbeschreibung} \\ \hline
		\$array & Array mit strukturierten Eingabedaten \\ \hline
	\end{tabular}
\end{table}
\paragraph{Beschreibung} Die Funktion generiert aus einem Array ein Daten im JSON-Format. Die Antwort wird als String an den Aufrufer zurückgegeben.
\subsubsection{generateError}
\paragraph{Parameter} Die Funktion besitzt folgende Parameter:
\begin{table}[H]
	\begin{tabular}{|c|p{11cm}|}
		\hline
		\textbf{Parametername} & \textbf{Parameterbeschreibung} \\ \hline
		\$input & Optionale Fehlermeldung oder andere Daten \\ \hline
	\end{tabular}
\end{table}
\paragraph{Beschreibung} Die Funktion generiert ein Array, welches eine Fehlermeldung darstellt. Die Antwort wird als strukturiertes Array an den Aufrufer zurückgegeben.
\subsubsection{generateSuccess}
\paragraph{Parameter} Die Funktion besitzt keine Parameter.
\paragraph{Beschreibung} Die Funktion generiert ein Array, welches eine Antwort auf einen erfolgreichen API-Aufruf darstellt. Die Antwort wird als strukturiertes Array an den Aufrufer zurückgegeben.
\subsubsection{addRankPoints}
\paragraph{Parameter} Die Funktion besitzt folgende Parameter:
\begin{table}[H]
	\begin{tabular}{|c|p{11cm}|}
		\hline
		\textbf{Parametername} & \textbf{Parameterbeschreibung} \\ \hline
		\$username & Nutzername \\ \hline
		\$token    & Authentifizierungstoken eines Moduls \\ \hline
		\$reason   & Begründung der Rangpunkte \\ \hline
		\$points   & Anzahl der Rangpunkte \\ \hline
	\end{tabular}
\end{table}
\paragraph{Beschreibung} Die Funktion fügt dem angegebenen Benutzer die angegebenen Rangpunkte hinzu. Die Funktion hat Auswirkungen auf folgende Quellen:
\begin{itemize}
	\item Tabelle mit Rangpunkten der Nutzer
\end{itemize}
Die Antwort wird als strukturiertes Array an den Aufrufer zurückgegeben.
\subsubsection{getSourceTypesAPI}
\paragraph{Parameter} Die Funktion besitzt keine Parameter.
\paragraph{Beschreibung} Die Funktion alle Typen von Quellen ab. Die Funktion hat Auswirkungen auf folgende Quellen:
\begin{itemize}
	\item Tabelle mit Typen von Quellen
\end{itemize}
Die Antwort wird als strukturiertes Array an den Aufrufer zurückgegeben.
\subsubsection{checkMailAddressExistentMoudleAPI}
\paragraph{Parameter} Die Funktion besitzt folgende Parameter:
\begin{table}[H]
	\begin{tabular}{|c|p{11cm}|}
		\hline
		\textbf{Parametername} & \textbf{Parameterbeschreibung} \\ \hline
		\$email & Nutzername \\ \hline
	\end{tabular}
\end{table}
\paragraph{Beschreibung} Die Funktion fügt dem angegebenen Benutzer die angegebenen Rangpunkte hinzu. Die Funktion nutzt folgende Quellen:
\begin{itemize}
	\item Tabelle mit Nutzerdaten
\end{itemize}
Die Antwort wird als strukturiertes Array an den Aufrufer zurückgegeben.