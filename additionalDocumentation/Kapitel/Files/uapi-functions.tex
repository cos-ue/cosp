\subsection{Allgemeines} Diese Datei enthält alle Funktionen, welche durch die Nutzer-API aufgerufen werden.
\begin{table}[H]
	\begin{tabular}{|c|p{11cm}|}
		\hline
		\textbf{Einbindungspunkt} & inc.php \\ \hline
		\textbf{Einbindungspunkt} & inc-sub.php \\ \hline
	\end{tabular}
\end{table}
Die Datei ist nicht direkt durch den Nutzer aufrufbar, dies wird durch folgenden Code-Ausschnitt sichergestellt:
\begin{lstlisting}[language=php]
if (!defined('NICE_PROJECT')) {
	die('Permission denied.');
}
\end{lstlisting}
Der Globale Wert {\glqq NICE\_PROJECT\grqq} wird durch für den Nutzer valide Aufrufpunkte festgelegt, z.B. {\glqq api.php\grqq}.
\newpage
\subsection{Funktionen}
\subsubsection{getPreviewPictureAPI}
\paragraph{Parameter} Die Funktion besitzt folgende Parameter:
\begin{table}[H]
	\begin{tabular}{|c|p{11cm}|}
		\hline
		\textbf{Parametername} & \textbf{Parameterbeschreibung} \\ \hline
		\$token & alphanumerischer Identifikator eines Bildes \\ \hline
	\end{tabular}
\end{table}
\paragraph{Beschreibung} Die Funktion liefert einen Base64-Codiertes Vorschaubild zurück. Die Funktion nutzt folgende Quellen:
\begin{itemize}
	\item Tabelle mit Bildern
\end{itemize}
Die Antwort wird als Zeichenkette an den Aufrufer zurückgegeben.
\subsubsection{getPictureFullsizeAPI}
\paragraph{Parameter} Die Funktion besitzt folgende Parameter:
\begin{table}[H]
	\begin{tabular}{|c|p{11cm}|}
		\hline
		\textbf{Parametername} & \textbf{Parameterbeschreibung} \\ \hline
		\$token & alphanumerischer Identifikator eines Bildes \\ \hline
	\end{tabular}
\end{table}
\paragraph{Beschreibung} Die Funktion liefert das Ursprungsbild als Binary zurück. Die Funktion nutzt folgende Quellen:
\begin{itemize}
	\item Tabelle mit Bildern
	\item Dateisystem
\end{itemize}
Die Antwort wird als Binary an den Aufrufer zurückgegeben.
\subsubsection{generateErrorUAPI}
\paragraph{Parameter} Die Funktion besitzt keine Parameter.
\paragraph{Beschreibung} Die Funktion liefert einen generischen Fehler auf eine API-Anfrage. Die Antwort wird als strukturiertes Array an den Aufrufer zurückgegeben.
\subsubsection{generateSuccessUAPI}
\paragraph{Parameter} Die Funktion besitzt folgende Parameter:
\begin{table}[H]
	\begin{tabular}{|c|p{11cm}|}
		\hline
		\textbf{Parametername} & \textbf{Parameterbeschreibung} \\ \hline
		\$message & Nachricht (optional) \\ \hline
	\end{tabular}
\end{table}
\paragraph{Beschreibung} Die Funktion liefert einen generischen Erfolg auf eine API-Anfrage. Die Antwort wird als strukturiertes Array an den Aufrufer zurückgegeben.
\subsubsection{generateJsonUAPI}
\paragraph{Parameter} Die Funktion besitzt folgende Parameter:
\begin{table}[H]
	\begin{tabular}{|c|p{11cm}|}
		\hline
		\textbf{Parametername} & \textbf{Parameterbeschreibung} \\ \hline
		\$array & Array mit Daten \\ \hline
	\end{tabular}
\end{table}
\paragraph{Beschreibung} Die Funktion generiert das JSON für Antwort auf einen API-Anfrage. Die Antwort wird direkt Ausgegeben.
\subsubsection{getStoryDataUAPI}
\paragraph{Parameter} Die Funktion besitzt folgende Parameter:
\begin{table}[H]
	\begin{tabular}{|c|p{11cm}|}
		\hline
		\textbf{Parametername} & \textbf{Parameterbeschreibung} \\ \hline
		\$token & alphanumerischer Token einer Geschichte \\ \hline
	\end{tabular}
\end{table}
\paragraph{Beschreibung} Die Funktion ruft alle Daten einer Geschichte ab und sendet diese zurück an den Aufrufer. Die Funktion nutzt folgende Quellen:
\begin{itemize}
	\item Tabelle mit Geschichten
\end{itemize}
Die Antwort wird als strukturiertes Array an den Aufrufer zurückgegeben.
\subsubsection{getStoriesDataUAPI}
\paragraph{Parameter} Die Funktion besitzt folgende Parameter:
\begin{table}[H]
	\begin{tabular}{|c|p{11cm}|}
		\hline
		\textbf{Parametername} & \textbf{Parameterbeschreibung} \\ \hline
		\$tokens & alphanumerische Identifikatoren von Geschichten \\ \hline
	\end{tabular}
\end{table}
\paragraph{Beschreibung} Die Funktion ruft die Daten zu allen angegeben Geschichten ab. Die Funktion nutzt folgende Quellen:
\begin{itemize}
	\item Tabelle mit Geschichten
\end{itemize}
Die Antwort wird als strukturiertes Array an den Aufrufer zurückgegeben.
\subsubsection{buildUserStoryArray}
\paragraph{Parameter} Die Funktion besitzt folgende Parameter:
\begin{table}[H]
	\begin{tabular}{|c|p{11cm}|}
		\hline
		\textbf{Parametername} & \textbf{Parameterbeschreibung} \\ \hline
		\$token           & alphanumerischer Identifikator einer Geschichte \\ \hline
		\$username        & Nutzername des Abfragenden \\ \hline
		\$ValidationValue & Validierungswert des Abfragenden \\ \hline
		\$apphashtoken    & Hash des alphanumerischen Modulidentifikators \\ \hline
	\end{tabular}
\end{table}
\paragraph{Beschreibung} Die Funktion generiert ein Array mit allen Daten einer Geschichte. Die Funktion nutzt folgende Quellen:
\begin{itemize}
	\item Tabelle mit Geschichten
	\item Tabelle mit Nutzerdaten
\end{itemize}
Die Antwort wird als strukturiertes Array an den Aufrufer zurückgegeben.
\subsubsection{validateStory}
\paragraph{Parameter} Die Funktion besitzt folgende Parameter:
\begin{table}[H]
	\begin{tabular}{|c|p{11cm}|}
		\hline
		\textbf{Parametername} & \textbf{Parameterbeschreibung} \\ \hline
		\$data & Zeichenkette mit Informationen \\ \hline
	\end{tabular}
\end{table}
\paragraph{Beschreibung} Die Funktion validiert die angegebene Geschichte. Die Funktion hat Auswirkungen auf folgende Quellen:
\begin{itemize}
	\item Tabelle mit Validierungsinformationen zu Geschichten
\end{itemize}
Die Antwort wird als strukturiertes Array an den Aufrufer zurückgegeben.
\subsubsection{validatePicture}
\paragraph{Parameter} Die Funktion besitzt folgende Parameter:
\begin{table}[H]
	\begin{tabular}{|c|p{11cm}|}
		\hline
		\textbf{Parametername} & \textbf{Parameterbeschreibung} \\ \hline
		\$data & Zeichenkette mit Informationen \\ \hline
	\end{tabular}
\end{table}
\paragraph{Beschreibung} Die Funktion validiert die angegebene Bilder. Die Funktion hat Auswirkungen auf folgende Quellen:
\begin{itemize}
	\item Tabelle mit Validierungsinformationen zu Bildern
\end{itemize}
Die Antwort wird als strukturiertes Array an den Aufrufer zurückgegeben.