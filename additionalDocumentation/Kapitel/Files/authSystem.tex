\subsection{Allgemeines} Diese Datei enthält alle Funktionen um Nutzer zu Verwalten, Authentifizieren und Anzulegen.
\begin{table}[H]
	\begin{tabular}{|c|p{11cm}|}
		\hline
		\textbf{Einbindungspunkt} & inc.php \\ \hline
		\textbf{Einbindungspunkt} & inc-sub.php \\ \hline
	\end{tabular}
\end{table}
Die Datei ist nicht direkt durch den Nutzer aufrufbar, dies wird durch folgenden Code-Ausschnitt sichergestellt:
\begin{lstlisting}[language=php]
if (!defined('NICE_PROJECT')) {
	die('Permission denied.');
}
\end{lstlisting}
Der Globale Wert {\glqq NICE\_PROJECT\grqq} wird durch für den Nutzer valide Aufrufpunkte festgelegt, z.B. {\glqq api.php\grqq}.
\newpage
\subsection{Funktionen}
\subsubsection{createNewUser}
\paragraph{Parameter} Die Funktion besitzt folgende Parameter:
\begin{table}[H]
	\begin{tabular}{|c|p{11cm}|}
		\hline
		\textbf{Parametername} & \textbf{Parameterbeschreibung} \\ \hline
		\$name      & Nutzername \\ \hline
		\$passwd    & Passwort des neuen Nutzers als Klartext \\ \hline
		\$email     & Nutzername \\ \hline
		\$firstname & Nutzername \\ \hline
		\$lastname  & Nutzername \\ \hline
	\end{tabular}
\end{table}
\paragraph{Beschreibung} Die Funktion fügt einen neuen Benutzer in das System ein. Die Funktion hat Auswirkungen auf folgende Quellen:
\begin{itemize}
	\item Nutzerdaten-Tabelle
\end{itemize}
Die Funktion hat keinen Rückgabewert.
\subsubsection{updateUserPassword}
\paragraph{Parameter} Die Funktion besitzt folgende Parameter:
\begin{table}[H]
	\begin{tabular}{|c|p{11cm}|}
		\hline
		\textbf{Parametername} & \textbf{Parameterbeschreibung} \\ \hline
		\$uid      & Identifikator des Nutzers \\ \hline
		\$password & Passwort des neuen Nutzers als Klartext \\ \hline
	\end{tabular}
\end{table}
\paragraph{Beschreibung} Die Funktion ändert das Passwort eines Nutzers. Die Funktion hat Auswirkungen auf folgende Quellen:
\begin{itemize}
	\item Nutzerdaten-Tabelle
\end{itemize}
Die Funktion hat keinen Rückgabewert.
\subsubsection{checkPassword}
\paragraph{Parameter} Die Funktion besitzt folgende Parameter:
\begin{table}[H]
	\begin{tabular}{|c|p{11cm}|}
		\hline
		\textbf{Parametername} & \textbf{Parameterbeschreibung} \\ \hline
		\$password & eingegebenes Passwort als Klartext \\ \hline
		\$username & Nutzername \\ \hline
	\end{tabular}
\end{table}
\paragraph{Beschreibung} Die Funktion prüft ein durch den Nutzer zur Authentifizierung angegebenes Passwort auf Korrektheit. Des Weiteren werden auch alle Session-Daten gesetzt. Die Funktion nutzt folgende Quellen:
\begin{itemize}
	\item Nutzerdaten-Tabelle
\end{itemize}
Die Antwort ist ein Boolean.
\subsubsection{checkPasswordOnly}
\paragraph{Parameter} Die Funktion besitzt folgende Parameter:
\begin{table}[H]
	\begin{tabular}{|c|p{11cm}|}
		\hline
		\textbf{Parametername} & \textbf{Parameterbeschreibung} \\ \hline
		\$password & eingegebenes Passwort als Klartext \\ \hline
		\$username & Nutzername \\ \hline
	\end{tabular}
\end{table}
\paragraph{Beschreibung} Die Funktion prüft ein durch den Nutzer zur Authentifizierung angegebenes Passwort auf Korrektheit. Es werden keine Session-Daten gesetzt. Die Funktion nutzt folgende Quellen:
\begin{itemize}
	\item Nutzerdaten-Tabelle
\end{itemize}
Die Antwort ist ein Boolean.
\subsubsection{inspectPassword}
\paragraph{Parameter} Die Funktion besitzt folgende Parameter:
\begin{table}[H]
	\begin{tabular}{|c|p{11cm}|}
		\hline
		\textbf{Parametername} & \textbf{Parameterbeschreibung} \\ \hline
		\$PasswordField1Val & Passwort Feld 1 \\ \hline
		\$PasswordField2Val & Passwort Feld 2 \\ \hline
	\end{tabular}
\end{table}
\paragraph{Beschreibung} Die Funktion prüft, ob ein Passwort die gewünschten Anforderungen erfüllt. Die Funktion nutzt folgende Quellen:
\begin{itemize}
	\item Nutzerdaten-Tabelle
\end{itemize}
Die Antwort ist ein Boolean.
\subsubsection{updateUser}
\paragraph{Parameter} Die Funktion besitzt folgende Parameter:
\begin{table}[H]
	\begin{tabular}{|c|p{11cm}|}
		\hline
		\textbf{Parametername} & \textbf{Parameterbeschreibung} \\ \hline
		\$firstname & Vorname \\ \hline
		\$lastname  & Nachname \\ \hline
		\$EMail     & E-Mailadresse \\ \hline
	\end{tabular}
\end{table}
\paragraph{Beschreibung} Die Funktion aktualisiert die Session-Daten eines Nutzers. Die Funktion hat keinen Rückgabewert.
\subsubsection{logLogin}
\paragraph{Parameter} Die Funktion besitzt folgende Parameter:
\begin{table}[H]
	\begin{tabular}{|c|p{11cm}|}
		\hline
		\textbf{Parametername} & \textbf{Parameterbeschreibung} \\ \hline
		\$type & Typ des Logins ({\glqq Gast\grqq} oder {\glqq user\grqq}) \\ \hline
	\end{tabular}
\end{table}
\paragraph{Beschreibung} Die Funktion loggt den Typ des Logins für statistische Zwecke. Die Funktion hat Auswirkungen auf folgende Quellen:
\begin{itemize}
	\item Tabelle mit statistischen Login-Daten
\end{itemize}
Die Funktion hat keinen Rückgabewert.