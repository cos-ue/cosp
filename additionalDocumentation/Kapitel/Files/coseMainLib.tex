\subsection{Allgemeines} Diese Datei enthält alle Funktionen, welche an diversen stellen im JavaScript dieser Seite verwendet werden.
Es wird auch die zum Bereitstellen von Tooltipps benötigten Befehle ausgeführt:
\begin{lstlisting}[language=JavaScript]
$(document).ready(function () {
	$('[data-toggle="tooltip"]').tooltip();
});
\end{lstlisting}
\subsection{Funktionen}
\subsubsection{AddNewRole}
\paragraph{Parameter} Die Funktion besitzt keine Parameter.
\paragraph{Beschreibung} Die Funktion sendet eine API-Anfrage zum hinzufügen einer neuen Rolle. Die Funktion hat Auswirkungen auf folgende Quellen:
\begin{itemize}
	\item Management-API
\end{itemize}
\subsubsection{sendApiRequest}
\paragraph{Parameter} Die Funktion besitzt folgende Parameter:
\begin{table}[H]
	\begin{tabular}{|c|p{11cm}|}
		\hline
		\textbf{Parametername} & \textbf{Parameterbeschreibung} \\ \hline
		json   & Daten der Anfrage (JSON-Format) \\ \hline
		reload & Legt fest, ob aktuelle Seite nach erfolgreicher Anfrage neu geladen werden soll \\ \hline
	\end{tabular}
\end{table}
\paragraph{Beschreibung} Die Funktion führt eine Anfrage mit den gegebenen Daten aus und lädt, wenn gewünscht, die Seite neu.
\subsubsection{openEditRoleModal}
\paragraph{Parameter} Die Funktion besitzt folgende Parameter:
\begin{table}[H]
	\begin{tabular}{|c|p{11cm}|}
		\hline
		\textbf{Parametername} & \textbf{Parameterbeschreibung} \\ \hline
		name  & Rollenname \\ \hline
		value & Rollenwert \\ \hline
		id    & Identifikator der Rolle \\ \hline
	\end{tabular}
\end{table}
\paragraph{Beschreibung} Die Funktion öffnet das Modal zum bearbeiten einer Rolle.
\subsubsection{saveEditRoleModal}
\paragraph{Parameter} Die Funktion besitzt keine Parameter.
\paragraph{Beschreibung} Die Funktion speichert die geänderten Werte einer Rolle. Die Funktion hat Auswirkungen auf folgende Quellen:
\begin{itemize}
	\item Management-API
\end{itemize}
\subsubsection{deleteRole}
\paragraph{Parameter} Die Funktion besitzt folgende Parameter:
\begin{table}[H]
	\begin{tabular}{|c|p{11cm}|}
		\hline
		\textbf{Parametername} & \textbf{Parameterbeschreibung} \\ \hline
		id & Identifikator einer Rolle \\ \hline
	\end{tabular}
\end{table}
\paragraph{Beschreibung} Die Funktion löscht eine Rolle. Die Funktion hat Auswirkungen auf folgende Quellen:
\begin{itemize}
	\item Management-API
\end{itemize}
\subsubsection{resetPasswordModalShow}
\paragraph{Parameter} Die Funktion besitzt folgende Parameter:
\begin{table}[H]
	\begin{tabular}{|c|p{11cm}|}
		\hline
		\textbf{Parametername} & \textbf{Parameterbeschreibung} \\ \hline
		username & Nutzername \\ \hline
		id       & Identifikator eines Nutzers \\ \hline
	\end{tabular}
\end{table}
\paragraph{Beschreibung} Die Funktion öffnet ein Modal um das Passwort des ausgewählten Nutzers zu ändern. 
\subsubsection{resetPasswordModalSave}
\paragraph{Parameter} Die Funktion besitzt keine Parameter.
\paragraph{Beschreibung} Die Funktion speichert das geänderte Passwort eines ausgewählten Nutzers. Die Funktion hat Auswirkungen auf folgende Quellen:
\begin{itemize}
	\item Management-API
	\item Tabelle mit Nutzerdaten
\end{itemize}
\subsubsection{enableDisableUser}
\paragraph{Parameter} Die Funktion besitzt folgende Parameter:
\begin{table}[H]
	\begin{tabular}{|c|p{11cm}|}
		\hline
		\textbf{Parametername} & \textbf{Parameterbeschreibung} \\ \hline
		id & Identifikator eines Nutzers \\ \hline
	\end{tabular}
\end{table}
\paragraph{Beschreibung} Die Funktion aktiviert oder deaktiviert einen Nutzer (je nach aktuellem Status der Aktivierung). Die Funktion hat Auswirkungen auf folgende Quellen:
\begin{itemize}
	\item Management-API
	\item Tabelle mit Nutzerdaten
\end{itemize}
\subsubsection{sendPasswordResetMail}
\paragraph{Parameter} Die Funktion besitzt folgende Parameter:
\begin{table}[H]
	\begin{tabular}{|c|p{11cm}|}
		\hline
		\textbf{Parametername} & \textbf{Parameterbeschreibung} \\ \hline
		id & Identifikator eines Nutzers \\ \hline
	\end{tabular}
\end{table}
\paragraph{Beschreibung} Die Funktion sendet dem ausgewählten Nutzer eine Mail zum zurücksetzen des Passwortes.
\subsubsection{AddNewRank}
\paragraph{Parameter} Die Funktion besitzt keine Parameter.
\paragraph{Beschreibung} Die Funktion fügt dem System einen neuen Rang hinzu. Die Funktion hat Auswirkungen auf folgende Quellen:
\begin{itemize}
	\item Management-API
\end{itemize}
\subsubsection{deleteRank}
\paragraph{Parameter} Die Funktion besitzt folgende Parameter:
\begin{table}[H]
	\begin{tabular}{|c|p{11cm}|}
		\hline
		\textbf{Parametername} & \textbf{Parameterbeschreibung} \\ \hline
		id & Identifikator eines Ranges \\ \hline
	\end{tabular}
\end{table}
\paragraph{Beschreibung} Die Funktion löscht den gegebenen Rang. Die Funktion hat Auswirkungen auf folgende Quellen:
\begin{itemize}
	\item Management-API
\end{itemize}
\subsubsection{openEditRankModal}
\paragraph{Parameter} Die Funktion besitzt folgende Parameter:
\begin{table}[H]
	\begin{tabular}{|c|p{11cm}|}
		\hline
		\textbf{Parametername} & \textbf{Parameterbeschreibung} \\ \hline
		name  & Rangname \\ \hline
		value & Rangwert \\ \hline
		id    & Identifikator eines Ranges \\ \hline
	\end{tabular}
\end{table}
\paragraph{Beschreibung} Die Funktion öffnet das Modal zum bearbeiten eines Ranges.
\subsubsection{saveEditRankModal}
\paragraph{Parameter} Die Funktion besitzt keine Parameter.
\paragraph{Beschreibung} Die Funktion speichert die geänderten Informationen eines Ranges. Die Funktion hat Auswirkungen auf folgende Quellen:
\begin{itemize}
	\item Management-API
\end{itemize}
\subsubsection{loadCaptchaContact}
\paragraph{Parameter} Die Funktion besitzt keine Parameter.
\paragraph{Beschreibung} Die Funktion lädt einen Captcha-Code für das Kontaktformular. Die Funktion nutzt folgende Quellen:
\begin{itemize}
	\item Management-API
\end{itemize}
\subsubsection{submitContact}
\paragraph{Parameter} Die Funktion besitzt keine Parameter.
\paragraph{Beschreibung} Die Funktion sendet eine API-Anfrage zum versenden einer Kontaktnachricht an die Mitarbeiter des Projekts.
\subsubsection{setErrorOnInputContact}
\paragraph{Parameter} Die Funktion besitzt folgende Parameter:
\begin{table}[H]
	\begin{tabular}{|c|p{11cm}|}
		\hline
		\textbf{Parametername} & \textbf{Parameterbeschreibung} \\ \hline
		elementid & Identifikator des HTML-Elementes\\ \hline
		state & Status der gesetzt werden soll\\ \hline
		tootip & Tooltipp der angezeigt werden soll\\ \hline
	\end{tabular}
\end{table}
\paragraph{Beschreibung} Die Funktion setzt die Darstellung eines Fehler-Status.
\subsubsection{getCookie}
\paragraph{Parameter} Die Funktion besitzt folgende Parameter:
\begin{table}[H]
	\begin{tabular}{|c|p{11cm}|}
		\hline
		\textbf{Parametername} & \textbf{Parameterbeschreibung} \\ \hline
		name & Name eines Cookies \\ \hline
	\end{tabular}
\end{table}
\paragraph{Beschreibung} Die Funktion fragt den Wert eines Cookies ab.
\subsubsection{testCookie}
\paragraph{Parameter} Die Funktion besitzt folgende Parameter:
\begin{table}[H]
	\begin{tabular}{|c|p{11cm}|}
		\hline
		\textbf{Parametername} & \textbf{Parameterbeschreibung} \\ \hline
		name & Name eines Cookies \\ \hline
	\end{tabular}
\end{table}
\paragraph{Beschreibung} Die Funktion prüft, ob ein bestimmter Cookie gesetzt ist.
\subsubsection{setCookie}
\paragraph{Parameter} Die Funktion besitzt folgende Parameter:
\begin{table}[H]
	\begin{tabular}{|c|p{11cm}|}
		\hline
		\textbf{Parametername} & \textbf{Parameterbeschreibung} \\ \hline
		name   & Name des Cookies \\ \hline
		value  & Wert beziehungsweise Inhalt des Cookies \\ \hline
		exdays & Ablaufdatum \\ \hline
	\end{tabular}
\end{table}
\paragraph{Beschreibung} Die Funktion setzt einen Cookie.
\subsubsection{deleteCookie}
\paragraph{Parameter} Die Funktion besitzt folgende Parameter:
\begin{table}[H]
	\begin{tabular}{|c|p{11cm}|}
		\hline
		\textbf{Parametername} & \textbf{Parameterbeschreibung} \\ \hline
		name & Name eines Cookies \\ \hline
	\end{tabular}
\end{table}
\paragraph{Beschreibung} Die Funktion löscht einen gesetzten Cookie.