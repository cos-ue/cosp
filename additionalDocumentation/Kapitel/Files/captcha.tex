\subsection{Allgemeines} Diese Datei enthält alle Funktionen um ein Captcha zu generieren.
\begin{table}[H]
	\begin{tabular}{|c|p{11cm}|}
		\hline
		\textbf{Einbindungspunkt} & inc.php \\ \hline
		\textbf{Einbindungspunkt} & inc-sub.php \\ \hline
	\end{tabular}
\end{table}
Die Datei ist nicht direkt durch den Nutzer aufrufbar, dies wird durch folgenden Code-Ausschnitt sichergestellt:
\begin{lstlisting}[language=php]
if (!defined('NICE_PROJECT')) {
	die('Permission denied.');
}
\end{lstlisting}
Der Globale Wert {\glqq NICE\_PROJECT\grqq} wird durch für den Nutzer valide Aufrufpunkte festgelegt, z.B. {\glqq api.php\grqq}.
\subsection{Funktionen}
\subsubsection{generateCaptcha}
\paragraph{Parameter} Die Funktion besitzt folgende Parameter:
\begin{table}[H]
	\begin{tabular}{|c|p{11cm}|}
		\hline
		\textbf{Parametername} & \textbf{Parameterbeschreibung} \\ \hline
		\$special & Flag zum Nutzen von Sonderzeichen \\ \hline
		\$ext     & Flag für externen Aufruf durch Modul \\ \hline
	\end{tabular}
\end{table}
\paragraph{Beschreibung} Die Funktion generiert ein Captcha. Die Antwort wird als strukturiertes Array an den Aufrufer zurückgegeben. Das Captcha wird Base64 codiert.
