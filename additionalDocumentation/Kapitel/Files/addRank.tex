\lstset{
	language=JavaScript,
	extendedchars=true,
	basicstyle= \small\ttfamily,
	showstringspaces=true,
	showspaces=false,
	tabsize=2,
	breaklines=true,
	showtabs=false,
	captionpos=b,
	showlines=true,
	xleftmargin=4.0ex,
	extendedchars=true,
	literate={ä}{{\"a}}1 {ö}{{\"o}}1 {ü}{{\"u}}1 {Ä}{{\"A}}1 {Ö}{{\"O}}1 {Ü}{{\"U}}1,
	breaklines=true,
	postbreak=\mbox{\textcolor{red}{$\hookrightarrow$}\space},
}
\subsection{Allgemeines} Diese Datei enthält alle Funktionen die zusätzlich auf der Seite des Rangmanagements benötigt werden.
Es wird auch die benötigte Variable {\glqq ranknames\grqq} mittels folgendem Code gesetzt:
\begin{lstlisting}[language=JavaScript]
var ranknames = sendApiRequest({type: "grn"}, false).data;
\end{lstlisting}
\subsection{Funktionen}
\subsubsection{checkRankName}
\paragraph{Parameter} Die Funktion besitzt folgende Parameter:
\begin{table}[H]
	\begin{tabular}{|c|p{11cm}|}
		\hline
		\textbf{Parametername} & \textbf{Parameterbeschreibung} \\ \hline
		modify & Wenn Wahr, existiert Rang bereits \\ \hline
	\end{tabular}
\end{table}
\paragraph{Beschreibung} Die Funktion prüft, ob ein Rangname bereits verwendet wird, sofern dieser geändert oder neu gesetzt wird. Die Funktion nutzt folgende Quellen:
\begin{itemize}
	\item Management-API
\end{itemize}
\subsubsection{openModuleBasedRankName}
\paragraph{Parameter} Die Funktion besitzt folgende Parameter:
\begin{table}[H]
	\begin{tabular}{|c|p{11cm}|}
		\hline
		\textbf{Parametername} & \textbf{Parameterbeschreibung} \\ \hline
		id & Identifikator eines Ranges \\ \hline
	\end{tabular}
\end{table}
\paragraph{Beschreibung} Die Funktion lädt alle Daten welche zum Öffnen des Modals zur Verwaltung von Modulbasierten Rangnamen benötigt wird. Die Funktion nutzt folgende Quellen:
\begin{itemize}
	\item Management-API
\end{itemize}
\subsubsection{saveModulBasedRankName}
\paragraph{Parameter} Die Funktion besitzt keine Parameter.
\paragraph{Beschreibung} Die Funktion speichert einen neuen modulbasierten Rangnamen. Die Funktion hat Auswirkungen auf folgende Quellen:
\begin{itemize}
	\item Management-API
\end{itemize}
\subsubsection{deleteModuleBasedRankName}
\paragraph{Parameter} Die Funktion besitzt folgende Parameter:
\begin{table}[H]
	\begin{tabular}{|c|p{11cm}|}
		\hline
		\textbf{Parametername} & \textbf{Parameterbeschreibung} \\ \hline
		id & Identifikator eines modulbasierten Rangnamens \\ \hline
	\end{tabular}
\end{table}
\paragraph{Beschreibung} Die Funktion löscht einen neuen modulbasierten Rangnamen. Die Funktion hat Auswirkungen auf folgende Quellen:
\begin{itemize}
	\item Management-API
\end{itemize}