\subsection{Allgemeines} Diese Datei ermöglicht es, Session-Daten innerhalb einer Datenbank zu speichern und initialisiert die Session.
\begin{table}[H]
	\begin{tabular}{|c|p{11cm}|}
		\hline
		\textbf{Einbindungspunkt} & inc.php \\ \hline
		\textbf{Einbindungspunkt} & inc-sub.php \\ \hline
	\end{tabular}
\end{table}
Die Datei ist nicht direkt durch den Nutzer aufrufbar, dies wird durch folgenden Code-Ausschnitt sichergestellt:
\begin{lstlisting}[language=php]
if (!defined('NICE_PROJECT')) {
	die('Permission denied.');
}
\end{lstlisting}
Der Globale Wert {\glqq NICE\_PROJECT\grqq} wird durch für den Nutzer valide Aufrufpunkte festgelegt, z.B. {\glqq api.php\grqq}.
\newpage
\subsection{Allgemeines}
Der Sessionstart erfolgt mittels nachfolgendem Code:
\begin{lstlisting}[language=php]
session_set_save_handler('ses_open', 'ses_close', 'ses_read', 'ses_write', 'ses_destroy', 'ses_gc');
register_shutdown_function('session_write_close');
session_start();
\end{lstlisting}
Desweiteren wird auch die Variable \$LOGIN gesetzt:
\begin{lstlisting}[language=php]
$LOGIN = false;
if (isset($_SESSION)) {
	if (isset($_SESSION['name'])) {
		$LOGIN = true;
	}
}
\end{lstlisting}
\subsection{Besonderheiten}
\newpage
\subsection{Funktionen}
\subsubsection{ses\_open}
\paragraph{Parameter} Die Funktion besitzt folgende Parameter:
\begin{table}[H]
	\begin{tabular}{|c|p{11cm}|}
		\hline
		\textbf{Parametername} & \textbf{Parameterbeschreibung} \\ \hline
		\$path & irgendein Pfad (wird nicht genutzt, aber benötigt) \\ \hline
		\$name & Irgendein Name (wird nicht genutzt, aber benötigt) \\ \hline
	\end{tabular}
\end{table}
\paragraph{Beschreibung} Die Funktion eröffnet eine Session und gibt stets {\glqq true\grqq} zurück.
\subsubsection{ses\_close}
\paragraph{Parameter} Die Funktion besitzt keine Parameter.
\paragraph{Beschreibung} Die Funktion schließt eine Session und gibt stets {\glqq true\grqq} zurück.
\subsubsection{ses\_read}
\paragraph{Parameter} Die Funktion besitzt folgende Parameter:
\begin{table}[H]
	\begin{tabular}{|c|p{11cm}|}
		\hline
		\textbf{Parametername} & \textbf{Parameterbeschreibung} \\ \hline
		\$ses\_id & Identifikator einer Session \\ \hline
	\end{tabular}
\end{table}
\paragraph{Beschreibung} Die Funktion liest eine Session aus der Datenbank. Die Funktion nutzt folgende Quellen:
\begin{itemize}
	\item Tabelle mit Sessiondaten
\end{itemize}
Die Antwort wird als Zeichenkette an den Aufrufer zurückgegeben.
\subsubsection{ses\_write}
\paragraph{Parameter} Die Funktion besitzt folgende Parameter:
\begin{table}[H]
	\begin{tabular}{|c|p{11cm}|}
		\hline
		\textbf{Parametername} & \textbf{Parameterbeschreibung} \\ \hline
		\$ses\_id & Identifikator der Session \\ \hline
		\$data    & Daten der Session \\ \hline
	\end{tabular}
\end{table}
\paragraph{Beschreibung} Die Funktion schreibt die Sessiondaten in die Datenbank. Die Funktion hat Auswirkungen auf folgende Quellen:
\begin{itemize}
	\item Tabelle mit Sessiondaten.
\end{itemize}
Der Rückgabewert ist stets {\glqq true\grqq}.
\subsubsection{ses\_destroy}
\paragraph{Parameter} Die Funktion besitzt folgende Parameter:
\begin{table}[H]
	\begin{tabular}{|c|p{11cm}|}
		\hline
		\textbf{Parametername} & \textbf{Parameterbeschreibung} \\ \hline
		\$ses\_id & Identifikator der Session \\ \hline
	\end{tabular}
\end{table}
\paragraph{Beschreibung} Die Funktion löscht die Sessiondaten aus der Datenbank. Die Funktion hat Auswirkungen auf folgende Quellen:
\begin{itemize}
	\item Tabelle mit Sessiondaten.
\end{itemize}
Der Rückgabewert ist stets {\glqq true\grqq}.
\subsubsection{ses\_gc}
\paragraph{Parameter} Die Funktion besitzt folgende Parameter:
\begin{table}[H]
	\begin{tabular}{|c|p{11cm}|}
		\hline
		\textbf{Parametername} & \textbf{Parameterbeschreibung} \\ \hline
		\$life & Lebenszeit der Sessions in Sekunden \\ \hline
	\end{tabular}
\end{table}
\paragraph{Beschreibung} Die Funktion löscht alle Sessiondaten aus der Datenbank, welche älter das jetzt minus die angegebenen Sekunden sind. Die Funktion hat Auswirkungen auf folgende Quellen:
\begin{itemize}
	\item Tabelle mit Sessiondaten.
\end{itemize}
Der Rückgabewert ist stets {\glqq true\grqq}.
\subsubsection{checkLoginDeny}
\paragraph{Parameter} Die Funktion besitzt folgende Parameter:
\begin{table}[H]
	\begin{tabular}{|c|p{11cm}|}
		\hline
		\textbf{Parametername} & \textbf{Parameterbeschreibung} \\ \hline
		\$login & Status des Login eines Nutzers \\ \hline
	\end{tabular}
\end{table}
\paragraph{Beschreibung} Die Funktion beendet die Ausführung des PHP, wenn der Loginstatus nicht korrekt gesetzt ist. Die Antwort wird als Boolean an den Aufrufer zurückgegeben.
