\newpage
\section{Nutzer API-Spezifikation}\label{uapi}
\subsection{Beschreibung}Diese API dient der Kommunikation zwischen einem Nutzer eines Moduls und den Inhalten die durch {\glqq COSP\grqq} bereit gestellt werden. Hauptsächlich unterstützt sie das Abrufen von Bildern und anderen Informationen. Durch diese API können keine Informationen geändert, gelöscht oder neu hinzugefügt werden. Jedoch können Geschichten und Bilder validiert werden. Diese API wird durch ein HMAC-Verfahren geschützt.
\subsection{Befehlsübersicht}
\begin{longtable}[H]{|c|p{12cm}|}
		\hline
		\textbf{Api-Befehl} & \textbf{Kurzbeschreibung}              \\ \hline
		gpp                 & Vollbild laden          \\ \hline
		gpf                 & Vorschaubild laden            \\ \hline
		gus                 & Geschichte laden \\ \hline
		gas                 & Alle Geschichten einer Liste laden \\ \hline
		vas                 & Geschichte Validieren \\ \hline
		vap                 & Bild validieren \\ \hline
\end{longtable}
\newpage
\subsection{Befehle}
\subsubsection{Vollbildladen}
\paragraph{Kurzbeschreibung}Dieser API-Request wird dazu genutzt um ein Vollbild zu laden.
\paragraph{Anfrage}Folgende Daten werden zu Anfrage benötigt:
\begin{table}[H]
	\begin{tabular}{|c|c|c|p{6.5cm}|}
		\hline
		\textbf{Paramtername} & \textbf{Datentyp} & \textbf{Konstante} & \textbf{Kurzbeschreibung}                                                                                               \\ \hline
		type                & string            & gpp                & Vollbild abrufen \\ \hline
		data                & string            &                    & Token \\ \hline
		seccode             & string            &                    & Security Code \\ \hline
		time                & int               &                    & Timestamp \\ \hline
	\end{tabular}
\end{table}
\paragraph{Antwort}Die Antwort ist das Bild mit einem entsprechendem Header.
\subsubsection{Vorschaubild laden}
\paragraph{Kurzbeschreibung}Dieser API-Request wird dazu genutzt um ein Vorschaubild zu laden.
\paragraph{Anfrage}Folgende Daten werden zu Anfrage benötigt:
\begin{table}[H]
	\begin{tabular}{|c|c|c|p{6.5cm}|}
		\hline
		\textbf{Paramtername} & \textbf{Datentyp} & \textbf{Konstante} & \textbf{Kurzbeschreibung}                                                                                               \\ \hline
		type                & string            & gpf                & Vorschaubild abrufen \\ \hline
		data                & string            &                    & Token \\ \hline
		seccode             & string            &                    & Security Code \\ \hline
		time                & int               &                    & Timestamp \\ \hline
	\end{tabular}
\end{table}
\paragraph{Antwort}Die Antwort ist das Bild mit einem entsprechendem Header.

\subsubsection{Geschichte laden}
\paragraph{Kurzbeschreibung}Dieser API-Request wird dazu genutzt um eine einzelne Geschichte zu laden.
\paragraph{Anfrage}Folgende Daten werden zu Anfrage benötigt:
\begin{table}[H]
	\begin{tabular}{|c|c|c|p{6.5cm}|}
		\hline
		\textbf{Paramtername} & \textbf{Datentyp} & \textbf{Konstante} & \textbf{Kurzbeschreibung}                                                                                               \\ \hline
		type                & string            & gus                & Geschichte abrufen \\ \hline
		data                & string            &                    & Token \\ \hline
		seccode             & string            &                    & Security Code \\ \hline
		time                & int               &                    & Timestamp \\ \hline
	\end{tabular}
\end{table}
\paragraph{Antwort}Die Antwort ist wie folgt aufgebaut:
\begin{table}[H]
	\begin{tabular}{|c|c|c|p{6.5cm}|}
		\hline
		\textbf{Paramtername} & \textbf{Datentyp} & \textbf{Konstante} & \textbf{Kurzbeschreibung}            \\ \hline                
		result              & string           &                 & Erfolgreich wenn Wert {\glqq ack\grqq} ist \\ \hline
		Code                & int              &                 & Erfolgreich wenn Wert {\glqq 0\grqq} ist \\ \hline
		data                & array            &                 & Abgefragter Inhalt \\ \hline
	\end{tabular}
\end{table}
\subparagraph{data}Dieses Array enthält Einträge in der nachstehend dargestellten Form haben:
\begin{table}[H]
	\begin{tabular}{|c|c|c|p{6.5cm}|}
		\hline
		\textbf{Paramtername} & \textbf{Datentyp} & \textbf{Konstante} & \textbf{Kurzbeschreibung}    \\ \hline
		token              & string            &                 & Identifikator der Geschichte \\ \hline
		story              & string            &                 & Inhalt der Geschichte \\ \hline
		title              & string            &                 & Titel der Geschichte \\ \hline
		name               & string            &                 & Nutzername des Erstellers \\ \hline
		date               & timestamp         &                 & Erstellungsdatum \\ \hline
		validatedByUser    & bool              &                 & Wahr, wenn Nutzer bereits Geschichte validiert hat \\ \hline
		validate           & bool              &                 & Wahr, Geschichte validiert ist \\ \hline
		valLink            & string            &                 & Validierungslink \\ \hline
		approval           & bool              &                 & Freischaltstatus \\ \hline
		editable           & bool              &                 & Wahr, wenn Geschichte änderbar \\ \hline
		deleted            & bool              &                 & Wahr, wenn Geschichte als gelöscht gilt \\ \hline
	\end{tabular}
\end{table}

\subsubsection{Alle Geschichten einer Liste laden}
\paragraph{Kurzbeschreibung}Dieser API-Request wird dazu genutzt um  alle Geschichten einer Liste zu laden.
\paragraph{Anfrage}Folgende Daten werden zu Anfrage benötigt:
\begin{table}[H]
	\begin{tabular}{|c|c|c|p{6.5cm}|}
		\hline
		\textbf{Paramtername} & \textbf{Datentyp} & \textbf{Konstante} & \textbf{Kurzbeschreibung}                                                                                               \\ \hline
		type                & string            & gas                & Geschichten abrufen \\ \hline
		data                & string            &                    & Token \\ \hline
		seccode             & string            &                    & Security Code \\ \hline
		time                & int               &                    & Timestamp \\ \hline
	\end{tabular}
\end{table}
\paragraph{Antwort}Die ist ein Array mit Elementen, welche folgende Einträge haben:
\begin{table}[H]
	\begin{tabular}{|c|c|c|p{6.5cm}|}
		\hline
		\textbf{Paramtername} & \textbf{Datentyp} & \textbf{Konstante} & \textbf{Kurzbeschreibung}            \\ \hline                
		token              & string            &                 & Identifikator der Geschichte \\ \hline
		story              & string            &                 & Inhalt der Geschichte \\ \hline
		title              & string            &                 & Titel der Geschichte \\ \hline
		name               & string            &                 & Nutzername des Erstellers \\ \hline
		date               & timestamp         &                 & Erstellungsdatum \\ \hline
		validatedByUser    & bool              &                 & Wahr, wenn Nutzer bereits Geschichte validiert hat \\ \hline
		validate           & bool              &                 & Wahr, Geschichte validiert ist \\ \hline
		valLink            & string            &                 & Validierungslink \\ \hline
		approval           & bool              &                 & Freischaltstatus \\ \hline
		editable           & bool              &                 & Wahr, wenn Geschichte änderbar \\ \hline
		deleted            & bool              &                 & Wahr, wenn Geschichte als gelöscht gilt \\ \hline
	\end{tabular}
\end{table}

\subsubsection{Geschichte validieren}
\paragraph{Kurzbeschreibung}Dieser API-Request wird dazu genutzt um eine einzelne Geschichte zu validieren.
\paragraph{Anfrage}Folgende Daten werden zu Anfrage benötigt:
\begin{table}[H]
	\begin{tabular}{|c|c|c|p{6.5cm}|}
		\hline
		\textbf{Paramtername} & \textbf{Datentyp} & \textbf{Konstante} & \textbf{Kurzbeschreibung}                                                                                               \\ \hline
		type                & string            & vas                & Geschichte validieren \\ \hline
		data                & string            &                    & Token \\ \hline
		seccode             & string            &                    & Security Code \\ \hline
		time                & int               &                    & Timestamp \\ \hline
	\end{tabular}
\end{table}
\paragraph{Antwort}Die Antwort ist wie folgt aufgebaut:
\begin{table}[H]
	\begin{tabular}{|c|c|c|p{6.5cm}|}
		\hline
		\textbf{Paramtername} & \textbf{Datentyp} & \textbf{Konstante} & \textbf{Kurzbeschreibung}            \\ \hline                
		result              & string           &                 & Erfolgreich wenn Wert {\glqq ack\grqq} ist \\ \hline
		Code                & int              &                 & Erfolgreich wenn Wert {\glqq 0\grqq} ist \\ \hline
	\end{tabular}
\end{table}

\subsubsection{Bild validieren}
\paragraph{Kurzbeschreibung}Dieser API-Request wird dazu genutzt um eine einzelnes Bild zu validieren.
\paragraph{Anfrage}Folgende Daten werden zu Anfrage benötigt:
\begin{table}[H]
	\begin{tabular}{|c|c|c|p{6.5cm}|}
		\hline
		\textbf{Paramtername} & \textbf{Datentyp} & \textbf{Konstante} & \textbf{Kurzbeschreibung}                                                                                               \\ \hline
		type                & string            & vap                & Bild validieren \\ \hline
		data                & string            &                    & Token \\ \hline
		seccode             & string            &                    & Security Code \\ \hline
		time                & int               &                    & Timestamp \\ \hline
	\end{tabular}
\end{table}
\paragraph{Antwort}Die Antwort ist wie folgt aufgebaut:
\begin{table}[H]
	\begin{tabular}{|c|c|c|p{6.5cm}|}
		\hline
		\textbf{Paramtername} & \textbf{Datentyp} & \textbf{Konstante} & \textbf{Kurzbeschreibung}            \\ \hline                
		result              & string           &                 & Erfolgreich wenn Wert {\glqq ack\grqq} ist \\ \hline
		Code                & int              &                 & Erfolgreich wenn Wert {\glqq 0\grqq} ist \\ \hline
	\end{tabular}
\end{table}